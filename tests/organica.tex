\documentclass{article}

\usepackage{polyglossia}
    \setdefaultlanguage{spanish}

\usepackage{fontspec}
    \setmainfont{Fira Sans}

\usepackage{chemfig}
\usepackage{chemformula}

\newcounter{gexformulation}

\newenvironment{gemyboxed}
  {\centering\setcounter{gexformulation}{1}\tabularx{\textwidth}{rlXX}
    \toprule\midrule}
  {\endtabularx}

\newenvironment{gorganictable}
  {\centering\setcounter{gexformulation}{1}\tabularx{\textwidth}{rlX}
    \toprule\midrule}
  {\endtabularx}

\newcommand{\gexbinary}[3]{
  \thegexformulation\stepcounter{gexformulation} & \ch{#1} & #2 & #3 \\ \midrule
}
\newcommand{\gexTernary}[4]{
  \thegexformulation\stepcounter{gexformulation} & \ch{#1} & #2 & #3 & #4 \\ \midrule
}

\newcommand{\gexOrgCf}[2]{
  \thegexformulation\stepcounter{gexformulation} & {#1} & \ch{#2} \\ \midrule
}

\newcommand{\gexOrgCh}[2]{
  \thegexformulation\stepcounter{gexformulation} & {#1} & \chemfig{#2} \\ \midrule
}

  \setchemfig{atom sep=2em}  % para una versión más moderna de chemfig

\usepackage{tabularx}
\usepackage{booktabs}



\begin{document}

\begin{table}
  \caption{Alcanos}
  \begin{gorganictable}
    \gexOrgCh{metano}{CH_4}
    \gexOrgCh{etano}{CH_3-CH_3}
    \gexOrgCh{propano}{CH_3-CH_2-CH_3}
    \gexOrgCh{butano}{CH_3-CH_2-CH_2-CH_3}
    \gexOrgCh{pentano}{CH_3-CH_2-CH_2-CH_2-CH_3}
    \gexOrgCh{hexano}{CH_3-CH_2-CH_2-CH_2-CH_2-CH_3}
    \gexOrgCh{heptano}{CH_3-CH_2-CH_2-CH_2-CH_2-CH_2-CH_3}
    \bottomrule
  \end{gorganictable}
\end{table}

\begin{table}
  \caption{Alquenos}
  \begin{gorganictable}
    \gexOrgCh{eteno}{CH_2=CH_2}
    \gexOrgCh{propeno}{CH_2=CH-CH_3}
    \gexOrgCh{pen-2-eno}{CH_3-CH=CH-CH_2-CH_3}
    \gexOrgCh{penta-1,3-dieno}{CH_2=CH-CH=CH-CH_3}
    \bottomrule
  \end{gorganictable}
\end{table}


\begin{table}
  \caption{Alquinos}
  \begin{gorganictable}
    \gexOrgCh{etino}{CH~CH}
    \gexOrgCh{propino}{CH_3-C~CH}
    \gexOrgCh{hepta-1,3,5-triino}{formula}
    \gexOrgCh{pent-2-ino}{CH_3-C~C-CH_2-CH_3}
    \gexOrgCh{pent-3-en-1-ino}{CH~C-CH=CH-CH_3}
    \bottomrule
  \end{gorganictable}
\end{table}


\begin{table}
  \caption{Aromáticos}
  \begin{gorganictable}
    \gexOrgCh{benceno}{*6(-=-=-=)}
    \gexOrgCh{benceno}{**6(------)}
    \gexOrgCh{naftaleno}{*6(-=*6(-=-=--)-=-=)}
    \gexOrgCh{naftaleno}{**6(--**6(------)----)}
    \gexOrgCh{antraceno}{**6(--**6(--**6(------)----)----)}
    \bottomrule
  \end{gorganictable}
\end{table}

\begin{table}
  \caption{Grupos ramificados habituales}
  \begin{gorganictable}
    \gexOrgCh{metilo}{-CH_3}
    \gexOrgCh{etilo}{-CH_2-CH_3}
    \gexOrgCh{propilo}{-CH_2-CH_2-CH_3}
    \gexOrgCh{isopropilo}{-CH(-[7]CH_3)-[1]CH_3}
    \gexOrgCh{isobutilo}{-CH_2-CH(-[7]CH_3)-[1]CH_3}
    \gexOrgCh{\textit{terc}-butilo}{-C(-[2]CH_3)(-[6]CH_3)-CH_3}
    \gexOrgCh{etenilo (vinilo)}{-CH=CH_2}
    \gexOrgCh{prop-2-enilo (alilo)}{-CH_2-CH=CH_2}
    \gexOrgCh{fenilo}{-**6(------)}
    \bottomrule
  \end{gorganictable}
\end{table}

\begin{table}
  \caption{Alcanos ramificados}
  \begin{gorganictable}
    \gexOrgCh{metilpropano (isobutano)}{CH_3-CH_2(-[6]CH_2)-CH_3}
    \gexOrgCh{2-metilbutano}{CH_3-CH_2(-[6]CH_3)-CH_2-CH_3}
    \gexOrgCh{2,2-dimetilbutano}{CH_3-CH_2(-[2]CH_3)(-[6]CH_3)-CH_2-CH_3}
    \bottomrule
  \end{gorganictable}
\end{table}

\begin{table}
  \caption{Hidrocarburos aromáticos ramificados}
  \begin{gorganictable}
    \gexOrgCh{metilbenceno (tolueno)}{**6(---(-CH_3)---)}
    \gexOrgCh{1,2-dimetilbenceno \textit{o}-dimetilbenceno (1,2-xileno)}{**6(--(-CH_3)-(-CH_3)---)}
    \gexOrgCh{1,3-dimetilbenceno \textit{m}-dimetilbenceno (1,3-xileno)}{**6(-(-CH_3)--(-CH_3)---)}
    \gexOrgCh{1,4-dimetilbenceno \textit{p}-dimetilbenceno (1,4-xileno)}{**6((-CH_3)---(-CH_3)---)}
    \bottomrule
  \end{gorganictable}
\end{table}

\section{Ejercicios de SANTILLANA}

\begin{table}
  \caption{EJ. 2, página 33}
  \begin{gorganictable}
    \gexOrgCh{metano}{CH_4}
    \gexOrgCh{etino}{CH~CH}
    \gexOrgCh{propino}{CH_3-C~CH}
    \gexOrgCh{ciclohexano}{*6(------)}
    \gexOrgCh{propano}{CH_3-CH_2-CH_3}
    \gexOrgCh{buta-1,3-dieno}{CH_2=CH-CH=CH_2}
    \gexOrgCh{hepta-1,3,5-triino}{CH_3-C~C-C~C-C~CH}
    \gexOrgCh{ciclohepta-2,3-dieno}{*7(-=---=-)}
    \gexOrgCh{hepta-1,5-diino}{CH_3-C~C-CH_2-CH_2-C~CH}
    \gexOrgCh{hepta-2,4-dieno}{CH_3-CH=CH-CH=CH-CH_2-CH_3}
    \gexOrgCh{hepta-1,6-dien-3-ino}{CH_2=CH-C~C-CH_2-CH=CH_2}
    \gexOrgCh{ciclopropano}{*3(---)}
    \gexOrgCh{hept-2-ino}{CH_3-C~C-CH_2-CH_2-CH_2-CH_3}
    \gexOrgCh{hept-3-eno}{CH_3-CH_2-CH_2-CH=CH-CH_2-CH_3}
    \gexOrgCh{heptano}{CH_3-CH_2-CH_2-CH_2-CH_2-CH_2-CH_3}
    \gexOrgCh{antraceno}{**6(--**6(--**6(------)----)----)}
    \bottomrule
  \end{gorganictable}
\end{table}

\begin{table}
  \caption{EJ. 3, página 34}
  \begin{gorganictable}
    \gexOrgCh{octano}{CH_3-CH_2-CH_2-CH_2-CH_2-CH_2-CH_2-CH_3}
    \gexOrgCh{benceno}{}
    \gexOrgCh{decano}{CH_3-CH_2-CH_2-CH_2-CH_2-CH_2-CH_2-CH_2-CH_2-CH_3}
    \gexOrgCh{nona-2,4,6-triino}{CH_3-C~C-C~C-C~C-CH_2-CH_3}
    \gexOrgCh{but-2-ino}{CH_3-C~C-CH_3}
    \gexOrgCh{ciclohexa-1,3-diino}{*6(-~---~-)}
    \gexOrgCh{naftaleno}{}
    \gexOrgCh{ciclohepta-1,3-dieno}{*7(=-=----)}
    \gexOrgCh{octa-1,5-diino}{CH~C-CH_2-CH_2-C~C-CH_2-CH_3}
    \gexOrgCh{butano}{}
    \gexOrgCh{ciclobuteno}{*4(--=-)}
    \gexOrgCh{eteno}{CH_2=CH_2}
    \gexOrgCh{ciclopenteno}{*5(--=--)}
    \gexOrgCh{propeno}{}
    \gexOrgCh{hexano}{}
    \gexOrgCh{hept-1-en-5-ino}{CH_2=CH-CH_2-CH_2-C~C-CH_3}
    \bottomrule
  \end{gorganictable}
\end{table}

\begin{table}
  \caption{EJ. 5, página 36}
  \begin{gorganictable}
    \gexOrgCh{}{CH_2(-[4]*3(---))-CH(-[6]CH_3)-CH_3}
    \gexOrgCh{}{CH~C-**6(------)}
    \gexOrgCh{}{CH_3-CH_2-CH_2-C(=[2]CH_2)-C~CH}
    \bottomrule
  \end{gorganictable}
\end{table}


\end{document}
