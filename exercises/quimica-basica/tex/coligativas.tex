\section{Propiedades coligativas}

\begin{exercise}[
    tags    = {termodinámica, entalpía, entalpia de reacción, calor},
    topics  = {química, termoquímica, termodinámica},
    source  = {FQ 1B OXF 2015, p79, e41},
  ]
  Calcula la temperatura de congelación de una disolución formada por \SI{9.5}{\gram} de etilenglicol (anticongelante usado en los automóviles cuya fórmula es \ch{CH2OH-CH2OH}) y \SI{20}{\gram} de agua.
\end{exercise}

\begin{solution}
  \SI{-14,25}{\celsius}
\end{solution}



\begin{exercise}[
    tags    = {termodinámica, entalpía, entalpia de reacción, calor},
    topics  = {química, termoquímica, termodinámica},
    source  = {FQ 1B SAN 2015, p94, e51},
  ]
  Cuál será, a \SI{80}{\celsius}, la presión de vapor de una disolución que se prepara disolviendo \SI{30}{\milli\liter} de glicerina, \ch{C3H8O3}, en \SI{70}{\milli\liter} de agua.

  \begin{gexdatos}
    \( p_{\mathrm{agua}}(\SI{80}{\celsius}) = \SI{355}{\mmHg} \),
    \( d_{\mathrm{agua}} = \SI{1}{\gram\per\milli\liter} \),
    \( d_{\mathrm{glicerina}} = \SI{1.26}{\gram\per\milli\liter} \). % REVIEW revisar si mathrm formatea bien el texto en recto
  \end{gexdatos}
\end{exercise}

\begin{solution}
  \SI{321}{\mmHg}
\end{solution}




\begin{exercise}[
    tags    = {termodinámica, entalpía, entalpia de reacción, calor},
    topics  = {química, termoquímica, termodinámica},
    source  = {FQ 1B OXF 2015, p79, e44},
  ]
  Suponiendo un comportamiento ideal, ¿cuál sería la presión de vapor de la disolución obtenida al mezclar \SI{500}{\milli\liter} de agua y \SI{90}{\gram} de glucosa (\ch{C6H12O6}) si la presión de vapor del agua a la temperatura de la mezcla es de \SI{55.3}{\mmHg}?
\end{exercise}

\begin{solution}
  \SI{54.32}{\mmHg}
\end{solution}




\begin{exercise}[
    tags    = {termodinámica, entalpía, entalpia de reacción, calor},
    topics  = {química, termoquímica, termodinámica},
    source  = {FQ 1B OXF 2015, p79, e46},
  ]
  Un cierto compuesto contiene 43,2\% de \ch{C}, 16,6\% de \ch{N}, 2,4\% de \ch{H} y 37,8\% de \ch{O}. La adición de \SI{6.45}{\gram} de esa sustancia en \SI{50}{\milli\liter} de benceno (\ch{C6H6}), cuya densidad es \SI{0.88}{\gram\per\cubic\centi\meter}, hace bajar el punto de congelación del benceno de \SI{5.51}{\celsius} a \SI{1.25}{\celsius}. Halla la fórmula molecular de ese compuesto.

  \begin{gexdatos}
    \( K_c (\ch{C6H6}) = \SI{5.02}{\celsius\kilo\gram\per\mole} \)
  \end{gexdatos}
\end{exercise}

\begin{solution}
  \ch{C6N2O4H4}
\end{solution}



\begin{exercise}[
    tags    = {termodinámica, entalpía, entalpia de reacción, calor},
    topics  = {química, termoquímica, termodinámica},
    source  = {FQ 1B SAN 2015, p94, e54},
  ]
  Determina la masa molar de una sustancia si al disolver \SI{17}{\gram} de la misma en \SI{150}{\gram} de benceno se obtiene una mezcla que se congela a \SI{-4}{\celsius}.

  \begin{gexdatos}
    \( K_c (\ch{C6H6}) = \SI{5.07}{\celsius\kilo\gram\per\mole} \),
    \( T_f (\ch{C6H6}) = \SI{6}{\celsius} \).
  \end{gexdatos}
\end{exercise}

\begin{solution}
  \SI{57.46}{\gram\per\mole}
\end{solution}




\begin{exercise}[
    tags    = {termodinámica, entalpía, entalpia de reacción, calor},
    topics  = {química, termoquímica, termodinámica},
    source  = {FQ 1B OXF 2015, p79, e48},
  ]
  La presión osmótica de una disolución es \SI{4.2}{atm} a \SI{20}{\celsius}. ¿Qué presión osmótica tendrá a \SI{50}{\celsius}?
\end{exercise}

\begin{solution}
  \SI{4.6}{atm}
\end{solution}




\begin{exercise}[
    tags    = {termodinámica, entalpía, entalpia de reacción, calor},
    topics  = {química, termoquímica, termodinámica},
    source  = {FQ 1B SAN 2015, p94, e59},
  ]
  La albúmina es una proteína del huevo. Calcula la masa
  molar de la albúmina si una disolución de \SI{50}{\gram} de albúmina
  por litro de agua ejerce una presión osmótica de
  \SI{14}{\mmHg} a \SI{25}{\celsius}.
\end{exercise}

\begin{solution}
  \SI{6.63e4}{\gram\per\mole}
\end{solution}




\begin{exercise}[
    tags    = {termodinámica, entalpía, entalpia de reacción, calor},
    topics  = {química, termoquímica, termodinámica},
    source  = {FQ 1B OXF 2015, p79, e50},
  ]
  Una muestra de \SI{2}{\gram} de un compuesto orgánico disuelto en \SI{100}{\cubic\centi\meter} de disolución se encuentra a una presión de \SI{1.31}{atm}, en el equilibrio osmótico. Sabiendo que la disolución está a \SI{0}{\celsius}, calcula la masa molar del compuesto orgánico.
\end{exercise}

\begin{solution}
  \SI{342}{\gram\per\mole}
\end{solution}
