\section{Disoluciones}

\begin{exercise}[
    tags    = {termodinámica, entalpía, entalpia de reacción, calor},
    topics  = {química, termoquímica, termodinámica},
    source  = {FQ 1B OXF 2015, p78, e9},
  ]
  Se disuelven \SI{10}{\gram} de sacarosa en \SI{250}{\gram} de agua. Indica la concentración de la disolución en:
  \begin{enumerate}
    \item Masa (\si{\gram}) de soluto/\SI{100}{\gram} de disolvente.
    \item Masa (\si{\gram}) de soluto/\SI{100}{\gram} de disolución.
  \end{enumerate}
\end{exercise}

\begin{solution}
  \begin{enumerate*}
    \item 4;
    \item 3,85.
  \end{enumerate*}
\end{solution}



\begin{exercise}[
    tags    = {termodinámica, entalpía, entalpia de reacción, calor},
    topics  = {química, termoquímica, termodinámica},
    source  = {FQ 1B SAN 2015, p92, e25},
  ]
  El suero fisiológico es una disolución de sal en agua al 0,9\%
  (porcentaje en masa). Calcula la cantidad de sal y de agua
  que necesitas para preparar \SI{2}{\kilo\gram} de suero fisiológico.
\end{exercise}

\begin{solution}
  \SI{18}{\gram} de sal y \SI{1982}{\gram} de agua.
\end{solution}




\begin{exercise}[
    tags    = {termodinámica, entalpía, entalpia de reacción, calor},
    topics  = {química, termoquímica, termodinámica},
    source  = {FQ 1B SAN 2015, p92, e28},
  ]
  Necesitamos preparar \SI{500}{\milli\liter} de una disolución de hidróxido de sodio \SI{2}{M}. Calcula qué cantidad de soluto necesitas y explica cómo la prepararás si se dispone de un producto comercial del 95\% de riqueza en \ch{NaOH}.
\end{exercise}

\begin{solution}
  Necesitas \SI{42.11}{\gram} de \ch{NaOH} comercial.
\end{solution}




\begin{exercise}[
    tags    = {termodinámica, entalpía, entalpia de reacción, calor},
    topics  = {química, termoquímica, termodinámica},
    source  = {FQ 1B SAN 2015, p92, e33},
  ]
  En el laboratorio tenemos un ácido clorhídrico del 37\% de riqueza en masa y \SI{1.18}{\gram\per\milli\liter} de densidad. Si cogemos \SI{70}{\milli\liter} del contenido de esa botella, ¿Cuánto ácido
  clorhídrico estaremos tomando?
\end{exercise}

\begin{solution}
  \SI{30.6}{\gram}.
\end{solution}




\begin{exercise}[
    tags    = {termodinámica, entalpía, entalpia de reacción, calor},
    topics  = {química, termoquímica, termodinámica},
    source  = {FQ 1B OXF 2015, p78, e11},
  ]
  Se prepara una disolución con \SI{5}{\gram} de \ch{NaOH} en \SI{25}{\gram} de agua destilada. Si el volumen final es de \SI{27.1}{\cubic\centi\meter}, calcula la concentración de la disolución en:
  \begin{enumerate}
    \item Porcentaje en masa.
    \item Masa (\si{\gram}) por litro.
    \item Molaridad.
    \item Molalidad.
  \end{enumerate}
\end{exercise}

\begin{solution}
  \begin{enumerate*}
    \item 16,7\%:
    \item \SI{184.5}{\gram\per\liter};
    \item \SI{4.6}{M};
    \item \SI{5}{m}.
  \end{enumerate*}
\end{solution}



\begin{exercise}[
    tags    = {termodinámica, entalpía, entalpia de reacción, calor},
    topics  = {química, termoquímica, termodinámica},
    source  = {FQ 1B SAN 2015, p94, e63},
  ]
  Se ha preparado una disolución mezclando \SI{100}{\milli\liter} de \ch{CaCl2} \SI{2}{M} con \SI{150}{\milli\liter} de \ch{NaCl} \SI{1.5}{M}.
  ¿Cuál será la concentración de los iones cloruro en la disolución resultante? Para calcularlo, supón que los volúmenes son aditivos.
\end{exercise}

\begin{solution}
    \SI{2.5}{M};
\end{solution}
