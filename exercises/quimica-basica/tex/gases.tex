\section{Leyes de los gases}

\begin{exercise}[
    tags    = {termodinámica, entalpía, entalpia de reacción, calor},
    topics  = {química, termoquímica, termodinámica},
    source  = {FQ 1B OXF 2015, p60, e46},
  ]
  Un gas ocupa un volumen de \SI{2}{\liter} en condiciones normales de presión y temperatura. ¿Qué volumen ocupará la misma masa de gas a \SI{2}{atm} de presión y \SI{50}{\celsius} de temperatura?
\end{exercise}

\begin{solution}
  \SI{1.18}{\liter}.
\end{solution}




\begin{exercise}[
    tags    = {termodinámica, entalpía, entalpia de reacción, calor},
    topics  = {química, termoquímica, termodinámica},
    source  = {FQ 1B OXF 2015, p60, e16},
  ]
  Un gas ocupa un volumen de \SI{80}{\cubic\centi\meter} a \SI{10}{\celsius} y \SI{715}{\mmHg}. ¿Qué volumen ocupará este gas en CN?
\end{exercise}

\begin{solution}
  \SI{72.6}{\cubic\centi\meter}.
\end{solution}




\begin{exercise}[
    tags    = {termodinámica, entalpía, entalpia de reacción, calor},
    topics  = {química, termoquímica, termodinámica},
    source  = {FQ 1B OXF 2015, p60, e19},
  ]
  La densidad de un gas en condiciones normales es \SI{1.48}{\gram\per\liter}. ¿Cuál será su densidad a \SI{320}{\kelvin} y \SI{730}{\mmHg}?
\end{exercise}

\begin{solution}
  \SI{1.21}{\gram\per\liter}.
\end{solution}




\begin{exercise}[
    tags    = {termodinámica, entalpía, entalpia de reacción, calor},
    topics  = {química, termoquímica, termodinámica},
    source  = {FQ 1B SAN 2015, p66, e37},
  ]
  En una ampolla de \SI{750}{\milli\liter} tenemos un gas que ejerce una presión de \SI{1.25}{atm} a \SI{50}{\celsius}. Lo conectamos a una segunda ampolla de \SI{2}{\liter}. ¿Qué presión leeremos ahora en el manómetro si no varía la temperatura?
\end{exercise}

\begin{solution}
  \SI{259}{\mmHg}.
\end{solution}



\begin{exercise}[
    tags    = {termodinámica, entalpía, entalpia de reacción, calor},
    topics  = {química, termoquímica, termodinámica},
    source  = {FQ 1B SAN 2015, p66, e39},
  ]
  Tenemos un gas encerrado en un recipiente rígido de \SI{5}{\liter}. ¿En cuánto cambia su temperatura si su presión pasa de \SI{300}{\mmHg} a \SI{600}{\mmHg}?
\end{exercise}

\begin{solution}
  Se duplica su temperatura absoluta.
\end{solution}




\begin{exercise}[
    tags    = {termodinámica, entalpía, entalpia de reacción, calor},
    topics  = {química, termoquímica, termodinámica},
    source  = {FQ 1B SAN 2015, p66, e39},
  ]
  Una pieza de una máquina está formada por un pistón que tiene un gas en su interior. En un momento dado, el volumen del pistón es de \SI{225}{\milli\liter} y la temperatura del gas es de \SI{50}{\celsius}. ¿Cuánto debe cambiar la temperatura para que el volumen sea de \SI{275}{\milli\liter} si la presión no varía?
\end{exercise}

\begin{solution}
  \( \Delta T = \SI{+71.7}{\kelvin} \)
\end{solution}




\begin{exercise}[
    tags    = {termodinámica, entalpía, entalpia de reacción, calor},
    topics  = {química, termoquímica, termodinámica},
    source  = {FQ 1B OXF 2015, p60, e22},
  ]
  Se dispone de \SI{45.0}{\gram} de metano (\ch{CH4}) a \SI{800}{\mmHg} y \SI{27}{\celsius}. Calcula:
  \begin{enumerate}
    \item El volumen que ocupa en las citadas condiciones.
    \item El número de moléculas existentes
  \end{enumerate}
\end{exercise}

\begin{solution}
  \begin{enumerate}
    \item \SI{66}{\liter}
    \item \SI{1.7e24}{moléculas}.
  \end{enumerate}
\end{solution}




\begin{exercise}[
    tags    = {termodinámica, entalpía, entalpia de reacción, calor},
    topics  = {química, termoquímica, termodinámica},
    source  = {FQ 1B OXF 2015, p61, e24-25},
  ]
  En un matraz de \SI{1}{\liter} están contenidos \SI{0.9}{\gram} de un gas a una temperatura de \SI{25}{\celsius}. Un manómetro acoplado al matraz indica \SI{600}{\mmHg}. Calcula la masa  molecular del gas. ¿Qué presión indicará el manómetro anterior si calentamos el gas hasta \SI{80}{\celsius}?
\end{exercise}

\begin{solution}
  \SI{27.9}{u}; \SI{710.7}{\mmHg}
\end{solution}



\begin{exercise}[
    tags    = {termodinámica, entalpía, entalpia de reacción, calor},
    topics  = {química, termoquímica, termodinámica},
    source  = {FQ 1B OXF 2015, p61, e29},
  ]
  Un recipiente contiene \SI{50}{\liter} de un gas de densidad \SI{1.45}{\gram\per\liter}. La temperatura a la que se encuentra el gas es \SI{353}{\kelvin}, y su presión \SI{10}{atm}. Calcula:
  \begin{enumerate}
    \item Los moles que contiene el recipiente.
    \item La masa de un mol del gas.
  \end{enumerate}
\end{exercise}

\begin{solution}
  \begin{enumerate*}
    \item \SI{18.87}{\mole};
    \item \SI{3.8}{\gram}.
  \end{enumerate*}
\end{solution}




\begin{exercise}[
    tags    = {termodinámica, entalpía, entalpia de reacción, calor},
    topics  = {química, termoquímica, termodinámica},
    source  = {FQ 1B SAN 2015, p67, e49},
  ]
  Una bombona de butano, \ch{C4H10}, tiene una capacidad de \SI{26}{\liter}, y cuando está llena su masa es \SI{12.5}{\kilo\gram} mayor que cuando está vacía. ¿Qué presión ejercería el butano que hay en su interior si estuviese en fase gaseosa? consideramos que la temperatura es de \SI{20}{\celsius}.
\end{exercise}

\begin{solution}
  \SI{198.9}{atm}
\end{solution}




\begin{exercise}[
    tags    = {termodinámica, entalpía, entalpia de reacción, calor},
    topics  = {química, termoquímica, termodinámica},
    source  = {FQ 1B SAN 2015, p68, e56},
  ]
  En una bombona tenemos una mezcla de gas hidrógeno y gas nitrógeno al 50\% en volumen. Si la presión de la mezcla es de \SI{800}{\mmHg}, ¿cuál es la presión parcial que ejerce cada gas?
\end{exercise}

\begin{solution}
  \( p_{\ch{H2}} = p_{\ch{N2}} = \SI{400}{\mmHg} \)
\end{solution}



\begin{exercise}[
    tags    = {termodinámica, entalpía, entalpia de reacción, calor},
    topics  = {química, termoquímica, termodinámica},
    source  = {FQ 1B SAN 2015, p68, e57},
  ]
  En un recipiente tenemos \SI{3.2}{\gram} de oxígeno que ejercen una presión de \SI{500}{\mmHg}. Sin que varíen la temperatura ni el volumen, añadimos al mismo recipiente \SI{4.2}{\gram} de gas hidrógeno. ¿Cuál será el valor
  de la presión ahora?
\end{exercise}

\begin{solution}
  \SI{11000}{\mmHg}.
\end{solution}




\begin{exercise}[
    tags    = {termodinámica, entalpía, entalpia de reacción, calor},
    topics  = {química, termoquímica, termodinámica},
    source  = {FQ 1B SAN 2015, p68, ejercicio resuelto 16},
  ]
  El aire seco es, fundamentalmente, una mezcla de \ch{N2} y \ch{O2}, cuya composición en masa es 75,5\% de \ch{N2} y 24,3\% de \ch{O2}. Cierto día la presión atmosférica es de \SI{720}{\mmHg}. ¿Qué presión ejerce el \ch{N2} ese día?
\end{exercise}

\begin{solution}
  \SI{562}{\mmHg}.
\end{solution}




\begin{exercise}[
    tags    = {termodinámica, entalpía, entalpia de reacción, calor},
    topics  = {química, termoquímica, termodinámica},
    source  = {FQ 1B SAN 2015, p68, e61},
  ]
  En un recipiente cerrado tenemos \SI{0.5}{\gram}de gas hidrógeno a \SI{150}{\celsius} y \SI{2}{atm}. A continuación,
  y sin modificar el volumen ni la temperatura,
  añadimos \SI{0.1}{\mole} de oxígeno.
  \begin{enumerate}
    \item Calcula la presión que ejerce la mezcla.
    \item Los dos gases reaccionan para dar agua (vapor), hasta
    que se consume todo el oxígeno. Calcula la presión
    en el recipiente al finalizar el proceso, suponiendo que
    no cambia la temperatura ni el volumen.
  \end{enumerate}
\end{exercise}

\begin{solution}
  \begin{enumerate*}
    \item \SI{2.8}{atm};
    \item \SI{2}{atm}.
  \end{enumerate*}
\end{solution}
