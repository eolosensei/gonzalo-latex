\section{Composición y fórmulas}

\begin{exercise}[
    tags    = {termodinámica, entalpía, entalpia de reacción, calor},
    topics  = {química, termoquímica, termodinámica},
    source  = {FQ 1B OXF 2015, p43, e40},
  ]
  El azufre, el oxígeno y el cinc forman el sulfato de cinc en la siguiente relación S:O:Zn; 1:1,99:2,04. Calcula la composición centesimal.
\end{exercise}

\begin{solution}
  19,9\% de \ch{S}; 39,6\% de \ch{O}; 40,5\% de \ch{Zn}.
\end{solution}




\begin{exercise}[
    tags    = {termodinámica, entalpía, entalpia de reacción, calor},
    topics  = {química, termoquímica, termodinámica},
    source  = {FQ 1B OXF 2015, p43, e41},
  ]
  Tenemos \SI{25}{\kilo\gram} de un abono nitrogenado de una riqueza en nitrato de potasio, \ch{KNO3}, del 60\%. Calcula la cantidad de nitrógeno, en \si{\kilo\gram}, que contiene el abono.
\end{exercise}

\begin{solution}
  \SI{2.1}{\kilo\gram}
\end{solution}



\begin{exercise}[
    tags    = {termodinámica, entalpía, entalpia de reacción, calor},
    topics  = {química, termoquímica, termodinámica},
    source  = {FQ 1B SAN 2015, p43, e43},
  ]
  Determina la composición centesimal de la glucosa, \ch{C6H12O6}.

  \begin{gexdatos}
    \( M(\ch{H}) = \SI{1.008}{\gram\per\mole} \),
    \( M(\ch{C}) = \SI{12.00}{\gram\per\mole} \),
    \( M(\ch{O}) = \SI{16.00}{\gram\per\mole} \).
  \end{gexdatos}
\end{exercise}

\begin{solution}
  39.98\% de \ch{C}; 6.72\% de \ch{H}; 53.30\% de \ch{O}
\end{solution}




\begin{exercise}[
    tags    = {termodinámica, entalpía, entalpia de reacción, calor},
    topics  = {química, termoquímica, termodinámica},
    source  = {FQ 1B SAN 2015, p43, e44},
  ]
  En el carbonato de sodio, por cada gramo de carbono se combinan \SI{4}{\gram} de oxígeno y \SI{3.83}{\gram} de sodio. Calcula su composición centesimal.
\end{exercise}

\begin{solution}
  11.3\% de \ch{C}; 45.3\% de \ch{O}; 43.4\% de \ch{Na}
\end{solution}




\begin{exercise}[
    tags    = {termodinámica, entalpía, entalpia de reacción, calor},
    topics  = {química, termoquímica, termodinámica},
    source  = {FQ 1B OXF 2015, p43, e42},
  ]
  Calcula la composición centesimal del sulfato de aluminio, \ch{Al2(SO4)3}.
\end{exercise}

\begin{solution}
  15,8\% de \ch{Al}; 28,1\% de \ch{S}; 56,1\% de \ch{O}.
\end{solution}



\begin{exercise}[
    tags    = {termodinámica, entalpía, entalpia de reacción, calor},
    topics  = {química, termoquímica, termodinámica},
    source  = {FQ 1B OXF 2015, p43, e43},
  ]
  Calcula la composición centesimal del \ch{KNO3}.
\end{exercise}

\begin{solution}
  38,6\% de \ch{K}; 13.9\% de \ch{N}; 47.5\% de \ch{O}.
\end{solution}




\begin{exercise}[
    tags    = {termodinámica, entalpía, entalpia de reacción, calor},
    topics  = {química, termoquímica, termodinámica},
    source  = {FQ 1B OXF 2015, p43, e45},
  ]
  Un óxido de vanadio que pesa \SI{3,53}{\gram} se redujo con dihidrógeno, y se obtuvo agua y otro óxido de vanadioque pesaba \SI{2,909}{\gram}. Este segundo óxido se volvió a reducir hasta obtener \SI{1,979}{\gram} de metal.
  \begin{enumerate}
    \item ¿Cuáles son las fórmulas empíricas de ambos óxidos?
    \item ¿Cuáles la cantidad total de agua formada en las dos reacciones?
  \end{enumerate}
\end{exercise}

\begin{solution}
  \begin{enumerate*}
    \item \ch{V2O3}; \ch{V2O5};
    \item \SI{1.74}{\gram}
  \end{enumerate*}
\end{solution}




\begin{exercise}[
    tags    = {termodinámica, entalpía, entalpia de reacción, calor},
    topics  = {química, termoquímica, termodinámica},
    source  = {FQ 1B OXF 2015, p43, e46},
  ]
  El análisis de un compuesto de carbono dio los siguientes porcentajes: 30,45\% de \ch{C}, 3,83\% de \ch{H}, 45,69\% de \ch{Cl} y 20,23\% de \ch{O}. Se sabe que la masa molar del compuesto es \SI{157}{\gram\per\mole}. ¿Cuál es la fórmula molecular del compuesto de carbono?
\end{exercise}

\begin{solution}
  \ch{C4H6O2Cl2}
\end{solution}




\begin{exercise}[
    tags    = {termodinámica, entalpía, entalpia de reacción, calor},
    topics  = {química, termoquímica, termodinámica},
    source  = {FQ 1B SAN 2015, p44, e52},
  ]
  Para determinar la fórmula química del mármol se descompone una muestra de \SI{2}{\gram} del mismo y se obtienen \SI{900}{\milli\gram} de calcio y \SI{240}{\milli\gram} de carbono; se sabe que el resto es oxígeno, ¿cuál es la fórmula?

  \begin{gexdatos}
    \( M(\ch{Ca}) = \SI{40.08}{\gram\per\mole} \),
    \( M(\ch{C}) = \SI{12.00}{\gram\per\mole} \),
    \( M(\ch{O}) = \SI{16.00}{\gram\per\mole} \).
  \end{gexdatos}
\end{exercise}

\begin{solution}
  \ch{CaCO3}
\end{solution}




\begin{exercise}[
    tags    = {termodinámica, entalpía, entalpia de reacción, calor},
    topics  = {química, termoquímica, termodinámica},
    source  = {FQ 1B SAN 2015, p44, e53},
  ]
  El hierro se oxida cuando se combina con oxígeno. Para determinar la fórmula del óxido resultante se calientan \SI{223.2}{\milli\gram} de hierro en presencia de exceso de oxígeno, obteniéndose una cantidad máxima de \SI{319.2}{\milli\gram} de óxido. ¿Cuál es la fórmula del compuesto que se formó?

  \begin{gexdatos}
    \( M(\ch{Fe}) = \SI{55.85}{\gram\per\mole} \),
    \( M(\ch{O}) = \SI{16.00}{\gram\per\mole} \).
  \end{gexdatos}
\end{exercise}

\begin{solution}
  \ch{Fe2O3}
\end{solution}
