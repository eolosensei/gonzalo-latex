\section{Leyes ponderales}

\begin{exercise}[
    tags    = {termodinámica, entalpía, entalpia de reacción, calor},
    topics  = {química, termoquímica, termodinámica},
    source  = {FQ 1B SAN 2015, p42, e24},
  ]
  El \ch{Mg} es un metal que se utiliza en la fabricación
  de fuegos artificiales, pues al arder produce fuertes
  destellos. En el proceso se forma \ch{MgO}, un compuesto
  en el que se combinan \SI{2.21}{\gram} de \ch{Mg} por cada \SI{1.45}{\gram} de \ch{O}. En un cohete se han colocado \SI{7}{\gram} de cinta de \ch{Mg}, ¿qué cantidad de \ch{MgO} se formará cuando el cohete arda?
\end{exercise}

\begin{solution}
  \SI{11.6}{\gram}
\end{solution}




\begin{exercise}[
    tags    = {termodinámica, entalpía, entalpia de reacción, calor},
    topics  = {química, termoquímica, termodinámica},
    source  = {FQ 1B OXF 2015, p42, e23},
  ]
  Se ha comprobado experimentalmente que \SI{4.7}{g} de elemento A reaccionan por completo con \SI{12.8}{\gram} del elemento B para dar \SI{17.5}{\gram} de un compuesto. ¿Qué cantidad de compuesto se formará si reaccionan \SI{4.7}{\gram} de A con \SI{11.5}{\gram} de B?
\end{exercise}

\begin{solution}
  \SI{15.7}{\gram}
\end{solution}




\begin{exercise}[
    tags    = {termodinámica, entalpía, entalpia de reacción, calor},
    topics  = {química, termoquímica, termodinámica},
    source  = {FQ 1B SAN 2015, p43, e34},
  ]
  Tenemos \SI{3.999e22}{átomos} de un metal cuya masa es de \SI{13.32}{\gram}. Consulta en la tabla periódica para averiguar
  qué metal es.

  \begin{gexdatos}
    \( N_A = \SI{6.022e23}{partículas}\)
  \end{gexdatos}
\end{exercise}

\begin{solution}
  PENDIENTE % FIXME pendiente
\end{solution}




\begin{exercise}[
    tags    = {termodinámica, entalpía, entalpia de reacción, calor},
    topics  = {química, termoquímica, termodinámica},
    source  = {FQ 1B OXF 2015, p43, e31},
  ]
  ¿Cuál de las siguientes muestras contiene mayor número de átomos?
  \begin{enumerate}
    \item \SI{10}{\gram} de \ch{Na};
    \item \SI{10}{\gram} de \ch{CO2};
    \item \SI{2}{\mole} de \ch{NH3};
  \end{enumerate}
\end{exercise}

\begin{solution}
  PENDIENTE % FIXME pendiente
\end{solution}




\begin{exercise}[
    tags    = {termodinámica, entalpía, entalpia de reacción, calor},
    topics  = {química, termoquímica, termodinámica},
    source  = {FQ 1B OXF 2015, p43, e34},
  ]
  Sabiendo que la densidad del \ch{H2O} es \SI{1}{\gram\per\cubic\centi\meter}, indica cuántos moles son:
  \begin{enumerate}
    \item \SI{3.42}{\gram} de \ch{H2O};
    \item \SI{10}{\cubic\centi\meter} de \ch{H2O};
    \item \SI{1.82e23}{moléculas} de \ch{H2O};
  \end{enumerate}
\end{exercise}

\begin{solution}
  \begin{enumerate*}
    \item \SI{0.19}{\mole};
    \item \SI{0.56}{\mole};
    \item \SI{0.3}{\mole}.
  \end{enumerate*}
\end{solution}



\begin{exercise}[
    tags    = {termodinámica, entalpía, entalpia de reacción, calor},
    topics  = {química, termoquímica, termodinámica},
    source  = {FQ 1B SAN 2015, p43, e37},
  ]
  La urea es un compuesto de fórmula \ch{CO(NH2)2}. Si tenemos \SI{5e24}{moléculas} de urea:
  \begin{enumerate}
    \item ¿Cuántos gramos de urea tenemos?
    \item ¿Cuántos moles de oxígeno?
    \item ¿Cuántos gramos de nitrógeno?
    \item ¿Cuántos átomos de hidrógeno?
  \end{enumerate}

  \begin{gexdatos}
    \( M(\ch{H}) = \SI{1.008}{\gram\per\mole} \),
    \( M(\ch{C}) = \SI{12.00}{\gram\per\mole} \),
    \( M(\ch{N}) = \SI{14.01}{\gram\per\mole} \),
    \( M(\ch{O}) = \SI{16.00}{\gram\per\mole} \),
    \( N_A = \SI{6.022e23}{partículas}\).
  \end{gexdatos}
\end{exercise}

\begin{solution}
  \begin{enumerate*}
    \item \SI{498,6}{\gram}; \item \SI{8,3}{\mole}; \item \SI{232,6}{\gram}; \item \SI{2e25}{átomos}.
  \end{enumerate*}
\end{solution}




\begin{exercise}[
    tags    = {termodinámica, entalpía, entalpia de reacción, calor},
    topics  = {química, termoquímica, termodinámica},
    source  = {FQ 1B SAN 2015, p43, e39},
  ]
  El aluminio se extrae de un mineral denominado bauxita,
  cuyo componente fundamental es el óxido de aluminio,
  \ch{Al2O3}. ¿Qué cantidad, en gramos, de óxido de aluminio
  necesitamos para obtener \SI{0.5}{\kilo\gram} de aluminio?

  \begin{gexdatos}
    \( M(\ch{Al}) = \SI{26.98}{\gram\per\mole} \),
    \( M(\ch{O}) = \SI{16.00}{\gram\per\mole} \),
    \( N_A = \SI{6.022e23}{partículas}\).
  \end{gexdatos}
\end{exercise}

\begin{solution}
  \SI{944.8}{\gram}
\end{solution}




\begin{exercise}[
    tags    = {termodinámica, entalpía, entalpia de reacción, calor},
    topics  = {química, termoquímica, termodinámica},
    source  = {FQ 1B OXF 2015, p43, e34},
  ]
  Sabiendo que la densidad del \ch{H2O} es \SI{1}{\gram\per\cubic\centi\meter}, indica cuántos moles son:
  \begin{enumerate}
    \item \SI{3.42}{\gram} de \ch{H2O};
    \item \SI{10}{\cubic\centi\meter} de \ch{H2O};
    \item \SI{1.82e23}{moléculas} de \ch{H2O};
  \end{enumerate}
\end{exercise}

\begin{solution}
  \begin{enumerate*}
    \item \SI{0.19}{\mole};
    \item \SI{0.56}{\mole};
    \item \SI{0.3}{\mole}.
  \end{enumerate*}
\end{solution}




\begin{exercise}[
    tags    = {termodinámica, entalpía, entalpia de reacción, calor},
    topics  = {química, termoquímica, termodinámica},
    source  = {FQ 1B OXF 2015, p43, e37},
  ]
  En una muestra de fósforo hay \SI{e24}{átomos}. Calcula:
  \begin{enumerate}
    \item La cantidad, en mol, de átomos de fósforo que hay en la muestra.
    \item La cantidad, en mol, de moléculas de fósforo que hay en la muestra (la molécula de fósforo es \ch{P4}).
  \end{enumerate}
\end{exercise}

\begin{solution}
  \begin{enumerate*}
    \item \SI{1.66}{\mole};
    \item \SI{0.415}{\mole};
  \end{enumerate*}
\end{solution}
