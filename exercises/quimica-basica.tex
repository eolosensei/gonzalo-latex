\documentclass[10pt]{article}
\title{Ejercicios Química Básica}
\author{Gonzalo Esteban}

\usepackage{polyglossia}
    \setdefaultlanguage{spanish}
\usepackage{fontspec}
    \setmainfont{Fira Sans}
\usepackage{amsmath, amsthm, amssymb}
\usepackage{unicode-math}
  \unimathsetup{
    math-style  = ISO,
    bold-style  = ISO
  }
  \setmathfont{Fira Math}
\usepackage{multicol}
  \setlength{\columnsep}{1cm}
  % \setcounter{tracingmulticols}{4}
\usepackage[top=3cm, bottom=3cm, left=2.5cm, right=2.5cm]{geometry}
\usepackage{xsim}
% Configuración general de xsim
  \loadxsimstyle{layouts}
  \xsimsetup{
  path                  = {xsim-files},
  exercise/template     = {runin},
  exercise/name         = {},
  exercise/print        = {false},
  solution/template     = {runin},
  solution/name         = {S},
  solution/print        = {true},
  exercise/within       = section,
  exercise/the-counter  = \thesection.\arabic{exercise}
  }
  \DeclareExerciseProperty{source}
\usepackage{siunitx}
% Exponent symbol options: \times for the typical cross
  \sisetup{
    per-mode                = symbol,
    output-decimal-marker   = {,},
    group-digits            = decimal,
    exponent-product        = \cdot,
    text-celsius            = ^^b0\kern -\scriptspace C,  % soluciona problemas con el símbolo de grados
    math-celsius            = ^^b0\kern -\scriptspace C,
    list-final-separator    = { y },
    list-pair-separator     = { y },
    range-phrase            = { \translate{to (numerical range)} },
    qualifier-mode          = brackets,
  }
\usepackage[inline]{enumitem}
% Configuración general de enumitem
% Establece la configuración por defecto en a), b), c)
  \setlist[enumerate,1]{
    label   = \alph*),
    itemsep = 0.3\itemsep,
  }
\usepackage{chemformula}
\usepackage{chemfig}
  \setchemfig{atom sep=2em}
\usepackage{booktabs}
\usepackage{icomma}    % resuelve el problema de espaciado excesivo en la coma decimal
\usepackage{fancyhdr}
  \pagestyle{fancy}
  \lhead{\textbf{Química básica}}
  \chead{}
  \rhead{1 BACH}
  \lfoot{}
  \cfoot{\thepage}
  \rfoot{}
  \renewcommand{\headrulewidth}{0.2pt}
  \renewcommand{\footrulewidth}{0pt}




  \newenvironment{gexdatos}{
      \vspace{2pt}
      \noindent\textit{Datos:}
    }{\vspace{5pt}}






\begin{document}

\maketitle

% 1. Leyes ponderales
\begin{multicols}{2}[
  \section{Leyes ponderales}
  ]

  \begin{exercise}[
      tags    = {termodinámica, entalpía, entalpia de reacción, calor},
      topics  = {química, termoquímica, termodinámica},
      source  = {FQ 1B SAN 2015, p42, e24},
    ]
    El \ch{Mg} es un metal que se utiliza en la fabricación
    de fuegos artificiales, pues al arder produce fuertes
    destellos. En el proceso se forma \ch{MgO}, un compuesto
    en el que se combinan \SI{2.21}{\gram} de \ch{Mg} por cada \SI{1.45}{\gram} de \ch{O}. En un cohete se han colocado \SI{7}{\gram} de cinta de \ch{Mg}, ¿qué cantidad de \ch{MgO} se formará cuando el cohete arda?
  \end{exercise}

  \begin{solution}
    \SI{11.6}{\gram}
  \end{solution}




  \begin{exercise}[
      tags    = {termodinámica, entalpía, entalpia de reacción, calor},
      topics  = {química, termoquímica, termodinámica},
      source  = {FQ 1B OXF 2015, p42, e23},
    ]
    Se ha comprobado experimentalmente que \SI{4.7}{g} de elemento A reaccionan por completo con \SI{12.8}{\gram} del elemento B para dar \SI{17.5}{\gram} de un compuesto. ¿Qué cantidad de compuesto se formará si reaccionan \SI{4.7}{\gram} de A con \SI{11.5}{\gram} de B?
  \end{exercise}

  \begin{solution}
    \SI{15.7}{\gram}
  \end{solution}




  \begin{exercise}[
      tags    = {termodinámica, entalpía, entalpia de reacción, calor},
      topics  = {química, termoquímica, termodinámica},
      source  = {FQ 1B SAN 2015, p43, e34},
    ]
    Tenemos \SI{3.999e22}{átomos} de un metal cuya masa es de \SI{13.32}{\gram}. Consulta en la tabla periódica para averiguar
    qué metal es.

    \begin{gexdatos}
      \( N_A = \SI{6.022e23}{partículas}\)
    \end{gexdatos}
  \end{exercise}

  \begin{solution}
    PENDIENTE % FIXME pendiente
  \end{solution}




  \begin{exercise}[
      tags    = {termodinámica, entalpía, entalpia de reacción, calor},
      topics  = {química, termoquímica, termodinámica},
      source  = {FQ 1B OXF 2015, p43, e31},
    ]
    ¿Cuál de las siguientes muestras contiene mayor número de átomos?
    \begin{enumerate}
      \item \SI{10}{\gram} de \ch{Na};
      \item \SI{10}{\gram} de \ch{CO2};
      \item \SI{2}{\mole} de \ch{NH3};
    \end{enumerate}
  \end{exercise}

  \begin{solution}
    PENDIENTE % FIXME pendiente
  \end{solution}




  \begin{exercise}[
      tags    = {termodinámica, entalpía, entalpia de reacción, calor},
      topics  = {química, termoquímica, termodinámica},
      source  = {FQ 1B OXF 2015, p43, e34},
    ]
    Sabiendo que la densidad del \ch{H2O} es \SI{1}{\gram\per\cubic\centi\meter}, indica cuántos moles son:
    \begin{enumerate}
      \item \SI{3.42}{\gram} de \ch{H2O};
      \item \SI{10}{\cubic\centi\meter} de \ch{H2O};
      \item \SI{1.82e23}{moléculas} de \ch{H2O};
    \end{enumerate}
  \end{exercise}

  \begin{solution}
    \begin{enumerate*}
      \item \SI{0.19}{\mole};
      \item \SI{0.56}{\mole};
      \item \SI{0.3}{\mole}.
    \end{enumerate*}
  \end{solution}



  \begin{exercise}[
      tags    = {termodinámica, entalpía, entalpia de reacción, calor},
      topics  = {química, termoquímica, termodinámica},
      source  = {FQ 1B SAN 2015, p43, e37},
    ]
    La urea es un compuesto de fórmula \ch{CO(NH2)2}. Si tenemos \SI{5e24}{moléculas} de urea:
    \begin{enumerate}
      \item ¿Cuántos gramos de urea tenemos?
      \item ¿Cuántos moles de oxígeno?
      \item ¿Cuántos gramos de nitrógeno?
      \item ¿Cuántos átomos de hidrógeno?
    \end{enumerate}

    \begin{gexdatos}
      \( M(\ch{H}) = \SI{1.008}{\gram\per\mole} \),
      \( M(\ch{C}) = \SI{12.00}{\gram\per\mole} \),
      \( M(\ch{N}) = \SI{14.01}{\gram\per\mole} \),
      \( M(\ch{O}) = \SI{16.00}{\gram\per\mole} \),
      \( N_A = \SI{6.022e23}{partículas}\).
    \end{gexdatos}
  \end{exercise}

  \begin{solution}
    \begin{enumerate*}
      \item \SI{498,6}{\gram}; \item \SI{8,3}{\mole}; \item \SI{232,6}{\gram}; \item \SI{2e25}{átomos}.
    \end{enumerate*}
  \end{solution}




  \begin{exercise}[
      tags    = {termodinámica, entalpía, entalpia de reacción, calor},
      topics  = {química, termoquímica, termodinámica},
      source  = {FQ 1B SAN 2015, p43, e39},
    ]
    El aluminio se extrae de un mineral denominado bauxita,
    cuyo componente fundamental es el óxido de aluminio,
    \ch{Al2O3}. ¿Qué cantidad, en gramos, de óxido de aluminio
    necesitamos para obtener \SI{0.5}{\kilo\gram} de aluminio?

    \begin{gexdatos}
      \( M(\ch{Al}) = \SI{26.98}{\gram\per\mole} \),
      \( M(\ch{O}) = \SI{16.00}{\gram\per\mole} \),
      \( N_A = \SI{6.022e23}{partículas}\).
    \end{gexdatos}
  \end{exercise}

  \begin{solution}
    \SI{944.8}{\gram}
  \end{solution}




  \begin{exercise}[
      tags    = {termodinámica, entalpía, entalpia de reacción, calor},
      topics  = {química, termoquímica, termodinámica},
      source  = {FQ 1B OXF 2015, p43, e34},
    ]
    Sabiendo que la densidad del \ch{H2O} es \SI{1}{\gram\per\cubic\centi\meter}, indica cuántos moles son:
    \begin{enumerate}
      \item \SI{3.42}{\gram} de \ch{H2O};
      \item \SI{10}{\cubic\centi\meter} de \ch{H2O};
      \item \SI{1.82e23}{moléculas} de \ch{H2O};
    \end{enumerate}
  \end{exercise}

  \begin{solution}
    \begin{enumerate*}
      \item \SI{0.19}{\mole};
      \item \SI{0.56}{\mole};
      \item \SI{0.3}{\mole}.
    \end{enumerate*}
  \end{solution}




  \begin{exercise}[
      tags    = {termodinámica, entalpía, entalpia de reacción, calor},
      topics  = {química, termoquímica, termodinámica},
      source  = {FQ 1B OXF 2015, p43, e37},
    ]
    En una muestra de fósforo hay \SI{e24}{átomos}. Calcula:
    \begin{enumerate}
      \item La cantidad, en mol, de átomos de fósforo que hay en la muestra.
      \item La cantidad, en mol, de moléculas de fósforo que hay en la muestra (la molécula de fósforo es \ch{P4}).
    \end{enumerate}
  \end{exercise}

  \begin{solution}
    \begin{enumerate*}
      \item \SI{1.66}{\mole};
      \item \SI{0.415}{\mole};
    \end{enumerate*}
  \end{solution}
\end{multicols}





% 2. Composición y fórmulas
\begin{multicols}{2}[
  \section{Composición y fórmulas}
  ]

  \begin{exercise}[
      tags    = {termodinámica, entalpía, entalpia de reacción, calor},
      topics  = {química, termoquímica, termodinámica},
      source  = {FQ 1B OXF 2015, p43, e40},
    ]
    El azufre, el oxígeno y el cinc forman el sulfato de cinc en la siguiente relación S:O:Zn; 1:1,99:2,04. Calcula la composición centesimal.
  \end{exercise}

  \begin{solution}
    19,9\% de \ch{S}; 39,6\% de \ch{O}; 40,5\% de \ch{Zn}.
  \end{solution}




  \begin{exercise}[
      tags    = {termodinámica, entalpía, entalpia de reacción, calor},
      topics  = {química, termoquímica, termodinámica},
      source  = {FQ 1B OXF 2015, p43, e41},
    ]
    Tenemos \SI{25}{\kilo\gram} de un abono nitrogenado de una riqueza en nitrato de potasio, \ch{KNO3}, del 60\%. Calcula la cantidad de nitrógeno, en \si{\kilo\gram}, que contiene el abono.
  \end{exercise}

  \begin{solution}
    \SI{2.1}{\kilo\gram}
  \end{solution}



  \begin{exercise}[
      tags    = {termodinámica, entalpía, entalpia de reacción, calor},
      topics  = {química, termoquímica, termodinámica},
      source  = {FQ 1B SAN 2015, p43, e43},
    ]
    Determina la composición centesimal de la glucosa, \ch{C6H12O6}.

    \begin{gexdatos}
      \( M(\ch{H}) = \SI{1.008}{\gram\per\mole} \),
      \( M(\ch{C}) = \SI{12.00}{\gram\per\mole} \),
      \( M(\ch{O}) = \SI{16.00}{\gram\per\mole} \).
    \end{gexdatos}
  \end{exercise}

  \begin{solution}
    39.98\% de \ch{C}; 6.72\% de \ch{H}; 53.30\% de \ch{O}
  \end{solution}




  \begin{exercise}[
      tags    = {termodinámica, entalpía, entalpia de reacción, calor},
      topics  = {química, termoquímica, termodinámica},
      source  = {FQ 1B SAN 2015, p43, e44},
    ]
    En el carbonato de sodio, por cada gramo de carbono se combinan \SI{4}{\gram} de oxígeno y \SI{3.83}{\gram} de sodio. Calcula su composición centesimal.
  \end{exercise}

  \begin{solution}
    11.3\% de \ch{C}; 45.3\% de \ch{O}; 43.4\% de \ch{Na}
  \end{solution}




  \begin{exercise}[
      tags    = {termodinámica, entalpía, entalpia de reacción, calor},
      topics  = {química, termoquímica, termodinámica},
      source  = {FQ 1B OXF 2015, p43, e42},
    ]
    Calcula la composición centesimal del sulfato de aluminio, \ch{Al2(SO4)3}.
  \end{exercise}

  \begin{solution}
    15,8\% de \ch{Al}; 28,1\% de \ch{S}; 56,1\% de \ch{O}.
  \end{solution}



  \begin{exercise}[
      tags    = {termodinámica, entalpía, entalpia de reacción, calor},
      topics  = {química, termoquímica, termodinámica},
      source  = {FQ 1B OXF 2015, p43, e43},
    ]
    Calcula la composición centesimal del \ch{KNO3}.
  \end{exercise}

  \begin{solution}
    38,6\% de \ch{K}; 13.9\% de \ch{N}; 47.5\% de \ch{O}.
  \end{solution}




  \begin{exercise}[
      tags    = {termodinámica, entalpía, entalpia de reacción, calor},
      topics  = {química, termoquímica, termodinámica},
      source  = {FQ 1B OXF 2015, p43, e45},
    ]
    Un óxido de vanadio que pesa \SI{3,53}{\gram} se redujo con dihidrógeno, y se obtuvo agua y otro óxido de vanadioque pesaba \SI{2,909}{\gram}. Este segundo óxido se volvió a reducir hasta obtener \SI{1,979}{\gram} de metal.
    \begin{enumerate}
      \item ¿Cuáles son las fórmulas empíricas de ambos óxidos?
      \item ¿Cuáles la cantidad total de agua formada en las dos reacciones?
    \end{enumerate}
  \end{exercise}

  \begin{solution}
    \begin{enumerate*}
      \item \ch{V2O3}; \ch{V2O5};
      \item \SI{1.74}{\gram}
    \end{enumerate*}
  \end{solution}




  \begin{exercise}[
      tags    = {termodinámica, entalpía, entalpia de reacción, calor},
      topics  = {química, termoquímica, termodinámica},
      source  = {FQ 1B OXF 2015, p43, e46},
    ]
    El análisis de un compuesto de carbono dio los siguientes porcentajes: 30,45\% de \ch{C}, 3,83\% de \ch{H}, 45,69\% de \ch{Cl} y 20,23\% de \ch{O}. Se sabe que la masa molar del compuesto es \SI{157}{\gram\per\mole}. ¿Cuál es la fórmula molecular del compuesto de carbono?
  \end{exercise}

  \begin{solution}
    \ch{C4H6O2Cl2}
  \end{solution}




  \begin{exercise}[
      tags    = {termodinámica, entalpía, entalpia de reacción, calor},
      topics  = {química, termoquímica, termodinámica},
      source  = {FQ 1B SAN 2015, p44, e52},
    ]
    Para determinar la fórmula química del mármol se descompone una muestra de \SI{2}{\gram} del mismo y se obtienen \SI{900}{\milli\gram} de calcio y \SI{240}{\milli\gram} de carbono; se sabe que el resto es oxígeno, ¿cuál es la fórmula?

    \begin{gexdatos}
      \( M(\ch{Ca}) = \SI{40.08}{\gram\per\mole} \),
      \( M(\ch{C}) = \SI{12.00}{\gram\per\mole} \),
      \( M(\ch{O}) = \SI{16.00}{\gram\per\mole} \).
    \end{gexdatos}
  \end{exercise}

  \begin{solution}
    \ch{CaCO3}
  \end{solution}




  \begin{exercise}[
      tags    = {termodinámica, entalpía, entalpia de reacción, calor},
      topics  = {química, termoquímica, termodinámica},
      source  = {FQ 1B SAN 2015, p44, e53},
    ]
    El hierro se oxida cuando se combina con oxígeno. Para determinar la fórmula del óxido resultante se calientan \SI{223.2}{\milli\gram} de hierro en presencia de exceso de oxígeno, obteniéndose una cantidad máxima de \SI{319.2}{\milli\gram} de óxido. ¿Cuál es la fórmula del compuesto que se formó?

    \begin{gexdatos}
      \( M(\ch{Fe}) = \SI{55.85}{\gram\per\mole} \),
      \( M(\ch{O}) = \SI{16.00}{\gram\per\mole} \).
    \end{gexdatos}
  \end{exercise}

  \begin{solution}
    \ch{Fe2O3}
  \end{solution}
\end{multicols}




% 3. Leyes de los gases
\begin{multicols}{2}[
  \section{Leyes de los gases}
  ]

  \begin{exercise}[
      tags    = {termodinámica, entalpía, entalpia de reacción, calor},
      topics  = {química, termoquímica, termodinámica},
      source  = {FQ 1B OXF 2015, p60, e46},
    ]
    Un gas ocupa un volumen de \SI{2}{\liter} en condiciones normales de presión y temperatura. ¿Qué volumen ocupará la misma masa de gas a \SI{2}{atm} de presión y \SI{50}{\celsius} de temperatura?
  \end{exercise}

  \begin{solution}
    \SI{1.18}{\liter}.
  \end{solution}




  \begin{exercise}[
      tags    = {termodinámica, entalpía, entalpia de reacción, calor},
      topics  = {química, termoquímica, termodinámica},
      source  = {FQ 1B OXF 2015, p60, e16},
    ]
    Un gas ocupa un volumen de \SI{80}{\cubic\centi\meter} a \SI{10}{\celsius} y \SI{715}{\mmHg}. ¿Qué volumen ocupará este gas en CN?
  \end{exercise}

  \begin{solution}
    \SI{72.6}{\cubic\centi\meter}.
  \end{solution}




  \begin{exercise}[
      tags    = {termodinámica, entalpía, entalpia de reacción, calor},
      topics  = {química, termoquímica, termodinámica},
      source  = {FQ 1B OXF 2015, p60, e19},
    ]
    La densidad de un gas en condiciones normales es \SI{1.48}{\gram\per\liter}. ¿Cuál será su densidad a \SI{320}{\kelvin} y \SI{730}{\mmHg}?
  \end{exercise}

  \begin{solution}
    \SI{1.21}{\gram\per\liter}.
  \end{solution}




  \begin{exercise}[
      tags    = {termodinámica, entalpía, entalpia de reacción, calor},
      topics  = {química, termoquímica, termodinámica},
      source  = {FQ 1B SAN 2015, p66, e37},
    ]
    En una ampolla de \SI{750}{\milli\liter} tenemos un gas que ejerce una presión de \SI{1.25}{atm} a \SI{50}{\celsius}. Lo conectamos a una segunda ampolla de \SI{2}{\liter}. ¿Qué presión leeremos ahora en el manómetro si no varía la temperatura?
  \end{exercise}

  \begin{solution}
    \SI{259}{\mmHg}.
  \end{solution}



  \begin{exercise}[
      tags    = {termodinámica, entalpía, entalpia de reacción, calor},
      topics  = {química, termoquímica, termodinámica},
      source  = {FQ 1B SAN 2015, p66, e39},
    ]
    Tenemos un gas encerrado en un recipiente rígido de \SI{5}{\liter}. ¿En cuánto cambia su temperatura si su presión pasa de \SI{300}{\mmHg} a \SI{600}{\mmHg}?
  \end{exercise}

  \begin{solution}
    Se duplica su temperatura absoluta.
  \end{solution}




  \begin{exercise}[
      tags    = {termodinámica, entalpía, entalpia de reacción, calor},
      topics  = {química, termoquímica, termodinámica},
      source  = {FQ 1B SAN 2015, p66, e39},
    ]
    Una pieza de una máquina está formada por un pistón que tiene un gas en su interior. En un momento dado, el volumen del pistón es de \SI{225}{\milli\liter} y la temperatura del gas es de \SI{50}{\celsius}. ¿Cuánto debe cambiar la temperatura para que el volumen sea de \SI{275}{\milli\liter} si la presión no varía?
  \end{exercise}

  \begin{solution}
    \( \Delta T = \SI{+71.7}{\kelvin} \)
  \end{solution}




  \begin{exercise}[
      tags    = {termodinámica, entalpía, entalpia de reacción, calor},
      topics  = {química, termoquímica, termodinámica},
      source  = {FQ 1B OXF 2015, p60, e22},
    ]
    Se dispone de \SI{45.0}{\gram} de metano (\ch{CH4}) a \SI{800}{\mmHg} y \SI{27}{\celsius}. Calcula:
    \begin{enumerate}
      \item El volumen que ocupa en las citadas condiciones.
      \item El número de moléculas existentes
    \end{enumerate}
  \end{exercise}

  \begin{solution}
    \begin{enumerate}
      \item \SI{66}{\liter}
      \item \SI{1.7e24}{moléculas}.
    \end{enumerate}
  \end{solution}




  \begin{exercise}[
      tags    = {termodinámica, entalpía, entalpia de reacción, calor},
      topics  = {química, termoquímica, termodinámica},
      source  = {FQ 1B OXF 2015, p61, e24-25},
    ]
    En un matraz de \SI{1}{\liter} están contenidos \SI{0.9}{\gram} de un gas a una temperatura de \SI{25}{\celsius}. Un manómetro acoplado al matraz indica \SI{600}{\mmHg}. Calcula la masa  molecular del gas. ¿Qué presión indicará el manómetro anterior si calentamos el gas hasta \SI{80}{\celsius}?
  \end{exercise}

  \begin{solution}
    \SI{27.9}{u}; \SI{710.7}{\mmHg}
  \end{solution}



  \begin{exercise}[
      tags    = {termodinámica, entalpía, entalpia de reacción, calor},
      topics  = {química, termoquímica, termodinámica},
      source  = {FQ 1B OXF 2015, p61, e29},
    ]
    Un recipiente contiene \SI{50}{\liter} de un gas de densidad \SI{1.45}{\gram\per\liter}. La temperatura a la que se encuentra el gas es \SI{353}{\kelvin}, y su presión \SI{10}{atm}. Calcula:
    \begin{enumerate}
      \item Los moles que contiene el recipiente.
      \item La masa de un mol del gas.
    \end{enumerate}
  \end{exercise}

  \begin{solution}
    \begin{enumerate*}
      \item \SI{18.87}{\mole};
      \item \SI{3.8}{\gram}.
    \end{enumerate*}
  \end{solution}




  \begin{exercise}[
      tags    = {termodinámica, entalpía, entalpia de reacción, calor},
      topics  = {química, termoquímica, termodinámica},
      source  = {FQ 1B SAN 2015, p67, e49},
    ]
    Una bombona de butano, \ch{C4H10}, tiene una capacidad de \SI{26}{\liter}, y cuando está llena su masa es \SI{12.5}{\kilo\gram} mayor que cuando está vacía. ¿Qué presión ejercería el butano que hay en su interior si estuviese en fase gaseosa? consideramos que la temperatura es de \SI{20}{\celsius}.
  \end{exercise}

  \begin{solution}
    \SI{198.9}{atm}
  \end{solution}




  \begin{exercise}[
      tags    = {termodinámica, entalpía, entalpia de reacción, calor},
      topics  = {química, termoquímica, termodinámica},
      source  = {FQ 1B SAN 2015, p68, e56},
    ]
    En una bombona tenemos una mezcla de gas hidrógeno y gas nitrógeno al 50\% en volumen. Si la presión de la mezcla es de \SI{800}{\mmHg}, ¿cuál es la presión parcial que ejerce cada gas?
  \end{exercise}

  \begin{solution}
    \( p_{\ch{H2}} = p_{\ch{N2}} = \SI{400}{\mmHg} \)
  \end{solution}



  \begin{exercise}[
      tags    = {termodinámica, entalpía, entalpia de reacción, calor},
      topics  = {química, termoquímica, termodinámica},
      source  = {FQ 1B SAN 2015, p68, e57},
    ]
    En un recipiente tenemos \SI{3.2}{\gram} de oxígeno que ejercen una presión de \SI{500}{\mmHg}. Sin que varíen la temperatura ni el volumen, añadimos al mismo recipiente \SI{4.2}{\gram} de gas hidrógeno. ¿Cuál será el valor
    de la presión ahora?
  \end{exercise}

  \begin{solution}
    \SI{11000}{\mmHg}.
  \end{solution}




  \begin{exercise}[
      tags    = {termodinámica, entalpía, entalpia de reacción, calor},
      topics  = {química, termoquímica, termodinámica},
      source  = {FQ 1B SAN 2015, p68, ejercicio resuelto 16},
    ]
    El aire seco es, fundamentalmente, una mezcla de \ch{N2} y \ch{O2}, cuya composición en masa es 75,5\% de \ch{N2} y 24,3\% de \ch{O2}. Cierto día la presión atmosférica es de \SI{720}{\mmHg}. ¿Qué presión ejerce el \ch{N2} ese día?
  \end{exercise}

  \begin{solution}
    \SI{562}{\mmHg}.
  \end{solution}




  \begin{exercise}[
      tags    = {termodinámica, entalpía, entalpia de reacción, calor},
      topics  = {química, termoquímica, termodinámica},
      source  = {FQ 1B SAN 2015, p68, e61},
    ]
    En un recipiente cerrado tenemos \SI{0.5}{\gram}de gas hidrógeno a \SI{150}{\celsius} y \SI{2}{atm}. A continuación,
    y sin modificar el volumen ni la temperatura,
    añadimos \SI{0.1}{\mole} de oxígeno.
    \begin{enumerate}
      \item Calcula la presión que ejerce la mezcla.
      \item Los dos gases reaccionan para dar agua (vapor), hasta
      que se consume todo el oxígeno. Calcula la presión
      en el recipiente al finalizar el proceso, suponiendo que
      no cambia la temperatura ni el volumen.
    \end{enumerate}
  \end{exercise}

  \begin{solution}
    \begin{enumerate*}
      \item \SI{2.8}{atm};
      \item \SI{2}{atm}.
    \end{enumerate*}
  \end{solution}
\end{multicols}





% 4. Disoluciones
\begin{multicols}{2}[
  \section{Disoluciones}
  ]

  \begin{exercise}[
      tags    = {termodinámica, entalpía, entalpia de reacción, calor},
      topics  = {química, termoquímica, termodinámica},
      source  = {FQ 1B OXF 2015, p78, e9},
    ]
    Se disuelven \SI{10}{\gram} de sacarosa en \SI{250}{\gram} de agua. Indica la concentración de la disolución en:
    \begin{enumerate}
      \item Masa (\si{\gram}) de soluto/\SI{100}{\gram} de disolvente.
      \item Masa (\si{\gram}) de soluto/\SI{100}{\gram} de disolución.
    \end{enumerate}
  \end{exercise}

  \begin{solution}
    \begin{enumerate*}
      \item 4;
      \item 3,85.
    \end{enumerate*}
  \end{solution}



  \begin{exercise}[
      tags    = {termodinámica, entalpía, entalpia de reacción, calor},
      topics  = {química, termoquímica, termodinámica},
      source  = {FQ 1B SAN 2015, p92, e25},
    ]
    El suero fisiológico es una disolución de sal en agua al 0,9\%
    (porcentaje en masa). Calcula la cantidad de sal y de agua
    que necesitas para preparar \SI{2}{\kilo\gram} de suero fisiológico.
  \end{exercise}

  \begin{solution}
    \SI{18}{\gram} de sal y \SI{1982}{\gram} de agua.
  \end{solution}




  \begin{exercise}[
      tags    = {termodinámica, entalpía, entalpia de reacción, calor},
      topics  = {química, termoquímica, termodinámica},
      source  = {FQ 1B SAN 2015, p92, e28},
    ]
    Necesitamos preparar \SI{500}{\milli\liter} de una disolución de hidróxido de sodio \SI{2}{M}. Calcula qué cantidad de soluto necesitas y explica cómo la prepararás si se dispone de un producto comercial del 95\% de riqueza en \ch{NaOH}.
  \end{exercise}

  \begin{solution}
    Necesitas \SI{42.11}{\gram} de \ch{NaOH} comercial.
  \end{solution}




  \begin{exercise}[
      tags    = {termodinámica, entalpía, entalpia de reacción, calor},
      topics  = {química, termoquímica, termodinámica},
      source  = {FQ 1B SAN 2015, p92, e33},
    ]
    En el laboratorio tenemos un ácido clorhídrico del 37\% de riqueza en masa y \SI{1.18}{\gram\per\milli\liter} de densidad. Si cogemos \SI{70}{\milli\liter} del contenido de esa botella, ¿Cuánto ácido
    clorhídrico estaremos tomando?
  \end{exercise}

  \begin{solution}
    \SI{30.6}{\gram}.
  \end{solution}




  \begin{exercise}[
      tags    = {termodinámica, entalpía, entalpia de reacción, calor},
      topics  = {química, termoquímica, termodinámica},
      source  = {FQ 1B OXF 2015, p78, e11},
    ]
    Se prepara una disolución con \SI{5}{\gram} de \ch{NaOH} en \SI{25}{\gram} de agua destilada. Si el volumen final es de \SI{27.1}{\cubic\centi\meter}, calcula la concentración de la disolución en:
    \begin{enumerate}
      \item Porcentaje en masa.
      \item Masa (\si{\gram}) por litro.
      \item Molaridad.
      \item Molalidad.
    \end{enumerate}
  \end{exercise}

  \begin{solution}
    \begin{enumerate*}
      \item 16,7\%:
      \item \SI{184.5}{\gram\per\liter};
      \item \SI{4.6}{M};
      \item \SI{5}{m}.
    \end{enumerate*}
  \end{solution}



  \begin{exercise}[
      tags    = {termodinámica, entalpía, entalpia de reacción, calor},
      topics  = {química, termoquímica, termodinámica},
      source  = {FQ 1B SAN 2015, p94, e63},
    ]
    Se ha preparado una disolución mezclando \SI{100}{\milli\liter} de \ch{CaCl2} \SI{2}{M} con \SI{150}{\milli\liter} de \ch{NaCl} \SI{1.5}{M}.
    ¿Cuál será la concentración de los iones cloruro en la disolución resultante? Para calcularlo, supón que los volúmenes son aditivos.
  \end{exercise}

  \begin{solution}
      \SI{2.5}{M};
  \end{solution}

  % \vfill\null
\end{multicols}






% 5. Propiedades coligativas
\begin{multicols}{2}[
  \section{Propiedades coligativas}
  ]

  \begin{exercise}[
      tags    = {termodinámica, entalpía, entalpia de reacción, calor},
      topics  = {química, termoquímica, termodinámica},
      source  = {FQ 1B OXF 2015, p79, e41},
    ]
    Calcula la temperatura de congelación de una disolución formada por \SI{9.5}{\gram} de etilenglicol (anticongelante usado en los automóviles cuya fórmula es \ch{CH2OH-CH2OH}) y \SI{20}{\gram} de agua.
  \end{exercise}

  \begin{solution}
    \SI{-14,25}{\celsius}
  \end{solution}



  \begin{exercise}[
      tags    = {termodinámica, entalpía, entalpia de reacción, calor},
      topics  = {química, termoquímica, termodinámica},
      source  = {FQ 1B SAN 2015, p94, e51},
    ]
    Cuál será, a \SI{80}{\celsius}, la presión de vapor de una disolución que se prepara disolviendo \SI{30}{\milli\liter} de glicerina, \ch{C3H8O3}, en \SI{70}{\milli\liter} de agua.

    \begin{gexdatos}
      \( p_{\mathrm{agua}}(\SI{80}{\celsius}) = \SI{355}{\mmHg} \),
      \( d_{\mathrm{agua}} = \SI{1}{\gram\per\milli\liter} \),
      \( d_{\mathrm{glicerina}} = \SI{1.26}{\gram\per\milli\liter} \). % REVIEW revisar si mathrm formatea bien el texto en recto
    \end{gexdatos}
  \end{exercise}

  \begin{solution}
    \SI{321}{\mmHg}
  \end{solution}




  \begin{exercise}[
      tags    = {termodinámica, entalpía, entalpia de reacción, calor},
      topics  = {química, termoquímica, termodinámica},
      source  = {FQ 1B OXF 2015, p79, e44},
    ]
    Suponiendo un comportamiento ideal, ¿cuál sería la presión de vapor de la disolución obtenida al mezclar \SI{500}{\milli\liter} de agua y \SI{90}{\gram} de glucosa (\ch{C6H12O6}) si la presión de vapor del agua a la temperatura de la mezcla es de \SI{55.3}{\mmHg}?
  \end{exercise}

  \begin{solution}
    \SI{54.32}{\mmHg}
  \end{solution}




  \begin{exercise}[
      tags    = {termodinámica, entalpía, entalpia de reacción, calor},
      topics  = {química, termoquímica, termodinámica},
      source  = {FQ 1B OXF 2015, p79, e46},
    ]
    Un cierto compuesto contiene 43,2\% de \ch{C}, 16,6\% de \ch{N}, 2,4\% de \ch{H} y 37,8\% de \ch{O}. La adición de \SI{6.45}{\gram} de esa sustancia en \SI{50}{\milli\liter} de benceno (\ch{C6H6}), cuya densidad es \SI{0.88}{\gram\per\cubic\centi\meter}, hace bajar el punto de congelación del benceno de \SI{5.51}{\celsius} a \SI{1.25}{\celsius}. Halla la fórmula molecular de ese compuesto.

    \begin{gexdatos}
      \( K_c (\ch{C6H6}) = \SI{5.02}{\celsius\kilo\gram\per\mole} \)
    \end{gexdatos}
  \end{exercise}

  \begin{solution}
    \ch{C6N2O4H4}
  \end{solution}



  \begin{exercise}[
      tags    = {termodinámica, entalpía, entalpia de reacción, calor},
      topics  = {química, termoquímica, termodinámica},
      source  = {FQ 1B SAN 2015, p94, e54},
    ]
    Determina la masa molar de una sustancia si al disolver \SI{17}{\gram} de la misma en \SI{150}{\gram} de benceno se obtiene una mezcla que se congela a \SI{-4}{\celsius}.

    \begin{gexdatos}
      \( K_c (\ch{C6H6}) = \SI{5.07}{\celsius\kilo\gram\per\mole} \),
      \( T_f (\ch{C6H6}) = \SI{6}{\celsius} \).
    \end{gexdatos}
  \end{exercise}

  \begin{solution}
    \SI{57.46}{\gram\per\mole}
  \end{solution}




  \begin{exercise}[
      tags    = {termodinámica, entalpía, entalpia de reacción, calor},
      topics  = {química, termoquímica, termodinámica},
      source  = {FQ 1B OXF 2015, p79, e48},
    ]
    La presión osmótica de una disolución es \SI{4.2}{atm} a \SI{20}{\celsius}. ¿Qué presión osmótica tendrá a \SI{50}{\celsius}?
  \end{exercise}

  \begin{solution}
    \SI{4.6}{atm}
  \end{solution}




  \begin{exercise}[
      tags    = {termodinámica, entalpía, entalpia de reacción, calor},
      topics  = {química, termoquímica, termodinámica},
      source  = {FQ 1B SAN 2015, p94, e59},
    ]
    La albúmina es una proteína del huevo. Calcula la masa
    molar de la albúmina si una disolución de \SI{50}{\gram} de albúmina
    por litro de agua ejerce una presión osmótica de
    \SI{14}{\mmHg} a \SI{25}{\celsius}.
  \end{exercise}

  \begin{solution}
    \SI{6.63e4}{\gram\per\mole}
  \end{solution}




  \begin{exercise}[
      tags    = {termodinámica, entalpía, entalpia de reacción, calor},
      topics  = {química, termoquímica, termodinámica},
      source  = {FQ 1B OXF 2015, p79, e50},
    ]
    Una muestra de \SI{2}{\gram} de un compuesto orgánico disuelto en \SI{100}{\cubic\centi\meter} de disolución se encuentra a una presión de \SI{1.31}{atm}, en el equilibrio osmótico. Sabiendo que la disolución está a \SI{0}{\celsius}, calcula la masa molar del compuesto orgánico.
  \end{exercise}

  \begin{solution}
    \SI{342}{\gram\per\mole}
  \end{solution}
  \vfill\null
\end{multicols}

\end{document}
