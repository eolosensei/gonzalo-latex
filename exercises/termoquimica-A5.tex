\documentclass[10pt,a5paper,twoside]{article}
\title{Ejercicios Termoquímica}
\author{Gonzalo Esteban}

\usepackage{polyglossia}
    \setdefaultlanguage{spanish}
\usepackage{fontspec}
    \setmainfont{Fira Sans}
\usepackage{amsmath, amsthm, amssymb}
\usepackage{unicode-math}
  \unimathsetup{
    math-style  = ISO,
    bold-style  = ISO
  }
  \setmathfont{Fira Math}
\usepackage[top=3cm, bottom=2.5cm, inner=2.2cm, outer=2.2cm]{geometry}
\usepackage{xsim}
% Configuración general de xsim
  \loadxsimstyle{layouts}
  \xsimsetup{
  path                  = {xsim-files},
  exercise/template     = {margin},
  exercise/name         = {},
  exercise/print        = {true},
  solution/template     = {runin},
  solution/name         = {S},
  solution/print        = {false},
  exercise/within       = section,
  exercise/the-counter  = \thesection.\arabic{exercise}
  }
  \DeclareExerciseProperty{source}
\usepackage{siunitx}
% Exponent symbol options: \times for the typical cross
  \sisetup{
    per-mode                = symbol,
    output-decimal-marker   = {,},
    group-digits            = decimal,
    exponent-product        = \cdot,
    text-celsius            = ^^b0\kern -\scriptspace C,  % soluciona problemas con el símbolo de grados
    math-celsius            = ^^b0\kern -\scriptspace C,
    list-final-separator    = { y },
    list-pair-separator     = { y },
    range-phrase            = { \translate{to (numerical range)} },
    qualifier-mode          = brackets,
  }
\usepackage[inline]{enumitem}
% Configuración general de enumitem
% Establece la configuración por defecto en a), b), c)
  \setlist[enumerate,1]{
    label   = \alph*),
    itemsep = 0.3\itemsep,
  }
\usepackage{chemformula}
\usepackage{chemfig}
  \setchemfig{atom sep=2em}
\usepackage{booktabs}
\usepackage{icomma}    % resuelve el problema de espaciado excesivo en la coma decimal
\usepackage{fancyhdr}
  \pagestyle{fancy}
  \fancyhead{}
  \fancyhead[LE]{\textbf{Termoquímica}}
  \fancyhead[RO]{1 BACH}
  \fancyfoot{}
  \fancyfoot[C]{\thepage}
  \renewcommand{\headrulewidth}{0.2pt}
  \renewcommand{\footrulewidth}{0pt}




  \newenvironment{gexdatos}{
      \vspace{4pt}
      \noindent\small\textit{Datos:}
    }{\vspace{5pt}}






\begin{document}

\maketitle

\section{Calor y entalpía}

  \subsection*{Entalpía de reacción}


    \begin{exercise}[
        tags    = {termodinámica, entalpía, entalpia de reacción, calor},
        topics  = {química, termoquímica, termodinámica},
        source  = {FQ 1B ANA 2016, p164, e5},
      ]
      Escribe las ecuaciones termoquímicas que describen los procesos de formación de las siguientes sustancias a partir de sus elementos constituyentes en estado estándar:
      \begin{enumerate}
        \item \ch{CO} (g).
        \item \ch{H2O} (g).
        \item \ch{H2O} (l).
      \end{enumerate}
    \end{exercise}

    \begin{solution}
      \begin{enumerate}
        \item \ch{C (grafito) + 1/2 O2 (g) -> CO (g)} \( \Delta H^0_f = \SI{-110.5}{\kilo\joule\per\mole} \).
        \item \ch{H2 (g) + 1/2 O2 (g) -> H2O (g)} \( \Delta H^0_f = \SI{-241.8}{\kilo\joule\per\mole} \).
        \item ch{H2 (g) + 1/2 O2 (g) -> H2O (l)} \( \Delta H^0_f = \SI{-285.8}{\kilo\joule\per\mole} \).
      \end{enumerate}
    \end{solution}




    \begin{exercise}[
        tags    = {termodinámica, entalpía, entalpia de reacción, calor},
        topics  = {química, termoquímica, termodinámica},
        source  = {FQ 1B ANA 2016, p165, e14},
      ]
      Calcula la entalpía de la reacción:
      \[ \ch{CO (g) + 1/2 O2 (g) -> CO2 (g)} \]
      sabiendo que en la formación de \SI{1,0}{\gram} de producto se desprenden \SI{6,43}{\kilo\joule}.
    \end{exercise}

    \begin{solution}
      \( \Delta H_r = \SI{-283.0}{\kilo\joule} \)
    \end{solution}




    \begin{exercise}[
        tags    = {termodinámica, entalpía, entalpia de reacción, calor},
        topics  = {química, termoquímica, termodinámica},
        source  = {FQ 1B ANA 2016, p165, e17},
      ]
      Calcula la entalpía de formación del ácido acético, \ch{CH3COOH}, sabiendo que su entalpía de combustión es \SI{-870,3}{\kilo\joule\per\mole} y que las entalpías estándar de formación del \ch{CO2} y del \ch{H2O} (l) son \SIlist{-393.5;-285.8}{\kilo\joule\per\mole}, respectivamente.
    \end{exercise}

    \begin{solution}
      \( \Delta H^0_r = \SI{-488.3}{\kilo\joule\per\mole} \)
    \end{solution}




    \begin{exercise}[
        tags    = {termodinámica, entalpía, entalpia de reacción, calor},
        topics  = {química, termoquímica, termodinámica},
        source  = {FQ 1B ANA 2016, p165, e23},
      ]
      Calcula la entalpía de la reacción de hidrogenación de eteno, \ch{C2H4}, a etano, \ch{C2H6}:
      \[ \ch{C2H4 (g) + H2 (g) -> C2H6 (g)} \]
      sabiendo que las entalpías de combustión, en \si{kcal/mol}, del eteno, del etano y del hidrógeno son \numlist{-337.3;-372.9;-68.38}, respectivamente. 
    \end{exercise}

    \begin{solution}
      \( \Delta H_r = \SI{-137}{\kilo\joule} \)
    \end{solution}




  \begin{exercise}[
      tags    = {termodinámica, entalpía, entalpia de reacción, calor},
      topics  = {química, termoquímica, termodinámica},
      source  = {},
    ]
    Utilizando los datos tabulados en tu libro de texto (o buscándolos en internet), calcular la variación de entalpía en la formación de las siguientes sustancias.
    \begin{enumerate}
      \item \SI{180}{\gram} de agua en estado gaseoso.
      \item \SI{300}{\gram} de óxido de nitrógeno(II) gaseoso.
    \end{enumerate}
  \end{exercise}

  \begin{solution}
    \begin{enumerate*}
      \item \SI{-2420}{\kilo\joule}; \item \SI{900}{\kilo\joule}
    \end{enumerate*}
  \end{solution}









  \begin{exercise}[
      tags    = {termodinámica, entalpía, entalpia de reacción, calor},
      topics  = {química, termoquímica, termodinámica},
      source  = {},
    ]
    La energía que proporciona una golosina de glucosa, \ch{C6H12O6}, es la misma si la digerimos que si la quemamos en el aire. La diferencia es la velocidad de liberación de la energía.
    \begin{enumerate}
      \item Escribir la ecuación termoquímica de la combustión de la glucosa si la entalpía de combustión es \SI{2800}{kJ/mol}.
      \item Calcular la energía que proporciona una golosina con \SI{18}{\gram} de glucosa.
    \end{enumerate}
  \end{exercise}

  \begin{solution}
    b) \SI{280}{\kilo\joule}
  \end{solution}




  \begin{exercise}[
      tags    = {termodinámica, entalpía, ley de Hess},
      topics  = {química, termoquímica, termodinámica},
      source  = {},
    ]
    Alguien ha propuesto fabricar diamantes a partir de la oxidación del metano, según la reacción:
    \[ \ch{CH4(g) + O2(g) -> C(diamante) + 2 H2O(l)}  \]
    Calcular la variación de entalpía del proceso conocidas las variaciones de entalpía de las siguientes reacciones:
    \begin{multline*}
      \mathbf{A:}\quad\ch{CH4(g) + 2 O2(g) -> CO2(g) + 2 H2O(l)} \\
        \Delta H_A = \SI{-890}{\kilo\joule}
    \end{multline*}
    \begin{multline*}
      \mathbf{B:}\quad\ch{C(diamante) + O2(g) -> CO2(g) + 2 H2O(l)} \\
        \Delta H_B = \SI{-395}{\kilo\joule}
    \end{multline*}


  \end{exercise}

  \begin{solution}
    \SI{-495}{\kilo\joule}
  \end{solution}




  \begin{exercise}[
      tags    = {termodinámica, entalpía, ley de Hess},
      topics  = {química, termoquímica, termodinámica},
      source  = {},
    ]
    La variación de entalpía de la reacción de síntesis del metano no se puede calcular directamente:
    \[ \ch{C(grafito) + 2 H2(g) -> CH4(g)}  \]
    Calcular su valor a partir de las siguientes reacciones:
    \begin{multline*}
      \mathbf{A:} \quad \ch{C(grafito) + O2(g) -> CO2(g) + 2 H2O(l)} \\
      \Delta H_A = \SI{-394}{\kilo\joule}
    \end{multline*}
    \begin{multline*}
      \mathbf{B:} \quad \ch{2 H2(g) + O2(g)-> H2O(g)} \\
      \Delta H_B = \SI{-572}{\kilo\joule}
    \end{multline*}
    \begin{multline*}
      \mathbf{C:} \quad \ch{CH4(g) + 2 O2(g) -> CO2(g) + 2 H2O(l)} \\
      \Delta H_C = \SI{-890}{\kilo\joule}
    \end{multline*}
  \end{exercise}

  \begin{solution}
    \SI{-76}{\kilo\joule}
  \end{solution}




  \begin{exercise}[
      tags    = {termodinámica, entalpía, ley de Hess},
      topics  = {química, termoquímica, termodinámica},
      source  = {},
    ]
    La aluminiotermia es una técnica utilizada para obtener hierro, a partir de aluminio y óxido de hierro(III), según la reacción:
    \[ \ch{2 Al(s) + Fe2O3(s) -> 2 Fe(l) + Al2O3(s)} \]
    \begin{enumerate}
      \item Deducir la entalpía de dicha reacción a partir de las ecuaciones termoquímicas:
        \begin{multline*}
          \mathbf{A:} \quad \ch{CH4(g) + 2 O2(g) -> CO2(g) + 2 H2O(l)} \\
          \Delta H_A = \SI{-890}{\kilo\joule}
        \end{multline*}
        \begin{multline*}
          \mathbf{B:} \quad \ch{CH4(g) + 2 O2(g) -> CO2(g) + 2 H2O(l)} \\
          \Delta H_B = \SI{-890}{\kilo\joule}
        \end{multline*}
      \item Calcular el calor desprendido en la aluminiotermia de \SI{16}{\gram} de óxido de hierro(III).
      \item ¿Qué cantidad de hierro se habrá formado cuando se liberan 195 kJ?
    \end{enumerate}
  \end{exercise}

  \begin{solution}
    \begin{enumerate}
      \item \SI{-780}{\kilo\joule};
      \item \SI{78}{\kilo\joule};
      \item \SI{0.5}{\mole\of{Fe}}.
    \end{enumerate}
  \end{solution}






\section{Entropía y espontaneidad}

  \subsection*{Entropía}

  \begin{exercise}[
      tags    = {termodinámica, entropía},
      topics  = {química, termoquímica, termodinámica},
      source  = {},
    ]
    Razona el signo que cabe esperar para la variación de entropía de los siguientes procesos:
    \begin{enumerate}
      \item \ch{I2(s) -> I2(g)}
      \item \ch{Na(s) + 1/2 Cl2(g) -> NaCl(s)}
      \item \ch{NaCl(s) -> NaCl(aq)}
      \item \ch{3 H2(g) + N2(g) -> 2 NH3(g)}
      \item \ch{HBr(g) -> Br2(g) + H2(g)}
    \end{enumerate}
  \end{exercise}

  \begin{solution}
    \begin{enumerate*}
      \item Aumenta;
      \item disminuye;
      \item aumenta;
      \item disminuye;
      \item aumenta.
    \end{enumerate*}
  \end{solution}




  \begin{exercise}[
      tags    = {termodinámica, entropía},
      topics  = {química, termoquímica, termodinámica},
      source  = {},
    ]
    Predice el cambio de entropía de las siguientes reacciones:
    \begin{enumerate}
      \item \ch{2 H2O(l) -> 2 H2(g) + O2(g)}
      \item \ch{3 H2(g) + N2(g) -> 2 NH3(g)}
      \item \ch{C(grafito) + 2 H2(g) + 1/2 O2(g) -> CH3OH(l)}
    \end{enumerate}
  \end{exercise}

  \begin{solution}
    \begin{enumerate*}
      \item Aumenta;
      \item disminuye;
      \item disminuye.
    \end{enumerate*}
  \end{solution}



  \begin{exercise}[
      tags    = {termodinámica, entropía},
      topics  = {química, termoquímica, termodinámica},
      source  = {},
    ]
    Indica el signo de la variación de la entropía cuando un charco de agua se hiela en invierno.
  \end{exercise}

  \begin{solution}
    La entropía disminuirá, debido a que el hielo es una estructura más ordenada que el agua líquida.
  \end{solution}




  \begin{exercise}[
      tags    = {termodinámica, entropía},
      topics  = {química, termoquímica, termodinámica},
      source  = {},
    ]
    Sabiendo la entropía molar estándar de las sustancias que intervienen en las siguientes reacciones, calcula la variación de entropía de ambas:
    \begin{enumerate}
      \item \ch{2 NO(g) + O2(g) -> 2 NO2(g)}
      \item \ch{HgO(s) -> Hg(l) + 1/2 O2(g)}
    \end{enumerate}
  \end{exercise}

  \begin{solution}
    \begin{enumerate*}
      \item \SI{-146.5}{\joule\per\kelvin}
      \item \SI{108.3}{\joule\per\kelvin}
    \end{enumerate*}
  \end{solution}




  \subsection*{Espontaneidad y energía libre de Gibbs}

  \begin{exercise}[
      tags    = {termodinámica, espontaneidad, Gibbs},
      topics  = {química, termoquímica, termodinámica},
      source  = {},
    ]
    Razona si una reacción será espontánea a \SI{25}{\celsius}, conociendo las variaciones de entalpía y de entropía a dicha temperatura: \( \Delta H^0 = \SI{-2}{\kilo\joule} \); \( \Delta S^0 = \SI{-3}{\joule\per\kelvin} \).
  \end{exercise}

  \begin{solution}
    \( \Delta G^0 = \SI{-1106}{\joule} \)
  \end{solution}




  \begin{exercise}[
      tags    = {termodinámica, espontaneidad},
      topics  = {química, termoquímica, termodinámica},
      source  = {},
    ]
    Calcula la variación de entropía en la síntesis del amoniaco, a partir de nitrógeno e hidrógeno, a \SI{25}{\celsius} y \SI{1}{atm}, e indica si favorece su espontaneidad. Consulta los valores de entropía estándar en la tabla.
  \end{exercise}

  \begin{solution}
    \( \Delta S^0 = \SI{-99.3}{\joule\per\kelvin} \)
  \end{solution}




  \begin{exercise}[
      tags    = {termodinámica, espontaneidad, Gibbs},
      topics  = {química, termoquímica, termodinámica},
      source  = {},
    ]
    Determina la variación de energía libre estándar para la reacción de descomposición del carbonato de calcio. Indica a partir de qué temperatura será espontáneo el proceso.

    \begin{gexdatos}
      \( \Delta H^0 = \SI{178}{\kilo\joule} \); \( \Delta S^0 = \SI{161}{\joule\per\kelvin} \).
    \end{gexdatos}
  \end{exercise}

  \begin{solution}
    \( \Delta G^0 = \SI{130}{\kilo\joule} \), \( T = \SI{833}{\celsius} \).
  \end{solution}




  \begin{exercise}[
      tags    = {termodinámica, espontaneidad, Gibbs},
      topics  = {química, termoquímica, termodinámica},
      source  = {},
    ]
    En la reacción \ch{F2(g) + 2 HCl(g) -> 2 HF(g) + Cl2(g)}, la variación de entropía es \( \Delta S^0_r = \SI{-6.04}{\joule\per\kelvin} \). Si sabemos que en la reacción de \SI{2}{\liter} de \ch{F2(g)} se desprenden \SI{28.87}{\kilo\joule}, ¿la reacción es espontánea a \SI{25}{\celsius}?
  \end{exercise}

  \begin{solution}
    \( \Delta G^0 = \SI{-350.3}{\kilo\joule} \).
  \end{solution}




  \begin{exercise}[
      tags    = {termodinámica, espontaneidad, Gibbs},
      topics  = {química, termoquímica, termodinámica},
      source  = {},
    ]
    El proceso que ocurre en las llamadas “bolsas de calor” es el siguiente:
    \begin{multline*}
      \ch{CaCl2(s) + H2O(l) -> Ca^{2+}(aq) + 2 Cl-(aq)} \\
      \Delta H = \SI{-83}{\kilo\joule}
    \end{multline*}
    Indica la veracidad o falsedad de las siguientes afirmaciones:
    \begin{enumerate}
      \item El proceso sólo es espontáneo si \( T < |∆H|/|∆S| \).
      \item El proceso siempre será espontáneo.
    \end{enumerate}

  \end{exercise}

  \begin{solution}
    AÚN SIN SOLUCIÓN % FIXME sin solución
  \end{solution}




  \begin{exercise}[
      tags    = {termodinámica, espontaneidad, Gibbs},
      topics  = {química, termoquímica, termodinámica},
      source  = {Química 1B VV 2015, p133, e40},
    ]
    Mediante la fotosíntesis, el dióxido de carbono se combina con el agua transformándose en hidratos de carbono, como la glucosa, y oxígeno molecular. Su fuente de energía es la luz del sol.
    \begin{enumerate}
      \item Escribe la ecuación para \SI{1}{\mole} de glucosa.
      \item Calcula la mínima energía solar necesaria para formar \SI{100}{\liter} de oxígeno a \SI{25}{\celsius} y \SI{1}{atm}.
      \item ¿Se trata de un proceso espontáneo a \SI{298}{\kelvin}? Razona y justifica la respuesta.
    \end{enumerate}

    \begin{gexdatos}
      \begin{tabular}{ccc}
        Sust. & \( \Delta H^0_f (\si{\kilo\joule\per\mole}) \) & \( S^0 (\si{\joule\per\kelvin\per\mole}) \) \\
        \toprule
        \ch{CO2(g)} & \( -393,5 \) & \( 213,6 \) \\
        \ch{H2O(l)} & \( -285,8 \) & \( 69,9 \) \\
        \ch{C6H12O6(s)} & \( -1273,5 \) & \( 212,1 \) \\
        \ch{O2(g)} & \( 0 \) & \( 205 \) \\
        \bottomrule
      \end{tabular}
    \end{gexdatos}
  \end{exercise}

  \begin{solution}
    \begin{enumerate*}
      \item \ch{6 CO2 + 6 H2O -> 6 O2 + C6H12O6};
      \item \SI{1910.2}{\kilo\joule};
      \item \( \Delta G^0 = \SI{2.88e6}{\joule\per\mole}\)
    \end{enumerate*}
  \end{solution}




  \begin{exercise}[
      tags    = {termodinámica, espontaneidad, Gibbs},
      topics  = {química, termoquímica, termodinámica},
      source  = {Química 1B VV 2015, p133, e41},
    ]
    Sabiendo que la temperatura de ebullición de un líquido es la temperatura a la que el líquido puro y el gas puro coexisten en equilibrio a \SI{1}{atm} de presión, es decir, \( \Delta G^0 = 0 \), y considerando la evaporación del bromo como:
    \[ \ch{Br2(l) <> Br2(g)} \]
    \begin{enumerate}
      \item Calcula \( \Delta H^0 \) a \SI{25}{\celsius}.
      \item Calcula \( S^0 \).
      \item Calcula \( \Delta G^0 \) a \SI{25}{\celsius} e indica si el proceso es espontáneo a dicha temperatura.
      \item Determina la temperatura de ebullición del \ch{Br_2} suponiendo que \( \Delta H^0 \) y \( \Delta S^0 \) no varían con la temperatura.
    \end{enumerate}

    \begin{gexdatos}
      \begin{tabular}{ccc}
        Sust. & \( \Delta H^0_f (\si{\kilo\joule\per\mole}) \) & \( S^0 (\si{\joule\per\kelvin\per\mole}) \) \\
        \toprule
        \ch{Br2 (l)} & \( 0 \) & \( 152,2 \) \\
        \ch{Br2 (g)} & \( 30,61 \) & \( 245,4 \) \\
        \bottomrule
      \end{tabular}
    \end{gexdatos}
  \end{exercise}

  \begin{solution}
    \begin{enumerate*}
      \item \SI{30.61}{\kilo\joule\per\mole};
      \item \SI{93.2}{\kilo\joule\per\kelvin\per\mole};
      \item \SI{2836.4}{\kilo\joule\per\mole};
      \item \SI{328.4}{\kelvin} (\SI{55}{\celsius}).
    \end{enumerate*}
  \end{solution}




  \begin{exercise}[
      tags    = {termodinámica, entalpía, entalpia de reacción, calor},
      topics  = {química, termoquímica, termodinámica},
      source  = {FQ 1B ANA 2016, p166, e32},
    ]
    Utilizando los valores que aparecen en la tabla, todos ellos obtenidos a la temperatura de \SI{25}{\celsius}, para la siguiente reacción de obtención del fosgeno:
    \[ \ch{CO (g) + Cl2 (g) -> COCl2 (g)} \]

    \begin{enumerate}
      \item Indica si será o no espontánea y si este hecho depende de la temperatura.
      \item Calcula la energía transferida al formarse \SI{5}{\gram} de fosgeno e indica, justificando tu respuesta, si se desprende o se absorbe la energía en el proceso.
    \end{enumerate}

    \begin{gexdatos}
      \begin{tabular}{ccc}
        Sust. & \( \Delta H^0_f \) (\si{\kilo\joule\per\mole}) & \( S^0 \) (\si{\joule\per\kelvin\per\mole}) \\
        \toprule
        \ch{CO (g)} & \( -110,4 \) & \( 197,7 \) \\
        \ch{Cl2 (g)} & \( 0,0 \) & \( 223,1 \) \\
        \ch{COCl2 (g)} & \( -222,8 \) & \( 288,8 \) \\
        \bottomrule
      \end{tabular}
    \end{gexdatos}

  \end{exercise}

  \begin{solution}
    \begin{enumerate}
      \item La reacción es espontánea a \SI{289}{\kelvin}.
      \item \( \Delta H_r = \SI{-5.68}{\kilo\joule} \).
    \end{enumerate}
  \end{solution}





  \begin{exercise}[
    tags    = {termodinámica, entalpía, entalpia de reacción, calor},
    topics  = {química, termoquímica, termodinámica},
    source  = {FQ 1B ANA 2016, p167, e34},
  ]
  La fermentación alcohólica supone la transformación de la glucosa sólida en etanol líquido y dióxido de carbono gas. Sabiendo que para esta reacción es \( \Delta H^0 = \SI{-69.4}{\kilo\joule} \) a \SI{25}{\celsius}, razona si el proceso será espontáneo a
  cualquier temperatura y calcula \( \Delta G^0 \) a \SI{25}{\celsius}.

  \begin{gexdatos}
    \( S^0 (\ch{C6H12O6}) = \SI{182.4}{\joule\per\mole\per\kelvin} \);
    \( S^0 (\ch{C2H6O}) = \SI{160.7}{\joule\per\mole\per\kelvin} \);
    \( S^0 (\ch{CO2}) = \SI{213.7}{\joule\per\mole\per\kelvin} \).
  \end{gexdatos}

\end{exercise}

\begin{solution}
  El proceso es espontáneo a cualquier temperatura. \( \Delta S_r = \SI{566.4}{\joule\per\kelvin} \); \( \Delta G_r (\SI{25}{\celsius}) = \SI{-238.2}{\kilo\joule} \)
\end{solution}







\end{document}
