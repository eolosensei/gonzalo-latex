\section{Entropía y espontaneidad}

  \subsection*{Entropía}

  \begin{exercise}[
      tags    = {termodinámica, entropía},
      topics  = {química, termoquímica, termodinámica},
      source  = {},
    ]
    Razona el signo que cabe esperar para la variación de entropía de los siguientes procesos:
    \begin{enumerate}
      \item \ch{I2(s) -> I2(g)}
      \item \ch{Na(s) + 1/2 Cl2(g) -> NaCl(s)}
      \item \ch{NaCl(s) -> NaCl(aq)}
      \item \ch{3 H2(g) + N2(g) -> 2 NH3(g)}
      \item \ch{HBr(g) -> Br2(g) + H2(g)}
    \end{enumerate}
  \end{exercise}

  \begin{solution}
    \begin{enumerate*}
      \item Aumenta;
      \item disminuye;
      \item aumenta;
      \item disminuye;
      \item aumenta.
    \end{enumerate*}
  \end{solution}




  \begin{exercise}[
      tags    = {termodinámica, entropía},
      topics  = {química, termoquímica, termodinámica},
      source  = {},
    ]
    Predice el cambio de entropía de las siguientes reacciones:
    \begin{enumerate}
      \item \ch{2 H2O(l) -> 2 H2(g) + O2(g)}
      \item \ch{3 H2(g) + N2(g) -> 2 NH3(g)}
      \item \ch{C(grafito) + 2 H2(g) + 1/2 O2(g) -> CH3OH(l)}
    \end{enumerate}
  \end{exercise}

  \begin{solution}
    \begin{enumerate*}
      \item Aumenta;
      \item disminuye;
      \item disminuye.
    \end{enumerate*}
  \end{solution}



  \begin{exercise}[
      tags    = {termodinámica, entropía},
      topics  = {química, termoquímica, termodinámica},
      source  = {},
    ]
    Indica el signo de la variación de la entropía cuando un charco de agua se hiela en invierno.
  \end{exercise}

  \begin{solution}
    La entropía disminuirá, debido a que el hielo es una estructura más ordenada que el agua líquida.
  \end{solution}




  \begin{exercise}[
      tags    = {termodinámica, entropía},
      topics  = {química, termoquímica, termodinámica},
      source  = {},
    ]
    Sabiendo la entropía molar estándar de las sustancias que intervienen en las siguientes reacciones, calcula la variación de entropía de ambas:
    \begin{enumerate}
      \item \ch{2 NO(g) + O2(g) -> 2 NO2(g)}
      \item \ch{HgO(s) -> Hg(l) + 1/2 O2(g)}
    \end{enumerate}
  \end{exercise}

  \begin{solution}
    \begin{enumerate*}
      \item \SI{-146.5}{\joule\per\kelvin}
      \item \SI{108.3}{\joule\per\kelvin}
    \end{enumerate*}
  \end{solution}




  \subsection*{Espontaneidad y energía libre de Gibbs}

  \begin{exercise}[
      tags    = {termodinámica, espontaneidad, Gibbs},
      topics  = {química, termoquímica, termodinámica},
      source  = {},
    ]
    Razona si una reacción será espontánea a \SI{25}{\celsius}, conociendo las variaciones de entalpía y de entropía a dicha temperatura: \( \Delta H^0 = \SI{-2}{\kilo\joule} \); \( \Delta S^0 = \SI{-3}{\joule\per\kelvin} \).
  \end{exercise}

  \begin{solution}
    \( \Delta G^0 = \SI{-1106}{\joule} \)
  \end{solution}




  \begin{exercise}[
      tags    = {termodinámica, espontaneidad},
      topics  = {química, termoquímica, termodinámica},
      source  = {},
    ]
    Calcula la variación de entropía en la síntesis del amoniaco, a partir de nitrógeno e hidrógeno, a \SI{25}{\celsius} y \SI{1}{atm}, e indica si favorece su espontaneidad. Consulta los valores de entropía estándar en la tabla.
  \end{exercise}

  \begin{solution}
    \( \Delta S^0 = \SI{-99.3}{\joule\per\kelvin} \)
  \end{solution}




  \begin{exercise}[
      tags    = {termodinámica, espontaneidad, Gibbs},
      topics  = {química, termoquímica, termodinámica},
      source  = {},
    ]
    Determina la variación de energía libre estándar para la reacción de descomposición del carbonato de calcio. Indica a partir de qué temperatura será espontáneo el proceso.

    \begin{gexdatos}
      \( \Delta H^0 = \SI{178}{\kilo\joule} \); \( \Delta S^0 = \SI{161}{\joule\per\kelvin} \).
    \end{gexdatos}
  \end{exercise}

  \begin{solution}
    \( \Delta G^0 = \SI{130}{\kilo\joule} \), \( T = \SI{833}{\celsius} \).
  \end{solution}




  \begin{exercise}[
      tags    = {termodinámica, espontaneidad, Gibbs},
      topics  = {química, termoquímica, termodinámica},
      source  = {},
    ]
    En la reacción \ch{F2(g) + 2 HCl(g) -> 2 HF(g) + Cl2(g)}, la variación de entropía es \( \Delta S^0_r = \SI{-6.04}{\joule\per\kelvin} \). Si sabemos que en la reacción de \SI{2}{\liter} de \ch{F2(g)} se desprenden \SI{28.87}{\kilo\joule}, ¿la reacción es espontánea a \SI{25}{\celsius}?
  \end{exercise}

  \begin{solution}
    \( \Delta G^0 = \SI{-350.3}{\kilo\joule} \).
  \end{solution}




  \begin{exercise}[
      tags    = {termodinámica, espontaneidad, Gibbs},
      topics  = {química, termoquímica, termodinámica},
      source  = {},
    ]
    El proceso que ocurre en las llamadas “bolsas de calor” es el siguiente:
    \begin{multline*}
      \ch{CaCl2(s) + H2O(l) -> Ca^{2+}(aq) + 2 Cl-(aq)} \\
      \Delta H = \SI{-83}{\kilo\joule}
    \end{multline*}
    Indica la veracidad o falsedad de las siguientes afirmaciones:
    \begin{enumerate}
      \item El proceso sólo es espontáneo si \( T < |∆H|/|∆S| \).
      \item El proceso siempre será espontáneo.
    \end{enumerate}

  \end{exercise}

  \begin{solution}
    AÚN SIN SOLUCIÓN % FIXME sin solución
  \end{solution}




  \begin{exercise}[
      tags    = {termodinámica, espontaneidad, Gibbs},
      topics  = {química, termoquímica, termodinámica},
      source  = {Química 1B VV 2015, p133, e40},
    ]
    Mediante la fotosíntesis, el dióxido de carbono se combina con el agua transformándose en hidratos de carbono, como la glucosa, y oxígeno molecular. Su fuente de energía es la luz del sol.
    \begin{enumerate}
      \item Escribe la ecuación para \SI{1}{\mole} de glucosa.
      \item Calcula la mínima energía solar necesaria para formar \SI{100}{\liter} de oxígeno a \SI{25}{\celsius} y \SI{1}{atm}.
      \item ¿Se trata de un proceso espontáneo a \SI{298}{\kelvin}? Razona y justifica la respuesta.
    \end{enumerate}

    \begin{gexdatos}
      \begin{tabular}{ccc}
        Sust. & \( \Delta H^0_f (\si{\kilo\joule\per\mole}) \) & \( S^0 (\si{\joule\per\kelvin\per\mole}) \) \\
        \toprule
        \ch{CO2(g)} & \( -393,5 \) & \( 213,6 \) \\
        \ch{H2O(l)} & \( -285,8 \) & \( 69,9 \) \\
        \ch{C6H12O6(s)} & \( -1273,5 \) & \( 212,1 \) \\
        \ch{O2(g)} & \( 0 \) & \( 205 \) \\
        \bottomrule
      \end{tabular}
    \end{gexdatos}
  \end{exercise}

  \begin{solution}
    \begin{enumerate*}
      \item \ch{6 CO2 + 6 H2O -> 6 O2 + C6H12O6};
      \item \SI{1910.2}{\kilo\joule};
      \item \( \Delta G^0 = \SI{2.88e6}{\joule\per\mole}\)
    \end{enumerate*}
  \end{solution}




  \begin{exercise}[
      tags    = {termodinámica, espontaneidad, Gibbs},
      topics  = {química, termoquímica, termodinámica},
      source  = {Química 1B VV 2015, p133, e41},
    ]
    Sabiendo que la temperatura de ebullición de un líquido es la temperatura a la que el líquido puro y el gas puro coexisten en equilibrio a \SI{1}{atm} de presión, es decir, \( \Delta G^0 = 0 \), y considerando la evaporación del bromo como:
    \[ \ch{Br2(l) <> Br2(g)} \]
    \begin{enumerate}
      \item Calcula \( \Delta H^0 \) a \SI{25}{\celsius}.
      \item Calcula \( S^0 \).
      \item Calcula \( \Delta G^0 \) a \SI{25}{\celsius} e indica si el proceso es espontáneo a dicha temperatura.
      \item Determina la temperatura de ebullición del \ch{Br_2} suponiendo que \( \Delta H^0 \) y \( \Delta S^0 \) no varían con la temperatura.
    \end{enumerate}

    \begin{gexdatos}
      \begin{tabular}{ccc}
        Sust. & \( \Delta H^0_f (\si{\kilo\joule\per\mole}) \) & \( S^0 (\si{\joule\per\kelvin\per\mole}) \) \\
        \toprule
        \ch{Br2 (l)} & \( 0 \) & \( 152,2 \) \\
        \ch{Br2 (g)} & \( 30,61 \) & \( 245,4 \) \\
        \bottomrule
      \end{tabular}
    \end{gexdatos}
  \end{exercise}

  \begin{solution}
    \begin{enumerate*}
      \item \SI{30.61}{\kilo\joule\per\mole};
      \item \SI{93.2}{\kilo\joule\per\kelvin\per\mole};
      \item \SI{2836.4}{\kilo\joule\per\mole};
      \item \SI{328.4}{\kelvin} (\SI{55}{\celsius}).
    \end{enumerate*}
  \end{solution}




  \begin{exercise}[
      tags    = {termodinámica, entalpía, entalpia de reacción, calor},
      topics  = {química, termoquímica, termodinámica},
      source  = {FQ 1B ANA 2016, p166, e32},
    ]
    Utilizando los valores que aparecen en la tabla, todos ellos obtenidos a la temperatura de \SI{25}{\celsius}, para la siguiente reacción de obtención del fosgeno:
    \[ \ch{CO (g) + Cl2 (g) -> COCl2 (g)} \]

    \begin{enumerate}
      \item Indica si será o no espontánea y si este hecho depende de la temperatura.
      \item Calcula la energía transferida al formarse \SI{5}{\gram} de fosgeno e indica, justificando tu respuesta, si se desprende o se absorbe la energía en el proceso.
    \end{enumerate}

    \begin{gexdatos}
      \begin{tabular}{ccc}
        Sust. & \( \Delta H^0_f \) (\si{\kilo\joule\per\mole}) & \( S^0 \) (\si{\joule\per\kelvin\per\mole}) \\
        \toprule
        \ch{CO (g)} & \( -110,4 \) & \( 197,7 \) \\
        \ch{Cl2 (g)} & \( 0,0 \) & \( 223,1 \) \\
        \ch{COCl2 (g)} & \( -222,8 \) & \( 288,8 \) \\
        \bottomrule
      \end{tabular}
    \end{gexdatos}

  \end{exercise}

  \begin{solution}
    \begin{enumerate}
      \item La reacción es espontánea a \SI{289}{\kelvin}.
      \item \( \Delta H_r = \SI{-5.68}{\kilo\joule} \).
    \end{enumerate}
  \end{solution}




  \begin{exercise}[
        tags    = {termodinámica, entalpía, entalpia de reacción, calor},
        topics  = {química, termoquímica, termodinámica},
        source  = {FQ 1B ANA 2016, p167, e34},
    ]
    La fermentación alcohólica supone la transformación de la glucosa sólida en etanol líquido y dióxido de carbono gas. Sabiendo que para esta reacción es \( \Delta H^0 = \SI{-69.4}{\kilo\joule} \) a \SI{25}{\celsius}, razona si el proceso será espontáneo a
    cualquier temperatura y calcula \( \Delta G^0 \) a \SI{25}{\celsius}.

    \begin{gexdatos}
        \( S^0 (\ch{C6H12O6}) = \SI{182.4}{\joule\per\mole\per\kelvin} \);
        \( S^0 (\ch{C2H6O}) = \SI{160.7}{\joule\per\mole\per\kelvin} \);
        \( S^0 (\ch{CO2}) = \SI{213.7}{\joule\per\mole\per\kelvin} \).
    \end{gexdatos}
  \end{exercise}

  \begin{solution}
    El proceso es espontáneo a cualquier temperatura. \( \Delta S_r = \SI{566.4}{\joule\per\kelvin} \); \( \Delta G_r (\SI{25}{\celsius}) = \SI{-238.2}{\kilo\joule} \)
  \end{solution}
