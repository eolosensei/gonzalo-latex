\section{Calor y entalpía}

  \subsection*{Entalpía de reacción}


    \begin{exercise}[
        tags    = {termodinámica, entalpía, entalpia de reacción, calor},
        topics  = {química, termoquímica, termodinámica},
        source  = {FQ 1B ANA 2016, p164, e5},
      ]
      Escribe las ecuaciones termoquímicas que describen los procesos de formación de las siguientes sustancias a partir de sus elementos constituyentes en estado estándar:
      \begin{enumerate}
        \item \ch{CO} (g).
        \item \ch{H2O} (g).
        \item \ch{H2O} (l).
      \end{enumerate}
    \end{exercise}

    \begin{solution}
      \begin{enumerate}
        \item \ch{C (grafito) + 1/2 O2 (g) -> CO (g)} \( \Delta H^0_f = \SI{-110.5}{\kilo\joule\per\mole} \).
        \item \ch{H2 (g) + 1/2 O2 (g) -> H2O (g)} \( \Delta H^0_f = \SI{-241.8}{\kilo\joule\per\mole} \).
        \item ch{H2 (g) + 1/2 O2 (g) -> H2O (l)} \( \Delta H^0_f = \SI{-285.8}{\kilo\joule\per\mole} \).
      \end{enumerate}
    \end{solution}




    \begin{exercise}[
        tags    = {termodinámica, entalpía, entalpia de reacción, calor},
        topics  = {química, termoquímica, termodinámica},
        source  = {FQ 1B ANA 2016, p165, e14},
      ]
      Calcula la entalpía de la reacción:
      \[ \ch{CO (g) + 1/2 O2 (g) -> CO2 (g)} \]
      % sabiendo que en la formación de \SI{1.0}{\gram} de producto se desprenden \SI{6.43}{\kilo\joule}.
    \end{exercise}

    \begin{solution}
      \( \Delta H_r = \SI{-283.0}{\kilo\joule} \)
    \end{solution}




    \begin{exercise}[
        tags    = {termodinámica, entalpía, entalpia de reacción, calor},
        topics  = {química, termoquímica, termodinámica},
        source  = {FQ 1B ANA 2016, p165, e17},
      ]
      Calcula la entalpía de formación del ácido acético, \ch{CH3COOH}, sabiendo que su entalpía de combustión es \SI{-870,3}{\kilo\joule\per\mole} y que las entalpías estándar de formación del \ch{CO2} y del \ch{H2O} (l) son \SIlist{-393.5;-285.8}{\kilo\joule\per\mole}, respectivamente.
    \end{exercise}

    \begin{solution}
      \( \Delta H^0_r = \SI{-488.3}{\kilo\joule\per\mole} \)
    \end{solution}




    \begin{exercise}[
        tags    = {termodinámica, entalpía, entalpia de reacción, calor},
        topics  = {química, termoquímica, termodinámica},
        source  = {FQ 1B ANA 2016, p165, e23},
      ]
      Calcula la entalpía de la reacción de hidrogenación de eteno, \ch{C2H4}, a etano, \ch{C2H6}:
      \[ \ch{C2H4 (g) + H2 (g) -> C2H6 (g)} \]
      sabiendo que las entalpías de combustión, en \si{kcal/mol}, del eteno, del etano y del hidrógeno son \numlist{-337.3;-372.9;-68.38}, respectivamente.
    \end{exercise}

    \begin{solution}
      \( \Delta H_r = \SI{-137}{\kilo\joule} \)
    \end{solution}




  \begin{exercise}[
      tags    = {termodinámica, entalpía, entalpia de reacción, calor},
      topics  = {química, termoquímica, termodinámica},
      source  = {},
    ]
    Utilizando los datos tabulados en tu libro de texto (o buscándolos en internet), calcular la variación de entalpía en la formación de las siguientes sustancias.
    \begin{enumerate}
      \item \SI{180}{\gram} de agua en estado gaseoso.
      \item \SI{300}{\gram} de óxido de nitrógeno(II) gaseoso.
    \end{enumerate}
  \end{exercise}

  \begin{solution}
    \begin{enumerate*}
      \item \SI{-2420}{\kilo\joule}; \item \SI{900}{\kilo\joule}
    \end{enumerate*}
  \end{solution}









  \begin{exercise}[
      tags    = {termodinámica, entalpía, entalpia de reacción, calor},
      topics  = {química, termoquímica, termodinámica},
      source  = {},
    ]
    La energía que proporciona una golosina de glucosa, \ch{C6H12O6}, es la misma si la digerimos que si la quemamos en el aire. La diferencia es la velocidad de liberación de la energía.
    \begin{enumerate}
      \item Escribir la ecuación termoquímica de la combustión de la glucosa si la entalpía de combustión es \SI{2800}{kJ/mol}.
      \item Calcular la energía que proporciona una golosina con \SI{18}{\gram} de glucosa.
    \end{enumerate}
  \end{exercise}

  \begin{solution}
    b) \SI{280}{\kilo\joule}
  \end{solution}




  \begin{exercise}[
      tags    = {termodinámica, entalpía, ley de Hess},
      topics  = {química, termoquímica, termodinámica},
      source  = {},
    ]
    Alguien ha propuesto fabricar diamantes a partir de la oxidación del metano, según la reacción:
    \[ \ch{CH4(g) + O2(g) -> C(diamante) + 2 H2O(l)}  \]
    Calcular la variación de entalpía del proceso conocidas las variaciones de entalpía de las siguientes reacciones:
    \begin{multline*}
      \mathbf{A:}\quad\ch{CH4(g) + 2 O2(g) -> CO2(g) + 2 H2O(l)} \\
        \Delta H_A = \SI{-890}{\kilo\joule}
    \end{multline*}
    \begin{multline*}
      \mathbf{B:}\quad\ch{C(diamante) + O2(g) -> CO2(g) + 2 H2O(l)} \\
        \Delta H_B = \SI{-395}{\kilo\joule}
    \end{multline*}


  \end{exercise}

  \begin{solution}
    \SI{-495}{\kilo\joule}
  \end{solution}




  \begin{exercise}[
      tags    = {termodinámica, entalpía, ley de Hess},
      topics  = {química, termoquímica, termodinámica},
      source  = {},
    ]
    La variación de entalpía de la reacción de síntesis del metano no se puede calcular directamente:
    \[ \ch{C(grafito) + 2 H2(g) -> CH4(g)}  \]
    Calcular su valor a partir de las siguientes reacciones:
    \begin{multline*}
      \mathbf{A:} \quad \ch{C(grafito) + O2(g) -> CO2(g) + 2 H2O(l)} \\
      \Delta H_A = \SI{-394}{\kilo\joule}
    \end{multline*}
    \begin{multline*}
      \mathbf{B:} \quad \ch{2 H2(g) + O2(g)-> H2O(g)} \\
      \Delta H_B = \SI{-572}{\kilo\joule}
    \end{multline*}
    \begin{multline*}
      \mathbf{C:} \quad \ch{CH4(g) + 2 O2(g) -> CO2(g) + 2 H2O(l)} \\
      \Delta H_C = \SI{-890}{\kilo\joule}
    \end{multline*}
  \end{exercise}

  \begin{solution}
    \SI{-76}{\kilo\joule}
  \end{solution}




  \begin{exercise}[
      tags    = {termodinámica, entalpía, ley de Hess},
      topics  = {química, termoquímica, termodinámica},
      source  = {},
    ]
    La aluminiotermia es una técnica utilizada para obtener hierro, a partir de aluminio y óxido de hierro(III), según la reacción:
    \[ \ch{2 Al(s) + Fe2O3(s) -> 2 Fe(l) + Al2O3(s)} \]
    \begin{enumerate}
      \item Deducir la entalpía de dicha reacción a partir de las ecuaciones termoquímicas:
        \begin{multline*}
          \mathbf{A:} \quad \ch{CH4(g) + 2 O2(g) -> CO2(g) + 2 H2O(l)} \\
          \Delta H_A = \SI{-890}{\kilo\joule}
        \end{multline*}
        \begin{multline*}
          \mathbf{B:} \quad \ch{CH4(g) + 2 O2(g) -> CO2(g) + 2 H2O(l)} \\
          \Delta H_B = \SI{-890}{\kilo\joule}
        \end{multline*}
      \item Calcular el calor desprendido en la aluminiotermia de \SI{16}{\gram} de óxido de hierro(III).
      \item ¿Qué cantidad de hierro se habrá formado cuando se liberan 195 kJ?
    \end{enumerate}
  \end{exercise}

  \begin{solution}
    \begin{enumerate}
      \item \SI{-780}{\kilo\joule};
      \item \SI{78}{\kilo\joule};
      \item \SI{0.5}{\mole\of{Fe}}.
    \end{enumerate}
  \end{solution}
