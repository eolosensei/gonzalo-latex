\section{Bloque 4. Transformaciones energéticas y espontaneidad (tema 6 del libro)}

\subsection*{Ejercicios del tema}

  \begin{exercise}[
      tags    = {},
      topics  = {química, termodinámica, termoquímica},
      source  = {FQ 1B MGH 2016, p159, e5},
    ]
    (159.5) Determina la variación de energía interna que sufre un sistema
    cuando:
    \begin{enumerate}
      \item Realiza un trabajo de \SI{600}{\joule} y cede \SI{40}{cal} al entorno.
      \item Absorbe \SI{300}{cal} del entorno y se realiza un trabajo de compresión de \SI{5}{\kilo\joule}.
    \end{enumerate}
  \end{exercise}

  \begin{solution}
    \begin{enumerate*}
      \item \( \Delta U = \SI{-767}{\joule} \); \item \( \Delta U = \SI{6.25e3}{\joule} \)
    \end{enumerate*}
  \end{solution}




  \begin{exercise}[
      tags    = {},
      topics  = {química, termodinámica, termoquímica},
      source  = {FQ 1B MGH 2016, p165, e13},
    ]
    (165.13) La descomposición térmica del clorato de potasio (\ch{KClO3})
    origina cloruro de potasio (\ch{KCl}) y oxígeno molecular. Calcula
    el calor que se desprende cuando se obtienen \SI{150}{\liter} de
    oxígeno medidos a \SI{25}{\celsius} y \SI{1}{atm} de presión.

    \begin{gexdatos}
      \( \Delta H^0_f\,(\si{kJ/mol}) \): \( \ch{KClO3_{(s)}} = -91,2 \); \( \ch{KCl_{(s)}} = -436 \)
    \end{gexdatos}
  \end{exercise}

  \begin{solution}
    Se desprenden \SI{1.41e3}{kJ}.
  \end{solution}




  \begin{exercise}[
      tags    = {},
      topics  = {química, termodinámica, termoquímica},
      source  = {FQ 1B MGH 2016, p165, e6},
    ]
    (165.14) Las entalpías estándar de formación del propano (g),
    dióxido de carbono (g) y agua (l), son respectivamente:
    \SIlist{-103,8;-393,5;-285,8}{kJ/mol}. Calcula:
    \begin{enumerate}
      \item La entalpía de la reacción de combustión del propano.
      \item Las calorías generadas en la combustión de una bombona de propano de \SI{1.80}{\liter} a \SI{25}{\celsius} y \SI{4}{atm} de presión.
    \end{enumerate}
  \end{exercise}

  \begin{solution}
    \begin{enumerate*}
      \item \( \Delta H^0_c = \SI{-2.22e3}{kJ/mol} \) de propano;
      \item Se desprenden \SI{156}{kcal}
    \end{enumerate*}
  \end{solution}




  \begin{exercise}[
      tags    = {},
      topics  = {química, termodinámica, termoquímica},
      source  = {FQ 1B MGH 2016, p165, e15},
    ]
    (165.15) En la reacción del oxígeno molecular con el cobre para formar
    óxido de cobre(II) se desprenden \SI{2.30}{kJ} por cada gramo de
    cobre que reacciona, a \SI{298}{\kelvin} y \SI{760}{\mmHg}. Calcula:
    \begin{enumerate}
      \item La entalpía de formación del óxido de cobre(II).
      \item El calor desprendido a presión constante cuando reaccionan \SI{100}{\liter} de oxígeno, medidos a \SI{1.5}{atm} y \SI{27}{\celsius}
    \end{enumerate}
  \end{exercise}

  \begin{solution}
    \begin{enumerate*}
      \item \( \Delta H^0_f\,(\ch{CuO}) = \SI{-146}{kJ/mol} \);
      \item Se desprenden \SI{1.78e3}{kJ}.
    \end{enumerate*}
  \end{solution}




  \begin{exercise}[
      tags    = {},
      topics  = {química, termodinámica, termoquímica},
      source  = {FQ 1B MGH 2016, p165, e16},
    ]
    (165.16) En la combustión completa de \SI{1.00}{\gram} de etanol (\ch{CH3-CH2OH}) se desprenden \SI{29.8}{kJ} y en la combustión de \SI{1.00}{\gram}
    de ácido etanoico (\ch{CH3-COOH}) se desprenden \SI{14.5}{kJ}.
    Determina numéricamente:
    \begin{enumerate}
      \item Cuál de las dos sustancias tiene mayor entalpía de combustión.
      \item Cuál de las dos sustancias tiene mayor entalpía de formación.
    \end{enumerate}
  \end{exercise}

  \begin{solution}
    \begin{enumerate}
      \item \( \Delta H^○_c (\textit{etanol}) = \SI{-1.37e3}{kJ/mol} \); \newline
            \( \Delta H^\mdsmwhtcircle_c (\textit{ácido etanoico}) = \SI{-870}{kJ/mol} \) % REVIEW comportamiento experimental del círculo de estado normal
      \item \( \Delta H_c^^b0 (\textit{etanol}) = \SI{-273}{kJ/mol} \); \newline
            \( \Delta H_c^^b0 (\textit{ácido etanoico}) = \SI{-489}{kJ/mol} \) % REVIEW comportamiento experimental del círculo de estado normal
    \end{enumerate}
  \end{solution}




  \begin{exercise}[
      tags    = {},
      topics  = {química, termodinámica, termoquímica},
      source  = {FQ 1B MGH 2016, p168, e24},
    ]
    (168.24) Calcula la entalpía de la reacción:
    \ch{CH4_{(g)} + Cl2_{(g)} -> CH3Cl_{(g)} + HCl_{(g)}} a partir de:
    \begin{enumerate}
      \item Las energías de enlace.
      \item Las entalpías de formación.
    \end{enumerate}

    \begin{gexdatos}
      \begin{itemize}
        \item \textit{Energías de enlace (\si{kJ/mol})}:

        \( \ch{C-H} = 414 \); \( \ch{Cl-Cl} = 244 \); \( \ch{C-Cl} = 330 \); \( \ch{H-Cl} = 430 \).

        \item \textit{Entalpías de formación (\si{kJ/mol})}:

        \( \Delta H^0_f\,(\ch{CH4}) = -74,9 \); \( \Delta H^0_f\,(\ch{CH3Cl}) = -82,0 \); \( \Delta H^0_f\,(\ch{HCl}) = -92,3 \).
      \end{itemize}
    \end{gexdatos}

  \end{exercise}

  \begin{solution}
    \begin{enumerate*}
      \item \( \Delta H^0_R = \SI{-102}{kJ} \); \item \( \Delta H^0_R = \SI{-99.4}{kJ} \)
    \end{enumerate*}
  \end{solution}




  \begin{exercise}[
      tags    = {},
      topics  = {química, termodinámica, termoquímica},
      source  = {FQ 1B MGH 2016, p168, e25},
    ]
    (168.25) El eteno se hidrogena para dar etano, según:
    \[ \ch{CH2=CH2_{(g)} + H2_{(g)} -> CH3-CH3_{(g)}} \quad \Delta H^0_R = \SI{-130}{kJ} \]
    Calcula la energía del enlace \ch{C=C}, si las energías de los
    enlaces \ch{C-C}, \ch{H-H} y \ch{C-H} son, respectivamente, \SIlist{347;436;414}{kJ/mol}.
  \end{exercise}

  \begin{solution}
    \( \Delta H^0 (\ch{C=C}) = \SI{609}{kJ/mol} \)
  \end{solution}





  \begin{exercise}[
      tags    = {},
      topics  = {química, termodinámica, termoquímica},
      source  = {FQ 1B MGH 2016, p168, e26},
    ]
    (168.26) A partir de los siguientes datos:
    \begin{itemize}
      \item Entalpía estándar de sublimación del \( \ch{C_{(s)}} = \SI{717}{kJ/mol} \).
      \item Entalpía de formación del \( \ch{CH3-CH3_{(g)}} = \SI{-85,0}{kJ/mol} \).
      \item Entalpía media del enlace \( \ch{H-H} = \SI{436}{kJ/mol} \).
      \item Entalpía media del enlace \( \ch{C-C} = \SI{347}{kJ/mol} \).
    \end{itemize}
    Responde a las siguientes cuestiones: % REVIEW comportamiento del espaciado
    \begin{enumerate}
      \item Calcula la variación de entalpía de la reacción:
      \ch{2 C_{(g)} + 3 H2_{(s)} -> CH3-CH3_{(g)}} e indica si es exotérmica
      o endotérmica.
      \item Determina el valor medio del enlace \ch{C-H}.
    \end{enumerate}
  \end{exercise}

  \begin{solution}
    \begin{enumerate*}
      \item \( \Delta H^0_R = \SI{-1.52e3}{kJ} \); \item \( \Delta H^0 (\ch{C-H}) = \SI{413}{kJ/mol} \)
    \end{enumerate*}
  \end{solution}






  \subsection*{Problemas resueltos}

  \subsubsection*{Entalpias de formación, de reacción y de combustión}

  \begin{exercise}[
      tags    = {},
      topics  = {química, termodinámica, termoquímica},
      source  = {FQ 1B MGH 2016, p179, e4},
    ]
    (179.4) El sulfuro de carbono reacciona con el oxígeno según:
     \[ \ch{CS2_{(l)} + 3 O2_{(g)} -> CO2_{(g)} + 2 SO2_{(g)}} \quad \Delta H_R = \SI{-1072}{kJ} \]
    \begin{enumerate}
      \item Calcula la entalpía de formación del \ch{CS2}
      \item Halla el volumen de \ch{SO2} emitido a la atmósfera, a \SI{1}{atm} y \SI{25}{\celsius}, cuando se ha liberado una energía de \SI{6000}{kJ}
    \end{enumerate}

    \begin{gexdatos}
      \( \Delta H^0_f\,(\si{kJ/mol}) \): \( \ch{CO2_{(g)}} = -393,5 \); \( \ch{SO2_{(g)}} = -296,4 \).
    \end{gexdatos}

  \end{exercise}

  \begin{solution}
    \begin{enumerate*}
      \item \( \Delta H^0_f\,(\ch{CS2_{(l)}}) = \SI{85.7}{kJ/mol} \); \item \( V = \SI{274}{\liter} \)
    \end{enumerate*}
  \end{solution}




  \begin{exercise}[
      tags    = {},
      topics  = {química, termodinámica, termoquímica},
      source  = {FQ 1B MGH 2016, p179, e5},
    ]
    (179.5) El dióxido de manganeso se reduce a manganeso metal reaccionando con el aluminio según:
    \[ \ch{MnO2_{(s)} + Al_{(s)} -> Al2O3_{(s)} + Mn_{(s)}} \]

    \begin{enumerate}
      \item Halla la entalpía de esa reacción sabiendo que las entalpías
      de formación valen:
      \( \Delta H^0_f\,(\ch{Al2O3}) = \SI{-1676}{kJ/mol} \); \( \Delta H^0_f\,(\ch{MnO2}) = \SI{-520}{kJ/mol} \)
      \item ¿Qué energía se transfiere cuando reaccionan \SI{10.0}{\gram}
      \ch{MnO2} con \SI{10.0}{\gram} de \ch{Al}?
    \end{enumerate}
  \end{exercise}

  \begin{solution}
    \begin{enumerate*}
      \item \( \Delta H_R = \SI{-896}{kJ/mol}-896 \) de \ch{Al2O3};
      \item \SI{68.7}{kJ} se desprenden
    \end{enumerate*}
  \end{solution}




  \begin{exercise}[
      tags    = {},
      topics  = {química, termodinámica, termoquímica},
      source  = {FQ 1B MGH 2016, p179, e6},
    ]
    (179.6) Durante la fotosíntesis, las plantas verdes sintetizan la glucosa según la siguiente reacción:
    \[ \ch{6 CO2_{(g)} + 6 H2O_{(l)} -> C6H12O6_{(s)} + 6 O2_{(g)}} \quad \Delta H_R = \SI{2815}{kJ/mol} \]
    \begin{enumerate}
      \item ¿Cuál es la entalpía de formación de la glucosa?
      \item ¿Qué energía se requiere para obtener \SI{50.0}{\gram} de glucosa?
      \item ¿Cuántos litros de oxígeno, en condiciones estándar, se desprenden por cada gramo de glucosa formado?
    \end{enumerate}

    \begin{gexdatos}
      \( \Delta H^0_f\,(\si{kJ/mol}) \): \( \ch{H2O_{(l)}} = -285,8 \); \( \ch    {CO2_{(g)}} = -393,5 \)
    \end{gexdatos}
  \end{exercise}

  \begin{solution}
    \begin{enumerate*}
      \item \( \Delta H^0_f = \SI{-1.26e3}{kJ/mol} \);
      \item \SI{782}{kJ};
      \item \SI{0.8}{\liter}
    \end{enumerate*}
  \end{solution}




  \begin{exercise}[
      tags    = {},
      topics  = {química, termodinámica, termoquímica},
      source  = {FQ 1B MGH 2016, p179, e7},
    ]
    (179.7) Las entalpías de combustión del etano y del eteno son
    \SIlist{-1560;1410}{kJ/mol}, respectivamente. Determina:
    \begin{enumerate}
      \item El valor de \( \Delta H^0_f \) para el etano y el eteno.
      \item Razona si el proceso de hidrogenación del eteno a etano
      es un proceso endotérmico o exotérmico.
      \item Calcula el calor que se desprende en la combustión de
      \SI{50.0}{\gram} de cada gas.
    \end{enumerate}

    \begin{gexdatos}
      \( \Delta H^0_f\,(\si{kJ/mol}) \): \( \ch{CO2_{(g)}} = -393,5 \); \( \ch{H2O_{(l)}} = -285,9 \)
    \end{gexdatos}
  \end{exercise}

  \begin{solution}
    \begin{enumerate*}
      \item \( \Delta H^0_f\,(\ch{C2H6})= \SI{-84,7}{kJ/mol} \); \( \Delta H^0_f\,(\ch{C2H4}) = \SI{51.2}{kJ/mol} \); \item Exotérmico; \item etano: \SI{2.60e3}{kJ}; eteno: \SI{2.52e3}{kJ}.
    \end{enumerate*}
  \end{solution}




  \begin{exercise}[
      tags    = {},
      topics  = {química, termodinámica, termoquímica},
      source  = {FQ 1B MGH 2016, p179, e8},
    ]
    (179.8) La gasolina es una mezcla compleja de hidrocarburos que
    vamos a considerar como si estuviera formada únicamente
    por hidrocarburos saturados de fórmula (\ch{C8H18})
    \begin{enumerate}
      \item Calcula el calor que se desprende en la combustión de
      \SI{50.0}{\liter} litros de gasolina (\( d = \SI{0.78}{\gram\per\milli\liter} \)).
      \item Halla la masa de \ch{CO2} que se emite a la atmósfera en esa
      combustión.
      \item Si el consumo de un vehículo es de \SI{7.00}{\liter} por cada
      \SI{100}{km}, ¿qué energía necesita por cada \si{km} recorrido?
    \end{enumerate}

    \begin{gexdatos}
      \( \Delta H^0_f\,(\si{kJ/mol}) \): \( \ch{CO2_{(g)}} = -394 \); \( \ch{H2O_{(l)}} = -286 \); \( \ch{C8H18_{(l)}} = -250 \)
    \end{gexdatos}
  \end{exercise}

  \begin{solution}
    \begin{enumerate*}
      \item \SI{1.87e6}{kJ}; \item \SI{120}{\kilo\gram}; \item \SI{2.62e3}{kJ/km}
    \end{enumerate*}
  \end{solution}




  \begin{exercise}[
      tags    = {},
      topics  = {química, termodinámica, termoquímica},
      source  = {FQ 1B MGH 2016, p179, e10},
    ]
    (179.10) Se quema benceno (\ch{C6H6}) en exceso de oxígeno, liberando energía.

    \begin{enumerate}
      \item Formula la reacción de combustión del benceno.
      \item Calcula la entalpía de combustión estándar de un mol de
      benceno líquido.
      \item Calcula el volumen de oxígeno, medido a \SI{25}{\celsius} y \SI{5}{atm}, necesario para quemar \SI{1}{\liter} de benceno líquido.
      \item Calcula el calor necesario para evaporar \SI{10}{\liter} de benceno líquido.
    \end{enumerate}

    \begin{gexdatos}
      \( \Delta H^0_f\,(\si{kJ/mol}) \):
      \( \ch{C6H6_{(l)}}  = +49 \);
      \( \ch{C6H6_{(v)}}  = +83 \);
      \( \ch{H2O_{(l)}}   = -286 \);
      \( \ch{CO2_{(g)}}   = -393 \).
      Densidad \( \textrm{benceno}_{(l)} = \SI{0.879}{\gram\per\cubic\centi\meter} \)
    \end{gexdatos}
  \end{exercise}

  \begin{solution}
    \begin{enumerate*}
      \item \( \Delta H^0_C = \SI{3.27e3}{kJ/mol} \); \item \SI{413}{\liter}; \item \SI{3.83e3}{kJ}.
    \end{enumerate*}
  \end{solution}





  \subsubsection*{Ley de Hess}

  \begin{exercise}[
      tags    = {},
      topics  = {química, termodinámica, termoquímica},
      source  = {FQ 1B MGH 2016, p180, e11},
    ]
    (180.11) El motor de una máquina cortacésped funciona con una gasolina que podemos considerar de composición única octano (\ch{C8H18}). Calcula:

    \begin{enumerate}
      \item La entalpía estándar de combustión del octano, aplicando la ley de Hess.
      \item El calor que se desprende en la combustión de \SI{2.00}{\kilo\gram} de octano.
    \end{enumerate}

    \begin{gexdatos}
      \( \Delta H^0_f\,(\si{kJ/mol}) \):
      \( \ch{CO2_{(g)}}   = -393,8 \);
      \( \ch{C8H18_{(l)}} = -264,0 \);
      \( \ch{H2O_{(l)}}   = -285,8 \).
    \end{gexdatos}
  \end{exercise}

  \begin{solution}
    \begin{enumerate*}
      \item \( \Delta H^0_c = \SI{-5.46e3}{kJ/mol} \); \item \SI{9.58e4}{kJ}
    \end{enumerate*}
  \end{solution}




  \begin{exercise}[
      tags    = {},
      topics  = {química, termodinámica, termoquímica},
      source  = {FQ 1B MGH 2016, p180, e12},
    ]
    (180.12) Sabiendo que las entalpías estándar de combustión del
    hexano (l), del carbono (s) y del hidrógeno (g) son respectivamente:
    \SIlist{-4192;-393,5;-285,8}{kJ/mol}, halla:

    \begin{enumerate}
      \item La entalpía de formación del hexano líquido en esas condiciones.
      \item Los gramos de carbono consumidos en la formación del hexano cuando se han intercambiado \SI{50.0}{kJ}.
    \end{enumerate}
  \end{exercise}

  \begin{solution}
    \begin{enumerate*}
      \item \( \Delta H^0_f\,(\textit{hexano}) = \SI{-170}{kJ/mol} \);
      \item \( m = \SI{21.2}{\gram} \) de C.
    \end{enumerate*}
  \end{solution}




  \begin{exercise}[
      tags    = {},
      topics  = {química, termodinámica, termoquímica},
      source  = {FQ 1B MGH 2016, p180, e14},
    ]
    (180.14) El calor desprendido en el proceso de obtención del benceno a partir de etino es:

    \ch{3 C2H2_{(g)} -> C6H6_{(l)}} \( \Delta H^0_R = \SI{-631}{kJ} \)

    \begin{enumerate}
      \item Calcula la entalpía estándar de combustión del benceno, sabiendo que la del etino es \SI{-1302}{kJ/mol}.
      \item ¿Qué volumen de etino, medido a \SI{26}{\celsius} y \SI{15}{atm}, se necesita para obtener \SI{0.25}{\liter} de benceno?
    \end{enumerate}

    \begin{gexdatos}
      \( \textrm{densidad del benceno} = \SI{880}{\gram\per\liter} \)
    \end{gexdatos}
  \end{exercise}

  \begin{solution}
    \begin{enumerate}
      \item \( \Delta H^0_c (\ch{C6H6_{(l)}}) = \SI{-3.28e3}{kJ/mol} \); \item \SI{13.8}{\liter} de \ch{C2H2}
    \end{enumerate}
  \end{solution}






  \subsubsection*{Entalpías de enlace}

  \begin{exercise}[
      tags    = {},
      topics  = {química, termodinámica, termoquímica},
      source  = {FQ 1B MGH 2016, p180, e18},
    ]
    (180.18) Calcula la variación de entalpía estándar de la hidrogenación del etino a etano:

    \begin{enumerate}
      \item A partir de las energías de enlace.
      \item A partir de las entalpías de formación.
    \end{enumerate}

    \begin{gexdatos}
      Energías de enlace (\si{kJ/mol}): \( \ch{C-H} = 415 \); \( \ch{H-H} = 436 \); \( \ch{C-C} = 350 \): \( \ch{C+C} = 825 \).

      \( \Delta H^0_f\,(\si{kJ/mol}) \): \( \textrm{etino} = 227 \); \( \textrm{etano} = -85,0 \)
    \end{gexdatos}
  \end{exercise}

  \begin{solution}
    \begin{enumerate*}
      \item \( \Delta H_R = \SI{-313}{kJ/mol} \);
      \item \( \Delta H_R = \SI{-312}{kJ/mol} \)
    \end{enumerate*}
  \end{solution}






  \subsubsection*{Entropía y espontaneidad}

  \begin{exercise}[
      tags    = {},
      topics  = {química, termodinámica, termoquímica},
      source  = {FQ 1B MGH 2016, p181, e20},
    ]
    (181.20) Dadas las siguientes ecuaciones termoquímicas:
    \[ \ch{2 H2O2_{(l)} -> 2 H2O_{(l)} + O2_{(g)}}\quad\Delta H = \SI{-196}{kJ} \]
    \[ \ch{N2_{(g)} + 3 H2_{(g)} -> 2 NH3 _{(g)}}\quad\Delta H = \SI{-92.4}{kJ} \]

    \begin{enumerate}
      \item Define el concepto de entropía y explica el signo más probable de \( \Delta S \) en cada una de ellas.
      \item Explica si esos procesos serán o no espontáneos a cualquier
      temperatura, a temperaturas altas, a temperaturas bajas, o no serán nunca espontáneos.
    \end{enumerate}
  \end{exercise}

  \begin{solution}
    \begin{enumerate}
      \item Por \textit{entropía} se entiende la magnitud física que nos mide el grado de desorden de un sistema; es decir, a mayor desorden de las partículas del sistema, mayor entropía. Por lo tanto:

      En la primera reacción, previsiblemente \( \Delta S^0_R > 0 \) ya que se forma una sustancia gaseosa como producto de la reacción y no había gases entre los reactivos.

      En la segunda reacción, \( \Delta S^0_R < 0 \) ya que se forman dos moles de una sustancia gaseosa y había 4 moles gaseosos en los reactivos.

      \item La espontaneidad de una reacción viene dado por la ecuación de Gibbs: \( \Delta G^0_R = \Delta H^0_R - T\Delta S^0_R \) que establece la necesidad de que \( \Delta G^0_R < 0 \) para que un proceso sea espontáneo. Según eso:

      \begin{itemize}
        \item La primera reacción será siempre espontánea, ya que \( \Delta H^0_R < 0 \) y \( \Delta S^0_R > 0 \) por lo que \( \Delta G^0_R \) será negativo a cualquier temperatura.
        \item En la segunda reacción se cumple que \( \Delta H^0_R < 0 \) y \( \Delta S^0_R < 0 \). Para que sea espontánea, el factor entálpico debe ser mayor (en valor absoluto) que el factor entrópico; eso es más fácil de conseguir si la temperatura de la reacción es baja.
      \end{itemize}
    \end{enumerate}
  \end{solution}




  \begin{exercise}[
      tags    = {},
      topics  = {química, termodinámica, termoquímica},
      source  = {FQ 1B MGH 2016, p181, e21},
    ]
    (181.21) Dada la reacción: \ch{N2O_{(g)} -> N2_{(g)} + 1/2 O2_{(g)}}
    siendo \( \Delta H^0 = \SI{43.0}{kJ/mol} \) y \( \Delta S^0 = \SI{80.0}{\joule\per\mole\per\kelvin} \)
    \begin{enumerate}
      \item Justifica el signo positivo de la variación de entropía.
      \item ¿Será espontánea a \SI{25}{\celsius}? ¿A qué temperatura estará en
      equilibrio?
    \end{enumerate}
  \end{exercise}

  \begin{solution}
    \begin{enumerate}
      \item Aumentan los moles de sustancias gaseosas; \item \( \Delta G^0_R = \SI{19.5}{kJ} \) (no es espontánea); \( T = \SI{538}{\kelvin} \)
    \end{enumerate}
  \end{solution}




  \begin{exercise}[
      tags    = {},
      topics  = {química, termodinámica, termoquímica},
      source  = {FQ 1B MGH 2016, p181, e23},
    ]
    (181.23) Se pretende obtener etileno (eteno) a partir de grafito e
    hidrógeno, a \SI{25}{\celsius} y \SI{1}{atm}, según la reacción:
    \[  \ch{2 C_{(s)} + 2 H2_{(g)} -> C2H4_{(g)}} \]

    Calcula:
    \begin{enumerate}
      \item La entalpía de reacción en condiciones estándar. ¿La reacción es endotérmica o exotérmica?
      \item La variación de energía libre de Gibbs en condiciones
      estándar. ¿Es espontánea la reacción en esas condiciones?
    \end{enumerate}

    \begin{gexdatos}
      \( S^0\,[\si{\joule\per\mole\per\kelvin}] \):
      \( \ch{C_{(s)}} = 5,70 \);
      \( \ch{H2_{(g)}} = 130,6 \);
      \( \ch{C2H4_{(g)}}= 219,2 \)

      \( \Delta H^0_f\,(\si{kJ/mol}) \):
      \( \ch{C2H4_{(g)}} = +52,5 \)
    \end{gexdatos}
  \end{exercise}

  \begin{solution}
    \begin{enumerate}
      \item \( \Delta H^0_R = \SI{+52.5}{kJ} \) (endotérmica)
      \item \( \Delta G^0_R = \SI{+68.4}{kJ} \) (no es espontánea)
    \end{enumerate}
  \end{solution}






  \subsubsection*{Aplica lo aprendido}

  \begin{exercise}[
      tags    = {},
      topics  = {química, termodinámica, termoquímica},
      source  = {FQ 1B MGH 2016, p182, e35},
    ]
    (182.35) El acetileno o etino (\ch{C2H2}) se hidrogena para producir etano. Calcula a \SI{298}{\kelvin}:

    \begin{enumerate}
      \item La entalpía estándar de reacción.
      \item La energía de Gibbs estándar de reacción.
      \item La entropía estándar de reacción.
      \item La entropía molar del hidrógeno.
    \end{enumerate}

    \begin{gexdatos}
      \begin{tabular}{rccc}
        Comp. & \( \Delta H^0_f \) & \( \Delta G^0_f \) & \( S^0 \) \\
           & (\si{\kilo\joule\per\mole}) & (\si{\kilo\joule\per\mole}) & (\si{\joule\per\mole\kelvin}) \\
        \toprule
        \ch{C2H2} & 227 & 209 & 200 \\
        \ch{C2H6} & -85 & -33 & 230 \\
        \bottomrule
      \end{tabular}
    \end{gexdatos}
  \end{exercise}

  \begin{solution}
    \begin{enumerate}
      \item \( \Delta H^0_R = \SI{-312}{kJ} \);
      \item \( \Delta G^0_R = \SI{-242}{kJ} \);
      \item \( \Delta S^0_R = \SI{-235}{J/K} \);
      \item \( S^0_{\ch{H2}}  = \SI{132}{\joule\per\mole\per\kelvin} \)
    \end{enumerate}
  \end{solution}
