\section{Bloque 5. Química del carbono (tema 5 del libro)}

\subsection*{Ejercicios del tema}

  \begin{exercise}[
      tags    = {},
      topics  = {química, química orgánica, orgánica},
      source  = {FQ 1B MGH 2016, p126, e7},
    ]
    (126.7) Formula los siguientes alcanos:
    \begin{enumerate}
      \item n-pentano
      \item 2,3,5-trimetilheptano
      \item 4-etil-2,6-dimetiloctano
      \item 4,6-dietil-2,4,8-trimetilnonano
      \item 4-etil-2,2,5,8-tetrametil-6-propildecano
      \item 3,7-dietil-5-isopropildecano
    \end{enumerate}
  \end{exercise}

  \begin{solution}
    \begin{enumerate}
      \item \ch{CH3-CH2-CH2-CH2-CH3}
      \item \chemfig{CH_3-CH(-[6]CH_3)-CH(-[6]CH_3)-CH_2-CH(-[6]CH_3)-CH_2-CH_3}
      \item \chemfig{CH_3-CH(-[6]CH_3)-CH_2-CH(-[6]CH_2-[6]CH_3)-CH_2-CH(-[6]CH_3)-CH_2-CH_3}
      \item \chemfig{CH_3-C(-[2]CH_3)(-[6]CH_2-[6]CH_3)-CH_2-CH(-[6]CH_3)-CH_2-CH(-[6]CH_2-[6]CH_3)-CH_2-CH(-[6]CH_3)-CH_3}
      \item \chemfig{CH_3-C(-[2]CH_3)(-[6]CH_3)-CH_2-CH(-[6]CH_2-[6]CH_3)-CH(-[6]CH_3)-CH(-[6]CH_2-[6]CH_3)-CH_2-CH(-[6]CH_3)-CH_3}
      \item \chemfig{CH_3-CH_2-CH(-[6]CH_2-[6]CH_3)-CH_2-CH(-[6]C(-[4]CH_3)-[0]CH_3)-CH_2-CH(-[6]CH_2-[6]CH_3)-CH_2-CH_2-CH_3}
    \end{enumerate}
  \end{solution}




  \begin{exercise}[
      tags    = {},
      topics  = {química, química orgánica, orgánica},
      source  = {FQ 1B MGH 2016, p129, e9},
    ]
    (129.9) Formula los siguientes hidrocarburos insaturados:
    \begin{enumerate}
      \item But-1-eno
      \item Pent-2-eno
      \item Hexa-2,4-dieno
      \item 3-butilhexa-1,4-dieno
      \item But-2-ino
      \item 3,4-dimetilpent-1-ino
      \item 3,6-dimetilnona-1,4,7-triino
      \item Pent-1-en-3-ino
      \item Hept-3-en-1,6-diino
      \item 4-etilhexa-1,3-dien-5-ino
    \end{enumerate}
  \end{exercise}

  \begin{solution}
    \begin{enumerate}
      \item \ch{CH2=CH-CH2-CH3}
      \item \ch{CH3-CH=CH-CH2-CH3}
      \item \ch{CH3-CH=CH-CH=CH-CH3}
      \item \ch{CH2=CH-CH(C4H9)-CH=CH-CH3}
      \item \ch{CH3-C+C-CH3}
      \item \ch{CH+C-CH(CH3)-CH(CH3)-CH3}
      \item \ch{CH+C-CH(CH3)-C+C-CH(CH3)-C+C-CH3}
      \item \ch{CH2=CH-C+C-CH3}
      \item \ch{CH+C-CH=CH-CH2-C+CH}
      \item \ch{CH2=CH-CH=C(C2H5)-C+CH}
    \end{enumerate}
  \end{solution}




  \begin{exercise}[
      tags    = {},
      topics  = {química, química orgánica, orgánica},
      source  = {FQ 1B MGH 2016, p131, e11},
    ]
    (131.11) Formula los siguientes hidrocarburos cíclicos:
    \begin{enumerate}
      \item Etilciclohexano
      \item Ciclopenteno
      \item Ciclohexino
      \item 1,1,4,4-tetrametilciclohexano
      \item 3-etilciclopenteno
      \item 2,3-dimetilciclohexeno
      \item 4-ciclobutilpent-1-ino
      \item 3-ciclohexil-5-metilhex-2-eno
      \item Ciclohexa-1,3-dieno
      \item 3-ciclopentilprop-1-eno
    \end{enumerate}
  \end{exercise}

  \begin{solution}
    \begin{enumerate}
      \item \chemfig{[:-90]*6(----(-CH_2-CH_3)--)}
      \item \chemfig{[:-126]*5(--=--)}
      \item \chemfig{[:-120]*6(-~----)}
      \item \chemfig{[:-90]*6(-(-[3]CH_3)(-[5]CH_3)---(-[1]CH_3)(-[7]CH_3)--)}
      \item \chemfig{[:-126]*5(--=-(-CH_2-[0]CH_3)-)}
      \item \chemfig{[:-120]*6(---(-CH_3)-(-CH_3)=-)}
      \item \chemfig{CH~C-CH_2-CH(-[6]*4(----))-CH_3}
      \item \chemfig{CH_3-CH=C(-[6]*6(------))-CH_2-CH(-[6]CH_3)-CH_3}
      \item \chemfig{[:-120]*6(-=-=--)}
      \item \chemfig{CH_2=CH-CH_2-*5(-----)}
    \end{enumerate}
  \end{solution}




  \begin{exercise}[
      tags    = {},
      topics  = {química, química orgánica, orgánica},
      source  = {FQ 1B MGH 2016, p131, e12},
    ]
    (131.12) Nombra los siguientes hidrocarburos cíclicos:
    \begin{enumerate}
      \item \chemfig{[:-90]*4(----)}
      \item \chemfig{[:-120]*6(----=-)}
      \item \chemfig{[:-120]*6(---(-CH_3)--(-CH_2-[0]CH_3)-)}
      \item \chemfig{[:-120]*6(-=---=)}
      \item \chemfig{CH_3-CH(-[6]*6(------))-CH_3}
      \item \chemfig{CH_3-CH(-[6]*6(------))-CH_2-C(-[2]CH_3)(-[6]CH_3)-CH_2-CH_3}
      \item \chemfig{CH_3-CH(-[6]*5(-----))-CH=CH_2}
    \end{enumerate}
  \end{exercise}

  \begin{solution}
    \begin{enumerate}
      \item Ciclobutano
      \item Ciclohexeno
      \item 1-etil-3-metilciclohexano
      \item Ciclohexa-1,3-dieno
      \item Isopropil-ciclohexano
      \item 2-ciclohexil-4,4-dimetilhexano
      \item 3-ciclopentilbut-1-eno
    \end{enumerate}
  \end{solution}




  \begin{exercise}[
      tags    = {},
      topics  = {química, química orgánica, orgánica},
      source  = {FQ 1B MGH 2016, p132, e13},
    ]
    (132.13) Formula los siguientes hidrocarburos aromáticos:
    \begin{enumerate}
      \item Metilbenceno (tolueno)
      \item Etenilbenceno
      \item 1,3-dietilbenceno
      \item 1-butil-4-isopropilbenceno
      \item \textit{para}-propiltolueno
      \item 3-fenil-5-metilheptano
      \item 4-fenilpent-1-eno
      \item 2,4-difenil-3-metilhexano
    \end{enumerate}
  \end{exercise}

  \begin{solution}
    \begin{enumerate}
      \item \chemfig{[:-120]**6(-----(-CH_3)-)}
      \item \chemfig{[:-120]**6(-----(-CH=[0]CH_2)-)}
      \item \chemfig{CH_3-CH_2-**6(--(-CH_2-[0]CH_3)----)}
      \item \chemfig{CH_3-CH_2-CH_2-CH_2-**6(---(-CH_2(-[2]CH_3)-[6]CH_3)---)}
      \item \chemfig{CH_3-**6(---(-CH_2-CH_2-CH_3)---)}
      \item \chemfig{CH_3-CH_2-CH(-[6]**6(------))-CH_2-CH(-[6]CH_3)-CH_2-CH_3}
      \item \chemfig{CH_2=CH-CH_2-CH(-[6]**6(------))-CH_3}
      \item \chemfig{CH_3-CH(-[6]**6(------))-CH(-[2]CH_3)-CH(-[6]**6(------))-CH_2-CH_3}
    \end{enumerate}
  \end{solution}




  \begin{exercise}[
      tags    = {},
      topics  = {química, química orgánica, orgánica},
      source  = {FQ 1B MGH 2016, p133, e15},
    ]
    (133.15) Formula los siguientes derivados halogenados:
    \begin{enumerate}
      \item 2-cloropropano
      \item 1,3-dibromobenceno
      \item 1,1,2,2-tetrafluoretano
      \item 1,4-diclorociclohexano
      \item 4-bromopent-1-ino
      \item 3-flúor-5-metilhex-2-eno
      \item 1,4-dibromo-6-ciclopentiloct-2-eno
      \item 4-yodo-3,5-difenilpent-1-ino
      \item 4-clorobut-1-eno
      \item 1,2-dibromobenceno
    \end{enumerate}
  \end{exercise}

  \begin{solution}
    \begin{enumerate}
      \item \ch{CH3-CHCl-CH3}
      \item \chemfig{[:-90]**6(-(-Br)--(-Br)---)}
      \item \ch{F2HC-CHF2}
      \item \chemfig{Cl-*6(---(-Cl)---)}
      \item \ch{CH+C-CH2-CHBr-CH3}
      \item \ch{CH3-CH=CHF-CH2-CH(CH3)-CH3}
      \item \chemfig{CH_2Br-CH=CH-CHBr-CH_2-CH(-[6]*5(-----))-CH_2-CH_3}
      \item \chemfig{CH~C-CH(-[6]**6(------))-CH(-[6]I)-CH_2-[6]**6(------)}
      \item \ch{CH2=CH-CH2-CH2Cl}
      \item \chemfig{Br-[6]**6(-----(-Br)-)}
    \end{enumerate}
  \end{solution}




  \begin{exercise}[
      tags    = {},
      topics  = {química, química orgánica, orgánica},
      source  = {FQ 1B MGH 2016, p135, e17},
    ]
    (135.17) Formula los siguientes alcoholes y éteres:
    \begin{enumerate}
      \item 3-metilpentan-1-ol
      \item Butano-1,2,3-triol
      \item 2-fenilpropano-1,3-diol
      \item Ciclohexanol
      \item Hexa-3,5-dien-2-ol
      \item Fenol (Hidroxibenceno)
      \item 2-etilpentan-1-ol
      \item Pent-3-en-1-ol
      \item Etilisopropiléter
      \item Etenilfeniléter
      \item Dimetiléter
      \item Butilciclopentiléter
    \end{enumerate}
  \end{exercise}

  \begin{solution}[print=false]
    Todavía sin solución
  \end{solution}




  \begin{exercise}[
      tags    = {},
      topics  = {química, química orgánica, orgánica},
      source  = {FQ 1B MGH 2016, p135, e18},
    ]
    (135.18) Nombra los siguientes alcoholes y éteres:
    \begin{enumerate}
      \item \ch{CH3OH}
      \item \chemfig{CH_2OH-CH_2-CH(-[6]CH_2-[::0]CH_3)-CH=CH_2}
      \item \ch{CH3-CHOH-CHBr-CH2OH}
      \item \chemfig{CH_3-CH_2-CH(-[6]CH_3)-CHOH-CH_3}
      \item \ch{CH2=CH-CHOH-CH2OH}
      \item \ch{CH3-CH2-CH2-O-CH2-CH2-CH3}
      \item \chemfig{CH_2=CH-O-CH(-[6]CH_3)-CH_3}
      \item \ch{CH3-(CH2)3-CH2-O-C+CH}
    \end{enumerate}
  \end{exercise}

  \begin{solution}[print=false]
    Todavía sin solución
  \end{solution}




  \begin{exercise}[
      tags    = {},
      topics  = {química, química orgánica, orgánica},
      source  = {FQ 1B MGH 2016, p136, e19},
    ]
    (136.19) Formula los siguientes aldehídos y cetonas:
    \begin{enumerate}
      \item Etanal (acetaldehído)
      \item Benzaldehído
      \item 3-metilpentanal
      \item 2-metilpentanodial
      \item Propenal
      \item Hex-2-endial
      \item 5-ciclohexilpent-3-inal
      \item 3-metilpent-2-enal
      \item Hex-2-endial
      \item Pentan-2-ona
      \item Hexa-2,4-diona
      \item 3-clorobutanona
      \item 1,4-difenilpentan-2-ona
      \item Hexa-1,5-dien-3-ona
    \end{enumerate}
  \end{exercise}

  \begin{solution}[print=false]
    Todavía sin solución
  \end{solution}




  \begin{exercise}[
      tags    = {},
      topics  = {química, química orgánica, orgánica},
      source  = {FQ 1B MGH 2016, p136, e20},
    ]
    (136.20) Nombra los siguientes aldehídos y cetonas:

    \begin{enumerate}
      \item \ch{HCHO}
      \item \ch{CH3-CH2-CH2-CHO}
      \item \ch{OHC-CH=CH-CHO}
      \item \chemfig{CH_2=C(-[6]*5(-----))-CH_2-{(}CH_2{)}_4-CHO}
      \item \ch{OHC-CH=CH-CH2-CH(CH3)-CHO}
      \item \chemfig{CH_3-CH(-[6]C_6H_5)-CH=CH-CHO}
      \item \ch{CHO-CH2-C=C-CH2-CH2-CHO}
      \item \ch{CH3-CO-CH2-CH3}
      \item \ch{CH3-CH=CH-CH2-CO-CH3}
      \item \ch{CH3-CO-CH 2-CH2-CH2-CO-CH3}
      \item \ch{CH3-CH(CH3)-CO-CH2-CH(CH3)-CH3}
      \item \ch{CH2=CH-CO-CH=CH-CH3}
    \end{enumerate}
  \end{exercise}

  \begin{solution}[print=false]
    Todavía sin solución
  \end{solution}




  \begin{exercise}[
      tags    = {},
      topics  = {química, química orgánica, orgánica},
      source  = {FQ 1B MGH 2016, p138, e21},
    ]
    (138.21) Formula los siguientes ácidos y ésteres:
    \begin{enumerate}
      \item Ácido etanoico (ácido acético)
      \item Ácido 3-metilhexanoico
      \item Ácido 2-fenilpentanodioico
      \item Ácido tricloroetanoico
      \item Ácido but-3-enoico
      \item Ácido hepta-2,4-dienoico
      \item Ácido pent-2-enodioico
      \item Ácido benzoico
      \item Butanoato de metilo
      \item Propanoato de etilo
      \item Benzoato de propilo
      \item Etanoato de octilo
      \item 3-cloropentanoato de etenilo
      \item But-3-enoato de isopropilo
    \end{enumerate}
  \end{exercise}

  \begin{solution}[print=false]
    Todavía sin solución
  \end{solution}




  \begin{exercise}[
      tags    = {},
      topics  = {química, química orgánica, orgánica},
      source  = {FQ 1B MGH 2016, p140, e23},
    ]
    (140.23) Formula los siguientes compuest os con funciones nitrogenadas:
    \begin{enumerate}
      \item Isopropilamina
      \item Pentan-3-amina
      \item Buta-1,3-diamina
      \item 3-etilhexan-3-amina
      \item 3,5-dimetilhexan-1-amina
      \item Pent-3-en-2-amina
      \item N-metilfenilamina
      \item N-ciclopentilbutilamina
      \item Etanamida
      \item N-metiletanamida
      \item 4-fenilpentanamida
      \item N-etilhex-4-enamida
    \end{enumerate}
  \end{exercise}

  \begin{solution}[print=false]
    Todavía sin solución
  \end{solution}




  \begin{exercise}[
      tags    = {},
      topics  = {química, química orgánica, orgánica},
      source  = {FQ 1B MGH 2016, p140, e24},
    ]
    (140.24) Nombra los siguientes compuestos nitrogenados:
    \begin{enumerate}
      \item \chemfig{CH_3-CH(-[6]NH_2)-CH_2-CH_3}
      \item \ch{CH3-CH2-CH2-NH2}
      \item \chemfig{CH_3-CH(-[6]NH_2)-CH_2-CH(-[6]NH_2)-CH_2-CH_2(-[6]NH_2)}
      \item \chemfig{CH_3-CH(-[6]CH_3)-NH-CH=CH_2}
      \item \chemfig{CH_3-NH-*4(----)}
      \item \ch{CH3-CH2-CH2-CH2-CH2-CO-NH2}
      \item \ch{CH3-CH=CH-CH2-CO-NH2}
      \item \ch{CH3-CH2-CHBr-CH2-CH2-CO-NH-CH3}
    \end{enumerate}
  \end{exercise}

  \begin{solution}[print=false]
    Todavía sin solución
  \end{solution}




  \begin{exercise}[
      tags    = {},
      topics  = {química, química orgánica, orgánica},
      source  = {FQ 1B MGH 2016, p140, e25},
    ]
    (140.25) Formula los siguientes compuestos orgánicos:
    \begin{enumerate}
      \item 2,2-dimetilpentano
      \item Hepta-1,5-dieno
      \item 1-fenilpent-2-ino
      \item 3-isopropilciclohexeno
      \item 1-butil-3-metilbenceno
      \item Butano-1,3-diol
      \item Butileteniléter
      \item But-3 enal
      \item Hex-5-in-2-ona
      \item Ácido 3-isopropilhexanoico
      \item Pentanoato de metilo
      \item 5-meilhexan-2,4-diamina
      \item N-metiletilamina
      \item N,N-dietilbutilamina
      \item Hex-3-enamida
      \item N-metilbutanamida
    \end{enumerate}
  \end{exercise}

  \begin{solution}[print=false]
    Todavía sin solución
  \end{solution}




  \begin{exercise}[
      tags    = {},
      topics  = {química, química orgánica, orgánica},
      source  = {FQ 1B MGH 2016, p143, e27},
    ]
    (143.27) Formula y nombra:
    \begin{enumerate}
      \item Dos hidrocarburos alifáticos que presenten isomería de cadena.
      \item Dos aminas con isomería de posición.
      \item Dos compuestos oxigenados con isomería de función.
    \end{enumerate}
  \end{exercise}

  \begin{solution}[print=false]
    Todavía sin solución
  \end{solution}




  \begin{exercise}[
      tags    = {},
      topics  = {química, química orgánica, orgánica},
      source  = {FQ 1B MGH 2016, p143, e28},
    ]
    (143.28) Escribe y nombra:
    \begin{enumerate}
      \item Todos los isómeros de cadena de fórmula \ch{C5H12}.
      \item Cuatro isómeros de función de fórmula \ch{C4H80}.
      \item Tres isómeros de posición de la amina \ch{C5H13N}.
    \end{enumerate}
  \end{exercise}

  \begin{solution}[print=false]
    Todavía sin solución
  \end{solution}




  \begin{exercise}[
      tags    = {},
      topics  = {química, química orgánica, orgánica},
      source  = {FQ 1B MGH 2016, p143, e29},
    ]
    (143.29) Dados los siguientes compuestos, formúlalos y justifica cuáles
    de ellos presentan isomería geométrica y cuáles isomería
    óptica:
    \begin{enumerate}
      \item 2-clorobutano
      \item Pent-3-en-2-ol
      \item Pentan-3-amina
      \item 2-fenilpent-2-eno
    \end{enumerate}
  \end{exercise}

  \begin{solution}[print=false]
    Todavía sin solución
  \end{solution}






  \subsection*{Problemas propuestos}

  \subsubsection*{Grupos funcionales y series homólogas}

  \begin{exercise}[
      tags    = {},
      topics  = {química, química orgánica, orgánica},
      source  = {FQ 1B MGH 2016, p150, e7},
    ]
    (150.7) Escribe el número de carbonos y el grupo funcional al que
    corresponden los siguientes compuestos:
    \begin{enumerate}
      \item Octano
      \item Butanamina
      \item Pentinamida
      \item Ácido decanoico
      \item Hexenal
      \item Propanona
      \item Butino
      \item Hepteno
      \item Metanol
      \item Dietiléter
    \end{enumerate}
  \end{exercise}

  \begin{solution}[print=false]
    Todavía sin solución
  \end{solution}




  \begin{exercise}[
      tags    = {},
      topics  = {química, química orgánica, orgánica},
      source  = {FQ 1B MGH 2016, p150, e8},
    ]
    (150.8) Indica si la estructura de cada pareja representa el mismo
    compuesto o compuestos diferentes, identificando los grupos
    funcionales presentes:
    \begin{enumerate}
      \item \ch{CH3CH2OCH3} y \ch{CH3OCH2CH3}
      \item \ch{CH3CH2OCH3} y \ch{CH3CHOHCH3}
      \item \ch{CH3CH2CH2OH} y \ch{CH3CHOHCH3}
    \end{enumerate}
  \end{exercise}

  \begin{solution}[print=false]
    Todavía sin solución
  \end{solution}




  \begin{exercise}[
      tags    = {},
      topics  = {química, química orgánica, orgánica},
      source  = {FQ 1B MGH 2016, p150, e9},
    ]
    (150.9) Contesta a cada uno de los siguientes apartados referidos a
    compuestos de cadena abierta:
    \begin{enumerate}
      \item ¿Qué grupos funcionales pueden tener los compuestos de
      fórmula molecular \ch{C_nH_{2n+2}O}?
      \item ¿Qué compuestos tienen por fórmula molecular \ch{C_nH_{2n-2}}?
    \end{enumerate}
  \end{exercise}

  \begin{solution}[print=false]
    Todavía sin solución
  \end{solution}




  \begin{exercise}[
      tags    = {},
      topics  = {química, química orgánica, orgánica},
      source  = {FQ 1B MGH 2016, p150, e10},
    ]
    (150.10) Nombra y formula los siguientes compuestos orgánicos:
    \begin{enumerate}
      \item \ch{CH3-CH2-COOH}
      \item \ch{CH3-CH2-C+CH}
      \item \ch{CH3-CHOH-CH2-CH2-CH3}
      \item \ch{CH3-CH2-CO-CH2-CH2-CH3}
      \item \ch{C6H14}
      \item Metil etil éter.
      \item Metanoato de propilo.
      \item Dietilamina.
      \item Pentanal.
      \item Metilpropeno.
    \end{enumerate}
  \end{exercise}

  \begin{solution}[print=false]
    Todavía sin solución
  \end{solution}




  \begin{exercise}[
      tags    = {},
      topics  = {química, química orgánica, orgánica},
      source  = {FQ 1B MGH 2016, p150, e13},
    ]
    (150.13) Formula las siguientes especies químicas:
    \begin{enumerate}
      \item 1-bromo-2,2-diclorobutano
      \item Trimetilamina
      \item 2-metilhex-1,5-dien-3-ino
      \item Butanoato de 2-metilpropilo
      \item Tolueno (metilbenceno)
      \item Propanamida
      \item 2,3-dimetilbut-1-eno
      \item Ácido 2,3-dimetilpentanodioico
    \end{enumerate}
  \end{exercise}

  \begin{solution}[print=false]
    Todavía sin solución
  \end{solution}




  \begin{exercise}[
      tags    = {},
      topics  = {química, química orgánica, orgánica},
      source  = {FQ 1B MGH 2016, p151, e14},
    ]
    (151.14) Nombra las siguientes especies químicas:
    \begin{enumerate}
      \item \ch{H2C=CH-CH=CH-CHO}
      \item \ch{H3C-CO-CO-CH3}
      \item \ch{H2C=CH-CH=CH-CH2-COOH}
      \item \ch{H3C-CH2-NH-CH2-CH3}
      \item \ch{CH+C-CH2-COOH}
      \item \ch{CH3-CH2-CH(CH3)-CONH2}
      \item \ch{H3C-C(OH)2-CH2-CH2OH}
    \end{enumerate}
  \end{exercise}

  \begin{solution}[print=false]
    Todavía sin solución
  \end{solution}




  \begin{exercise}[
      tags    = {},
      topics  = {química, química orgánica, orgánica},
      source  = {FQ 1B MGH 2016, p151, e15},
    ]
    (151.15) Nombra y/o formula los siguientes compuestos:
    \begin{enumerate}
      \item \ch{CHCl3}
      \item \ch{CH3-CH2-CHO}
      \item \ch{CH3-CH2-CH2-CH2-CO-NH2}
      \item \ch{(CH3)2-CHOH}
      \item 2,2-dimetilbutano
      \item Para-diaminobenceno
      \item Ciclohexano
      \item Etil propil éter
    \end{enumerate}
  \end{exercise}

  \begin{solution}[print=false]
    Todavía sin solución
  \end{solution}




  \begin{exercise}[
      tags    = {},
      topics  = {química, química orgánica, orgánica},
      source  = {FQ 1B MGH 2016, p151, e16},
    ]
    (151.16) Formula o nombra, según corresponda:
    \begin{enumerate}
      \item 1-etil-3-metilbenceno
      \item 2-metilpropan-2-ol
      \item 2-metil-propanoato de etilo
      \item Pent-3-en-1-amina
      \item \ch{ClCH=CH-CH3}
      \item \ch{CH3-CH2-O-CH2-CH3}
      \item \ch{CH3-CH(CH3)-CO-CH2-CH(CH3)-CH3}
      \item \ch{CH2=CH-CH2-CO-NH-CH3}
    \end{enumerate}
  \end{exercise}

  \begin{solution}[print=false]
    Todavía sin solución
  \end{solution}




  \begin{exercise}[
      tags    = {},
      topics  = {química, química orgánica, orgánica},
      source  = {FQ 1B MGH 2016, p151, e20},
    ]
    (151.20) Formula o nombra los siguientes compuestos:
    \begin{enumerate}
      \item Cromato de cobre(II)
      \item Hidruro de magnesio
      \item Hidrogenosulfuro de bario
      \item Etanamina
      \item Propan-1,2-diol
      \item \ch{Fe(OH)2}
      \item \ch{H2SO3}
      \item \ch{N2O5}
      \item \chemfig{[:-120]**6(-----(-CH=[2]O)-)}
      \item \chemfig{CH_3-CH(-[6]CH_3)-CH(-[6]CH_3)-CH(-[6]CH_3)-CH_2-CH_3}
    \end{enumerate}
  \end{exercise}

  \begin{solution}[print=false]
    Todavía sin solución
  \end{solution}




  \begin{exercise}[
      tags    = {},
      topics  = {química, química orgánica, orgánica},
      source  = {FQ 1B MGH 2016, p151, e21},
    ]
    (151.21) Formula o nombra los siguientes compuestos orgánicos:
    \begin{enumerate}
      \item 3-etil-2-metilhexano
      \item 1-bromopent-2-ino
      \item 3-etilhe xano-1,5-diol
      \item 3-metilpentan-2,4-diamina
      \item \ch{CH2=CH-CH2-CO-O-CH3}
      \item \ch{C6H5-O-C6H5}
      \item \ch{CH3-CH2-CO-NH-CH2-CH3}
      \item \ch{COOH-CH2-CH2-CHBr-COOH}
    \end{enumerate}
  \end{exercise}

  \begin{solution}[print=false]
    Todavía sin solución
  \end{solution}






  \subsubsection*{Isomería estructural y espacial}

  \begin{exercise}[
      tags    = {},
      topics  = {química, química orgánica, orgánica},
      source  = {FQ 1B MGH 2016, p152, e23},
    ]
    (152.23) Formula los siguientes compuestos orgánicos:
    \begin{enumerate}
      \item But-3-en-2-ona
      \item Buta-1,3-dien-2-ol
      \item Dietiléter
    \end{enumerate}
    ¿Cuáles de ellos son isómeros entre sí?
  \end{exercise}

  \begin{solution}[print=false]
    Todavía sin solución
  \end{solution}




  \begin{exercise}[
      tags    = {},
      topics  = {química, química orgánica, orgánica},
      source  = {FQ 1B MGH 2016, p152, e24},
    ]
    (152.24) Escribe y nombra cinco isómeros de cadena de fórmula molecular \ch{C6H14}.
  \end{exercise}

  \begin{solution}[print=false]
    Todavía sin solución
  \end{solution}




  \begin{exercise}[
      tags    = {},
      topics  = {química, química orgánica, orgánica},
      source  = {FQ 1B MGH 2016, p152, e25},
    ]
    (152.25) Escribe y nombra cuatro isómeros de función de fórmula molecular \ch{C4H8O}.
  \end{exercise}

  \begin{solution}[print=false]
    Todavía sin solución
  \end{solution}




  \begin{exercise}[
      tags    = {},
      topics  = {química, química orgánica, orgánica},
      source  = {FQ 1B MGH 2016, p152, e28},
    ]
    (152.28) Escribe y nombra todos los isómeros estructurales de fórmula \ch{C5H10}
  \end{exercise}

  \begin{solution}[print=false]
    Todavía sin solución
  \end{solution}




  \begin{exercise}[
      tags    = {},
      topics  = {química, química orgánica, orgánica},
      source  = {FQ 1B MGH 2016, p152, e30},
    ]
    (152.30) Formula y nombra:
    \begin{enumerate}
      \item Dos isómeros de posición de fórmula \ch{C3H8O}
      \item Dos isómeros de función de fórmula \ch{C3H8O}
      \item Dos isómeros geométricos de fórmula \ch{C4H8}
      \item Un compuesto que tenga dos carbonos quirales (asimétricos) de fórmula \ch{C4H8BrCl}
    \end{enumerate}
  \end{exercise}

  \begin{solution}[print=false]
    Todavía sin solución
  \end{solution}




  \begin{exercise}[
      tags    = {},
      topics  = {química, química orgánica, orgánica},
      source  = {FQ 1B MGH 2016, p152, e31},
    ]
    (152.31) Un derivado halogenado etilénico que presenta isomería cis-trans está formado en un 22,4\% de C, un 2,8\% de H y un 74,8\% de bromo. Además, a \SI{130}{\celsius} y \SI{1}{atm} de presión, una muestra de \SI{12,9}{\gram} ocupa un volumen de \SI{2}{\liter}. Halla su fórmula molecular y escribe los posibles isómeros.
  \end{exercise}

  \begin{solution}
    \ch{C4H6Br2}
  \end{solution}




  \begin{exercise}[
      tags    = {},
      topics  = {química, química orgánica, orgánica},
      source  = {FQ 1B MGH 2016, p152, e32},
    ]
    (152.32) Un alcohol monoclorado está formado en un 38,1\% de C,
    un 7,4\% de H, un 37,6\% de Cl y el resto es oxígeno. Escribe
    su fórmula semidesarrollada sabiendo que tiene un carbono
    asimétrico y que su fórmula molecular y su fórmula empírica
    coinciden.
  \end{exercise}

  \begin{solution}
    \ch{C3H7OCl}
  \end{solution}




  \begin{exercise}[
      tags    = {},
      topics  = {química, química orgánica, orgánica},
      source  = {FQ 1B MGH 2016, p152, e33},
    ]
    (152.33) Un hidrocarburo monoinsaturado tiene un 87,8\% de carbono.
    Si su densidad en condiciones normales es \SI{3,66}{\gram\per\liter}, determina sus fórmulas empírica y molecular.
  \end{exercise}

  \begin{solution}
    Formula empírica: \ch{C3H5}; Fórmula molecular: \ch{C6H10}.
  \end{solution}
