\section{Bloque 2. Aspectos cualitativos de la Química (tema 3 del libro)}

  \subsection*{Ejercicios del tema}

    \begin{exercise}[
        tags    = {},
        topics  = {química,química básica},
        source  = {FQ 1B MGH 2016, p66, e4},
      ]
      (66.4) Sabemos que \SI{40}{uma} es la masa del átomo de calcio. Calcula:
      \begin{enumerate}
        \item La masa en gramos de \SI{1}{átomo} de Ca.
        \item ¿Cuál de las siguientes cantidades tienen mayor número de átomos? \SI{40}{g} de Ca; \SI{0,20}{moles} de Ca; \SI{5e23}{átomos} de Ca.
      \end{enumerate}
    \end{exercise}

    \begin{solution}
      \begin{enumerate*}
        \item \( m = \SI{6,6E-23}{\gram} \);
        \item \SI{40}{\gram} de Ca.
      \end{enumerate*}
    \end{solution}




    \begin{exercise}[
        tags    = {},
        topics  = {química,química básica},
        source  = {FQ 1B MGH 2016, p66, e5},
      ]
      (66.5) Si tenemos en cuenta que \SI{56}{uma} es la masa del átomo de hierro, calcula:
      \begin{enumerate}
        \item La masa atómica en gramos de \SI{1}{átomo} de~Fe.
        \item Cuál de las siguientes cantidades tiene mayor número de átomos de Fe: \SI{56}{\gram}, \SI{0,20}{moles} o \SI{5e23}{átomos}.
      \end{enumerate}
    \end{exercise}

    \begin{solution}
      \begin{enumerate*}
        \item \( m = \SI{9,3E-23}{\gram} \);
        \item \SI{56}{\gram} de Fe.
      \end{enumerate*}
    \end{solution}




    \begin{exercise}[
        tags    = {},
        topics  = {química,química básica},
        source  = {FQ 1B MGH 2016, p66, e6},
      ]
      (66.6) Responde a las siguientes cuestiones:
      \begin{enumerate}
        \item ¿En cuál de las siguientes cantidades de los elementos que se enumeran a continuación existe un mayor número de moles: \SI{100}{\gram} de hierro, \SI{100}{\gram} de oxígeno molecular, \SI{100}{\gram} de zinc o \SI{100}{\gram} de níquel?
        \item ¿Y un mayor número de átomos?
      \end{enumerate}
    \end{exercise}

    \begin{solution}
      \begin{enumerate*}
        \item En los \SI{100}{\gram} de oxígeno molecular;
        \item En los \SI{100}{\gram} de oxígeno molecular.
      \end{enumerate*}
    \end{solution}




    \begin{exercise}[
        tags    = {},
        topics  = {química,química básica},
        source  = {FQ 1B MGH 2016, p71, e10},
      ]
      (71.10) Sabiendo que un gas a \SI{1,5}{atm} y \SI{290}{\kelvin} tiene una densidad de \SI{1,178}{\gram\per\liter}, calcula su masa molecular.
    \end{exercise}

    \begin{solution}
      \( M = \SI{18,7}{\gram\per\mole} \)
    \end{solution}




    \begin{exercise}[
        tags    = {},
        topics  = {química,química básica},
        source  = {FQ 1B MGH 2016, p71, e11},
      ]
      (71.11) Calcula la densidad del metano (\ch{CH4}) a \SI{700}{\mmHg} y \SI{75}{\celsius}.
    \end{exercise}

    \begin{solution}
      \( d = \SI{0,52}{\gram\per\liter} \)
    \end{solution}



    \begin{exercise}[
        tags    = {},
        topics  = {química,química básica},
        source  = {FQ 1B MGH 2016, p71, e12},
      ]
      (71.12) Calcula el número de moléculas de \ch{CO2} que habrá en \SI{10}{\liter} del mismo gas medidos en condiciones normales.
    \end{exercise}

    \begin{solution}
      \( M = \SI{2,7e23}{moléculas} \).
    \end{solution}




    \begin{exercise}[
        tags    = {},
        topics  = {química,química básica},
        source  = {FQ 1B MGH 2016, p71, e13},
      ]
      (71.13) Calcula la masa en gramos de un mol de \ch{SO2} sabiendo que
      exactamente \SI{5}{\cubic\centi\meter} de dicho gas, medidos en condiciones
      normales, tienen una masa de \SI{0,01428}{\gram}.
    \end{exercise}

    \begin{solution}
      \( M = \SI{64}{\gram\per\mole} \)
    \end{solution}




    \begin{exercise}[
        tags    = {},
        topics  = {química,química básica},
        source  = {FQ 1B MGH 2016, p71, e14},
      ]
      (71.14) La masa de \SI{1,20}{\milli\gram} de una sustancia gaseosa pura equivale
      a \SI{1,2e19}{moléculas}. Calcula la masa en gramos de \SI{1}{\mole} de dicha sustancia.
    \end{exercise}

    \begin{solution}
      \( M = \SI{60,2}{\gram\per\mole} \)
    \end{solution}




    \begin{exercise}[
        tags    = {},
        topics  = {química,química básica},
        source  = {FQ 1B MGH 2016, p72, e15},
      ]
      (72.15) Se introducen, en un recipiente de \SI{5.0}{\liter}, \SI{10}{\gram} de alcohol etílico (\ch{C2H5OH}) y \SI{10}{\gram} de acetona (\ch{C3H6O}) y posteriormente se calienta el reactor a \SI{200}{\celsius}, con lo cual ambos líquidos pasan a la fase gaseosa. Calcula la presión en el interior del reactor, suponiendo comportamiento ideal, y la presión parcial de cada componente.
    \end{exercise}

    \begin{solution}
      \( p_\textrm{alcohol} = \SI{1.7}{atm} \);
      \( p_\textrm{acetona} = \SI{1.3}{atm} \);
      \( P_T = \SI{3.0}{atm} \).
    \end{solution}




    \begin{exercise}[
        tags    = {},
        topics  = {química,química básica},
        source  = {FQ 1B MGH 2016, p74, e16},
      ]
      (74.16) Calcula la composición centesimal de la molécula de propano (\ch{C3H8}).
    \end{exercise}

    \begin{solution}
      \num{81,8}\% de carbono; \num{18,2}\% de hidrógeno.
    \end{solution}




    \begin{exercise}[
        tags    = {},
        topics  = {química,química básica},
        source  = {FQ 1B MGH 2016, p77, e18},
      ]
      (77.18) Calcula la fracción molar de cada uno de los componentes
      de una disolución que se ha preparado mezclando \SI{90}{\gram} de alcohol etílico (\ch{C2H50H}) y \SI{110}{\gram} de agua.
    \end{exercise}

    \begin{solution}
      \( X_\textrm{alcohol} = 0,24 \),
      \( X_\textrm{agua} = 0,76 \).
    \end{solution}




    \begin{exercise}[
        tags    = {},
        topics  = {química,química básica},
        source  = {FQ 1B MGH 2016, p77, e20},
      ]
      (77.20) Una disolución de hidróxido de sodio en agua que contiene un 25\% de hidróxido tiene una densidad de \SI{1.25}{\gram\per\milli\liter}. Calcula
      su molaridad y su normalidad.
    \end{exercise}

    \begin{solution}
      \( M = \SI{7.8}{M} \)  y  \( N = \SI{7.8}{N} \).
    \end{solution}




    \begin{exercise}[
        tags    = {},
        topics  = {química,química básica},
        source  = {FQ 1B MGH 2016, p77, e22},
      ]
      (77.22) ¿Cuál es la molaridad de una disolución de ácido sulfúrico
      del 26\% de riqueza y densidad \SI{1.19}{\gram\per\milli\liter}?
    \end{exercise}

    \begin{solution}
      \( M = \SI{3.2}{M} \).
    \end{solution}




    \begin{exercise}[
        tags    = {},
        topics  = {química,química básica},
        source  = {FQ 1B MGH 2016, p80, e23},
      ]
      (80.23) El alcanfor puro tiene un punto de fusión de \SI{178}{\celsius} y una constante crioscópica de \SI{40}{\celsius\kilo\gram\per\mole}. La disolución resultante de añadir \SI{2}{\gram} de un soluto no volátil a
      \SI{10}{\gram} de alcanfor congela a \SI{158}{\celsius}. Calcula la masa molecular del soluto añadido.
    \end{exercise}

    \begin{solution}
      \( M = \SI{400}{\gram\per\mole} \).
    \end{solution}




    \begin{exercise}[
        tags    = {},
        topics  = {química,química básica},
        source  = {FQ 1B MGH 2016, p80, e24},
      ]
      (80.24) Tenemos \SI{100}{\milli\liter} de una disolución acuosa que contiene \SI{0.25}{\gram} de un polisacárido. Dicha disolución a \SI{25}{\celsius}, ejerce una presión osmótica de \SI{23.9}{\mmHg}. El polisacárido tiene la siguiente fórmula empírica \ch{(C6H10O5)_{n}}. Calcula el valor de la masa molecular del polisacárido.
    \end{exercise}

    \begin{solution}
      \( M = \SI{1938}{\gram\per\mole} \).
    \end{solution}







  \subsection*{Problemas propuestos}

    \subsubsection*{Leyes de los volúmenes de combinación. Hipótesis de Avogadro. Concepto de molécula. Mol}

      \begin{exercise}[
          tags    = {},
          topics  = {química,química básica},
          source  = {FQ 1B MGH 2016, p83, e8},
        ]
        (83.8) Determina la masa, \( M \), de un mol de un gas en los siguientes
        casos:
        \begin{enumerate}
          \item Su densidad en CN es de \SI{3.17}{\gram\per\liter}.
          \item Su densidad es de \SI{2.4}{g/L} a \SI{20}{\celsius} y \SI{1}{atm} de presión.
          \item Dos gramos de dicho gas ocupan un volumen de \SI{600}{\milli\liter}, medido a \SI{17}{\celsius} y \SI{1.8}{atm} de presión.
        \end{enumerate}
      \end{exercise}

      \begin{solution}
        \begin{enumerate*}
          \item \( M = \SI{71}{g/mol} \);
          \item \( M = \SI{58}{g/mol} \);
          \item \( M = \SI{44}{g/mol} \).
          \end{enumerate*}
      \end{solution}




      \begin{exercise}[
          tags    = {},
          topics  = {química,química básica},
          source  = {FQ 1B MGH 2016, p83, e9},
        ]
        (83.9) Realiza los siguientes cálculos numéricos:
        \begin{enumerate}
          \item Los átomos de oxígeno que hay en \SI{0.25}{moles} de sulfato de potasio (\ch{K2SO4}).
          \item Las moléculas de gasolina (\ch{C8H18}) que hay en un depósito de \SI{40}{\liter} (\( d = \SI{0.76}{g/mL} \)).
          \item Los gramos de calcio que hay en \SI{60}{g} de un carbonato de calcio (\ch{CaCO3}) del 80\% de riqueza.
          \item De una sustancia pura, sabemos que \SI{1.75e19}{moléculas} corresponden a una masa de \SI{2.73}{mg}. ¿Cuál será la masa de \SI{1}{mol}?
        \end{enumerate}
      \end{exercise}

      \begin{solution}
        \begin{enumerate*}
          \item \SI{6e23}{átomos}; \item \SI{1.6e26}{moléculas}; \item \SI{19}{g}; \item \( M = \SI{93}{g/mol} \).
        \end{enumerate*}
      \end{solution}




      \begin{exercise}[
          tags    = {},
          topics  = {química,química básica},
          source  = {FQ 1B MGH 2016, p83, e10},
        ]
        (83.10) Disponemos de \SI{3}{moles} de sulfuro de hidrógeno. Calcula,
        sabiendo que las masas atómicas son \( \ch{S} = 32 \) y \( \ch{H} = 1 \):
        \begin{enumerate}
          \item Cuántos gramos de \ch{H2S} hay en esos \SI{3}{moles}.
          \item El número de moléculas de \ch{H2S} que forman los \SI{3}{moles}.
          \item Los moles de \ch{H2} y de \ch{S} que tenemos en los \SI{3}{moles} de \ch{H2S}.
        \end{enumerate}
      \end{exercise}

      \begin{solution}
        \begin{enumerate*}
          \item \( m_{\ch{H2S}} = \SI{102}{\gram} \);
          \item \SI{1.8e24}{moléculas};
          \item \SI{3}{moles} de \ch{H2} y \SI{3}{moles} de \ch{S}.
        \end{enumerate*}
      \end{solution}




      \begin{exercise}[
          tags    = {},
          topics  = {química,química básica},
          source  = {FQ 1B MGH 2016, p84, e12},
        ]
        (84.12) ¿Dónde crees que habrá más moléculas, en \SI{15}{g} de \ch{H2} o en
        \SI{15}{g} de \ch{O2}? Justifica la respuesta.
      \end{exercise}

      \begin{solution}
        En \SI{15}{g} de \ch{H2}.
      \end{solution}




      \begin{exercise}[
          tags    = {},
          topics  = {química,química básica},
          source  = {FQ 1B MGH 2016, p84, e13},
        ]
        (84.13) ¿Cuál será el volumen de \ch{HCl}, medido en CN, que podremos
        obtener con \SI{6e22}{moléculas} de cloro?
      \end{exercise}

      \begin{solution}
        \( V = \SI{4.5}{\liter} \) de \ch{HCl}.
      \end{solution}




      \begin{exercise}[
          tags    = {},
          topics  = {química,química básica},
          source  = {FQ 1B MGH 2016, p84, e14},
        ]
        (84.14) Calcula los gramos de amoniaco que podrías obtener con \SI{10}{\liter}
        de \ch{N2}, medidos en CN.
      \end{exercise}

      \begin{solution}
        \( m_{\ch{NH3}} = \SI{15}{g} \).
      \end{solution}




      \begin{exercise}[
          tags    = {},
          topics  = {química,química básica},
          source  = {FQ 1B MGH 2016, p84, e15},
        ]
        (84.15) A \SI{20}{\celsius} la presión de un gas encerrado en un volumen V constante es de \SI{850}{\mmHg}. ¿Cuál será el valor de la presión si bajamos la temperatura a \SI{0}{\celsius}?
      \end{exercise}

      \begin{solution}
        \( p = \SI{792}{\mmHg} \).
      \end{solution}



    \subsubsection*{Leyes de los gases}

      \begin{exercise}[
          tags    = {},
          topics  = {química,química básica},
          source  = {FQ 1B MGH 2016, p84, e17},
        ]
        (84.17) Diez litros de un gas medidos en CN, ¿qué volumen ocuparán
        si cambiamos las condiciones a \SI{50}{\celsius} y \SI{4}{atm} de presión?
      \end{exercise}

      \begin{solution}
        \( V = \SI{2.76}{\liter}\).
      \end{solution}




      \begin{exercise}[
          tags    = {},
          topics  = {química,química básica},
          source  = {FQ 1B MGH 2016, p84, e18},
        ]
        (84.18) En un matraz de \SI{5}{\liter} hay \SI{42}{\gram} de \ch{N2} a \SI{27}{\celsius}. Se abre el recipiente hasta que su presión se iguala con la presión atmosférica, que es de \SI{1}{atm}.
        \begin{enumerate}
          \item ¿Cuántos gramos de \ch{N2} han salido a la atmósfera?
          \item ¿A qué \( T \) deberíamos poner el recipiente para igualar la presión inicial?
        \end{enumerate}
      \end{exercise}

      \begin{solution}
        \begin{enumerate*}
          \item \SI{36.3}{\gram} de \ch{N2} han salido;
          \item \( T' = \SI{2214}{\kelvin} \).
        \end{enumerate*}
      \end{solution}




      \begin{exercise}[
          tags    = {},
          topics  = {química,química básica},
          source  = {FQ 1B MGH 2016, p84, e20},
        ]
        (84.20) En una bombona se introducen \SI{0.21}{moles} de \ch{N2}, \SI{0.12}{moles}
        de \ch{H2} y \SI{2.32}{moles} de \ch{NH3}. Si la presión total es de \SI{12.4}{atm}, ¿cuál es la presión parcial de cada componente?
      \end{exercise}

      \begin{solution}
        \( p_{\ch{N2}} = \SI{0.98}{atm} \); \( p_{\ch{H2}} = \SI{0.56}{atm} \); \( p_{\ch{NH3}} = \SI{10.9}{atm} \).
      \end{solution}




      \begin{exercise}[
          tags    = {},
          topics  = {química,química básica},
          source  = {FQ 1B MGH 2016, p84, e21},
        ]
        (84.21) En CNTP, \SI{1}{mol} de \ch{NH3} ocupa \SI{22.4}{\liter} y contiene \SI{6.02e23}{moléculas}. Calcula:
        \begin{enumerate}
          \item ¿Cuántas moléculas habrá en \SI{37}{\gram} de amoniaco a \SI{142}{\celsius} y \SI{748}{\mmHg}?
          \item ¿Cuál es la densidad del amoniaco a \SI{142}{\celsius} y \SI{748}{\mmHg}?
        \end{enumerate}
      \end{exercise}

      \begin{solution}
        \begin{enumerate*}
          \item \SI{1.31e24}{moléculas} de \ch{NH3}; \item \( d = \SI{0.49}{\gram\per\liter} \).
        \end{enumerate*}
      \end{solution}




      \begin{exercise}[
          tags    = {},
          topics  = {química,química básica},
          source  = {FQ 1B MGH 2016, p84, e22},
        ]
        (84.22) Resuelve los siguientes ejercicios referidos a la ecuación de Clapeyron:
        \begin{enumerate}
          \item Un gas ocupa un volumen de \SI{15}{\liter} a \SI{60}{\celsius} y \SI{900}{\mmHg}. ¿Qué volumen ocuparía en CN?
          \item En una bombona de \SI{15.0}{\liter} hay gas helio a \SI{20}{\celsius}. Si el manómetro marca \SI{5.2}{atm}, ¿cuántos gramos de helio hay en la bombona? ¿A qué \( T \) estaría el gas si la presión fuera la atmosférica?
          \item Una cierta cantidad de aire ocupa un volumen de \SI{10}{\liter} a \SI{47}{\celsius} y \SI{900}{\mmHg}. Si la densidad del aire es de \SI{1.293}{g/L}, ¿qué masa de aire hay en el recipiente?
        \end{enumerate}
      \end{exercise}

      \begin{solution}
        \begin{enumerate*}
          \item \( V = \SI{14.6}{\liter} \);
          \item \( m = \SI{16}{\gram} \) de \ch{He}, \( T = \SI{56}{\kelvin} \); \item \( m = \SI{13}{\gram} \) de aire.
        \end{enumerate*}
      \end{solution}






  \subsubsection*{Composición centesimal. Fórmulas moleculares y empíricas}

  \begin{exercise}[
      tags    = {},
      topics  = {química,química básica},
      source  = {FQ 1B MGH 2016, p84, e23},
    ]
    (84.23) Un compuesto orgánico tiene la siguiente composición centesimal: \( \ch{C} = 24,24\% \), \( \ch{H} = 4,05\% \), \( \ch{Cl} = 71,71\% \). Calcula: % REVIEW revisar como reacciona el chemformula aquí
    \begin{enumerate}
      \item La fórmula empírica.
      \item Su fórmula molecular, sabiendo que \SI{0.942}{\gram} de dicho compuesto ocupan un volumen de \SI{213}{\milli\liter} medidos a \SI{1}{atm} y \SI{0}{\celsius}.
    \end{enumerate}
  \end{exercise}

  \begin{solution}
    \begin{enumerate*}
      \item \ch{(CH2Cl)_n};
      \item \ch{C2H4Cl2}
    \end{enumerate*}
  \end{solution}




  \begin{exercise}[
      tags    = {},
      topics  = {química,química básica},
      source  = {FQ 1B MGH 2016, p84, e24},
    ]
    (84.24) Resuelve los siguientes ejercicios:
    \begin{enumerate}
      \item Entre dos minerales de fórmulas \ch{Cu5FeS4} y \ch{Cu2S}, ¿cuál es más rico en cobre?
      \item De los siguientes fertilizantes indica cuál es más rico en nitrógeno: \ch{NH4NO3} o \ch{(NH4)3PO3}.
      \item Halla la composición centesimal del arseniato de cobre(II)
      y del sulfato de sodio decahidratado.
    \end{enumerate}
  \end{exercise}

  \begin{solution}
    \begin{enumerate*}
      \item \ch{Cu2S};
      \item \ch{NH4NO3};
      \item 40,7\% de Cu, 32\% de As, 27,3\% de O; 14,3\% de Na, 9,9\% de S, 69,6\% de O, 6,2\% de H.
    \end{enumerate*}
  \end{solution}






  \subsubsection*{Disoluciones y propiedades coligativas}
  \begin{exercise}[
      tags    = {},
      topics  = {química,química básica},
      source  = {FQ 1B MGH 2016, p85, e26},
    ]
    (85.26) Calcula la fracción molar de agua y alcohol etílico en una
    disolución preparada agregando \SI{50}{\gram} de alcohol etílico y \SI{100}{\gram} de agua.
  \end{exercise}

  \begin{solution}
    \( X_\textrm{alcohol} = 0,16 \), \( X_\textrm{agua} = 0,84 \).
  \end{solution}




  \begin{exercise}[
      tags    = {},
      topics  = {química,química básica},
      source  = {FQ 1B MGH 2016, p85, e29},
    ]
    (85.29) Un ácido sulfúrico diluido tiene una densidad de \SI{1.10}{\gram\per\milli\liter} y una riqueza del 65\% en masa. Calcula la molaridad y la normalidad de la disolución.
  \end{exercise}

  \begin{solution}
    \SI{7.3}{M}; \SI{14.6}{N}.
  \end{solution}




  \begin{exercise}[
      tags    = {},
      topics  = {química,química básica},
      source  = {FQ 1B MGH 2016, p85, e30},
    ]
    (85.30) Calcula los gramos de hidróxido de sodio comercial de un
    85\% de riqueza en masa que harán falta para preparar \SI{250}{\milli\liter} de una disolución de \ch{NaOH} \SI{0.5}{M}.
  \end{exercise}

  \begin{solution}
    \SI{5.9}{\gram}
  \end{solution}




  \begin{exercise}[
      tags    = {},
      topics  = {química,química básica},
      source  = {FQ 1B MGH 2016, p85, e31},
    ]
    (85.31) Una disolución de ácido sulfúrico está formada por \SI{12.0}{\gram} de
    ácido, \SI{19.2}{\gram} de agua y ocupa un volumen de \SI{27}{\milli\liter}. Calcula la densidad de la disolución, la concentración centesimal, la molaridad y la molalidad.
  \end{exercise}

  \begin{solution}
    \( d = \SI{1.16}{\gram\per\milli\liter} \); \( \%\textrm{masa} = 38,5\% \); \( M = \SI{4.5}{M} \); \( m = \SI{6.4}{m} \)
  \end{solution}




  \begin{exercise}[
      tags    = {},
      topics  = {química,química básica},
      source  = {FQ 1B MGH 2016, p85, e32},
    ]
    (85.32) En la etiqueta de un frasco de \ch{HCl} dice: densidad \SI{1.19}{\gram\per\milli\liter}, riqueza 37,1\% en peso. Calcula:
    \begin{enumerate}
      \item Masa de \SI{1}{\liter} de esta disolución.
      \item Concentración del ácido en \si{\gram\per\liter}.
      \item Molaridad del ácido.
    \end{enumerate}
  \end{exercise}

  \begin{solution}
    \begin{enumerate*}
      \item \SI{1,19}{\kilo\gram};
      \item \SI{441,5}{\gram\per\liter};
      \item \SI{12,1}{M}.
    \end{enumerate*}
  \end{solution}




  \begin{exercise}[
      tags    = {},
      topics  = {química,química básica},
      source  = {FQ 1B MGH 2016, p85, e33},
    ]
    (85.33) Cuando se agrega \SI{27.8}{\gram} de una sustancia a \SI{200}{\cubic\centi\meter} de agua, la presión de vapor baja de \SI{23.7}{\mmHg} a \SI{22.9}{\mmHg}. Calcula la masa molecular de la sustancia.
  \end{exercise}

  \begin{solution}
    \SI{71.7}{\gram\per\mole}.
  \end{solution}




  \begin{exercise}[
      tags    = {},
      topics  = {química,química básica},
      source  = {FQ 1B MGH 2016, p85, e34},
    ]
    (85.34) Una disolución compuesta por \SI{24}{\gram} de azúcar en \SI{75}{\cubic\centi\meter} de agua, congela a \SI{-1.8}{\celsius}. Calcula:
    \begin{enumerate}
      \item La masa molecular del azúcar.
      \item Si su fórmula empírica es \ch{CH2O}, ¿cuál es su fórmula molecular? Dato: \( K_c = \SI{1.86}{\celsius\kilo\gram\per\mole} \).
    \end{enumerate}
  \end{exercise}

  \begin{solution}
    \begin{enumerate*}
      \item \SI{330}{\gram};
      \item \ch{C11H22O11}
    \end{enumerate*}
  \end{solution}




  \begin{exercise}[
      tags    = {},
      topics  = {química,química básica},
      source  = {FQ 1B MGH 2016, p85, e35},
    ]
    (85.35) Una disolución que contiene \SI{25}{\gram} de albúmina de
    huevo por litro ejerce una presión osmótica de \SI{13.5}{\mmHg} a
    \SI{25}{\celsius}. Determina la masa molecular de esa proteína.
  \end{exercise}

  \begin{solution}
    \SI{3.44e4}{\gram\per\mole}
  \end{solution}



  \begin{exercise}[
      tags    = {},
      topics  = {química,química básica},
      source  = {FQ 1B MGH 2016, p85, e36},
    ]
    (85.36) Cuando llega el invierno y bajan las temperaturas decidimos fabricar nuestro propio anticongelante añadiendo \SI{3}{\liter} de etilenglicol (\ch{C2H6O2}), cuya densidad es de \SI{1.12}{\gram\per\cubic\centi\meter} a \SI{8}{\liter} de agua que vertemos al radiador del coche. ¿A qué temperatura podrá llegar la disolución del radiador sin que se congele?

    \begin{gexdatos}
      constante crioscópica molal del agua \( K_c = \SI{1.86}{\celsius\kilo\gram\per\mole} \).
    \end{gexdatos}
  \end{exercise}

  \begin{solution}
    \SI{-12.6}{\celsius}.
  \end{solution}





  \subsubsection*{Aplica lo aprendido}

  \begin{exercise}[
      tags    = {},
      topics  = {química,química básica},
      source  = {FQ 1B MGH 2016, p85, e38},
    ]
    (85.38) Razona en cuál de las siguientes cantidades habrá un mayor número de átomos:

    \begin{enumerate}
      \item \SI{20}{\gram} de hierro.
      \item \SI{20}{\gram} de azufre.
      \item \SI{20}{\gram} de oxígeno molecular.
      \item Todas tienen la misma cantidad de átomos.
    \end{enumerate}
  \end{exercise}

  \begin{solution}
    La c): \SI{7.53e23}{átomos} de O.
  \end{solution}




  \begin{exercise}[
      tags    = {},
      topics  = {química,química básica},
      source  = {FQ 1B MGH 2016, p85, e39},
    ]
    (85.39) Una determinada cantidad de aire a la presión de \SI{2}{atm} y
    temperatura de \SI{298}{\kelvin} ocupa un volumen de \SI{10}{\liter}. Calcula la masa molecular media del aire, sabiendo que el contenido del
    mismo en el matraz tiene una masa de \SI{23.6}{\gram}.
  \end{exercise}

  \begin{solution}
    \( m = \SI{28.8}{\gram\per\mole} \).
  \end{solution}




  \begin{exercise}[
      tags    = {},
      topics  = {química,química básica},
      source  = {FQ 1B MGH 2016, p86, e43},
    ]
    (86.43) Si tenemos encerrado aire en un recipiente de cristal, al
    calentarlo a \SI{20}{\celsius} la presión se eleva a \SI{1.2}{atm}. ¿Cuánto marcará el barómetro si elevamos la temperatura \SI{10}{\celsius}?
  \end{exercise}

  \begin{solution}
    \( p = \SI{1.24}{atm} \).
  \end{solution}




  \begin{exercise}[
      tags    = {},
      topics  = {química,química básica},
      source  = {FQ 1B MGH 2016, p86, e44},
    ]
    (86.44) Se queman completamente \SI{1.50}{\gram} de un compuesto orgánico
    formado por carbono, hidrógeno y oxígeno. En la combustión
    se obtuvieron \SI{0.71}{\gram} de agua y \SI{1.74}{\gram} de \ch{CO2}. Determina las fórmulas empírica y molecular del compuesto si
    \SI{1.03}{\gram} del mismo ocupan un volumen de \SI{350}{\milli\liter} a \SI{20}{\celsius} y \SI{750}{\mmHg}.
  \end{exercise}

  \begin{solution}
    Empírica \ch{C2H4O3}; molecular \ch{C2H4O3}
  \end{solution}




  \begin{exercise}[
      tags    = {},
      topics  = {química,química básica},
      source  = {FQ 1B MGH 2016, p86, e45},
    ]
    (86.45) Sabiendo que la densidad del aire en CN es de \SI{1.293}{\gram\per\liter}, calcula la masa de aire que contiene un recipiente de \SI{25}{\liter}, si hemos medido que la presión interior, cuando la temperatura es de \SI{77}{\celsius}, es de \SI{1.5}{atm}. Calcula, asimismo, el número de moles de aire que tenemos.
  \end{exercise}

  \begin{solution}
    \( m = \SI{37.82}{\gram} \); \( n = \SI{1.31}{\mole} \).
  \end{solution}




  \begin{exercise}[
      tags    = {},
      topics  = {química,química básica},
      source  = {FQ 1B MGH 2016, p86, e46},
    ]
    (86.46) A partir de los siguientes datos, determina la fórmula empírica y molecular de:

    \begin{enumerate}
      \item Un hidrocarburo con 82,76\% de C; si su densidad en CN es de \SI{2.59}{\gram\per\liter}.
      \item Un hidrocarburo formado por un 85,7\% de C; si \SI{651}{\gram} contienen \SI{15.5}{moles} del mismo
      \item Un compuesto con 57,1\% de C, 4,8\% de H y 38,1\% de S; si en \SI{10}{\gram} hay \SI{3.6e22}{moléculas}.
      \item Un compuesto con 55\% de Cl, 37,2\% de C y 7,8\% de H; si \SI{2.8}{\gram} del compuesto ocupan un volumen de \SI{1.15}{\liter} a \SI{27}{\celsius} y \SI{0.93}{atm} de presión.
    \end{enumerate}
  \end{exercise}

  \begin{solution}
    \begin{enumerate*}
      \item \ch{C4H10};
      \item \ch{C3H6};
      \item \ch{C8H8S2};
      \item \ch{C2H2Cl}.
    \end{enumerate*}
  \end{solution}




  \begin{exercise}[
      tags    = {},
      topics  = {química,química básica},
      source  = {FQ 1B MGH 2016, p86, e49},
    ]
    (86.49) Se dispone de tres recipientes que contienen \SI{1}{\liter} de \ch{CH4} gas, \SI{2}{\liter} de \ch{N2} gas y \SI{15}{\liter} de \ch{O2} gas, respectivamente, en condiciones normales de presión y temperatura. Indica razonadamente:

    \begin{enumerate}
      \item Cuál contiene mayor número de moléculas.
      \item Cuál contiene mayor número de átomos.
      \item Cuál tiene mayor densidad.
    \end{enumerate}

    \begin{gexdatos}
      masas atómicas: \( \textrm{H} = 1 \); \( \textrm{C} = 12 \); \( \textrm{N} = 14 \); \( \textrm{O} = 16 \).
    \end{gexdatos}
  \end{exercise}

  \begin{solution}
    \begin{enumerate*}
      \item El \ch{O2} (\SI{4.0e23}{moléculas});
      \item el \ch{O2} (\SI{8.0e23}{átomos});
      \item el \ch{O2} (\SI{1.4}{\gram\per\liter}).
    \end{enumerate*}
  \end{solution}





  \begin{exercise}[
      tags    = {},
      topics  = {química,química básica},
      source  = {FQ 1B MGH 2016, p86, e50},
    ]
    (86.50) Un frasco de \SI{1.0}{\liter} de capacidad está lleno de dióxido de carbono gaseoso a \SI{27}{\celsius}. Se hace vacío hasta que la presión del
    gas es \SI{10}{\mmHg}. Indica razonadamente:

    \begin{enumerate}
      \item Cuántos gramos de dióxido de carbono contiene el frasco.
      \item Cuántas moléculas hay en el frasco.
    \end{enumerate}

    \begin{gexdatos}
      \( R = \SI{0.082}{atm \liter\per\mole\per\kelvin} \); masas atómicas: \( \ch{C} = 12 \); \( \ch{O} = 16 \).
    \end{gexdatos}
  \end{exercise}

  \begin{solution}
    \begin{enumerate*}
      \item \SI{0.024}{\gram} \ch{CO2};
      \item \SI{3.2e20}{moléculas} de \ch{CO2}
    \end{enumerate*}
  \end{solution}
