\documentclass[10pt]{article}
\title{Ejercicios Charly}
\author{Gonzalo Esteban}

\usepackage{polyglossia}
    \setdefaultlanguage{spanish}
\usepackage{fontspec}
    \setmainfont{Fira Sans}
\usepackage{amsmath, amsthm, amssymb}
\usepackage{unicode-math}
  \unimathsetup{
    math-style  = ISO,
    bold-style  = ISO
  }
  \setmathfont{Fira Math}
\usepackage{multicol}
  \setlength{\columnsep}{1cm}
\usepackage[top=2cm, bottom=3cm, left=2.5cm, right=2cm]{geometry}
\usepackage{xsim}
% Configuración general de xsim
  \loadxsimstyle{layouts}
  \xsimsetup{
  path                  = {xsim-files},
  exercise/template     = {runin},
  exercise/name         = {},
  exercise/print        = {true},
  solution/template     = {runin},
  solution/name         = {S},
  solution/print        = {false},
  exercise/within       = section,
  exercise/the-counter  = \thesection.\arabic{exercise}
  }
  \DeclareExerciseProperty{source}
\usepackage{siunitx}
% Exponent symbol options: \times for the typical cross
  \sisetup{
    per-mode                = symbol,
    output-decimal-marker   = {,},
    exponent-product        = \cdot,
    text-celsius            = ^^b0\kern -\scriptspace C,  % soluciona problemas con el símbolo de grados
    math-celsius            = ^^b0\kern -\scriptspace C,
    list-final-separator    = { y },
    list-pair-separator     = { y },
    range-phrase            = { \translate{to (numerical range)} }
  }
\usepackage[inline]{enumitem}
% Configuración general de enumitem
% Establece la configuración por defecto en a), b), c)
  \setlist[enumerate,1]{
    label   = \alph*),
    itemsep = 0.3\itemsep,
  }
\usepackage{chemformula}
\usepackage{chemfig}
  \setchemfig{atom sep=2em}
\usepackage{booktabs}
\usepackage{fancyhdr}
  \pagestyle{fancy}
  \lhead{\textbf{Ejercicios química}}
  \chead{}
  \rhead{1 BACH}
  \lfoot{}
  \cfoot{\thepage}
  \rfoot{}
  \renewcommand{\headrulewidth}{0.2pt}
  \renewcommand{\footrulewidth}{0pt}





  \newenvironment{gexdatos}{
      \noindent\makebox[0pt][r]{\textit{Datos:}}
    }{\vspace{5pt}}








\begin{document}

\maketitle

\begin{multicols}{2}[
    \section{Bloque 2. Aspectos cualitativos de la Química (tema 3 del libro)}
  ]

\begin{exercise}[
    tags    = {},
    topics  = {química,química básica},
    source  = {FQ 1B MGH 2016, p66, e4},
  ]
  (4) Sabemos que \SI{40}{uma} es la masa del átomo de calcio. Calcula:

  \begin{enumerate}
    \item La masa en gramos de \SI{1}{átomo} de Ca.
    \item ¿Cuál de las siguientes cantidades tienen mayor número de átomos? \SI{40}{g} de Ca; \SI{0,20}{moles} de Ca; \SI{5e23}{átomos} de Ca.
  \end{enumerate}
\end{exercise}

\begin{solution}
  \begin{enumerate*}
    \item \( m = \SI{6,6E-23}{\gram} \);
    \item \SI{40}{\gram} de Ca.
  \end{enumerate*}
\end{solution}




\begin{exercise}[
    tags    = {},
    topics  = {química,química básica},
    source  = {FQ 1B MGH 2016, p66, e5},
  ]
  (5) Si tenemos en cuenta que \SI{56}{uma} es la masa del átomo de hierro, calcula:
  \begin{enumerate}
    \item La masa atómica en gramos de \SI{1}{átomo} de~Fe.
    \item Cuál de las siguientes cantidades tiene mayor número de átomos de Fe: \SI{56}{\gram}, \SI{0,20}{moles} o \SI{5e23}{átomos}.
  \end{enumerate}
\end{exercise}

\begin{solution}
  \begin{enumerate*}
    \item \( m = \SI{9,3E-23}{\gram} \);
    \item \SI{56}{\gram} de Fe.
  \end{enumerate*}
\end{solution}




\begin{exercise}[
    tags    = {},
    topics  = {química,química básica},
    source  = {FQ 1B MGH 2016, p66, e6},
  ]
  (6) Responde a las siguientes cuestiones:
  \begin{enumerate}
    \item ¿En cuál de las siguientes cantidades de los elementos que se enumeran a continuación existe un mayor número de moles: \SI{100}{\gram} de hierro, \SI{100}{\gram} de oxígeno molecular, \SI{100}{\gram} de cinc o \SI{100}{\gram} de níquel?
    \item ¿Y un mayor número de átomos?
  \end{enumerate}
\end{exercise}

\begin{solution}
  \begin{enumerate*}
    \item En los \SI{100}{\gram} de oxígeno molecular;
    \item En los \SI{100}{\gram} de oxígeno molecular.
  \end{enumerate*}
\end{solution}




\begin{exercise}[
    tags    = {},
    topics  = {química,química básica},
    source  = {FQ 1B MGH 2016, p71, e10},
  ]
  (10) Sabiendo que un gas a \SI{1,5}{atm} y \SI{290}{\kelvin} tiene una densidad de \SI{1,178}{\gram\per\liter}, calcula su masa molecular.
\end{exercise}

\begin{solution}
  \( M = \SI{18,7}{\gram\per\mole} \)
\end{solution}




\begin{exercise}[
    tags    = {},
    topics  = {química,química básica},
    source  = {FQ 1B MGH 2016, p71, e11},
  ]
  (11) Calcula la densidad del metano (\ch{CH4}) a \SI{700}{\mmHg} y \SI{75}{\celsius}.
\end{exercise}

\begin{solution}
  \( d = \SI{0,52}{\gram\per\liter} \)
\end{solution}



\begin{exercise}[
    tags    = {},
    topics  = {química,química básica},
    source  = {FQ 1B MGH 2016, p71, e12},
  ]
  (12) Calcula el número de moléculas de \ch{CO2} que habrá en \SI{10}{\liter} del mismo gas medidos en condiciones normales.
\end{exercise}

\begin{solution}
  \( M = \SI{2,7e23}{moléculas} \).
\end{solution}




\begin{exercise}[
    tags    = {},
    topics  = {química,química básica},
    source  = {FQ 1B MGH 2016, p71, e13},
  ]
  (13) Calcula la masa en gramos de un mol de \ch{SO2} sabiendo que
  exactamente \SI{5}{\cubic\centi\meter} de dicho gas, medidos en condiciones
  normales, tienen una masa de \SI{0,01428}{\gram}.
\end{exercise}

\begin{solution}
  \( M = \SI{64}{\gram\per\mole} \)
\end{solution}




\begin{exercise}[
    tags    = {},
    topics  = {química,química básica},
    source  = {FQ 1B MGH 2016, p71, e14},
  ]
  (14) La masa de \SI{1,20}{\milli\gram} de una sustancia gaseosa pura equivale
  a \SI{1,2e19}{moléculas}. Calcula la masa en gramos de \SI{1}{\mole} de dicha sustancia.
\end{exercise}

\begin{solution}
  \( M = \SI{60,2}{\gram\per\mole} \)
\end{solution}




\begin{exercise}[
    tags    = {},
    topics  = {química,química básica},
    source  = {FQ 1B MGH 2016, p72, e15},
  ]
  (15) Se introducen, en un recipiente de \SI{5.0}{\liter}, \SI{10}{\gram} de alcohol etílico (\ch{C2H5OH}) y \SI{10}{\gram} de acetona (\ch{C3H6O}) y posteriormente se calienta el reactor a \SI{200}{\celsius}, con lo cual ambos líquidos pasan a la fase gaseosa. Calcula la presión en el interior del reactor, suponiendo comportamiento ideal, y la presión parcial de cada componente.
\end{exercise}

\begin{solution}
  \( p_\textrm{alcohol} = \SI{1.7}{atm} \);
  \( p_\textrm{acetona} = \SI{1.3}{atm} \);
  \( P_T = \SI{3.0}{atm} \).
\end{solution}




\begin{exercise}[
    tags    = {},
    topics  = {química,química básica},
    source  = {FQ 1B MGH 2016, p74, e16},
  ]
  (16) Calcula la composición centesimal de la molécula de propano (\ch{C3H8}).
\end{exercise}

\begin{solution}
  \num{81,8}\% de carbono; \num{18,2}\% de hidrógeno.
\end{solution}




\begin{exercise}[
    tags    = {},
    topics  = {química,química básica},
    source  = {FQ 1B MGH 2016, p77, e18},
  ]
  (18) Calcula la fracción molar de cada uno de los componentes
  de una disolución que se ha preparado mezclando \SI{90}{\gram} de alcohol etílico (\ch{C2H50H}) y \SI{110}{\gram} de agua.
\end{exercise}

\begin{solution}
  \( X_\textrm{alcohol} = 0.24 \),
  \( X_\textrm{agua} = 0,76 \).
\end{solution}




\begin{exercise}[
    tags    = {},
    topics  = {química,química básica},
    source  = {FQ 1B MGH 2016, p77, e20},
  ]
  (20) Una disolución de hidróxido de sodio en agua que contiene un 25\% de hidróxido tiene una densidad de \SI{1.25}{\gram\per\milli\liter}. Calcula
  su molaridad y su normalidad.
\end{exercise}

\begin{solution}
  \( M = \SI{7.8}{M} \)  y  \( N = \SI{7.8}{N} \).
\end{solution}




\begin{exercise}[
    tags    = {},
    topics  = {química,química básica},
    source  = {FQ 1B MGH 2016, p77, e22},
  ]
  (22) ¿Cuál es la molaridad de una disolución de ácido sulfúrico
  del 26\% de riqueza y densidad \SI{1.19}{\gram\per\milli\liter}?
\end{exercise}

\begin{solution}
  \( M = \SI{3.2}{M} \).
\end{solution}




\begin{exercise}[
    tags    = {},
    topics  = {química,química básica},
    source  = {FQ 1B MGH 2016, p80, e23},
  ]
  (23) El alcanfor puro tiene un punto de fusión de \SI{178}{\celsius} y una constante crioscópica de \SI{40}{\celsius\kilo\gram\per\mole}. La disolución resultante de añadir \SI{2}{\gram} de un soluto no volátil a
  \SI{10}{\gram} de alcanfor congela a \SI{158}{\celsius}. Calcula la masa molecular del soluto añadido.
\end{exercise}

\begin{solution}
  \( M = \SI{400}{\gram\per\mole} \).
\end{solution}




\begin{exercise}[
    tags    = {},
    topics  = {química,química básica},
    source  = {FQ 1B MGH 2016, p80, e24},
  ]
  (24) Tenemos \SI{100}{\milli\liter} de una disolución acuosa que contiene \SI{0.25}{\gram} de un polisacárido. Dicha disolución a \SI{25}{\celsius}, ejerce una presión osmótica de \SI{23.9}{\mmHg}. El polisacárido tiene la siguiente fórmula empírica \ch{(C6H10O5)_{n}}. Calcula el valor de la masa molecular del polisacárido.
\end{exercise}

\begin{solution}
  \( M = \SI{1938}{\gram\per\mole} \).
\end{solution}







\subsection{Problemas propuestos}

\subsubsection{Leyes de los volúmenes de combinación. Hipótesis de Avogadro. Concepto de molécula. Mol}

\begin{exercise}[
    tags    = {},
    topics  = {química,química básica},
    source  = {FQ 1B MGH 2016, p83, e8},
  ]
  (8) Determina la masa, M, de un mol de un gas en los siguientes
  casos:
  \begin{enumerate}
    \item Su densidad en c.n. es de \SI{3.17}{\gram\per\liter}.
    \item Su densidad es de \SI{2.4}{g/L} a \SI{20}{\celsius} y \SI{1}{atm} de presión.
    \item Dos gramos de dicho gas ocupan un volumen de \SI{600}{\milli\liter}, medido a \SI{17}{\celsius} y \SI{1.8}{atm} de presión.
  \end{enumerate}
\end{exercise}

\begin{solution}
  \begin{enumerate*}
    \item \( M = \SI{71}{g/mol} \);
    \item \( M = \SI{58}{g/mol} \);
    \item \( M = \SI{44}{g/mol} \).
    \end{enumerate*}
\end{solution}




\begin{exercise}[
    tags    = {},
    topics  = {química,química básica},
    source  = {FQ 1B MGH 2016, p83, e9},
  ]
  (9) Realiza los siguientes cálculos numéricos:
  \begin{enumerate}
    \item
    \item Los átomos de oxígeno que hay en \SI{0.25}{moles} de sulfato
    de potasio (\ch{K2SO4}).
    \item Las moléculas de gasolina (\ch{C8H18}) que hay en un depósito
    de \SI{40}{\liter} (\( d = \SI{0.76}{g/mL} \)).
    \item Los gramos de calcio que hay en \SI{60}{g} de un carbonato de
    calcio (\ch{CaCO3}) del 80\% de riqueza.
    \item De una sustancia pura, sabemos que \SI{1.75e19}{moléculas} moléculas
    corresponden a una masa de \SI{2.73}{mg}. ¿Cuál será la masa de \SI{1}{mol}?
  \end{enumerate}
\end{exercise}

\begin{solution}
  \begin{enumerate*}
    \item \SI{6e23}{átomos}; \item \SI{1.6e26}{moléculas}; \item \SI{19}{g}; \item \( M = \SI{93}{g/mol} \).
  \end{enumerate*}
\end{solution}




\begin{exercise}[
    tags    = {},
    topics  = {química,química básica},
    source  = {FQ 1B MGH 2016, p83, e10},
  ]
  (10) Disponemos de \SI{3}{moles} de sulfuro de hidrógeno. Calcula,
  sabiendo que las masas atómicas son \( \ch{S} = 32 \) y \( \ch{H} = 1 \):
  \begin{enumerate}
    \item Cuántos gramos de \ch{H2S} hay en esos \SI{3}{moles}.
    \item El número de moléculas de \ch{H2S} que forman los \SI{3}{moles}.
    \item Los moles de \ch{H2} y de \ch{S} que tenemos en los \SI{3}{moles} de \ch{H2S}.
  \end{enumerate}
\end{exercise}

\begin{solution}
  \begin{enumerate*}
    \item \( m_{\ch{H2S}} = \SI{102}{\gram} \);
    \item \SI{1.8e24}{moléculas};
    \item \SI{3}{moles} de \ch{H2} y \SI{3}{moles} de \ch{S}.
  \end{enumerate*}
\end{solution}




\begin{exercise}[
    tags    = {},
    topics  = {química,química básica},
    source  = {FQ 1B MGH 2016, p84, e12},
  ]
  (12) ¿Dónde crees que habrá más moléculas, en \SI{15}{g} de \ch{H2} o en
  \SI{15}{g} de \ch{O2}? Justifica la respuesta.
\end{exercise}

\begin{solution}
  En \SI{15}{g} de \ch{H2}.
\end{solution}




\begin{exercise}[
    tags    = {},
    topics  = {química,química básica},
    source  = {FQ 1B MGH 2016, p84, e13},
  ]
  (13) ¿Cuál será el volumen de \ch{HCl}, medido en c.n., que podremos
  obtener con \SI{6e22}{moléculas} de cloro?
\end{exercise}

\begin{solution}
  \( V = \SI{4.5}{\liter} \) de \ch{HCl}.
\end{solution}




\begin{exercise}[
    tags    = {},
    topics  = {química,química básica},
    source  = {FQ 1B MGH 2016, p84, e14},
  ]
  (14) Calcula los gramos de amoniaco que podrías obtener con \SI{10}{\liter}
  de \ch{N2}, medidos en c.n.
\end{exercise}

\begin{solution}
  \( m_{\ch{NH}} = \SI{15}{g} \).
\end{solution}




\begin{exercise}[
    tags    = {},
    topics  = {química,química básica},
    source  = {FQ 1B MGH 2016, p84, e15},
  ]
  (15) A \SI{20}{\celsius} la presión de un gas encerrado en un volumen V constante es de \SI{850}{\mmHg}. ¿Cuál será el valor de la presión si bajamos la temperatura a \SI{0}{\celsius}?
\end{exercise}

\begin{solution}
  \( p = \SI{792}{\mmHg} \).
\end{solution}



\subsubsection{Leyes de los gases}

\begin{exercise}[
    tags    = {},
    topics  = {química,química básica},
    source  = {FQ 1B MGH 2016, p84, e17},
  ]
  (17) Diez litros de un gas medidos en c.n., ¿qué volumen ocuparán
  si cambiamos las condiciones a \SI{50}{\celsius} y \SI{4}{atm} de presión?
\end{exercise}

\begin{solution}
  V= 2,96 L.
\end{solution}




\begin{exercise}[
    tags    = {},
    topics  = {química,química básica},
    source  = {FQ 1B MGH 2016, p84, e18},
  ]
  (18) En un matraz de \SI{5}{\liter} hay si \SI{42}{\gram} de \ch{N2} a \SI{27}{\celsius}. Se abre el recipiente
  hasta que su presión se iguala con la presión atmosférica,
  que es de \SI{1}{atm}.
  \begin{enumerate}
    \item ¿Cuántos gramos de \ch{N2} han salido a la atmósfera?
    \item ¿A qué T deberíamos poner el recipiente para igualar la presión inicial?
  \end{enumerate}
\end{exercise}

\begin{solution}
  \begin{enumerate*}
    \item \SI{36.3}{\gram} de \ch{N2} han salido;
    \item \( T' = \SI{2214}{\kelvin} \).
  \end{enumerate*}
\end{solution}




\begin{exercise}[
    tags    = {},
    topics  = {química,química básica},
    source  = {FQ 1B MGH 2016, p84, e20},
  ]
  (20) En una bombona se introducen \SI{0.21}{moles} de \ch{N2}, \SI{0.12}{moles}
  de \ch{H2} y \SI{2.32}{moles} de \ch{NH3}. Si la presión total es de \SI{12.4}{atm}, ¿cuál es la presión parcial de cada componente?
\end{exercise}

\begin{solution}
  \( p_{\ch{N2}} = \SI{0.98}{atm} \); \( p_{\ch{H2}} = \SI{0.56}{atm} \); \( p_{\ch{NH3}} = \SI{10.9}{atm} \).
\end{solution}




\begin{exercise}[
    tags    = {},
    topics  = {química,química básica},
    source  = {FQ 1B MGH 2016, p84, e21},
  ]
  (21) En c.n. de p y T, \SI{1}{mol} de \ch{NH3} ocupa \SI{22.4}{\liter} y contiene \SI{6.02e23}{moléculas}. Calcula:
  \begin{enumerate}
    \item ¿Cuántas moléculas habrá en \SI{37}{\gram} de amoniaco a \SI{142}{\celsius} y \SI{748}{\mmHg}?
    \item ¿Cuál es la densidad del amoniaco a \SI{142}{\celsius} y \SI{748}{\mmHg}?
  \end{enumerate}
\end{exercise}

\begin{solution}
  \begin{enumerate*}
    \item \SI{1.31e24}{moléculas} de \ch{NH3}; \item \( d = \SI{0.49}{\gram\per\liter} \).
  \end{enumerate*}
\end{solution}




\begin{exercise}[
    tags    = {},
    topics  = {química,química básica},
    source  = {FQ 1B MGH 2016, p84, e22},
  ]
  (22) Resuelve los siguientes ejercicios referidos a la ecuación de Clapeyron:
  \begin{enumerate}
    \item Un gas ocupa un volumen de \SI{15}{\liter} a \SI{60}{\celsius} y \SI{900}{\mmHg}. ¿Qué volumen ocuparía en c.n.?
    \item En una bombona de \SI{15.0}{\liter} hay gas helio a \SI{20}{\celsius}. Si el manómetro marca \SI{5.2}{atm}, ¿cuántos gramos de helio hay en la bombona? ¿A qué T estaría el gas si la presión fuera la atmosférica?
    \item Una cierta cantidad de aire ocupa un volumen de \SI{10}{\liter} a \SI{47}{\celsius} y \SI{900}{\mmHg}. Si la densidad del aire es de \SI{1.293}{g/L}, ¿qué masa de aire hay en el recipiente?
  \end{enumerate}
\end{exercise}

\begin{solution}
  \begin{enumerate*}
    \item \( V = \SI{14.6}{\liter} \); \item \( m = \SI{16}{\gram} \) de \ch{He}, \( T = \SI{56}{\kelvin} \); \item \( m = \SI{13}{\gram} \) de aire.
  \end{enumerate*}
\end{solution}






\subsubsection{Composición centesimal. Fórmulas moleculares y empíricas}

\begin{exercise}[
    tags    = {},
    topics  = {química,química básica},
    source  = {FQ 1B MGH 2016, p84, e23},
  ]
  (23) Un compuesto orgánico tiene la siguiente composición centesimal: \( \ch{C} = 24,24\% \), \( \ch{H} = 4,05\% \), \( \ch{Cl} = 71,71\% \). Calcula: % REVIEW revisar como reacciona el chemformula aquí
  \begin{enumerate}
    \item La fórmula empírica.
    \item Su fórmula molecular, sabiendo que \SI{0.942}{\gram} de dicho compuesto ocupan un volumen de \SI{213}{\milli\liter} medidos a \SI{1}{atm} y \SI{0}{\celsius}.
  \end{enumerate}
\end{exercise}

\begin{solution}
  \begin{enumerate*}
    \item \ch{(CH2Cl)_n};
    \item \ch{C2H4Cl2}
  \end{enumerate*}
\end{solution}




\begin{exercise}[
    tags    = {},
    topics  = {química,química básica},
    source  = {FQ 1B MGH 2016, p84, e24},
  ]
  (24) Resuelve los siguientes ejercicios:
  \begin{enumerate}
    \item Entre dos minerales de fórmulas \ch{Cu5FeS4} y \ch{Cu2S}, ¿cuál es más rico en cobre?
    \item De los siguientes fertilizantes indica cuál es más rico en nitrógeno: \ch{NH4NO3} o \ch{(NH4)3PO3}.
    \item Halla la composición centesimal del arseniato de cobre(II)
    y del sulfato de sodio decahidratado.
  \end{enumerate}
\end{exercise}

\begin{solution}
  \begin{enumerate*}
    \item \ch{Cu2S};
    \item \ch{NH4NO3};
    \item 40,7\% de Cu, 32\% de As, 27,3\% de O; 14,3\% de Na, 9,9\% de S, 69,6\% de O, 6,2\% de H.
  \end{enumerate*}
\end{solution}






\subsubsection{Disoluciones y propiedades coligativas}

\begin{exercise}[
    tags    = {},
    topics  = {química,química básica},
    source  = {FQ 1B MGH 2016, p85, e26},
  ]
  (26) Calcula la fracción molar de agua y alcohol etílico en una
  disolución preparada agregando \SI{50}{\gram} de alcohol etílico y \SI{100}{\gram} de agua.
\end{exercise}

\begin{solution}
  \( X_\textrm{alcohol} = 0,16 \), \( X_\textrm{agua} = 0,84 \).
\end{solution}




\begin{exercise}[
    tags    = {},
    topics  = {química,química básica},
    source  = {FQ 1B MGH 2016, p85, e29},
  ]
  (29) Un ácido sulfúrico diluido tiene una densidad de \SI{1.10}{\gram\per\milli\liter} y una riqueza del 65\% en masa. Calcula la molaridad y la normalidad de la disolución.
\end{exercise}

\begin{solution}
  \SI{7.3}{M}; \SI{14.6}{N}.
\end{solution}




\begin{exercise}[
    tags    = {},
    topics  = {química,química básica},
    source  = {FQ 1B MGH 2016, p85, e30},
  ]
  (30) Calcula los gramos de hidróxido de sodio comercial de un
  85\% de riqueza en masa que harán falta para preparar \SI{250}{\milli\liter} de una disolución de \ch{NaOH} \SI{0.5}{M}.
\end{exercise}

\begin{solution}
\SI{5.9}{\gram}
\end{solution}




\begin{exercise}[
    tags    = {},
    topics  = {química,química básica},
    source  = {FQ 1B MGH 2016, p85, e31},
  ]
  (31) Una disolución de ácido sulfúrico está formada por \SI{12.0}{\gram} de
  ácido, \SI{19.2}{\gram} de agua y ocupa un volumen de \SI{27}{\milli\liter}. Calcula la densidad de la disolución, la concentración centesimal, la molaridad y la molalidad.
\end{exercise}

\begin{solution}
  \( d = \SI{1.16}{\gram\per\milli\liter} \); \( \%masa= 38,5\% \); \( M= \SI{4.5}{M} \); \( m= \SI{6.4}{m} \)
\end{solution}




\begin{exercise}[
    tags    = {},
    topics  = {química,química básica},
    source  = {FQ 1B MGH 2016, p85, e32},
  ]
  (32) En la etiqueta de un frasco de \ch{HCl} dice: densidad \SI{1.19}{\gram\per\milli\liter}, riqueza 37,1\% en peso. Calcula:
  \begin{enumerate}
    \item Masa de \SI{1}{\liter} de esta disolución.
    \item Concentración del ácido en \si{\gram\per\liter}.
    \item Molaridad del ácido.
  \end{enumerate}
\end{exercise}

\begin{solution}
  \begin{enumerate*}
    \item \SI{1,19}{\kilo\gram};
    \item \SI{441,5}{\gram\per\liter};
    \item \SI{12,1}{M}.
  \end{enumerate*}
\end{solution}




\begin{exercise}[
    tags    = {},
    topics  = {química,química básica},
    source  = {FQ 1B MGH 2016, p85, e33},
  ]
  (33) Cuando se agrega \SI{27.8}{\gram} de una sustancia a \SI{200}{\cubic\centi\meter} de agua, la presión de vapor baja de \SI{23.7}{\mmHg} a \SI{22.9}{\mmHg}. Calcula la masa molecular de la sustancia.
\end{exercise}

\begin{solution}
  \SI{71.7}{\gram\per\mole}.
\end{solution}




\begin{exercise}[
    tags    = {},
    topics  = {química,química básica},
    source  = {FQ 1B MGH 2016, p85, e34},
  ]
  (34) Una disolución compuesta por \SI{24}{\gram} de azúcar en \SI{75}{\cubic\centi\meter} de agua, congela a \SI{-1.8}{\celsius}. Calcula:
  \begin{enumerate}
    \item La masa molecular del azúcar,
    \item Si su fórmula empírica es \ch{CH2O}, ¿cuál es su fórmula molecular? Dato: \( K_c = \SI{1.86}{\celsius\kilo\gram\per\mole} \).
  \end{enumerate}
\end{exercise}

\begin{solution}
  \begin{enumerate*}
    \item \SI{330}{\gram};
    \item \ch{C11H22O11}
  \end{enumerate*}
\end{solution}




\begin{exercise}[
    tags    = {},
    topics  = {química,química básica},
    source  = {FQ 1B MGH 2016, p85, e35},
  ]
  (35) Una disolución que contiene \SI{25}{\gram} de albúmina de
  huevo por litro ejerce una presión osmótica de \SI{13.5}{\mmHg}, a
  \SI{25}{\celsius}. Determina la masa molecular de esa proteína.
\end{exercise}

\begin{solution}
  \SI{3.44e4}{\gram\per\mole}
\end{solution}




\begin{exercise}[
    tags    = {},
    topics  = {química,química básica},
    source  = {FQ 1B MGH 2016, p85, e36},
  ]
  (36) Cuando llega el invierno y bajan las temperaturas decidimos fabricar nuestro propio anticongelante añadiendo \SI{3}{\liter} de etilenglicol (\ch{C2H6O2}), cuya densidad es de \SI{1.12}{\gram\per\cubic\centi\meter} a \SI{8}{\liter} de agua que vertemos al radiador del coche. ¿A qué temperatura podrá llegar la disolución del radiador sin que se congele?

    Dato: constante crioscópica molal del agua \( K_c = \SI{1.86}{\celsius\kilo\gram\per\mole} \).
\end{exercise}

\begin{solution}
  \SI{-12.6}{\celsius}.
\end{solution}





\subsubsection{Aplica lo aprendido}

\begin{exercise}[
    tags    = {},
    topics  = {química,química básica},
    source  = {FQ 1B MGH 2016, p85, e38},
  ]
  (38) Razona en cuál de las siguientes cantidades habrá un mayor número de átomos:

  \begin{enumerate}
    \item \SI{20}{\gram} de hierro.
    \item \SI{20}{\gram} de azufre.
    \item \SI{20}{\gram} de oxígeno molecular.
    \item Todas tienen la misma cantidad de átomos.
  \end{enumerate}
\end{exercise}

\begin{solution}
  La c): \SI{7.53e23}{átomos} de O.
\end{solution}




\begin{exercise}[
    tags    = {},
    topics  = {química,química básica},
    source  = {FQ 1B MGH 2016, p85, e39},
  ]
  (39) Una determinada cantidad de aire a la presión de \SI{2}{atm} y
  temperatura de \SI{298}{\kelvin} ocupa un volumen de \SI{10}{\liter}. Calcula la masa molecular media del aire, sabiendo que el contenido del
  mismo en el matraz tiene una masa de \SI{23.6}{\gram}.
\end{exercise}

\begin{solution}
  \( m = \SI{28.8}{\gram\per\mole} \).
\end{solution}




\begin{exercise}[
    tags    = {},
    topics  = {química,química básica},
    source  = {FQ 1B MGH 2016, p86, e43},
  ]
  (43) Si tenemos encerrado aire en un recipiente de cristal, al
  calentarlo a \SI{20}{\celsius} la presión se eleva a \SI{1.2}{atm}. ¿Cuánto marcará el barómetro si elevamos la temperatura \SI{10}{\celsius}?
\end{exercise}

\begin{solution}
  \( p = \SI{1.24}{atm} \).
\end{solution}




\begin{exercise}[
    tags    = {},
    topics  = {química,química básica},
    source  = {FQ 1B MGH 2016, p86, e44},
  ]
  (44) Se queman completamente \SI{1.50}{\gram}1,50 g de un compuesto orgánico
  formado por carbono, hidrógeno y oxígeno. En la combustión
  se obtuvieron \SI{0.71}{\gram} de agua y \SI{1.74}{\gram} de \ch{CO2}. Determina las fórmulas empírica y molecular del compuesto si
  \SI{1.03}{\gram} del mismo ocupan un volumen de \SI{350}{\milli\liter} a \SI{20}{\celsius} y \SI{750}{\mmHg}.
\end{exercise}

\begin{solution}
  Empírica \ch{C2H4O3}; molecular \ch{C2H4O3}
\end{solution}




\begin{exercise}[
    tags    = {},
    topics  = {química,química básica},
    source  = {FQ 1B MGH 2016, p86, e45},
  ]
  (45) Sabiendo que la densidad del aire en c.n. es de \SI{1.293}{\gram\per\liter}, calcula la masa de aire que contiene un recipiente de \SI{25}{\liter}, si hemos medido que la presión interior, cuando la temperatura es de \SI{77}{\celsius}, es de \SI{1.5}{atm}. Calcula, asimismo, el número de moles de aire que tenemos.
\end{exercise}

\begin{solution}
  \( m = \SI{37.82}{\gram} \); \( n = \SI{1.31}{\mole} \).
\end{solution}




\begin{exercise}[
    tags    = {},
    topics  = {química,química básica},
    source  = {FQ 1B MGH 2016, p86, e46},
  ]
  (46) A partir de los siguientes datos, determina la fórmula empírica y molecular de:

  \begin{enumerate}
    \item Un hidrocarburo con 82,76\% de C; si su densidad en c.n. es de \SI{2.59}{\gram\per\liter}.
    \item Un hidrocarburo formado por un 85,7\% de C; si \SI{651}{\gram} contienen \SI{15.5}{moles} del mismo
    \item Un compuesto con 57,1\% de C, 4,8\% de H y 38,1\% de S; si en \SI{10}{\gram} hay \SI{3.6e22}{moléculas}.
    \item Un compuesto con 55\% de Cl, 37,2\% de C y 7,8\% de H; si \SI{2.8}{\gram} del compuesto ocupan un volumen de \SI{1.15}{\liter} a \SI{27}{\celsius} y \SI{0.93}{atm} de presión.
  \end{enumerate}
\end{exercise}

\begin{solution}
  \begin{enumerate*}
    \item \ch{C4H10};
    \item \ch{C3H6};
    \item \ch{C8H8S2};
    \item \ch{C2H2Cl}.
  \end{enumerate*}
\end{solution}




\begin{exercise}[
    tags    = {},
    topics  = {química,química básica},
    source  = {FQ 1B MGH 2016, p86, e49},
  ]
  (49) Se dispone de tres recipientes que contienen \SI{1}{\liter} de \ch{CH4} gas, \SI{2}{\liter} de \ch{N2} gas y \SI{15}{\liter} de \ch{O2} gas, respectivamente, en condiciones normales de presión y temperatura. Indica razonadamente:

  \begin{enumerate}
    \item Cuál contiene mayor número de moléculas.
    \item Cuál contiene mayor número de átomos.
    \item Cuál tiene mayor densidad.
  \end{enumerate}

  \begin{gexdatos}
    masas atómicas: \( \textrm{H} = 1 \); \( \textrm{C} = 12 \); \( \textrm{N} = 14 \); \( \textrm{O} = 16 \).
  \end{gexdatos}
\end{exercise}

\begin{solution}
  \begin{enumerate*}
    \item El \ch{O2} (\SI{4.0e23}{moléculas});
    \item el \ch{O2} (\SI{8.0e23}{átomos});
    \item el \ch{O2} (\SI{1.4}{\gram\per\liter}).
  \end{enumerate*}
\end{solution}





\begin{exercise}[
    tags    = {},
    topics  = {química,química básica},
    source  = {FQ 1B MGH 2016, p86, e50},
  ]
  (50) Un frasco de \SI{1.0}{\liter} de capacidad está lleno de dióxido de carbono gaseoso a \SI{27}{\celsius}. Se hace vacío hasta que la presión del
  gas es \SI{10}{\mmHg}. Indica razonadamente:

  \begin{enumerate}
    \item Cuántos gramos de dióxido de carbono contiene el frasco.
    \item Cuántas moléculas hay en el frasco.
  \end{enumerate}

  \begin{gexdatos}
    \( R = \SI{0.082}{atm\liter\per\mole\kelvin} \); masas atómicas: \( \ch{C} = 12 \); \( \ch{O} = 16 \).
  \end{gexdatos}
\end{exercise}

\begin{solution}
  \begin{enumerate*}
    \item \SI{0.024}{\gram} \ch{CO2};
    \item \SI{3.2e20}{moléculas} de \ch{CO2}
  \end{enumerate*}
\end{solution}

\end{multicols}






\begin{multicols}{2}[
  \section{BLOQUE 5. QUÍMICA DEL CARBONO (tema 5 del libro)}
  ]

\begin{exercise}[
    tags    = {},
    topics  = {química, química orgánica, orgánica},
    source  = {FQ 1B MGH 2016, p126, e7},
  ]
  (7) Formula los siguientes alcanos:
  \begin{enumerate}
    \item n-pentano
    \item 2,3,5-trimetilheptano
    \item 4-etil-2,6-dimetiloctano
    \item 4,6-dietil-2,4,8-trimetilnonano
    \item 4-etil-2,2,5,8-tetrametil-6-propildecano
    \item 3,7-dietil-5-isopropildecano
  \end{enumerate}
\end{exercise}

\begin{solution}[print=false]
  Todavía sin solución
\end{solution}




\begin{exercise}[
    tags    = {},
    topics  = {química, química orgánica, orgánica},
    source  = {FQ 1B MGH 2016, p129, e9},
  ]
  (9) Formula los siguientes hidrocarburos insaturados:
  \begin{enumerate}
    \item But-1-eno
    \item Pent-2-eno
    \item Hexa-2,4-dieno
    \item 3-butilhexa-1,4-dieno
    \item But-2-ino
    \item 3,4-dimetilpent-1-ino
    \item 3,6-dimetilnona-1,4,7-triino
    \item Pent-1-en-3-ino
    \item Hept-3-en-1,6-diino
    \item 4-etilhexa-1,3-dien-5-ino
  \end{enumerate}
\end{exercise}

\begin{solution}[print=false]
  Todavía sin solución
\end{solution}




\begin{exercise}[
    tags    = {},
    topics  = {química, química orgánica, orgánica},
    source  = {FQ 1B MGH 2016, p131, e11},
  ]
  (11) Formula los siguientes hidrocarburos cíclicos:
  \begin{enumerate}
    \item Etilciclohexano
    \item Ciclopenteno
    \item Ciclohexino
    \item 1,1,4,4-tetrametilciclohexano
    \item 3-etilciclopenteno
    \item 2,3-dimetilciclohexeno
    \item 4-ciclobutilpent-1-ino
    \item 3-ciclohexil-5-metilhex-2-eno
    \item Ciclohexa-1,3-dieno
    \item 3-ciclopentilprop-1-eno
  \end{enumerate}
\end{exercise}

\begin{solution}[print=false]
  Todavía sin solución
\end{solution}




\begin{exercise}[
    tags    = {},
    topics  = {química, química orgánica, orgánica},
    source  = {FQ 1B MGH 2016, p131, e12},
  ]
  (12) Nombra los siguientes hidrocarburos cíclicos:
  \begin{enumerate}
    \item \chemfig{*4(----)}
    \item \chemfig{*6(--=---)}
    \item \chemfig{*6(---(-CH_3)--(-CH_2-[2]CH_3)-)}
    \item \chemfig{*6(---=-=)}
    \item \chemfig{*6(----(-CH(-[4]CH_3)-[0]CH_3)--)}
    \item \chemfig{CH_3-CH(-[6]*6(------))-CH_2-C(-[2]CH_3)(-[6]CH_3)-CH_2-CH_3}
    \item \chemfig{CH_3-CH(-[6]*5(-----))-CH=CH_2}
  \end{enumerate}
\end{exercise}

\begin{solution}[print=false]
  Todavía sin solución
\end{solution}




\begin{exercise}[
    tags    = {},
    topics  = {química, química orgánica, orgánica},
    source  = {FQ 1B MGH 2016, p132, e13},
  ]
  (13) Formula los siguientes hidrocarburos aromáticos:
  \begin{enumerate}
    \item Metilbenceno (tolueno)
    \item Etenilbenceno
    \item 1,3-dietilbenceno
    \item 1-butil-4-isopropilbenceno
    \item Para-propiltolueno
    \item 3-fenil-5-metilheptano
    \item 4-fenilpent-1-eno
    \item 2,4-difenil-3-metilhexano
  \end{enumerate}
\end{exercise}

\begin{solution}[print=false]
  Todavía sin solución
\end{solution}




\begin{exercise}[
    tags    = {},
    topics  = {química, química orgánica, orgánica},
    source  = {FQ 1B MGH 2016, p133, e15},
  ]
  (15) Formula los siguientes derivados halogenados:
  \begin{enumerate}
    \item 2-cloropropano
    \item 1,3-dibromobenceno
    \item 1,1,2,2-tetrafluoretano
    \item 1,4-diclorociclohexano
    \item 4-bromopent-1-ino
    \item 3-flúor-5-metilhex-2-eno
    \item 1,4-dibromo-6-ciclopentiloct-2-eno
    \item 4-yodo-3,5-difenilpent-1-ino
    \item 4-clorobut-1-eno
    \item 1,2-dibromobenceno
  \end{enumerate}
\end{exercise}

\begin{solution}[print=false]
  Todavía sin solución
\end{solution}




\begin{exercise}[
    tags    = {},
    topics  = {química, química orgánica, orgánica},
    source  = {FQ 1B MGH 2016, p135, e17},
  ]
  (17) Formula los siguientes alcoholes y éteres:
  \begin{enumerate}
    \item 3-metilpentan-1-ol
    \item Butano-1,2,3-triol
    \item 2-fenilpropano-1,3-diol
    \item Ciclohexanol
    \item Hexa-3,5-dien-2-ol
    \item Fenol (Hidroxibenceno)
    \item 2-etilpentan-1-ol
    \item Pent-3-en-1-ol
    \item Etilisopropiléter
    \item Etenilfeniléter
    \item Dimetiléter
    \item Butilciclopentiléter
  \end{enumerate}
\end{exercise}

\begin{solution}[print=false]
  Todavía sin solución
\end{solution}




\begin{exercise}[
    tags    = {},
    topics  = {química, química orgánica, orgánica},
    source  = {FQ 1B MGH 2016, p135, e18},
  ]
  (18) Nombra los siguientes alcoholes y éteres:
  \begin{enumerate}
    \item \ch{CH3OH}
    \item \chemfig{CH_2OH-CH_2-CH(-[6]CH_2-[::0]CH_3)-CH=CH_2}
    \item \ch{CH3-CHOH-CHBr-CH2OH}
    \item \chemfig{CH_3-CH_2-CH(-[6]CH_3)-CHOH-CH_3}
    \item \ch{CH2=CH-CHOH-CH2OH}
    \item \ch{CH3-CH2-CH2-O-CH2-CH2-CH3}
    \item \chemfig{CH_2=CH-O-CH(-[6]CH_3)-CH_3}
    \item \ch{CH3-(CH2)3-CH2-O-C+CH}
  \end{enumerate}
  VER EN EL LIBRO, PÁGINA 135
\end{exercise}

\begin{solution}[print=false]
  Todavía sin solución
\end{solution}




\begin{exercise}[
    tags    = {},
    topics  = {química, química orgánica, orgánica},
    source  = {FQ 1B MGH 2016, p136, e19},
  ]
  (19) Formula los siguientes aldehídos y cetonas:
  \begin{enumerate}
    \item Etanal (acetaldehído)
    \item Benzaldehído
    \item 3-metilpentanal
    \item 2-metilpentanodial
    \item Propenal
    \item Hex-2-endial
    \item 5-ciclohexilpent-3-inal
    \item 3-metilpent-2-enal
    \item Hex-2-endial
    \item Pentan-2-ona
    \item Hexa-2,4-diona
    \item 3-clorobutanona
    \item 1,4-difenilpentan-2-ona
    \item Hexa-1,5-dien-3-ona
  \end{enumerate}
\end{exercise}

\begin{solution}[print=false]
  Todavía sin solución
\end{solution}




\begin{exercise}[
    tags    = {},
    topics  = {química, química orgánica, orgánica},
    source  = {FQ 1B MGH 2016, p136, e20},
  ]
  (20) Nombra los siguientes aldehídos y cetonas:

  \begin{enumerate}
    \item HCHO
    \item \ch{CH3-CH2-CH2-CHO}
    \item \ch{OHC-CH=CH-CHO}
    \item \chemfig{CH_2=C(-[6]*5(-----))-CH_2-{(}CH_2{)}_4-CHO}
    \item \ch{OHC-CH=CH-CH2-CH(CH3)-CHO}
    \item \chemfig{CH_3-CH(-[6]C_6H_5)-CH=CH-CHO}
    \item \ch{CHO-CH2-C=C-CH2-CH2-CHO}
    \item \ch{CH3-CO-CH2-CH3}
    \item \ch{CH3-CH=CH-CH2-CO-CH3}
    \item \ch{CH3-CO-CH 2-CH2-CH2-CO-CH3}
    \item \ch{CH3-CH(CH3)-CO-CH2-CH(CH3)-CH3}
    \item \ch{CH2=CH-CO-CH=CH-CH3}
  \end{enumerate}
\end{exercise}

\begin{solution}[print=false]
  Todavía sin solución
\end{solution}




\begin{exercise}[
    tags    = {},
    topics  = {química, química orgánica, orgánica},
    source  = {FQ 1B MGH 2016, p138, e21},
  ]
  (21) Formula los siguientes ácidos y ésteres:
  \begin{enumerate}
    \item Ácido etanoico (ácido acético)
    \item Ácido 3-metilhexanoico
    \item Ácido 2-fenilpentanodioico
    \item Ácido tricloroetanoico
    \item Ácido but-3-enoico
    \item Ácido hepta-2,4-dienoico
    \item Ácido pent-2-enodioico
    \item Ácido benzoico
    \item Butanoato de metilo
    \item Propanoato de etilo
    \item Benzoato de propilo
    \item Etanoato de octilo
    \item 3-cloropentanoato de etenilo
    \item But-3-enoato de isopropilo
  \end{enumerate}
\end{exercise}

\begin{solution}[print=false]
  Todavía sin solución
\end{solution}




\begin{exercise}[
    tags    = {},
    topics  = {química, química orgánica, orgánica},
    source  = {FQ 1B MGH 2016, p140, e23},
  ]
  (23) Formula los siguientes compuest os con funciones nitrogenadas:
  \begin{enumerate}
    \item Isopropilamina
    \item Pentan-3-amina
    \item Buta-1,3-diamina
    \item 3-etilhexan-3-amina
    \item 3,5-dimetilhexan-1-amina
    \item Pent-3-en-2-amina
    \item N-metilfenilamina
    \item N-ciclopentilbutilamina
    \item Etanamida
    \item N-metiletanamida
    \item 4-fenilpentanamida
    \item N-etilhex-4-enamida
  \end{enumerate}
\end{exercise}

\begin{solution}[print=false]
  Todavía sin solución
\end{solution}




\begin{exercise}[
    tags    = {},
    topics  = {química, química orgánica, orgánica},
    source  = {FQ 1B MGH 2016, p140, e24},
  ]
  (24) Nombra los siguientes compuestos nitrogenados:
  \begin{enumerate}
    \item \chemfig{CH_3-CH(-[6]NH_2)-CH_2-CH_3}
    \item \ch{CH3-CH2-CH2-NH2}
    \item \chemfig{CH_3-CH(-[6]NH_2)-CH_2-CH(-[6]NH_2)-CH_2-CH_2(-[6]NH_2)}
    \item \chemfig{CH_3-CH(-[6]CH_3)-NH-CH=CH_2}
    \item \chemfig{CH_3-NH-*4(----)}
    \item \ch{CH3-CH2-CH2-CH2-CH2-CO-NH2}
    \item \ch{CH3-CH=CH-CH2-CO-NH2}
    \item \ch{CH3-CH2-CHBr-CH2-CH2-CO-NH-CH3}
  \end{enumerate}
\end{exercise}

\begin{solution}[print=false]
  Todavía sin solución
\end{solution}




\begin{exercise}[
    tags    = {},
    topics  = {química, química orgánica, orgánica},
    source  = {FQ 1B MGH 2016, p140, e25},
  ]
  (25) Formula los siguientes compuestos orgánicos:
  \begin{enumerate}
    \item 2,2-dimetilpentano
    \item Hepta-1,5-dieno
    \item 1-fenilpent-2-ino
    \item 3-isopropilciclohexeno
    \item 1-butil-3-metilbenceno
    \item Butano-1,3-diol
    \item Butileteniléter
    \item But-3 enal
    \item Hex-5-in-2-ona
    \item Ácido 3-isopropilhexanoico
    \item Pentanoato de metilo
    \item 5-meilhexan-2,4-diamina
    \item N-metiletilamina
    \item N,N-dietilbutilamina
    \item Hex-3-enamida
    \item N-metilbutanamida
  \end{enumerate}
\end{exercise}

\begin{solution}[print=false]
  Todavía sin solución
\end{solution}




\begin{exercise}[
    tags    = {},
    topics  = {química, química orgánica, orgánica},
    source  = {FQ 1B MGH 2016, p143, e27},
  ]
  (27) Formula y nombra:
  \begin{enumerate}
    \item Dos hidrocarburos alifáticos que presenten isomería de cadena.
    \item Dos aminas con isomería de posición.
    \item Dos compuestos oxigenados con isomería de función.
  \end{enumerate}
\end{exercise}

\begin{solution}[print=false]
  Todavía sin solución
\end{solution}




\begin{exercise}[
    tags    = {},
    topics  = {química, química orgánica, orgánica},
    source  = {FQ 1B MGH 2016, p143, e28},
  ]
  (28) Escribe y nombra:
  \begin{enumerate}
    \item Todos los isómeros de cadena de fórmula \ch{C5H12}.
    \item Cuatro isómeros de función de fórmula \ch{C4H80}.
    \item Tres isómeros de posición de la amina \ch{C5H13N}.
  \end{enumerate}
\end{exercise}

\begin{solution}[print=false]
  Todavía sin solución
\end{solution}




\begin{exercise}[
    tags    = {},
    topics  = {química, química orgánica, orgánica},
    source  = {FQ 1B MGH 2016, p143, e29},
  ]
  (29) Dados los siguientes compuestos, formúlalos y justifica cuáles
  de ellos presentan isomería geométrica y cuáles isomería
  óptica:
  \begin{enumerate}
    \item 2-clorobutano
    \item Pent-3-en-2-ol
    \item Pentan-3-amina
    \item 2-fenilpent-2-eno
  \end{enumerate}
\end{exercise}

\begin{solution}[print=false]
  Todavía sin solución
\end{solution}






\subsection{Problemas propuestos}

\subsubsection{Grupos funcionales y series homólogas}

\begin{exercise}[
    tags    = {},
    topics  = {química, química orgánica, orgánica},
    source  = {FQ 1B MGH 2016, p150, e7},
  ]
  (7) Escribe el número de carbonos y el grupo funcional al que
  corresponden los siguientes compuestos:
  \begin{enumerate}
    \item Octano
    \item Butanamina
    \item Pentinamida
    \item Ácido decanoico
    \item Hexenal
    \item Propanona
    \item Butino
    \item Hepteno
    \item Metanol
    \item Dietiléter
  \end{enumerate}
\end{exercise}

\begin{solution}[print=false]
  Todavía sin solución
\end{solution}




\begin{exercise}[
    tags    = {},
    topics  = {química, química orgánica, orgánica},
    source  = {FQ 1B MGH 2016, p150, e8},
  ]
  (8) Indica si la estructura de cada pareja representa el mismo
  compuesto o compuestos diferentes, identificando los grupos
  funcionales presentes:
  \begin{enumerate}
    \item \ch{CH3CH2OCH3} y \ch{CH3OCH2CH3}
    \item \ch{CH3CH2OCH3} y \ch{CH3CHOHCH3}
    \item \ch{CH3CH2CH2OH} y \ch{CH3CHOHCH3}
  \end{enumerate}
\end{exercise}

\begin{solution}[print=false]
  Todavía sin solución
\end{solution}




\begin{exercise}[
    tags    = {},
    topics  = {química, química orgánica, orgánica},
    source  = {FQ 1B MGH 2016, p150, e9},
  ]
  (9) Contesta a cada uno de los siguientes apartados referidos a
  compuestos de cadena abierta:
  \begin{enumerate}
    \item ¿Qué grupos funcionales pueden tener los compuestos de
    fórmula molecular \ch{C_nH_{2n+2}O}?
    \item ¿Qué compuestos tienen por fórmula molecular \ch{C_nH_{2n-2}}?
  \end{enumerate}
\end{exercise}

\begin{solution}[print=false]
  Todavía sin solución
\end{solution}




\begin{exercise}[
    tags    = {},
    topics  = {química, química orgánica, orgánica},
    source  = {FQ 1B MGH 2016, p150, e10},
  ]
  (10) Nombra y formula los siguientes compuestos orgánicos:
  \begin{enumerate}
    \item \ch{CH3-CH2-COOH}
    \item \ch{CH3-CH2-C+CH}
    \item \ch{CH3-CHOH-CH2-CH2-CH3}
    \item \ch{CH3-CH2-CO-CH2-CH2-CH3}
    \item \ch{C6H14}
    \item Metil etil éte
    \item Metanoato de propil
    \item Dietilamin
    \item Pentana
    \item Metilpropen
  \end{enumerate}
\end{exercise}

\begin{solution}[print=false]
  Todavía sin solución
\end{solution}




\begin{exercise}[
    tags    = {},
    topics  = {química, química orgánica, orgánica},
    source  = {FQ 1B MGH 2016, p150, e13},
  ]
  (13) Formula las siguientes especies químicas:
  \begin{enumerate}
    \item 1-bromo-2,2-diclorobutano
    \item Trimetilamina
    \item 2-metilhex-1,5-dien-3-ino
    \item Butanoato de 2-metilpropilo
    \item Tolueno (metilbenceno)
    \item Propanamida
    \item 2,3-dimetilbut-1-eno
    \item Ácido 2,3-dimetilpentanodioico
  \end{enumerate}
\end{exercise}

\begin{solution}[print=false]
  Todavía sin solución
\end{solution}




\begin{exercise}[
    tags    = {},
    topics  = {química, química orgánica, orgánica},
    source  = {FQ 1B MGH 2016, p151, e14},
  ]
  (14) Nombra las siguientes especies químicas:
  \begin{enumerate}
    \item \ch{H2C=CH-CH=CH-CHO}
    \item \ch{H3C-CO-CO-CH3}
    \item \ch{H2C=CH-CH=CH-CH2-COOH}
    \item \ch{H3C-CH2-NH-CH2-CH3}
    \item \ch{CH+C-CH2-COOH}
    \item \ch{CH3-CH2-CH(CH3)-CONH2}
    \item \ch{H3C-C(OH)2-CH2-CH2OH}
  \end{enumerate}
\end{exercise}

\begin{solution}[print=false]
  Todavía sin solución
\end{solution}




\begin{exercise}[
    tags    = {},
    topics  = {química, química orgánica, orgánica},
    source  = {FQ 1B MGH 2016, p151, e15},
  ]
  (15) Nombra y/o formula los siguientes compuestos:
  \begin{enumerate}
    \item \ch{CHCl3}
    \item \ch{CH3-CH2-CHO}
    \item \ch{CH3-CH2-CH2-CH2-CO-NH2}
    \item \ch{(CH3)2-CHOH}
    \item 2,2-dimetilbutano
    \item Para-diaminobenceno
    \item Ciclohexano
    \item Etil propil éter
  \end{enumerate}
\end{exercise}

\begin{solution}[print=false]
  Todavía sin solución
\end{solution}




\begin{exercise}[
    tags    = {},
    topics  = {química, química orgánica, orgánica},
    source  = {FQ 1B MGH 2016, p151, e16},
  ]
  (16) Formula o nombra, según corresponda:
  \begin{enumerate}
    \item 1-etil-3-metilbenceno
    \item 2-metilpropan-2-ol
    \item 2-metil-propanoato de etilo
    \item Pent-3-en-1-amina
    \item \ch{ClCH=CH-CH3}
    \item \ch{CH3-CH2-O-CH2-CH3}
    \item \ch{CH3-CH(CH3)-CO-CH2-CH(CH3)-CH3}
    \item \ch{CH2=CH-CH2-CO-NH-CH3}
  \end{enumerate}
\end{exercise}

\begin{solution}[print=false]
  Todavía sin solución
\end{solution}




\begin{exercise}[
    tags    = {},
    topics  = {química, química orgánica, orgánica},
    source  = {FQ 1B MGH 2016, p151, e20},
  ]
  (20) Formula o nombra los siguientes compuestos:
  \begin{enumerate}
    \item Cromato de cobre(II)
    \item Hidruro de magnesio
    \item Hidrogenosulfuro de bario
    \item Etanamina
    \item Propan-1,2-diol
    \item \ch{Fe(OH)2}
    \item \ch{H2SO3}
    \item \ch{N2O5}
    \item \chemfig{**6(---(-CH=[2]O)---)}
    \item \chemfig{CH_3-CH(-[6]CH_3)-CH(-[6]CH_3)-CH(-[6]CH_3)-CH_2-CH_3}
  \end{enumerate}
\end{exercise}

\begin{solution}[print=false]
  Todavía sin solución
\end{solution}




\begin{exercise}[
    tags    = {},
    topics  = {química, química orgánica, orgánica},
    source  = {FQ 1B MGH 2016, p151, e21},
  ]
  (21) Formula o nombra los siguientes compuestos orgánicos:
  \begin{enumerate}
    \item 3-etil-2-metilhexano
    \item 1-bromopent-2-ino:
    \item 3-etilhe xano-1,5-diol:
    \item 3-metilpentan-2,4-diamina
    \item \ch{CH2=CH-CH2-CO-O-CH3}
    \item \ch{C6H5-O-C6H5}
    \item \ch{CH3-CH2-CO-NH-CH2-CH3}
    \item \ch{COOH-CH2-CH2-CHBr-COOH}
  \end{enumerate}
\end{exercise}

\begin{solution}[print=false]
  Todavía sin solución
\end{solution}






\subsubsection{Isomería estructural y espacial}

\begin{exercise}[
    tags    = {},
    topics  = {química, química orgánica, orgánica},
    source  = {FQ 1B MGH 2016, p152, e23},
  ]
  (23) Formula los siguientes compuestos orgánicos:
  \begin{enumerate}
    \item But-3-en-2-ona
    \item Buta-1,3-dien-2-ol
    \item Dietiléter
  \end{enumerate}
  ¿Cuáles de ellos son isómeros entre sí?
\end{exercise}

\begin{solution}[print=false]
  Todavía sin solución
\end{solution}




\begin{exercise}[
    tags    = {},
    topics  = {química, química orgánica, orgánica},
    source  = {FQ 1B MGH 2016, p152, e24},
  ]
  (24) Escribe y nombra cinco isómeros de cadena de fórmula molecular \ch{C6H14}.
\end{exercise}

\begin{solution}[print=false]
  Todavía sin solución
\end{solution}




\begin{exercise}[
    tags    = {},
    topics  = {química, química orgánica, orgánica},
    source  = {FQ 1B MGH 2016, p152, e25},
  ]
  (25) Escribe y nombra cuatro isómeros de función de fórmula molecular \ch{C4H8O}.
\end{exercise}

\begin{solution}[print=false]
  Todavía sin solución
\end{solution}




\begin{exercise}[
    tags    = {},
    topics  = {química, química orgánica, orgánica},
    source  = {FQ 1B MGH 2016, p152, e28},
  ]
  (28) Escribe y nombra todos los isómeros estructurales de fórmula C5H10
\end{exercise}

\begin{solution}[print=false]
  Todavía sin solución
\end{solution}




\begin{exercise}[
    tags    = {},
    topics  = {química, química orgánica, orgánica},
    source  = {FQ 1B MGH 2016, p152, e30},
  ]
  (30) Formula y nombra:
  \begin{enumerate}
    \item Dos isómeros de posición de fórmula \ch{C3H8O}
    \item Dos isómeros de función de fórmula \ch{C3H8O}
    \item Dos isómeros geométricos de fórmula \ch{C4H8}
    \item Un compuesto que tenga dos carbonos quirales (asimétricos) de fórmula \ch{C4H8BrCl}
  \end{enumerate}
\end{exercise}

\begin{solution}[print=false]
  Todavía sin solución
\end{solution}




\begin{exercise}[
    tags    = {},
    topics  = {química, química orgánica, orgánica},
    source  = {FQ 1B MGH 2016, p152, e31},
  ]
  (31) Un derivado halogenado etilénico que presenta isomería cis-trans está formado en un 22,4\% de C, un 2,8\% de H y un 74,8\% de bromo. Además, a \SI{130}{\celsius} y \SI{1}{atm} de presión, una muestra de \SI{12,9}{\gram} ocupa un volumen de \SI{2}{\liter}. Halla su fórmula molecular y escribe los posibles isómeros.
\end{exercise}

\begin{solution}
  \ch{C4H6Br2}
\end{solution}




\begin{exercise}[
    tags    = {},
    topics  = {química, química orgánica, orgánica},
    source  = {FQ 1B MGH 2016, p152, e32},
  ]
  (32) Un alcohol monoclorado está formado en un 38,1\% de C,
  un 7,4\% de H, un 37,6\% de Cl y el resto es oxígeno. Escribe
  su fórmula semidesarrollada sabiendo que tiene un carbono
  asimétrico y que su fórmula molecular y su fórmula empírica
  coinciden.
\end{exercise}

\begin{solution}
  \ch{C3H7OCl}
\end{solution}




\begin{exercise}[
    tags    = {},
    topics  = {química, química orgánica, orgánica},
    source  = {FQ 1B MGH 2016, p152, e33},
  ]
  (33) Un hidrocarburo monoinsaturado tiene un 87,8\% de carbono.
  Si su densidad en condiciones normales es \SI{3,66}{\gram\per\liter}, determina sus fórmulas empírica y molecular.
\end{exercise}

\begin{solution}
  Formula empírica: \ch{C3H5}; Fórmula molecular: \ch{C6H10}.
\end{solution}

\end{multicols}






\section{BLOQUE 4. Transformaciones energéticas y espontaneidad (tema 6 del libro)}

\begin{multicols}{2}


\begin{exercise}[
    tags    = {},
    topics  = {química, termodinámica, termoquímica},
    source  = {FQ 1B MGH 2016, p159, e5},
  ]
  (5) Determina la variación de energía interna que sufre un sistema
  cuando:
  \begin{enumerate}
    \item Realiza un trabajo de \SI{600}{\joule} y cede \SI{40}{cal} al entorno.
    \item Absorbe \SI{300}{cal} del entorno y se realiza un trabajo de compresión de \SI{5}{\kilo\joule}.
  \end{enumerate}
\end{exercise}

\begin{solution}
  \begin{enumerate*}
    \item \( \Delta U = \SI{-767}{\joule} \); \item \( \Delta U = \SI{6.25e3}{\joule} \)
  \end{enumerate*}
\end{solution}




\begin{exercise}[
    tags    = {},
    topics  = {química, termodinámica, termoquímica},
    source  = {FQ 1B MGH 2016, p165, e13},
  ]
  (13) La descomposición térmica del clorato de potasio (\ch{KClO3})
  origina cloruro de potasio (\ch{KCl}) y oxígeno molecular. Calcula
  el calor que se desprende cuando se obtienen \SI{150}{\liter} de
  oxígeno medidos a \SI{25}{\celsius} y \SI{1}{atm} de presión.

  \begin{gexdatos}
    \( \Delta H^0_f (\si{kJ/mol}) \): \( \ch{KClO3_{(s)}} = -91,2 \); \( \ch{KCl_{(s)}} = -436 \)
  \end{gexdatos}
\end{exercise}

\begin{solution}
  Se desprenden \SI{1.41e3}{kJ}.
\end{solution}




\begin{exercise}[
    tags    = {},
    topics  = {química, termodinámica, termoquímica},
    source  = {FQ 1B MGH 2016, p165, e6},
  ]
  (14) Las entalpías estándar de formación del propano (g),
  dióxido de carbono (g) y agua (l), son respectivamente:
  \SIlist{-103,8;-393,5;-285,8}{kJ/mol}. Calcula:
  \begin{enumerate}
    \item La entalpía de la reacción de combustión del propano.
    \item Las calorías generadas en la combustión de una bombona de propano de \SI{1.80}{\liter} a \SI{25}{\celsius} y \SI{4}{atm} de presión.
  \end{enumerate}
\end{exercise}

\begin{solution}
  \begin{enumerate*}
    \item \( \Delta H^0_c = \SI{-2.22e3}{kJ/mol} \) de propano;
    \item Se desprenden \SI{156}{kcal}
  \end{enumerate*}
\end{solution}




\begin{exercise}[
    tags    = {},
    topics  = {química, termodinámica, termoquímica},
    source  = {FQ 1B MGH 2016, p165, e15},
  ]
  (15) En la reacción del oxígeno molecular con el cobre para formar
  óxido de cobre(II) se desprenden \SI{2.30}{kJ} por cada gramo de
  cobre que reacciona, a \SI{298}{\kelvin} y \SI{760}{\mmHg}. Calcula:
  \begin{enumerate}
    \item La entalpía de formación del óxido de cobre(II).
    \item El calor desprendido a presión constante cuando reaccionan \SI{100}{\liter} de oxígeno, medidos a \SI{1.5}{atm} y \SI{27}{\celsius}
  \end{enumerate}
\end{exercise}

\begin{solution}
  \begin{enumerate*}
    \item \( \Delta H^0_f (\ch{CuO}) = \SI{-146}{kJ/mol} \);
    \item Se desprenden \SI{1.78e3}{kJ}.
  \end{enumerate*}
\end{solution}




\begin{exercise}[
    tags    = {},
    topics  = {química, termodinámica, termoquímica},
    source  = {FQ 1B MGH 2016, p165, e16},
  ]
  (16) En la combustión completa de \SI{1.00}{\gram} de etanol (\ch{CH3-CH2OH}) se desprenden \SI{29.8}{kJ} y en la combustión de \SI{1.00}{\gram}
  de ácido etanoico (\ch{CH3-COOH}) se desprenden \SI{14.5}{kJ}.
  Determina numéricamente:
  \begin{enumerate}
    \item Cuál de las dos sustancias tiene mayor entalpía de combustión.
    \item Cuál de las dos sustancias tiene mayor entalpía de formación.
  \end{enumerate}
\end{exercise}

\begin{solution}
  \begin{enumerate}
    \item \( \Delta H^0_c (\textit{etanol}) = \SI{-1.37e3}{kJ/mol} \);
          \( \Delta H^0_c (\textit{ácido etanoico}) = \SI{-870}{kJ/mol} \)
    \item \( \Delta H^0_f (\textit{etanol}) = \SI{-273}{kJ/mol} \); \newline
          \( \Delta H^0_f (\textit{ácido etanoico}) = \SI{-489}{kJ/mol} \)
  \end{enumerate}
\end{solution}




\begin{exercise}[
    tags    = {},
    topics  = {química, termodinámica, termoquímica},
    source  = {FQ 1B MGH 2016, p168, e24},
  ]
  (24) Calcula la entalpía de la reacción:
  \ch{CH4_{(g)} + Cl2_{(g)} -> CH3Cl_{(g)} + HCl_{(g)}} a partir de:
  \begin{enumerate}
    \item Las energías de enlace.
    \item Las entalpías de formación.
  \end{enumerate}

  \begin{gexdatos}
    \begin{itemize}
      \item \textit{Energías de enlace (\si{kJ/mol})}:
      \( \ch{C-H} = 414 \); \( \ch{Cl-Cl} = 244 \); \( \ch{C-Cl} = 330 \); \( \ch{H-Cl} = 430 \).
      \item \textit{Entalpías de formación (\si{kJ/mol})}:
      \( (\ch{CH4}) = -74,9 \); \( (\ch{CH3Cl}) = -82,0 \); \( \Delta H^0_f (\ch{HCl}) = -92,3 \).
    \end{itemize}
  \end{gexdatos}

\end{exercise}

\begin{solution}
  \begin{enumerate*}
    \item \( \Delta H^0_R = \SI{-102}{kJ} \); \item \( \Delta H^0_R = \SI{-99.4}{kJ} \)
  \end{enumerate*}
\end{solution}




\begin{exercise}[
    tags    = {},
    topics  = {química, termodinámica, termoquímica},
    source  = {FQ 1B MGH 2016, p168, e25},
  ]
  (25) El eteno se hidrogena para dar etano, según:
  \ch{CH2=CH2_{(g)} + H2_{(g)} -> CH3-CH3_{(g)}} \( \Delta H^0_R = \SI{-130}{kJ} \)
  Calcula la energía del enlace \ch{C=C}, si las energías de los
  enlaces \ch{C-C}, \ch{H-H} y \ch{C-H} son, respectivamente, \SIlist{347;436;414}{kJ/mol}.
\end{exercise}

\begin{solution}
  \( \Delta H^0 (\ch{C=C}) = \SI{609}{kJ/mol} \)
\end{solution}





\begin{exercise}[
    tags    = {},
    topics  = {química, termodinámica, termoquímica},
    source  = {FQ 1B MGH 2016, p168, e26},
  ]
  (26) A partir de los siguientes datos:
  \begin{itemize}
    \item Entalpía estándar de sublimación del \( \ch{C_{(s)}} = \SI{717}{kJ/mol} \).
    \item Entalpía de formación del \( \ch{CH3-CH3_{(g)}} = \SI{-85,0}{kJ/mol} \).
    \item Entalpía media del enlace \( \ch{H-H} = \SI{436}{kJ/mol} \).
    \item Entalpía media del enlace \( \ch{C-C} = \SI{347}{kJ/mol} \).
  \end{itemize}
  Responde a las siguientes cuestiones: % REVIEW comportamiento del espaciado
  \begin{enumerate}
    \item Calcula la variación de entalpía de la reacción:
    \ch{2 C_{(g)} + 3 H2_{(s)} -> CH3-CH3_{(g)}} e indica si es exotérmica
    o endotérmica.
    \item Determina el valor medio del enlace \ch{C-H}.
  \end{enumerate}
\end{exercise}

\begin{solution}
  \begin{enumerate*}
    \item \( \Delta H^0_R = \SI{-1.52e3}{kJ} \); \item \( \Delta H^0 (\ch{C-H}) = \SI{413}{kJ/mol} \)
  \end{enumerate*}
\end{solution}






\subsection{Problemas resueltos}

\subsubsection{Entalpias de formación, de reacción y de combustión}

\begin{exercise}[
    tags    = {},
    topics  = {química, termodinámica, termoquímica},
    source  = {FQ 1B MGH 2016, p179, e4},
  ]
  (4) El sulfuro de carbono reacciona con el oxígeno según:

  \ch{CS2_{(l)} + 3 O2_{(g)} -> CO2_{(g)} + 2 SO2_{(g)}} \( \Delta H_R = \SI{-1072}{kJ} \)

  \begin{enumerate}
    \item Calcula la entalpía de formación del \ch{CS2}
    \item Halla el volumen de \ch{SO2} emitido a la atmósfera, a \SI{1}{atm} y \SI{25}{\celsius}, cuando se ha liberado una energía de \SI{6000}{kJ}
  \end{enumerate}

  \begin{gexdatos}
    \( \Delta H^0_f (\si{kJ/mol}) \): \( \ch{CO2_{(g)}} = -393,5 \); \( \ch{SO2_{(g)}} = -296,4 \).
  \end{gexdatos}

\end{exercise}

\begin{solution}
  \begin{enumerate*}
    \item \( \Delta H^0_f \ch{CS2_{(l)}} = 85,7 kJ/mol \); \item \( V = \SI{274}{\liter} (\ch{SO2}) \)
  \end{enumerate*}
\end{solution}




\begin{exercise}[
    tags    = {},
    topics  = {química, termodinámica, termoquímica},
    source  = {FQ 1B MGH 2016, p179, e5},
  ]
  (5) El dióxido de manganeso se reduce a manganeso metal reaccionando
  con el aluminio según:

  \ch{MnO2_{(s)} + Al_{(s)} -> Al2O3_{(s)} + Mn_{(s)}}

  \begin{enumerate}
    \item Halla la entalpía de esa reacción sabiendo que las entalpías
    de formación valen:
    \( \Delta H^0_f (\ch{Al2O3}) = \SI{-1676}{kJ/mol} \); \( \Delta H^0_f (\ch{MnO2}) = \SI{-520}{kJ/mol} \)
    \item ¿Qué energía se transfiere cuando reaccionan \SI{10.0}{\gram}
    \ch{MnO2} con \SI{10.0}{\gram} de \ch{Al}?
  \end{enumerate}
\end{exercise}

\begin{solution}
  \begin{enumerate*}
    \item \( \Delta H_R = \SI{-896}{kJ/mol}-896 \) de \ch{Al2O3};
    \item \SI{68.7}{kJ} se desprenden
  \end{enumerate*}
\end{solution}




\begin{exercise}[
    tags    = {},
    topics  = {química, termodinámica, termoquímica},
    source  = {FQ 1B MGH 2016, p179, e6},
  ]
  (6) Durante la fotosíntesis, las plantas verdes sintetizan la glucosa
  según la siguiente reacción:

  \( \ch{6 CO2_{(g)} + 6 H2O_{(l)} -> C6H12O6_{(s)} + 6 O2_{(g)}} \Delta H_R = \SI{2815}{kJ/mol} \)

  \begin{enumerate}
    \item ¿Cuál es la entalpía de formación de la glucosa?
    \item ¿Qué energía se requiere para obtener \SI{50.0}{\gram} de glucosa?
    \item ¿Cuántos litros de oxígeno, en condiciones estándar, se desprenden por cada gramo de glucosa formado?
  \end{enumerate}

  \begin{gexdatos}
    \( \Delta H^0_f (\si{kJ/mol}) \): \( \ch{H2O_{(l)}} = -285,8 \); \( \ch    {CO2_{(g)}} = -393,5 \)
  \end{gexdatos}
\end{exercise}

\begin{solution}
  \begin{enumerate*}
    \item \( \Delta H^0_f = -1,26·10^3 kJ/mol \);
    \item \SI{782}{kJ};
    \item \SI{0.8}{\liter}
  \end{enumerate*}
\end{solution}




\begin{exercise}[
    tags    = {},
    topics  = {química, termodinámica, termoquímica},
    source  = {FQ 1B MGH 2016, p179, e7},
  ]
  (7) Las entalpías de combustión del etano y del eteno son
  \SIlist{-1560;1410}{kJ/mol}, respectivamente. Determina:
  \begin{enumerate}
    \item El valor de \( \Delta H^0_f \) para el etano y el eteno.
    \item Razona si el proceso de hidrogenación del eteno a etano
    es un proceso endotérmico o exotérmico.
    \item Calcula el calor que se desprende en la combustión de
    \SI{50.0}{\gram} de cada gas.
  \end{enumerate}

  \begin{gexdatos}
    \( \Delta H^0_f (\si{kJ/mol}) \): \( \ch{CO2_{(g)}} = -393,5 \); \( \ch{H2O_{(l)}} = -285,9 \)
  \end{gexdatos}
\end{exercise}

\begin{solution}
  \begin{enumerate*}
    \item \( \Delta H^0_f (\ch{C2H6})= \SI{-84,7}{kJ/mol} \); \( \Delta H^0_f (\ch{C2H4}) = \SI{51.2}{kJ/mol} \); \item Exotérmico; \item etano: \SI{2.60e3}{kJ}; eteno: \SI{2.52e3}{kJ}.
  \end{enumerate*}
\end{solution}




\begin{exercise}[
    tags    = {},
    topics  = {química, termodinámica, termoquímica},
    source  = {FQ 1B MGH 2016, p179, e8},
  ]
  (8) La gasolina es una mezcla compleja de hidrocarburos que
  vamos a considerar como si estuviera formada únicamente
  por hidrocarburos saturados de fórmula (\ch{C8H18})
  \begin{enumerate}
    \item Calcula el calor que se desprende en la combustión de
    \SI{50.0}{\liter} litros de gasolina (\( d = 0,78 g/mL \)).
    \item Halla la masa de \ch{CO2} que se emite a la atmósfera en esa
    combustión.
    \item Si el consumo de un vehículo es de \SI{7.00}{\liter} por cada
    \SI{100}{km}, ¿qué energía necesita por cada \si{km} recorrido?
  \end{enumerate}

  \begin{gexdatos}
    \( \Delta H^0_f (\si{kJ/mol}) \): \( \ch{CO2_{(g)}} = -394 \); \( \ch{H2O_{(l)}} = -286 \); \( \ch{C8H18_{(l)}} = -250 \)
  \end{gexdatos}
\end{exercise}

\begin{solution}
  \begin{enumerate*}
    \item \SI{1.87e6}{kJ}; \item \SI{120}{\kilo\gram}; \item \SI{2.62e3}{kJ/km}
  \end{enumerate*}
\end{solution}




\begin{exercise}[
    tags    = {},
    topics  = {química, termodinámica, termoquímica},
    source  = {FQ 1B MGH 2016, p179, e10},
  ]
  (10) Se quema benceno (\ch{C6H6}) en exceso de oxígeno, liberando energía.

  \begin{enumerate}
    \item Formula la reacción de combustión del benceno.
    \item Calcula la entalpía de combustión estándar de un mol de
    benceno líquido.
    \item Calcula el volumen de oxígeno, medido a \SI{25}{\celsius} y \SI{5}{atm}, necesario para quemar \SI{1}{\liter} de benceno líquido.
    \item Calcula el calor necesario para evaporar \SI{10}{\liter} de benceno líquido.
  \end{enumerate}

  \begin{gexdatos}
    \( \Delta H^0_f (\si{kJ/mol}) \):
    \( \ch{C6H6_{(l)}}  = +49 \);
    \( \ch{C6H6_{(v)}}  = +83 \);
    \( \ch{H2O_{(l)}}   = -286 \);
    \( \ch{CO2_{(g)}}   = -393 \)
    Densidad \( \textrm{benceno}_{(l)} = \SI{0.879}{\gram\per\cubic\centi\meter} \)
  \end{gexdatos}
\end{exercise}

\begin{solution}
  \begin{enumerate*}
    \item \( \Delta H^0_C = \SI{3.27e3}{kJ/mol} \); \item \SI{413}{\liter}; \item \SI{3.83e3}{kJ}.
  \end{enumerate*}
\end{solution}





\subsubsection{Ley de Hess}

\begin{exercise}[
    tags    = {},
    topics  = {química, termodinámica, termoquímica},
    source  = {FQ 1B MGH 2016, p180, e11},
  ]
  (11) El motor de una máquina cortacésped funciona con una gasolina que podemos considerar de composición única octano (\ch{C8H18}). Calcula:

  \begin{enumerate}
    \item La entalpía estándar de combustión del octano, aplicando la ley de Hess.
    \item El calor que se desprende en la combustión de \SI{2.00}{\kilo\gram} de octano.
  \end{enumerate}

  \begin{gexdatos}
    \( \Delta H^0_f (\si{kJ/mol}) \):
    \( \ch{CO2_{(g)}}   = -393,8 \)
    \( \ch{C8H18_{(l)}} = -264,0 \)
    \( \ch{H2O_{(l)}}   = -285,8 \)
  \end{gexdatos}
\end{exercise}

\begin{solution}
  \begin{enumerate*}
    \item \( \Delta H^0_C = \SI{-5.46e3}{kJ/mol} \); \item \SI{9.58e4}{kJ}
  \end{enumerate*}
\end{solution}




\begin{exercise}[
    tags    = {},
    topics  = {química, termodinámica, termoquímica},
    source  = {FQ 1B MGH 2016, p180, e12},
  ]
  (12) Sabiendo que las entalpías estándar de combustión del
  hexano (l), del carbono (s) y del hidrógeno (g) son respectivamente:
  \SIlist{-4192;-393,5;-285,8}{kJ/mol}, halla:

  \begin{enumerate}
    \item La entalpía de formación del hexano líquido en esas condiciones.
    \item Los gramos de carbono consumidos en la formación del hexano cuando se han intercambiado \SI{50.0}{kJ}.
  \end{enumerate}
\end{exercise}

\begin{solution}
  \begin{enumerate*}
    \item \( \Delta H^0_f (\textit{hexano}) = \SI{-170}{kJ/mol} \) \item \( m = \SI{21.2}{\gram} \) de C.
  \end{enumerate*}
\end{solution}




\begin{exercise}[
    tags    = {},
    topics  = {química, termodinámica, termoquímica},
    source  = {FQ 1B MGH 2016, p180, e14},
  ]
  (14) El calor desprendido en el proceso de obtención del benceno a partir de etino es:

  \ch{3 C2H2_{(g)} -> C6H6_{(l)}} \( \Delta H^0_R = \SI{-631}{kJ} \)

  \begin{enumerate}
    \item Calcula la entalpía estándar de combustión del benceno, sabiendo que la del etino es \SI{-1302}{kJ/mol}.
    \item ¿Qué volumen de etino, medido a \SI{26}{\celsius} y \SI{15}{atm}, se necesita para obtener \SI{0.25}{\liter} de benceno?
  \end{enumerate}

  \begin{gexdatos}
    \( \textrm(densidad del benceno) = \SI{880}{\gram\per\liter} \)
  \end{gexdatos}
\end{exercise}

\begin{solution}
  \begin{enumerate}
    \item \( \Delta H^0_C (\ch{C6H6_{(l)}}) = \SI{-3.28e3}{kJ/mol} \); \item \SI{13.8}{\liter} de \ch{C2H2}
  \end{enumerate}
\end{solution}






\subsubsection{Entalpías de enlace}

\begin{exercise}[
    tags    = {},
    topics  = {química, termodinámica, termoquímica},
    source  = {FQ 1B MGH 2016, p180, e18},
  ]
  (18) Calcula la variación de entalpía estándar de la hidrogenación del etino a etano:

  \begin{enumerate}
    \item A partir de las energías de enlace.
    \item A partir de las entalpías de formación.
  \end{enumerate}

  \begin{gexdatos}
    Energías de enlace (\si{kJ/mol}): \( \ch{C-H} = 415 \); \( \ch{H-H} = 436 \); \( \ch{C-C} = 350 \): \( \ch{C~C} = 825 \)

    \( \Delta H^0_f (\si{kJ/mol}) \): \( \textrm{etino} = 227 \); \( \textrm{etano} = -85,0 \)
  \end{gexdatos}
\end{exercise}

\begin{solution}
  \begin{enumerate*}
    \item \( \Delta H_R = \SI{-313}{kJ/mol} \);
    \item \( \Delta H_R = \SI{-312}{kJ/mol} \)
  \end{enumerate*}
\end{solution}






\subsubsection{Entropía y espontaneidad}

\begin{exercise}[
    tags    = {},
    topics  = {química, termodinámica, termoquímica},
    source  = {FQ 1B MGH 2016, p181, e20},
  ]
  (20) Dadas las siguientes ecuaciones termoquímicas:

  \( \ch{2 H2O2_{(l)} -> 2 H2O_{(l)} + O2_{(g)}} \Delta H = \SI{-196}{kJ} \)

  \( \ch{N2_{(g)} + 3 H2_{(g)} -> 2 NH3 _{(g)}} \Delta H = \SI{-92.4}{kJ} \)

  \begin{enumerate}
    \item Define el concepto de entropía y explica el signo más probable de \( \Delta S \) en cada una de ellas.
    \item Explica si esos procesos serán o no espontáneos a cualquier
    temperatura, a temperaturas altas, a temperaturas bajas, o no serán nunca espontáneos.
  \end{enumerate}
\end{exercise}

\begin{solution}[print=false]
  TODAVÍA SIN SOLUCIÓN
\end{solution}




\begin{exercise}[
    tags    = {},
    topics  = {química, termodinámica, termoquímica},
    source  = {FQ 1B MGH 2016, p181, e21},
  ]
  (21) Dada la reacción: \ch{N2O_{(g)} --> N2_{(g)} + 1/2 O2_{(g)}}
  siendo \( \Delta H^0 = \SI{43.0}{kJ/mol} \) y \( \Delta S^0 = \SI{80.0}{\joule\per\mole\kelvin} \) % REVIEW unidades de entropía

  \begin{enumerate}
    \item Justifica el signo positivo de la variación de entropía.
    \item ¿Será espontánea a \SI{25}{\celsius}? ¿A qué temperatura estará en
    equilibrio?
  \end{enumerate}
\end{exercise}

\begin{solution}
  \begin{enumerate*}
    \item Aumentan los moles de sustancias gaseosas; \item \( \Delta G^0_R = \SI{19.5}{kJ} \) (no es espontánea); \( T = \SI{538}{\kelvin} \)
  \end{enumerate*}
\end{solution}




\begin{exercise}[
    tags    = {},
    topics  = {química, termodinámica, termoquímica},
    source  = {FQ 1B MGH 2016, p181, e23},
  ]
  (23) Se pretende obtener etileno (eteno) a partir de grafito e
  hidrógeno, a \SI{25}{\celsius} y \SI{1}{atm}, según la reacción:

  \ch{2 C_{(s)} + 2 H2_{(g)} --> C2H4_{(g)}}

  Calcula:

  \begin{enumerate}
    \item La entalpía de reacción en condiciones estándar. ¿La reacción es endotérmica o exotérmica?
    \item La variación de energía libre de Gibbs en condiciones
    estándar. ¿Es espontánea la reacción en esas condiciones?
  \end{enumerate}

  \begin{gexdatos}
    \( S^0 (J mol-1 K-1) \):
    \( \ch{C_{(s)}} = 5,70 \);
    \( \ch{H2_{(g)}} = 130,6 \);
    \( \ch{C2H4_{(g)}}= 219,2 \)

    \( \Delta H^0_f (\si{kJ/mol}) \):
    \( \ch{C2H4_{(g)}} = +52,5 \)
  \end{gexdatos}
\end{exercise}

\begin{solution}
  \begin{enumerate*}
    \item \( \Delta H^0_R = \SI{+52.5}{kJ} \) (endotérmica)
    \item \( \Delta G^0_R = \SI{+68.4}{kJ} \) (no es espontánea)
  \end{enumerate*}
\end{solution}






\subsubsection{Aplica lo aprendido}

\begin{exercise}[
    tags    = {},
    topics  = {química, termodinámica, termoquímica},
    source  = {FQ 1B MGH 2016, p182, e35},
  ]
  (35) El acetileno o etino (\ch{C2H2}) se hidrogena para producir etano. Calcula a \SI{298}{\kelvin}:

  \begin{enumerate}
    \item La entalpía estándar de reacción.
    \item La energía de Gibbs estándar de reacción.
    \item La entropía estándar de reacción.
    \item La entropía molar del hidrógeno.
  \end{enumerate}

  \begin{tabular}{rccc}
    Comp. & \( \Delta H^0_f \) & \( \Delta G^0_f \) & \( S^0 \) \\
       & (\si{\kilo\joule\per\mole}) & (\si{\kilo\joule\per\mole}) & (\si{\kilo\joule\per\mole\kelvin}) \\
    \toprule
    \ch{C2H2} & 227 & 209 & 200 \\
    \ch{C2H6} & -85 & -33 & 230 \\
    \bottomrule
  \end{tabular}
\end{exercise}

\begin{solution}
  \begin{enumerate*}
    \item \( \Delta H^0_R = \SI{-312}{kJ} \);
    \item \( \Delta G^0_R = \SI{-242}{kJ} \);
    \item \( \Delta S^0_R = \SI{-235}{J/K} \);
    \item \( S^0_{\ch{H2}}  = \SI{132}{J/mol K} \)
  \end{enumerate*}
\end{solution}

\end{multicols}

\end{document}
