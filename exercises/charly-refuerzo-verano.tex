\documentclass[10pt]{article}

\usepackage{polyglossia}
    \setdefaultlanguage{spanish}

\usepackage{fontspec}
    \setmainfont{Fira Sans}

\usepackage{amsmath, amsthm, amssymb}
\usepackage{unicode-math}
  \unimathsetup{
    math-style  = ISO,
    bold-style  = ISO
  }
  \setmathfont{Fira Math}


\usepackage{multicol}
  \setlength{\columnsep}{2cm}

\usepackage[top=2cm, bottom=3cm, left=3cm, right=2.5cm]{geometry}


\usepackage{xsim}
% Configuración general de xsim
  \loadxsimstyle{layouts}
  \xsimsetup{
    solution/print    = {true},
    path              = {xsim-files},
    exercise/template = {margin},
    exercise/name     = {E},
    solution/template = {margin},
    solution/name     = {s}
  }

\usepackage{siunitx}
% Exponent symbol options: \times for the typical cross
  \sisetup{
    per-mode                = symbol,
    output-decimal-marker   = {,},
    exponent-product        = \cdot,
    text-celsius            = ^^b0\kern -\scriptspace C,  % soluciona problemas con el símbolo de grados
    math-celsius            = ^^b0\kern -\scriptspace C
  }

\usepackage[inline]{enumitem}
% Configuración general de enumitem
% Establece la configuración por defecto en a), b), c)
  \setlist[enumerate,1]{label=\alph*)}

\usepackage{chemformula}
\usepackage{chemfig}
  \setchemfig{atom sep=2em}

\begin{document}

\section{BLOQUE 2. ASPECTOS CUALITATIVOS DE LA QUÍMICA (tema 3 del libro)}

\begin{multicols}{2}

\begin{exercise}
4> Sabemos que \SI{40}{uma} es la masa del átomo de calcio. Calcula:
  \begin{enumerate}
    \item La masa en gramos de \SI{1}{átomo} de Ca.
    \item ¿Cuál de las siguientes cantidades tienen mayor número de átomos? \SI{40}{g} de Ca; \SI{0,20}{moles} de Ca; \SI{5e23}{átomos} de Ca.
  \end{enumerate}
\end{exercise}

\begin{solution}
  \begin{enumerate*}
    \item $m = \SI{6,6E-23}{\gram}$;
    \item \SI{40}{\gram} de Ca.
  \end{enumerate*}
\end{solution}

\begin{exercise}
  5> Si tenemos en cuenta que \SI{56}{uma} es la masa del átomo de hierro, calcula:
  \begin{enumerate}
    \item La masa atómica en gramos de \SI{1}{átomo} de Fe.
    \item Cuál de las siguientes cantidades tiene mayor número de átomos de Fe: \SI{56}{\gram}, \SI{0,20}{moles} o \SI{5e23}{átomos}.
  \end{enumerate}
\end{exercise}

\begin{solution}
  \begin{enumerate*}
    \item $m = \SI{9,3E-23}{\gram}$;
    \item \SI{65}{\gram} de Fe.
  \end{enumerate*}
\end{solution}

\begin{exercise}
  6> Responde a las siguientes cuestiones:
  \begin{enumerate}
    \item ¿En cuál de las siguientes cantidades de los elementos que se enumeran a continuación existe un mayor número de moles: \SI{100}{\gram} de hierro, \SI{100}{\gram} de oxígeno molecular, \SI{100}{\gram} de cinc o \SI{100}{\gram} de níquel?
    \item ¿Y un mayor número de átomos?
  \end{enumerate}
\end{exercise}

\begin{solution}
  \begin{enumerate*}
    \item En los \SI{100}{gramos} de oxígeno molecular;
    \item En los \SI{100}{gramos} de oxígeno molecular.
  \end{enumerate*}
\end{solution}

\begin{exercise}
  10> Sabiendo que un gas a \SI{1,5}{atm} y \SI{290}{\kelvin} tiene una densidad de \SI{1,178}{\gram\per\liter}, calcula su masa molecular.
\end{exercise}
\begin{solution}
  $M = \SI{18,7}{\gram\per\mole}$
\end{solution}

\begin{exercise}
  11> Calcula la densidad del metano (\ch{CH4}) a \SI{700}{\mmHg} y \SI{75}{\celsius}.
\end{exercise}
\begin{solution}
  $d = \SI{0,52}{\gram\per\liter}$
\end{solution}

\begin{exercise}
  12> Calcula el número de moléculas de \ch{CO2} que habrá en \SI{10}{\liter} del mismo gas medidos en condiciones normales.
\end{exercise}

\begin{solution}
  $M = \SI{2,7e23}{moléculas}$.
\end{solution}

\begin{exercise}
  13> Calcula la masa en gramos de un mol de \ch{SO2} sabiendo que
  exactamente \SI{5}{\cubic\centi\meter} de dicho gas, medidos en condiciones
  normales, tienen una masa de \SI{0,01428}{\gram}.
\end{exercise}
\begin{solution}
  $M = \SI{64}{\gram\per\mole}$
\end{solution}

\begin{exercise}
  14> La masa de \SI{1,20}{\milli\gram} de una sustancia gaseosa pura equivale
  a \SI{1,2e19}{moléculas}. Calcula la masa en gramos de \SI{1}{\mole} de dicha sustancia.
\end{exercise}
\begin{solution}
  $M = \SI{60,2}{\gram\per\mole}$
\end{solution}

\begin{exercise}
  15> Se introducen, en un recipiente de \SI{5.0}{\liter}, \SI{10}{\gram} de alcohol etílico (\ch{C2H5OH}) y \SI{10}{\gram} de acetona (\ch{C3H6O}) y posteriormente se calienta el reactor a \SI{200}{\celsius}, con lo cual
  ambos líquidos pasan a la fase gaseosa. Calcula la presión en el interior del reactor, suponiendo comportamiento ideal, y la presión parcial de cada componente.
\end{exercise}
\begin{solution}
  $p_\textrm{alcohol} = \SI{1.7}{atm}; p_\textrm{acetona} = \SI{1.3}{atm} ; P_T = \SI{3.0}{atm}$
\end{solution}

\begin{exercise}
  16> Calcula la composición centesimal de la molécula de propano (\ch{C3H8}).
\end{exercise}
\begin{solution}
  \num{81,8}\% de carbono; \num{18,2}\% de hidrógeno.
\end{solution}

\begin{exercise}
  18> Calcula la fracción molar de cada uno de los componentes
  de una disolución que se ha preparado mezclando \SI{90}{\gram} de alcohol etílico (\ch{C2H50H}) y \SI{110}{\gram} de agua.
\end{exercise}
\begin{solution}
  $X_\textrm{alcohol} = 0,24, X_\textrm{agua} = 0,76$.
\end{solution}

\begin{exercise}
  20> Una disolución de hidróxido de sodio en agua que contiene un 25\% de hidróxido tiene una densidad de \SI{1.25}{\gram\per\milli\liter}. Calcula
  su molaridad y su normalidad.
\end{exercise}
\begin{solution}
  $M = \SI{7.8}{M}$ y $N = \SI{7.8}{N}$.
\end{solution}

\begin{exercise}
  22> ¿Cuál es la molaridad de una disolución de ácido sulfúrico
  del 26\% de riqueza y densidad \SI{1.19}{\gram\per\milli\liter}?
\end{exercise}
\begin{solution}
  $M = \SI{3.2}{M}$.
\end{solution}

\begin{exercise}
  23> El alcanfor puro tiene un punto de fusión de \SI{178}{\celsius} y una constante crioscópica de \SI{40}{\celsius\kilo\gram\per\mole}. La disolución resultante de añadir \SI{2}{\gram} de un soluto no volátil a
  \SI{10}{\gram} de alcanfor congela a \SI{158}{\celsius}. Calcula la masa molecular del soluto añadido.
\end{exercise}
\begin{solution}
  $M = \SI{400}{\gram\per\mole}$.
\end{solution}

\begin{exercise}
  24> Tenemos \SI{100}{\milli\liter} de una disolución acuosa que contiene \SI{0.25}{\gram} de un polisacárido. Dicha disolución a \SI{25}{\celsius}, ejerce una presión osmótica de \SI{23.9}{\mmHg}. El polisacárido tiene la siguiente fórmula empírica \ch{(C6H10O5)_{n}}. Calcula el valor de la masa molecular del polisacárido.
\end{exercise}
\begin{solution}
  $M = \SI{1938}{\gram\per\mole}$.
\end{solution}



\subsection{Problemas propuestos}

\subsubsection{Leyes de los volúmenes de combinación. Hipótesis de Avogadro. Concepto de molécula. Mol}

\begin{exercise}
  8. Determina la masa, M, de un mol de un gas en los siguientes
  casos:
  \begin{enumerate}
    \item Su densidad en c.n. es de \SI{3.17}{\gram\per\liter}.
    \item Su densidad es de \SI{2.4}{g/L} a \SI{20}{\celsius} y \SI{1}{atm} de presión.
    \item Dos gramos de dicho gas ocupan un volumen de \SI{600}{\milli\liter}, medido a \SI{17}{\celsius} y \SI{1.8}{atm} de presión.
  \end{enumerate}
\end{exercise}
\begin{solution}
  \begin{enumerate*}
    \item $M = \SI{71}{g/mol}$;
    \item $M = \SI{58}{g/mol}$;
    \item $M = \SI{44}{g/mol}$.
    \end{enumerate*}
\end{solution}

\begin{exercise}
  9. Realiza los siguientes cálculos numéricos:
  \begin{enumerate}
    \item
    \item Los átomos de oxígeno que hay en \SI{0.25}{moles} de sulfato
    de potasio (\ch{K2SO4}).
    \item Las moléculas de gasolina (\ch{C8H18}) que hay en un depósito
    de \SI{40}{\liter} ($d = \SI{0.76}{g/mL}$) .
    \item Los gramos de calcio que hay en \SI{60}{g} de un carbonato de
    calcio (\ch{CaCO3}) del 80\% de riqueza.
    \item De una sustancia pura, sabemos que \SI{1.75e19}{moléculas} moléculas
    corresponden a una masa de \SI{2.73}{mg}. ¿Cuál será la masa de \SI{1}{mol}?
  \end{enumerate}
\end{exercise}
\begin{solution}
  \begin{enumerate*}
    \item \SI{6e23}{átomos}; \item \SI{1.6e26}{moléculas}; \item \SI{19}{g}; \item $M = \SI{93}{g/mol}$.
  \end{enumerate*}
\end{solution}

\begin{exercise}
  10. Disponemos de \SI{3}{moles} de sulfuro de hidrógeno. Calcula,
  sabiendo que las masas atómicas son $\ch{S} = 32$ y $\ch{H} = 1$:
  \begin{enumerate}
    \item Cuántos gramos de \ch{H2S} hay en esos \SI{3}{moles}.
    \item El número de moléculas de \ch{H2S} que forman los \SI{3}{moles}.
    \item Los moles de \ch{H2} y de \ch{S} que tenemos en los \SI{3}{moles} de \ch{H2S}.
  \end{enumerate}
\end{exercise}
\begin{solution}
  \begin{enumerate*}
    \item $m_{\ch{H2S}} = \SI{102}{\gram}$;
    \item \SI{1.8e24}{moléculas};
    \item \SI{3}{moles} de \ch{H2} y \SI{3}{moles} de \ch{S}.
  \end{enumerate*}
\end{solution}

\begin{exercise}
  12. ¿Dónde crees que habrá más moléculas, en \SI{15}{g} de \ch{H2} o en
  \SI{15}{g} de \ch{O2}? Justifica la respuesta.
\end{exercise}
\begin{solution}
  En \SI{15}{g} de \ch{H2}.
\end{solution}

\begin{exercise}
  13. ¿Cuál será el volumen de \ch{HCl}, medido en c.n., que podremos
  obtener con \SI{6e22}{moléculas} de cloro?
\end{exercise}
\begin{solution}
  $V = \SI{4.5}{\liter}$ de \ch{HCl}.
\end{solution}

\begin{exercise}
  14. Calcula los gramos de amoniaco que podrías obtener con \SI{10}{\liter}
  de \ch{N2}, medidos en c.n.
\end{exercise}
\begin{solution}
  $m_{\ch{NH}} = \SI{15}{g}$.
\end{solution}

\begin{exercise}
  15. A \SI{20}{\celsius} la presión de un gas encerrado en un volumen V constante es de \SI{850}{\mmHg}. ¿Cuál será el valor de la presión si bajamos la temperatura a \SI{0}{\celsius}?
\end{exercise}
\begin{solution}
  $p = \SI{792}{\mmHg}$.
\end{solution}



\subsubsection{Leyes de los gases}

\begin{exercise}
  17. Diez litros de un gas medidos en c.n., ¿qué volumen ocuparán
  si cambiamos las condiciones a \SI{50}{\celsius} y \SI{4}{atm} de presión?
\end{exercise}
\begin{solution}
  V= 2,96 L.
\end{solution}

\begin{exercise}
  18. En un matraz de \SI{5}{\liter} hay si \SI{42}{\gram} de \ch{N2} a \SI{27}{\celsius}. Se abre el recipiente
  hasta que su presión se iguala con la presión atmosférica,
  que es de \SI{1}{atm}.
  \begin{enumerate}
    \item ¿Cuántos gramos de \ch{N2} han salido a la atmósfera?
    \item ¿A qué T deberíamos poner el recipiente para igualar la presión inicial?
  \end{enumerate}
\end{exercise}
\begin{solution}
  \begin{enumerate*}
    \item \SI{36.3}{\gram} de \ch{N2} han salido;
    \item $T' = \SI{2214}{\kelvin}$.
  \end{enumerate*}
\end{solution}

\begin{exercise}
  20. En una bombona se introducen \SI{0.21}{moles} de \ch{N2}, \SI{0.12}{moles}
  de \ch{H2} y \SI{2.32}{moles} de \ch{NH3}. Si la presión total es de \SI{12.4}{atm}, ¿cuál es la presión parcial de cada componente?
\end{exercise}
\begin{solution}
  $p_{\ch{N2}} = \SI{0.98}{atm}$; $p_{\ch{H2}} = \SI{0.56}{atm}$; $p_{\ch{NH3}} = \SI{10.9}{atm}$.
\end{solution}

\begin{exercise}
  21. En c.n. de p y T, \SI{1}{mol} de \ch{NH3} ocupa \SI{22.4}{\liter} y contiene \SI{6.02e23}{moléculas}. Calcula:
  \begin{enumerate}
    \item ¿Cuántas moléculas habrá en \SI{37}{\gram} de amoniaco a \SI{142}{\celsius} y \SI{748}{\mmHg}?
    \item ¿Cuál es la densidad del amoniaco a \SI{142}{\celsius} y \SI{748}{\mmHg}?
  \end{enumerate}
\end{exercise}
\begin{solution}
  \begin{enumerate*}
    \item \SI{1.31e24}{moléculas} de \ch{NH3}; \item $d = \SI{0.49}{\gram\per\liter}$.
  \end{enumerate*}
\end{solution}

\begin{exercise}
  22. Resuelve los siguientes ejercicios referidos a la ecuación de Clapeyron:
  \begin{enumerate}
    \item Un gas ocupa un volumen de \SI{15}{\liter} a \SI{60}{\celsius} y \SI{900}{\mmHg}. ¿Qué volumen ocuparía en c.n.?
    \item En una bombona de \SI{15.0}{\liter} hay gas helio a \SI{20}{\celsius}. Si el manómetro marca \SI{5.2}{atm}, ¿cuántos gramos de helio hay en la bombona? ¿A qué T estaría el gas si la presión fuera la atmosférica?
    \item Una cierta cantidad de aire ocupa un volumen de \SI{10}{\liter} a \SI{47}{\celsius} y \SI{900}{\mmHg}. Si la densidad del aire es de \SI{1.293}{g/L}, ¿qué masa de aire hay en el recipiente?
  \end{enumerate}
\end{exercise}
\begin{solution}
  \begin{enumerate*}
    \item $V = \SI{14.6}{\liter}$; \item $m = \SI{16}{\gram}$ de \ch{He}, $T = \SI{56}{\kelvin}$; \item $m = \SI{13}{\gram}$ de aire.
  \end{enumerate*}
\end{solution}





\subsubsection{Composición centesimal. Fórmulas moleculares y empíricas}

\begin{exercise}
  23. Un compuesto orgánico tiene la siguiente composición centesimal: $\ch{C} = 24,24\%$, $\ch{H} = 4,05\%$, $\ch{Cl} = 71,71\%$. Calcula: % REVIEW revisar como reacciona el chemformula aquí
  \begin{enumerate}
    \item La fórmula empírica.
    \item Su fórmula molecular, sabiendo que \SI{0.942}{\gram} de dicho compuesto ocupan un volumen de \SI{213}{\milli\liter} medidos a \SI{1}{atm} y \SI{0}{\celsius}.
  \end{enumerate}
\end{exercise}
\begin{solution}
  \begin{enumerate*}
    \item \ch{(CH2Cl)_n};
    \item \ch{C2H4Cl2}
  \end{enumerate*}

\end{solution}

\begin{exercise}
  24. Resuelve los siguientes ejercicios:
  \begin{enumerate}
    \item Entre dos minerales de fórmulas \ch{Cu5FeS4} y \ch{Cu2S}, ¿cuál es más rico en cobre?
    \item De los siguientes fertilizantes indica cuál es más rico en nitrógeno: \ch{NH4NO3} o \ch{(NH4)3PO3}.
    \item Halla la composición centesimal del arseniato de cobre(II)
    y del sulfato de sodio decahidratado.
  \end{enumerate}
\end{exercise}
\begin{solution}
  \begin{enumerate*}
    \item \ch{Cu2S};
    \item \ch{NH4NO3};
    \item 40,7\% de Cu, 32\% de As, 27,3\% de O; 14,3\% de Na, 9,9\% de S, 69,6\% de O, 6,2\% de H.
  \end{enumerate*}
\end{solution}




\subsubsection{Disoluciones y propiedades coligativas}

\begin{exercise}
  26. Calcula la fracción molar de agua y alcohol etílico en una
  disolución preparada agregando \SI{50}{\gram} de alcohol etílico y \SI{100}{\gram} de agua.
\end{exercise}
\begin{solution}
  $X_\textrm{alcohol} = 0,16$, $X_\textrm{agua} = 0,84$.
\end{solution}

\begin{exercise}
  29. Un ácido sulfúrico diluido tiene una densidad de \SI{1.10}{\gram\per\milli\liter} y una riqueza del 65\% en masa. Calcula la molaridad y la normalidad de la disolución.
\end{exercise}
\begin{solution}
  \SI{7.3}{M}; \SI{14.6}{N}.
\end{solution}

\begin{exercise}
  30. Calcula los gramos de hidróxido de sodio comercial de un
  85\% de riqueza en masa que harán falta para preparar \SI{250}{\milli\liter} de una disolución de \ch{NaOH} \SI{0.5}{M}.
\end{exercise}
\begin{solution}
\SI{5.9}{\gram}
\end{solution}

\begin{exercise}
  31. Una disolución de ácido sulfúrico está formada por \SI{12.0}{\gram} de
  ácido, \SI{19.2}{\gram} de agua y ocupa un volumen de \SI{27}{\milli\liter}. Calcula la densidad de la disolución, la concentración centesimal, la molaridad y la molalidad.
\end{exercise}
\begin{solution}
  $d = \SI{1.16}{\gram\per\milli\liter}$; $\%masa= 38,5\%$; $M= \SI{4.5}{M}$; $m= \SI{6.4}{m}$
\end{solution}

\begin{exercise}
  32. En la etiqueta de un frasco de \ch{HCl} dice: densidad \SI{1.19}{\gram\per\milli\liter}, riqueza 37,1\% en peso. Calcula:
  \begin{enumerate}
    \item Masa de \SI{1}{\liter} de esta disolución.
    \item Concentración del ácido en \si{\gram\per\liter}.
    \item Molaridad del ácido.
  \end{enumerate}
\end{exercise}
\begin{solution}
  \begin{enumerate*}
    \item \SI{1,19}{\kilo\gram};
    \item \SI{441,5}{\gram\per\liter};
    \item \SI{12,1}{M}.
  \end{enumerate*}
\end{solution}

\begin{exercise}
  33. Cuando se agrega \SI{27.8}{\gram} de una sustancia a \SI{200}{\cubic\centi\meter} de agua, la presión de vapor baja de \SI{23.7}{\mmHg} a \SI{22.9}{\mmHg}. Calcula la masa molecular de la sustancia.
\end{exercise}
\begin{solution}
  \SI{71.7}{\gram\per\mole}.
\end{solution}

\begin{exercise}
  34. Una disolución compuesta por \SI{24}{\gram} de azúcar en \SI{75}{\cubic\centi\meter} de agua, congela a \SI{-1.8}{\celsius}. Calcula:
  \begin{enumerate}
    \item La masa molecular del azúcar,
    \item Si su fórmula empírica es \ch{CH2O}, ¿cuál es su fórmula molecular? Dato: $K_c = \SI{1.86}{\celsius\kilo\gram\per\mole}$.
  \end{enumerate}
\end{exercise}
\begin{solution}
  \begin{enumerate*}
    \item \SI{330}{\gram};
    \item \ch{C11H22O11}
  \end{enumerate*}
\end{solution}

\begin{exercise}
  35. Una disolución que contiene \SI{25}{\gram} de albúmina de
  huevo por litro ejerce una presión osmótica de \SI{13.5}{\mmHg}, a
  \SI{25}{\celsius}. Determina la masa molecular de esa proteína.
\end{exercise}
\begin{solution}
  \SI{3.44e4}{\gram\per\mole}
\end{solution}

\begin{exercise}
  36. Cuando llega el invierno y bajan las temperaturas decidimos
  fabricar nuestro propio anticongelante añadiendo \SI{3}{\liter} de
  etilenglicol (\ch{C2H6O2}), cuya densidad es de \SI{1.12}{\gram\per\cubic\centi\meter} a \SI{8}{\liter}
  de agua que vertemos al radiador del coche. ¿A qué temperatura
  podrá llegar la disolución del radiador sin que se congele?
  Dato: constante crioscópica molal del agua $K_c = \SI{1.86}{\celsius\kilo\gram\per\mole}$.
\end{exercise}
\begin{solution}
  \SI{-12.6}{\celsius}.
\end{solution}



\subsubsection{Aplica lo aprendido}

\begin{exercise}
  38. Razona en cuál de las siguientes cantidades habrá un mayor número de átomos:
  \begin{enumerate}
    \item \SI{20}{\gram} de hierro.
    \item \SI{20}{\gram} de azufre.
    \item \SI{20}{\gram} de oxígeno molecular.
    \item Todas tienen la misma cantidad de átomos.
  \end{enumerate}
\end{exercise}
\begin{solution}
  La c): \SI{7.53e23}{átomos} de O.
\end{solution}

\begin{exercise}
  39. Una determinada cantidad de aire a la presión de \SI{2}{atm} y
  temperatura de \SI{298}{\kelvin} ocupa un volumen de \SI{10}{\liter}. Calcula la masa molecular media del aire, sabiendo que el contenido del
  mismo en el matraz tiene una masa de \SI{23.6}{\gram}.
\end{exercise}
\begin{solution}
  $m = \SI{28.8}{\gram\per\mole}$.
\end{solution}

\begin{exercise}
  43. Si tenemos encerrado aire en un recipiente de cristal, al
  calentarlo a \SI{20}{\celsius} la presión se eleva a \SI{1.2}{atm}. ¿Cuánto marcará el barómetro si elevamos la temperatura \SI{10}{\celsius}?
\end{exercise}
\begin{solution}
  $p = \SI{1.24}{atm}$.
\end{solution}

\begin{exercise}
  44. Se queman completamente \SI{1.50}{\gram}1,50 g de un compuesto orgánico
  formado por carbono, hidrógeno y oxígeno. En la combustión
  se obtuvieron \SI{0.71}{\gram} de agua y \SI{1.74}{\gram} de \ch{CO2}. Determina las fórmulas empírica y molecular del compuesto si
  \SI{1.03}{\gram} del mismo ocupan un volumen de \SI{350}{\milli\liter} a \SI{20}{\celsius} y \SI{750}{\mmHg}.
\end{exercise}
\begin{solution}
  Empírica \ch{C2H4O3}; molecular \ch{C2H4O3}
\end{solution}

\begin{exercise}
  45. Sabiendo que la densidad del aire en c. n. es de \SI{1.293}{\gram\per\liter}, calcula la masa de aire que contiene un recipiente de \SI{25}{\liter}, si
  hemos medido que la presión interior, cuando la temperatura
  es de \SI{77}{\celsius}, es de \SI{1.5}{atm}. Calcula, asimismo, el número de
  moles de aire que tenemos.
\end{exercise}
\begin{solution}
  $m = \SI{37.82}{\gram}$; $n = \SI{1.31}{\mole}$.
\end{solution}

\begin{exercise}
46. A partir de los siguientes datos, determina la fórmula empírica y molecular de:
\begin{enumerate}
  \item Un hidrocarburo con 82,76\% de C; si su densidad en c.n. es de \SI{2.59}{\gram\per\liter}.
  \item Un hidrocarburo formado por un 85,7\% de C; si \SI{651}{\gram} contienen \SI{15.5}{moles} del mismo
  \item Un compuesto con 57,1\% de C, 4,8\% de H y 38,1\% de S; si en \SI{10}{\gram} hay \SI{3.6e22}{moléculas}.
  \item Un compuesto con 55\% de Cl, 37,2\% de C y 7,8\% de H; si \SI{2.8}{\gram} del compuesto ocupan un volumen de \SI{1.15}{\liter} a \SI{27}{\celsius} y \SI{0.93}{atm} de presión.
\end{enumerate}
\end{exercise}
\begin{solution}
  \begin{enumerate*}
    \item \ch{C4H10};
    \item \ch{C3H6};
    \item \ch{C8H8S2};
    \item \ch{C2H2Cl}.
  \end{enumerate*}
\end{solution}

\begin{exercise}
  49. Se dispone de tres recipientes que contienen \SI{1}{\liter} de \ch{CH4} gas,
  \SI{2}{\liter} de \ch{N2} gas y \SI{15}{\liter} de \ch{O2} gas, respectivamente, en condiciones normales de presión y temperatura. Indica razonadamente:
  \begin{enumerate}
    \item Cuál contiene mayor número de moléculas.
    \item Cuál contiene mayor número de átomos.
    \item Cuál tiene mayor densidad.
  \end{enumerate}
  Datos: masas atómicas: $\textrm{H} = 1$; $\textrm{C} = 12$; $\textrm{N} = 14$; $\textrm{O} = 16$.
\end{exercise}
\begin{solution}
  \begin{enumerate*}
    \item El \ch{O2} (\SI{4.0e23}{moléculas});
    \item el \ch{O2} (\SI{8.0e23}{átomos});
    \item el \ch{O2} (\SI{1.4}{\gram\per\liter}).
  \end{enumerate*}
\end{solution}

\begin{exercise}
  50. Un frasco de \SI{1.0}{\liter} de capacidad está lleno de dióxido de carbono gaseoso a \SI{27}{\celsius}. Se hace vacío hasta que la presión del
  gas es \SI{10}{\mmHg}. Indica razonadamente:
  \begin{enumerate}
    \item Cuántos gramos de dióxido de carbono contiene el frasco.
    \item Cuántas moléculas hay en el frasco.
  \end{enumerate}
  Datos: R = 0,082 atm L mol-1 K-1; masas atómicas: C = 12; O= 16.
\end{exercise}
\begin{solution}
  \begin{enumerate*}
    \item \SI{0.024}{\gram} \ch{CO2};
    \item \SI{3.2e20}{moléculas} de \ch{CO2}
  \end{enumerate*}
\end{solution}

\end{multicols}



\begin{multicols}{2}[
  \section{BLOQUE 5. QUÍMICA DEL CARBONO (tema 5 del libro)}
  ]

\begin{exercise}
  7> Formula los siguientes alcanos:
  \begin{enumerate}
    \item n-pentano
    \item 2,3,5-trimetilheptano
    \item 4-etil-2,6-dimetiloctano
    \item 4,6-dietil-2,4,8-trimetilnonano
    \item 4-etil-2,2,5,8-tetrametil-6-propildecano
    \item 3,7-dietil-5-isopropildecano
  \end{enumerate}
\end{exercise}

\begin{solution}[print=false]
  Todavía sin solución
\end{solution}

\begin{exercise}
  9> Formula los siguientes hidrocarburos insaturados:
  \begin{enumerate}
    \item But-1-eno
    \item Pent-2-eno
    \item Hexa-2,4-dieno
    \item 3-butilhexa-1,4-dieno
    \item But-2-ino
    \item 3,4-dimetilpent-1-ino
    \item 3,6-dimetilnona-1,4,7-triino
    \item Pent-1-en-3-ino
    \item Hept-3-en-1,6-diino
    \item 4-etilhexa-1,3-dien-5-ino
  \end{enumerate}
\end{exercise}

\begin{solution}[print=false]
  Todavía sin solución
\end{solution}

\begin{exercise}
  11> Formula los siguientes hidrocarburos cíclicos:
  \begin{enumerate}
    \item Etilciclohexano
    \item Ciclopenteno
    \item Ciclohexino
    \item 1,1,4,4-tetrametilciclohexano
    \item 3-etilciclopenteno
    \item 2,3-dimetilciclohexeno
    \item 4-ciclobutilpent-1-ino
    \item 3-ciclohexil-5-metilhex-2-eno
    \item Ciclohexa-1,3-dieno
    \item 3-ciclopentilprop-1-eno
  \end{enumerate}
\end{exercise}

\begin{solution}[print=false]
  Todavía sin solución
\end{solution}

\begin{exercise}
  12> Nombra los siguientes hidrocarburos cíclicos:
  \begin{enumerate}
    \item \chemfig{*4(----)}
    \item \chemfig{*6(--=---)}
    \item \chemfig{*6(---(-CH_3)--(-CH_2-[2]CH_3)-)}
    \item \chemfig{*6(---=-=)}
    \item \chemfig{*6(----(-CH(-[4]CH_3)-[0]CH_3)--)}
    \item \chemfig{CH_3-CH(-[6]*6(------))-CH_2-C(-[2]CH_3)(-[6]CH_3)-CH_2-CH_3}
    \item \chemfig{CH_3-CH(-[6]*5(-----))-CH=CH_2}
  \end{enumerate}
\end{exercise}

\begin{solution}[print=false]
  Todavía sin solución
\end{solution}

\begin{exercise}
  13> Formula los siguientes hidrocarburos aromáticos:
  \begin{enumerate}
    \item Metilbenceno (tolueno)
    \item Etenilbenceno
    \item 1,3-dietilbenceno
    \item 1-butil-4-isopropilbenceno
    \item Para-propiltolueno
    \item 3-fenil-5-metilheptano
    \item 4-fenilpent-1-eno
    \item 2,4-difenil-3-metilhexano
  \end{enumerate}
\end{exercise}

\begin{solution}[print=false]
  Todavía sin solución
\end{solution}

\begin{exercise}
  15> Formula los siguientes derivados halogenados:
  \begin{enumerate}
    \item 2-cloropropano
    \item 1,3-dibromobenceno
    \item 1,1,2,2-tetrafluoretano
    \item 1,4-diclorociclohexano
    \item 4-bromopent-1-ino
    \item 3-flúor-5-metilhex-2-eno
    \item 1,4-dibromo-6-ciclopentiloct-2-eno
    \item 4-yodo-3,5-difenilpent-1-ino
    \item 4-clorobut-1-eno
    \item 1,2-dibromobenceno
  \end{enumerate}
\end{exercise}

\begin{solution}[print=false]
  Todavía sin solución
\end{solution}

\begin{exercise}
  17> Formula los siguientes alcoholes y éteres: 18> Nomb ra los siguientes alcoholes y éteres:
  \begin{enumerate}
    \item 3-metilpentan-1-ol
    \item Butano-1,2,3-triol
    \item 2-fenilpropano-1,3-diol
    \item Ciclohexanol
    \item Hexa-3,5-dien-2-ol
    \item Fenol (Hidroxibenceno)
    \item 2-etilpentan-1-ol
    \item Pent-3-en-1-ol
    \item Etilisopropiléter
    \item Etenilfeniléter
    \item Dimetiléter
    \item Butilciclopentiléter
  \end{enumerate}
\end{exercise}

\begin{solution}[print=false]
  Todavía sin solución
\end{solution}

\begin{exercise}
  18> Nombra los siguientes alcoholes y éteres:
  \begin{enumerate}
    \item \ch{CH3OH}
    \item \chemfig{CH_2OH-CH_2-CH(-[6]CH_2-[::0]CH_3)-CH=CH_2}
    \item \ch{CH3-CHOH-CHBr-CH2OH}
    \item \chemfig{CH_3-CH_2-CH(-[6]CH_3)-CHOH-CH_3}
    \item \ch{CH2=CH-CHOH-CH2OH}
    \item \ch{CH3-CH2-CH2-O-CH2-CH2-CH3}
    \item \chemfig{CH_2=CH-O-CH(-[6]CH_3)-CH_3}
    \item \ch{CH3-(CH2)3-CH2-O-C+CH}
  \end{enumerate}
  VER EN EL LIBRO, PÁGINA 135
\end{exercise}

\begin{solution}[print=false]
  Todavía sin solución
\end{solution}


\begin{exercise}
  % 19> Formula los siguientes aldehídos y cetonas:
  \begin{enumerate}
    \item Etanal (acetaldehído)
    \item Benzaldehído
    \item 3-metilpentanal
    \item 2-metilpentanodial
    \item Propenal
    \item Hex-2-endial
    \item 5-ciclohexilpent-3-inal
    \item 3-metilpent-2-enal
    \item Hex-2-endial
    \item Pentan-2-ona
    \item Hexa-2,4-diona
    \item 3-clorobutanona
    \item 1,4-difenilpentan-2-ona
    \item Hexa-1,5-dien-3-ona
  \end{enumerate}
\end{exercise}

\begin{solution}[print=false]
  Todavía sin solución
\end{solution}

\begin{exercise}
  20> Nombra los siguientes aldehídos y cetonas:

  VER EN EL LIBRO, PÁGINA 136
  \begin{enumerate}
    \item HCHO
    \item \ch{CH3-CH2-CH2-CHO}
    \item \ch{OHC-CH=CH-CHO}
    \item \chemfig{CH_2=C(-[6]*5(-----))-CH_2-{(}CH_2{)}_4-CHO}
    \item \ch{OHC-CH=CH-CH2-CH(CH3)-CHO}
    \item \chemfig{CH_3-CH(-[6]C_6H_5)-CH=CH-CHO}
    \item \ch{CHO-CH2-C=C-CH2-CH2-CHO}
    \item \ch{CH3-CO-CH2-CH3}
    \item \ch{CH3-CH=CH-CH2-CO-CH3}
    \item \ch{CH3-CO-CH 2-CH2-CH2-CO-CH3}
    \item \ch{CH3-CH(CH3)-CO-CH2-CH(CH3)-CH3}
    \item \ch{CH2=CH-CO-CH=CH-CH3}
  \end{enumerate}

\end{exercise}

\begin{solution}[print=false]
  Todavía sin solución
\end{solution}

\begin{exercise}
  21> Formula los siguientes ácidos y ésteres:
  \begin{enumerate}
    \item Ácido etanoico (ácido acético)
    \item Ácido 3-metilhexanoico
    \item Ácido 2-fenilpentanodioico
    \item Ácido tricloroetanoico
    \item Ácido but-3-enoico
    \item Ácido hepta-2,4-dienoico
    \item Ácido pent-2-enodioico
    \item Ácido benzoico
    \item Butanoato de metilo
    \item Propanoato de etilo
    \item Benzoato de propilo
    \item Etanoato de octilo
    \item 3-cloropentanoato de etenilo
    \item But-3-enoato de isopropilo
  \end{enumerate}
\end{exercise}

\begin{solution}[print=false]
  Todavía sin solución
\end{solution}

\begin{exercise}
  23> Formula los siguientes compuest os con funciones nitrogenadas:
  \begin{enumerate}
    \item Isopropilamina
    \item Pentan-3-amina
    \item Buta-1,3-diamina
    \item 3-etilhexan-3-amina
    \item 3,5-dimetilhexan-1-amina
    \item Pent-3-en-2-amina
    \item N-metilfenilamina
    \item N-ciclopentilbutilamina
    \item Etanamida
    \item N-metiletanamida
    \item 4-fenilpentanamida
    \item N-etilhex-4-enamida
  \end{enumerate}
\end{exercise}

\begin{solution}[print=false]
  Todavía sin solución
\end{solution}

\begin{exercise}
  24> Nombra los siguientes compuestos nitrogenados:
  \begin{enumerate}
    \item \chemfig{CH_3-CH(-[6]NH_2)-CH_2-CH_3}
    \item \ch{CH3-CH2-CH2-NH2}
    \item \chemfig{CH_3-CH(-[6]NH_2)-CH_2-CH(-[6]NH_2)-CH_2-CH_2(-[6]NH_2)}
    \item \chemfig{CH_3-CH(-[6]CH_3)-NH-CH=CH_2}
    \item \chemfig{CH_3-NH-*4(----)}
    \item \ch{CH3-CH2-CH2-CH2-CH2-CO-NH2}
    \item \ch{CH3-CH=CH-CH2-CO-NH2}
    \item \ch{CH3-CH2-CHBr-CH2-CH2-CO-NH-CH3}
  \end{enumerate}
\end{exercise}

\begin{solution}[print=false]
  Todavía sin solución
\end{solution}

\begin{exercise}
  25> Formula los siguientes compuestos orgánicos:
  \begin{enumerate}
    \item 2,2-dimetilpentano
    \item Hepta-1,5-dieno
    \item 1-fenilpent-2-ino
    \item 3-isopropilciclohexeno
    \item 1-butil-3-metilbenceno
    \item Butano-1,3-diol
    \item Butileteniléter
    \item But-3 enal
    \item Hex-5-in-2-ona
    \item Ácido 3-isopropilhexanoico
    \item Pentanoato de metilo
    \item 5-meilhexan-2,4-diamina
    \item N-metiletilamina
    \item N,N-dietilbutilamina
    \item Hex-3-enamida
    \item N-metilbutanamida
  \end{enumerate}
\end{exercise}

\begin{solution}[print=false]
  Todavía sin solución
\end{solution}

\begin{exercise}
  27> Formula y nombra:
  \begin{enumerate}
    \item Dos hidrocarburos alifáticos que presenten isomería de cadena.
    \item Dos aminas con isomería de posición.
    \item Dos compuestos oxigenados con isomería de función.
  \end{enumerate}
\end{exercise}

\begin{solution}[print=false]
  Todavía sin solución
\end{solution}

\begin{exercise}
  28> Escribe y nombra:
  \begin{enumerate}
    \item Todos los isómeros de cadena de fórmula \ch{C5H12}.
    \item Cuatro isómeros de función de fórmula \ch{C4H80}.
    \item Tres isómeros de posición de la amina \ch{C5H13N}.
  \end{enumerate}
\end{exercise}

\begin{solution}[print=false]
  Todavía sin solución
\end{solution}

\begin{exercise}
  Dados los siguientes compuestos, formúlalos y justifica cuáles
  de ellos presentan isomería geométrica y cuáles isomería
  óptica:
  \begin{enumerate}
    \item 2-clorobutano
    \item Pent-3-en-2-ol
    \item Pentan-3-amina
    \item 2-fenilpent-2-eno
  \end{enumerate}
\end{exercise}

\begin{solution}[print=false]
  Todavía sin solución
\end{solution}


\subsection{Problemas propuestos}

\subsubsection{Grupos funcionales y series homólogas}

\begin{exercise}
  7. Escribe el número de carbonos y el grupo funcional al que
  corresponden los siguientes compuestos:
  \begin{enumerate}
    \item Octano
    \item Butanamina
    \item Pentinamida
    \item Ácido decanoico
    \item Hexenal
    \item Propanona
    \item Butino
    \item Hepteno
    \item Metanol
    \item Dietiléter
  \end{enumerate}
\end{exercise}

\begin{solution}[print=false]
  Todavía sin solución
\end{solution}

\begin{exercise}
  8. Indica si la estructura de cada pareja representa el mismo
  compuesto o compuestos diferentes, identificando los grupos
  funcionales presentes:
  \begin{enumerate}
    \item \ch{CH3CH2OCH3} y \ch{CH3OCH2CH3}
    \item \ch{CH3CH2OCH3} y \ch{CH3CHOHCH3}
    \item \ch{CH3CH2CH2OH} y \ch{CH3CHOHCH3}
  \end{enumerate}
\end{exercise}

\begin{solution}[print=false]
  Todavía sin solución
\end{solution}

\begin{exercise}
  9. Contesta a cada uno de los siguientes apartados referidos a
  compuestos de cadena abierta:
  \begin{enumerate}
    \item ¿Qué grupos funcionales pueden tener los compuestos de
    fórmula molecular \ch{C_nH_{2n+2}O}?
    \item ¿Qué compuestos tienen por fórmula molecular \ch{C_nH_{2n-2}}?
  \end{enumerate}
\end{exercise}

\begin{solution}[print=false]
  Todavía sin solución
\end{solution}

\begin{exercise}
  10. Nombra y formula los siguientes compuestos orgánicos:
  \begin{enumerate}
    \item \ch{CH3-CH2-COOH}
    \item \ch{CH3-CH2-C+CH}
    \item \ch{CH3-CHOH-CH2-CH2-CH3}
    \item \ch{CH3-CH2-CO-CH2-CH2-CH3}
    \item \ch{C6H14}
    \item Metil etil éte
    \item Metanoato de propil
    \item Dietilamin
    \item Pentana
    \item Metilpropen
  \end{enumerate}
\end{exercise}

\begin{solution}[print=false]
  Todavía sin solución
\end{solution}

\begin{exercise}
  13. Formula las siguientes especies químicas:
  \begin{enumerate}
    \item 1-bromo-2,2-diclorobutano
    \item Trimetilamina
    \item 2-metilhex-1,5-dien-3-ino
    \item Butanoato de 2-metilpropilo
    \item Tolueno (metilbenceno)
    \item Propanamida
    \item 2,3-dimetilbut-1-eno
    \item Ácido 2,3-dimetilpentanodioico
  \end{enumerate}
\end{exercise}

\begin{solution}[print=false]
  Todavía sin solución
\end{solution}

\begin{exercise}
  14. Nombra las siguientes especies químicas:
  \begin{enumerate}
    \item \ch{H2C=CH-CH=CH-CHO}
    \item \ch{H3C-CO-CO-CH3}
    \item \ch{H2C=CH-CH=CH-CH2-COOH}
    \item \ch{H3C-CH2-NH-CH2-CH3}
    \item \ch{CH+C-CH2-COOH}
    \item \ch{CH3-CH2-CH(CH3)-CONH2}
    \item \ch{H3C-C(OH)2-CH2-CH2OH}
  \end{enumerate}
\end{exercise}

\begin{solution}[print=false]
  Todavía sin solución
\end{solution}

\begin{exercise}
  15. Nombra y/o formula los siguientes compuestos:
  \begin{enumerate}
    \item \ch{CHCl3}
    \item \ch{CH3-CH2-CHO}
    \item \ch{CH3-CH2-CH2-CH2-CO-NH2}
    \item \ch{(CH3)2-CHOH}
    \item 2,2-dimetilbutano
    \item Para-diaminobenceno
    \item Ciclohexano
    \item Etil propil éter
  \end{enumerate}
\end{exercise}

\begin{solution}[print=false]
  Todavía sin solución
\end{solution}

\begin{exercise}
  16. Formula o nombra, según corresponda:
  \begin{enumerate}
    \item 1-etil-3-metilbenceno
    \item 2-metilpropan-2-ol
    \item 2-metil-propanoato de etilo
    \item Pent-3-en-1-amina
    \item \ch{ClCH=CH-CH3}
    \item \ch{CH3-CH2-O-CH2-CH3}
    \item \ch{CH3-CH(CH3)-CO-CH2-CH(CH3)-CH3}
    \item \ch{CH2=CH-CH2-CO-NH-CH3}
  \end{enumerate}
\end{exercise}

\begin{solution}[print=false]
  Todavía sin solución
\end{solution}

\begin{exercise}
  20. Formula o nombra los siguientes compuestos:
  \begin{enumerate}
    \item Cromato de cobre(II)
    \item Hidruro de magnesio
    \item Hidrogenosulfuro de bario
    \item Etanamina
    \item Propan-1,2-diol
    \item \ch{Fe(OH)2}
    \item \ch{H2SO3}
    \item \ch{N2O5}
    \item \chemfig{**6(---(-CH=[2]O)---)}
    \item \chemfig{CH_3-CH(-[6]CH_3)-CH(-[6]CH_3)-CH(-[6]CH_3)-CH_2-CH_3}
  \end{enumerate}
\end{exercise}

\begin{solution}[print=false]
  Todavía sin solución
\end{solution}

\begin{exercise}
  21. Formula o nombra los siguientes compuestos orgánicos :
  \begin{enumerate}
    \item 3-etil-2-metilhexano
    \item 1-bromopent-2-ino:
    \item 3-etilhe xano-1,5-diol:
    \item 3-metilpentan-2,4-diamina
    \item \ch{CH2=CH-CH2-CO-O-CH3}
    \item \ch{C6H5-O-C6H5}
    \item \ch{CH3-CH2-CO-NH-CH2-CH3}
    \item \ch{COOH-CH2-CH2-CHBr-COOH}
  \end{enumerate}
\end{exercise}

\begin{solution}[print=false]
  Todavía sin solución
\end{solution}

\subsubsection{Isomería estructural y espacial}

\begin{exercise}
  23. Formula los siguientes compuestos orgánicos:
  \begin{enumerate}
    \item But-3-en-2-ona
    \item Buta-1,3-dien-2-ol
    \item Dietiléter
  \end{enumerate}
  ¿Cuáles de ellos son isómeros entre sí?
\end{exercise}

\begin{solution}[print=false]
  Todavía sin solución
\end{solution}

\begin{exercise}
  24. Escribe y nombra cinco isómeros de cadena de fórmula molecular \ch{C6H14}.
\end{exercise}

\begin{solution}[print=false]
  Todavía sin solución
\end{solution}

\begin{exercise}
  25. Escribe y nombra cuatro isómeros de función de fórmula molecular \ch{C4H8O}.
\end{exercise}

\begin{solution}[print=false]
  Todavía sin solución
\end{solution}

\begin{exercise}
  28. Escribe y nombra todos los isómeros estructurales de fórmula C5H10
\end{exercise}

\begin{solution}[print=false]
  Todavía sin solución
\end{solution}

\begin{exercise}
  Formula y nombra:
  \begin{enumerate}
    \item Dos isómeros de posición de fórmula \ch{C3H8O}
    \item Dos isómeros de función de fórmula \ch{C3H8O}
    \item Dos isómeros geométricos de fórmula \ch{C4H8}
    \item Un compuesto que tenga dos carbonos quirales (asimétricos) de fórmula \ch{C4H8BrCl}
  \end{enumerate}
\end{exercise}

\begin{solution}[print=false]
  Todavía sin solución
\end{solution}

\begin{exercise}
  31. Un derivado halogenado etilénico que presenta isomería cis-trans
  está formado en un 22,4\% de C, un 2,8\% de H y un
  74,8\% de bromo. Además, a \SI{130}{\celsius} y \SI{1}{atm} de presión, una muestra de \SI{12,9}{\gram} ocupa un volumen de \SI{2}{\liter}. Halla su fórmula molecular y escribe los posibles isómeros.
\end{exercise}

\begin{solution}
  \ch{C4H6Br2}
\end{solution}

\begin{exercise}
  32. Un alcohol monoclorado está formado en un 38,1\% de C,
  un 7,4\% de H, un 37,6\% de Cl y el resto es oxígeno. Escribe
  su fórmula semidesarrollada sabiendo que tiene un carbono
  asimétrico y que su fórmula molecular y su fórmula empírica
  coinciden.
\end{exercise}

\begin{solution}
  \dh{C3H7OCl}
\end{solution}

\begin{exercise}
  33. Un hidrocarburo monoinsaturado tiene un 87,8\% de carbono.
  Si su densidad en condiciones normales es \SI{3,66}{\gram\per\liter}, determina sus fórmulas empírica y molecular.
\end{exercise}

\begin{solution}
  Formula empírica: \ch{C3H5}; Fórmula molecular: \ch{C6H10}.
\end{solution}

\end{multicols}

\section{BLOQUE 4. Transformaciones energéticas y espontaneidad (tema 6 del libro)}
%
% 5> Determina la variación de energía interna que sufre un sistema
% cuando:
% a) Realiza un trabajo de 600 J y cede 40 calorías al entorno.
% b) Absorbe 300 calorías del entorno y se realiza un trabajo de compresión de 5 kJ.
% S: a) dU = -767 J; b) dU = 6,25 · 10^3 J
%
% 13> La descomposición térmica del clorato de potasio (KClO3)
% origina cloruro de potasio (KCl) y oxígeno molecular. Calcula
% el calor que se desprende cuando se obtienen 150 litros de
% oxígeno medidos a 25ºC y 1 atm de presión.
% Datos: dH^0_f (kJ/mol): KClO3 (s) = -91,2; KCl (s) = -436
% S: Se desprenden 1,41·10^3 kJ
%
% 14> Las entalpías estándar de formación del propano (g),
% dióxido de carbono (g) y agua (l), son respectivamente :
% -103,8; -393,5 y -285,8 kJ/mol. Calcula:
% a) La entalpía de la reacción de combustión del propano.
% b) Las calorías generadas en la combustión de una bombona
% de propano de 1,80 litros a 25ºC y 4 atm de presión.
% S: a) dH^0_c = -2,22 · 10^3 kJ/mol de propano; b) Se desprenden
% 156 kcal
%
% 15> En la reacción del oxígeno molecular con el cobre para formar
% óxido de cobre(II) se desprenden 2,30 kJ por cada gramo de
% cobre que reacciona, a 298 K y 760 mm Hg. Calcula :
% a) La entalpía de formación del óxido de cobre(II). '
% b) El calor desprendido a presión constante cuando reaccionan
% 100 L de oxígeno, medidos a 1,5 atm y 27 °C
% S: a) dH^0_f (CuO) = -146 kJ/mol;
% b) Se desprenden 1,78·10^3 kJ
%
% 16> En la combustión completa de 1,00 g de etanol (CH3-CH2OH) se desprenden 29,8 kJ y en la combustión de 1,00 g
% de ácido etanoico (CH3-COOH) se desprenden 14,5 kJ.
% Determina numéricamente :
% a) Cuál de las dos sustancias tiene mayor entalpía de combustión.
% b) Cuál de las dos sustancias tiene mayor entalpía de formación
% S: a) dH^0_c etanol = -1,37·10^3 kJ/mol; dH^0_c ácido etanoico = -870 kJ/mol
% b) dH^0_f etanol = -273 kJ/mol; dH^0_f ácido etanoico = -489 kJ/mol
%
% 24> Calcula la entalpía de la reacción:
% CH4 (g) + Cl2 (g) ➔ CH3Cl (g) + HCl (g) a partir de:
% a) Las energías de enlace.
% b) Las entalpías de formación.
% Datos: Energías de enlace (kJ/mol):
% C-H = 414; Cl-Cl = 244; C-Cl = 330; H-Cl = 430.
% Entalpías de formación (kJ/mol) :
% (CH4) = -74,9; (CH3Cl) = -82,0; dH^0_f (HCl) = -92,3.
% S: a) dH^0_R = -102 kJ; b) dH^0_R = -99,4 kJ
%
% 25> El eteno se hidrogena para dar etano, según:
% CH2=CH2 (g) + H2 (g) ➔ CH3-CH3 (g) dH^0_R = -130 kJ
% Calcula la energía del enlace C=C, si las energías de los
% enlaces C-C, H-H y C-H son, respectivamente, 347, 436
% y 414 kJ/mol. S: dH^0 (C=C) = 609 kJ/mol
%
% 26> A partir de los siguientes datos:
% Entalpía estándar de sublimación del C (s) = 717 kJ/mol.
% Entalpía de formación del CH3-CH3 (g) = -85,0 kJ/mol.
% Entalpía media del enlace H-H = 436 kJ/mol.
% Entalpía media del enlace C-C = 347 kJ/mol.
% a) Calcula la variación de entalpía de la reacción:
% 2C (g) + 3H2 (g) ➔ CH3-CH3 (g) e indica si es exotérmica
% o endotérmica.
% b) Determina el valor medio del enlace C-H.
% S: a) dH^0_R = -1,52·10^3 kJ; b) dH^0 (C-H) = 413 kJ/mol
%
%
% \subsection{Problemas resueltos}
%
% \subsubsection{Entalpias de formación, de reacción y de combustión}
%
% 4. El sulfuro de carbono reacciona con el oxígeno según:
% CS2 (l) + 3 O2(9) -> CO2 (g) + 2 SO2 (g) dH_R = -1072 kJ
% a) Calcula la entalpía de formación del CS2
% b) Halla el volumen de SO2 emitido a la atmósfera, a 1 atm y
% 25 °C, cuando se ha liberado una energía de 6000 kJ
% Datos: dH^0_f (kJ/mol): CO2(g) = -393,5; SO2(g) = -296,4.
% S: a) dH^0_f CS2(l) = 85,7 kJ/mol; b) V = 274 L (SO2)
%
% 5. El dióxido de manganeso se reduce a manganeso metal reaccionando
% con el aluminio según:
% MnO2 (s) + Al (s) --> Al2O3 (s) + Mn (s)
% a) Halla la entalpía de esa reacción sabiendo que las entalpías
% de formación valen:
% dH^0_f (Al2O3) = -1676 kJ/mol; dH^0_f (MnO2) = -520 kJ/mol
% b) ¿Qué energía se transfiere cuando reaccionan 10,0 g de
% MnO2 con 10,0 g de Al?
% S: a) dH_R = -896 kJ/mol de Al2O3 ; b) 68,7 kJ se desprenden
%
% 6. Durante la fotosíntesis, las plantas verdes sintetizan la glucosa
% según la siguiente reacción:
% 6 CO2(g) + 6 H2O(l) --> C6H12O6 (s) + 6 O2(g) dH_R = 2815 kJ/mol
% a) ¿Cuál es la entalpía de formación de la glucosa?
% b) ¿Qué energía se requiere para obtener 50,0 g de glucosa?
% c) ¿Cuántos litros de oxígeno, en condiciones estándar, se
% desprenden por cada gramo de glucosa formado?
% Datos : dH^0_f (kJ/mol): H2O(l) = -285,8; CO2(g) = -393,5
% S: a) dH^0_f = -1,26·10^3 kJ/mol; b) 782 kJ; e) 0,8 L
%
% Las entalpías de combustión del etano y del eteno son
% -1560 kJ/mol y -1410 kJ/ mol respectivamente. Determina:
% a) El valor de dH^0_f para el etano y el eteno.
% b) Razona si el proceso de hidrogenación del eteno a etano
% es un proceso endotérmico o exotérmico.
% c) Calcula el calor que se desprende en la combustión de
% 50,0 g de cada gas.
% Datos: dH^0_f (kJ/mol) : CO2(g) = -393,5; H2O(l) = -285,9
% S: a) dH^0_f (C2H6)= -84,7 kJ/mol; dH^0_f (C2H4 = 51,2 kJ/mol; b) Exotérmico; e) etano: 2,60·10^3 kJ; eteno: 2,52 • 10^3 kJ.
%
% 8. La gasolina es una mezcla compleja de hidrocarburos que
% vamos a considerar como si estuviera formada únicamente
% por hidrocarburos saturados de fórmula (C8H18)
% a) Calcula el calor que se desprende en la combustión de
% 50,0 litros de gasolina (d = 0,78 g/mL).
% b) Halla la masa de CO2 que se emite a la atmósfera en esa
% combustión.
% c) Si el consumo de un vehículo es de 7,00 litros por cada
% 100 km, ¿qué energía necesita por cada km recorrido?
% Datos: dH^0_f (kJ/mol): CO2(g) = -394; H2O(l) = -286; C8H18 (l) = -250
% S: 1,87·10^6 kJ; b) 120 kg; e) 2,62·10^3 kJ/km
%
% Se quema benceno (C6H6) en exceso de oxígeno, liberando energía.
% a) Formula la reacción de combustión del benceno.
% b) Calcula la entalpía de combustión estándar de un mol de
% benceno líquido.
% c) Calcula el volumen de oxígeno, medido a 25 ºC y 5 atm,
% necesario para quemar 1 L de benceno líquido.
% d) Calcula el calor necesario para evaporar 10 L de benceno
% líquido.
% Datos: dH^0_f (kJ/mol): C6H6 (l)= +49; C6H6 (v)= +83;
% H2O(l)= -286; CO2 (g)= -393;
% Densidad benceno(l) = 0,879 g cm^-3
% S: b) dH^0_C = -3,27·10^3 kJ/mol; e) 413 L; d) 3,83·10^3 kJ
%
%
%
%
%
% \subsubsection{Ley de Hess}
%
% 11. El motor de una máquina cortacésped funciona con una gasolina que podemos considerar de composición única octano
% (C8H18). Calcula:
% a) La entalpía estándar de combustión del octano, aplicando la ley de Hess.
% b) El calor que se desprende en la combustión de 2,00 kg de octano.
% Datos: dH^0_f (kJ/mol): CO2(g)= -393,8; C8H18 (l)= -264,0;
% H2O(l)= -285,8
% S: a) dH^0_C = -5,46·10^3 kJ/mol; b) 9,58·10^4 kJ
%
% 12. Sabiendo que las entalpías estándar de combustión del
% hexano (l), del carbono (s) y del hidrógeno (g) son respectivamente:
% -4192; -393,5; y -285,8 kJ/mol, halla:
% a) La entalpía de formación del hexano líquido en esas condiciones.
% b) Los gramos de carbono consumidos en la formación del
% hexano cuando se han intercambiado 50,0 kJ.
% S: a) dH^0_f (hexano) = -170 kJ/mol b) m = 21,2 g de C
%
% 14. El calor desprendido en el proceso de obtención del benceno a partir de etino es:
% 3 C2H2 (g) -> C6H6 (l) dH^0_R = -631 kJ
% a) Calcula la entalpía estándar de combustión del benceno,
% sabiendo que la del etino es -1302 kJ/mol.
% b) ¿Qué volumen de etino, medido a 26ºC y 15 atm, se necesita
% para obtener 0,25 L de benceno?
% Datos: densidad del benceno = 880 g L^-1
% S: a) dH^0_C C6H6 (l) = -3,28·10^3 kJ/mol; b) 13,8 L C2H2
%
%
%
%
% \subsubsection{Entalpías de enlace}
%
% 18. Calcula la variación de entalpía estándar de la hidrogenación
% del etino a etano:
% a) A partir de las energías de enlace.
% b) A partir de las entalpías de formación.
% Datos: Energías de enlace (kJ/mol): C-H = 415; H-H = 436; C-C = 350: C~C = 825
% dH^0_f (kJ/mol): etino = 227; etano = -85,0
% S: a) dH_R = -313 kJ/mol; b) dH_R = -312 kJ/mol
%
%
%
%
% \subsubsection{Entropía y espontaneidad}
%
% 20. Dadas las siguientes ecuaciones termoquímicas:
% 2 H2O2 (l) --> 2 H2O(l) + O2(g) dH = -196 kJ
% N2 (g) + 3 H2(g) --> 2 NH3 (g) dH = -92,4 kJ
% a) Define el concepto de entropía y explica el signo más probable de dS en cada una de ellas .
% b) Explica si esos procesos serán o no espontáneos a cualquier
% temperatura, a temperaturas altas, a temperaturas
% bajas, o no serán nunca espontáneos.
%
% 21. Dada la reacción: N2O(g) --> N2(g) + 1/2 O2 (g)
% siendo dH^0 = 43,0 kJ/mol y dS^0 = 80,0 J/mol K
% a) Justifica el signo positivo de la variación de entropía.
% b) ¿Será espontánea a 25 ºC? ¿A qué temperatura estará en
% equilibrio?
% S: a) Aumentan los moles de sustancias gaseosas b) dG^0_R = 19,2 kJ (no es espontánea); T = 538 K
%
% 23. Se pretende obtener etileno (eteno) a partir de grafito e
% hidrógeno, a 25ºC y 1 atm, según la reacción:
% 2 C(s) + 2 H2(g) --> C2H4 (g). Calcula:
% a) La entalpía de reacción en condiciones estándar. ¿La reacción
% es endotérmica o exotérmica?
% b) La variación de energía libre de Gibbs en condiciones
% estándar. ¿Es espontánea la reacción en esas condiciones?
% Datos: S^0 (J mol-1 K-1): C(s) = 5, 70; H2(g) = 130,6;
% C2H4(g)= 219,2 dH^0_f (kJ/mol) : C2H4 (g) = +52,5
% S: a) dH^0_R = +52,5 kJ (endotérmica)
% b) dG^0_R = +68,4 kJ (no es espontánea)
%
%
%
%
% \subsubsection{Aplica lo aprendido}
%
%
% El acetileno o etino (C 2HJ se hidrogena para producir etano. Calcula a 298 K:
% a) La entalpía estándar de reacción.
% b) La energía de Gibbs estándar de reacción.
% e) La entropía estándar de reacción.
% d) La entropía molar del hidrógeno.
%
% Sustancia & dH^0_f (kJ mol^-1) & dG^0_f (kJ mol^-1) & S^0 (kJ mol^-1 K^-1) \\
% C2H2 & 227 & 209 & 200 \\
% C2H6 & -85 & -33 & 230 \\
%
% S: a) dH^0_R = -312 kJ; b) dG^0_R = -242 kJ; c) dS^0_R = -235 J/K d) S^0_H2, = 132 J/mol K



\end{document}
