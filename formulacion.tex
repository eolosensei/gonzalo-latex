\documentclass{article}

\usepackage{polyglossia}
    \setdefaultlanguage{spanish}

\usepackage{fontspec}
    \setmainfont{Fira Sans}

%\usepackage[version=4]{mhchem}
\usepackage{chemfig}
\usepackage{chemformula}

\usepackage{tabularx}
\usepackage{booktabs}

\begin{document}


\setchemfig{atom sep=1.5em}  % para una versión más moderna de chemfig

\begin{table}
    \centering
    \begin{tabularx}{\textwidth}{rXX}
        \toprule\midrule
        ej: & \ch{CH2=CH-O-CH3}                     & metoxieteno (metil vinil éter)                        \\ \midrule
        a) & \ch{C6H5-CH2OH}                        & fenilmetanol (alcohol bencílico)                      \\ \midrule
        b) & \ch{CH3-C+C-CH2-COOH}                  & ácido pent-3-inoico                                   \\ \midrule
        c) & \ch{CHO-CHO}                           & etanodial                                             \\ \midrule
        d) & \chemfig{HCOO-CH*4(-CH_2-CH_2-CH_2-)}  & metanoato de ciclobutilo                              \\ \midrule
        e) & \ch{CH3-CHOH-CH(CH3)2}                 & 3-metilbutan-2-ol                                     \\ \midrule
        f) & \ch{CH3-CO-CH2-CO-CH2-CH2-CH3}         & heptano-2,4-diona                                     \\ \midrule
        g) & \ch{(CH3)2CH-CO-CH(CH3)2}              & 2,4-dimetilpentan-3-ona                               \\ \midrule
        h) & \ch{CH3-CH2-O-C6H5}                    & etoxibenceno (etil fenil éter)                        \\ \midrule
        i) & \ch{CH3-COO-C(CH3)3}                   & etanoato de terc-butilo (acetato de terc-butilo)      \\ \midrule
        j) & \ch{CH2=CH-CH2-COOH}                   & ácido but-3-enoico                                    \\ \midrule
        \bottomrule
    \end{tabularx}
    \caption{Ejercicio 33, página 51, Formulación Santillana}
\end{table}

\begin{table}
    \centering
    \begin{tabularx}{\textwidth}{rXX}
           & Nombre erróneo                                 & Nombre correcto                   \\ \toprule
        a) & ácido 2-isopropil-2-metilpropanoico            & ácido 2,2,3-trimetilbutanoico     \\ \midrule
        b) & \ch{CH3-CH2-O-CH3} propanol                    & etil metil éter                   \\ \midrule
        c) & ciclopropanal                                  & no existe (el grupo funcional aldehído tiene que estar en un carbono terminal) \\ \midrule
        d) & \ch{CH3-CH2-CH2-COO-CH3} propanoato de etilo   & butanoato de metilo               \\ \midrule
        e) & ácido 2-\textit{terc}-butiletanoico            & ácido 3,3-dimetilbutanoico        \\ \midrule
        f) & 1-metiletan-1-ol                               & propan-2-ol                       \\ \midrule
        \bottomrule
    \end{tabularx}
    \caption{Ejercicio 34, página 51, Formulación Santillana}
\end{table}

\begin{table}
    \centering
    \begin{tabularx}{\textwidth}{rXX}
        \toprule\midrule
        a) & \ch{CH2=CH-NH-CH3}                               & N-metiletenamina                \\ \midrule
        b) & \ch{CH3-CN}                                      & etanonitrilo                    \\ \midrule
        c) & \ch{CH3-CO-CH3}                                  & propanona                       \\ \midrule
        d) & \ch{CH3-CHOH-CH2OH}                              & propano-1,2-diol                \\ \midrule
        e) & \ch{C6H5-O-CH2-CH3}                              & etil fenil éter                 \\ \midrule
        f) & \chemfig{[:30]*6(-=(-(=[:60]O)-[:-60]OH)-=-=)}   & ácido benzoico                  \\ \midrule
        g) & \ch{(CH3)3C-O-CH3}                               & \textit{terc}-butil metil éter  \\ \midrule
        h) & \chemfig{H_3C-[1]CH_2-[7]CH_2-[1]C(=[2]C(-[3]H_3C)-[1]CH(-[2]CH_3)-[7]CH_3)-[7]CH_2-[1]CH=[7]CH_2}   &    5,6-dimetil-4-propilhepta-1,4-dieno \\ \midrule
        i) & \ch{CHO-CH=CH-CH=CH-CH2-CHO}                     & hept-2,4-dienal                 \\ \midrule
        \bottomrule
    \end{tabularx}
    \caption{Ejercicio 35, página 51, Formulación Santillana}
\end{table}

\begin{table}
    \centering
    \begin{tabularx}{\textwidth}{rXX}
        \toprule\midrule
        a) & \ch{CH3-COOH}                                & ácido etanoico                          \\ \midrule
        b) & \ch{NC-CH2-CH2-CH2-CH3}                      & pentanotrilo                            \\ \midrule
        c) & \chemfig{H_3C-*7(-=--~--)}                   & 5-metilciclohept-1-en-4-ino             \\ \midrule
        d) & \ch{CH3-C+C-CH2-CH2-COOH}                    & ácido hex-4-inoico                      \\ \midrule
        e) & \chemfig{N(-[:-90]CH_3)(-[:150]*5(-----))-[:30]C(=[::60]O)-[::-60]CH_3}  & \textit{N}-ciclopentil-\textit{N}-metiletanoamida     \\ \midrule
        f) & \chemfig{N(-[6]CH_3)(-[:150]*3(---))-[:30](-[2]CH_3)-[:-30]CH_3}         & \textit{N}-isopropil-\textit{N}-metilciclopropanamina \\ \midrule
        g) & \chemfig{[:30]*6(=-(-OH)=(-CH_3)-=(-OH)-)}   & 2-metilbenceno-1,4-diol                 \\ \midrule
        h) & \ch{CH3-CO-CH2-C+CH}                         & pent-4-in-2-ona                         \\ \midrule
        i) & \chemfig{*6(=(-Cl)-=(-Cl)-=(-Cl)-)}          & 1,3,5-triclorobenceno                   \\ \midrule
        j) & \chemfig{O=-[::-60]N(-CH_3)-[::-60]H_3C}     & \textit{N},\textit{N}-dimetilformamida  \\ \midrule
        \bottomrule
    \end{tabularx}
    \caption{Ejercicio 36, página 52, Formulación Santillana}
\end{table}

\begin{table}
    \centering
    \begin{tabularx}{\textwidth}{rllX}
           & Fórmula & Nombre tradicional & Nombre de hidrógeno \\
        \toprule
        a) & \ch{HF}        & ácido fluorhídrico    & hidrogeno(fluoruro)                 \\ \midrule
        b) & \ch{H2Se}      & ácido selenhídrico    & dihidrogeno(selenuro)               \\ \midrule
        c) & \ch{HClO}      & ácido hipocloroso     & hidrógeno(oxidoclorato)             \\ \midrule
        d) & \ch{H2SO4}     & ácido sulfúrico       & dihidrogeno(tetraoxidosulfato)      \\ \midrule
        e) & \ch{HMnO4}     & ácido permangánico    & hidrogeno(tetraoxidomanganato)      \\ \midrule
        f) & \ch{HI}        & ácido yodhídrico      & hidrogeno(yoduro)                   \\ \midrule
        g) & \ch{H2Te}      & ácido telurhídrico    & dihidrogeno(telururo)               \\ \midrule
        h) & \ch{H2Cr2O7}   & ácido dicrómico       & dihidrogeno(heptaoxidodicromato)    \\ \midrule
        i) & \ch{H2CrO4}    & ácido crómico         & dihidrogeno(tetraoxidocromato)      \\ \midrule
        j) & \ch{HBrO4}     & ácido perbrómico      & hidrogeno(tetraoxidobromato)        \\ \midrule
        \bottomrule
    \end{tabularx}
    \caption{Ejercicio 34, página 28, Formulación Santillana}
\end{table}

\begin{table}
    \centering
    \begin{tabularx}{\textwidth}{rllX}
           & Fórmula & Nombre tradicional & Nombre de hidrógeno \\
        \toprule
        a) & \ch{HCl}     & ácido clorhídrico     & cloruro de hidrógeno                \\ \midrule
        b) & \ch{H2MnO4}  & ácido mangánico       & dihidrogeno(tetraoxidomanganato)    \\ \midrule
        c) & \ch{H2S}     & ácido sulfhídrico     & sulfuro de dihidrógeno              \\ \midrule
        d) & \ch{H3As04}  & ácido arsénico        & trihidrogeno(tetraoxidoarsenato)    \\ \midrule
        e) & \ch{H3PO4}   & ácido fosfórico       & trihidrogeno(tetraoxidofosfato)     \\ \midrule
        f) & \ch{H3PO3}   & ácido fosforoso       & trihidrogeno(trioxidofosfato)       \\ \midrule
        g) & \ch{HNO}     & ácido hiponitroso     & hidrogeno(óxidonitrato)             \\ \midrule
        h) & \ch{H3AsO3}  & ácido arsenoso        & trihidrogeno(trioxidoarsenato)      \\ \midrule
        i) & \ch{H2TeO2}  & ácido hipoteluroso    & dihidrogeno(dióxidotelurato)        \\ \midrule
        j) & \ch{HBr}     & ácido bromhídrico     & bromuro de hidrógeno                \\ \midrule
        \bottomrule
    \end{tabularx}
    \caption{Ejercicio 35, página 28, Formulación Santillana}
\end{table}


\section{Ácidos carboxílicos}

Esto son ejercicios de la página 43 del Santillana

a) \ch{HOOC-COOH}
ácido etanodioico

b) \ch{CH+C-CH2-COOH}
ácido but-3-inoico

c) \ch{HOOC-CH=CH-C+C-CH3}
ácido hex-2-en-4-inoico

d) \chemfig{CH_2(-[4]*6(------))-CH_2-C(=[:50]O)-[:-50]OH}
ácido 3-ciclohexilpropanoico

e) \ch{HOOC-CH2-CH2-CH3}
ácido butanoico

f) \ch{CH3-CH2-CH2-CH2-COOH}
ácido pentanoico

g) \ch{CH3-COOH}
ácido etanoico (ácido acético)

h)
ácido isopropilpropanodioico

a) ácido fórmico
\ch{HCOOH}

b) ácido but-3-enoico
\ch{CH2=Ch-CH2-COOH}

c) ácido ciclobutiletanoico

d) ácido feniletanoico

e) ácido butanodioico
\ch{HOOC-CH2-CH2-COOH}

f) ácido but-2-enoico
\ch{CH3-CH=CH-COOH}

g) ácido 3-isopropilbenzoico


h) ácido 2,4-diclorobenzoico

\end{document}
