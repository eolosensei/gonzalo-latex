\section{Ejercicios de nomenclatura}

  \begin{exercise}[
      tags    = {orgánica,formulación,múltiple,2B},
      topics  = {química orgánica,formulación,nomenclatura},
      source  = {Química 2B SAN 2016, p391, e11},
    ]
    Nombra los siguientes compuestos:
  %  \renewcommand\printatom[1]{\fbox{\ensuremath{\mathrm#1}}}
    \begin{enumerate}
      \item \chemfig{CH_2=CH-CH(-[6]CH_3)-CH_3}
      \item \ch{CH3-CH2-O-CH2-CH3}
      \item \chemfig{CH_3-CH(-[2]CH_3)-CH_2-CH_3}
      \item \ch{CH3-CH2-C+C-CH3}
      \item \ch{CH3-CH2-CH2-CH2-CH2-CH2-NH2}
      \item \ch{CH3-CO-CH3}
      \item \ch{CH3-CH2OH}
      \item \ch{CH3-COO-CH2-CH3}
    \end{enumerate}
  \end{exercise}

  \begin{solution}
    \begin{enumerate}
      \item 3-metilbut-1-eno
      \item dietil éter / etoxietano
      \item metilbutano
      \item pent-2-ino
      \item hexanamina
      \item dimetil cetona / propanona
      \item etanol
      \item etanoato de etilo / acetato de etilo
    \end{enumerate}
  \end{solution}

  \begin{exercise}[
      tags    = {orgánica,formulación,múltiple,2B},
      topics  = {química orgánica,formulación,nomenclatura},
      source  = {Química 2B SAN 2016, p391, e12},
    ]
    Nombra los siguientes compuestos:

    \begin{enumerate}
      \item \ch{CH3-CH2-CH2-COOH}
      \item \ch{CH3-CHCl2}
      \item \ch{CH+CH}
      \item \ch{CH3-CO-(CH2)4-CH3}
      \item \ch{CH3-CHOH-CH3}
      \item \chemfig{[:-120]**6(-----(-COOH)-)}
      \item \chemfig{CH_3-CH(-[6]CH_3)-CH(-[6]CH_3)-CH_2-CH(-[6]CH_3)-CH_3}
      \item \ch{CHCl=CH-CH_3}
    \end{enumerate}
  \end{exercise}

\begin{solution}
  \begin{enumerate}
    \item ácido butanoico
    \item 1,1-dicloroetano
    \item etino
    \item pentil metil cetona / heptan-2-ona
    \item propan-2-ol
    \item ácido benzoico / ácido bencenocarboxílico
    \item 2,3,5-trimetilhexano
    \item 1-cloropropeno
  \end{enumerate}
\end{solution}


\begin{exercise}[
    tags    = {orgánica,formulación,múltiple,2B},
    topics  = {química orgánica,formulación,nomenclatura},
    source  = {Química 2B SAN 2016, p391, e13},
  ]
  Nombra los siguientes compuestos:

  \begin{enumerate}
    \item \chemfig{H_3C-CH(-[6]CH_3)-CH(=[6]O)-CH_2-CH(-[6]CH_3)-CH_3}
    \item \ch{CH3-CH2-CO-CH3}
    \item \ch{CH2=CHOH}
    \item \ch{CH2=CH2}
    \item \ch{CH3-CH2-NH-CH3}
    \item \ch{CH3-CH=CH-CO-CH3}
    \item \ch{CH3-CH2-CO-NH2}
    \item \ch{CH3-CH2-CH=CH-C+C-CH=CH-CH3}
  \end{enumerate}
\end{exercise}

\begin{solution}
  \begin{enumerate}
    \item 2,5-dimetilhexan-3-ona / isobutil isopropil cetona
    \item butanona / etil metil cetona
    \item etenol
    \item eteno
    \item etil metil amina / N-metiletanamina
    \item pent-3-en-2-ona / metil prop-1-enil cetona
    \item propanamida
    \item nona-2,6-dien-4-ino
  \end{enumerate}
\end{solution}

\begin{exercise}[
    tags    = {orgánica,formulación,múltiple,2B},
    topics  = {química orgánica,formulación,nomenclatura},
    source  = {Química 2B SAN 2016, p391, e14},
  ]
  Nombra los siguientes compuestos:

  \begin{enumerate}
    \item \ch{CH3-CH2-CH2-CH2-CHO}
    \item \ch{CH3-COOH}
    \item \ch{CH2=CH-CH=CH-CH2-COOH}
    \item \chemfig{CH_3-C(-[2]OH)(-[6]OH)-CH_2-CH_2-CHO}
    \item \ch{CH3-CH2-NH-CH2-CH3}
    \item \chemfig{CH_3-CH(-[6]CH_3)-COO-CH_2-CH_3}
    \item \ch{CH2=CH-CH=CH-CHO}
    \item \ch{CH+C-CH2-COOH}
  \end{enumerate}
\end{exercise}

\begin{solution}
  \begin{enumerate}
    \item pentanal
    \item ácido etanoico / ácido acético
    \item ácido hexa-3,5-dienoico
    \item 4,4-dihidroxipentanal
    \item dietil amina / N-etiletanamina
    \item 2-metilpropanoato de etilo
    \item penta-2,4-dienal
    \item ácido but-3-inoico
  \end{enumerate}
\end{solution}


\begin{exercise}[
    tags    = {orgánica,formulación,múltiple,2B},
    topics  = {química orgánica,formulación,nomenclatura},
    source  = {Química 2B SAN 2016, p391, e15},
  ]
  Nombra los siguientes compuestos:

  \begin{enumerate}
    \item \ch{CH3-CO-CO-CH3}
    \item \ch{CH3-CH2-CHOH-CONH2}
    \item \chemfig{CH~C-CH(-[6]CH_3)-CO-CH_2-CH_3}
    \item \ch{CH3-CH2-CH2-NH2}
    \item \ch{CH+C-CH=CH-CHCl-C+CH}
    \item \chemfig{NC-CH_2-CH(-[6]CH_3)-CH_3}
    \item \ch{CH3-CO-CHOH-CH3}
    \item \ch{CH2=CH-CH2-CH=CH2}
  \end{enumerate}
  \end{exercise}

  \begin{solution}
    \begin{enumerate}
      \item butanodiona
      \item 2-hidroxibutanamida
      \item 4-metilhex-5-in-3-ona
      \item propanamina
      \item 5-clorohept-3-eno-1,6-diino
      \item 3-metilbutanonitrilo
      \item 3-hidroxibutanona
      \item penta-1,4-dieno
    \end{enumerate}
  \end{solution}


  \begin{exercise}[
      tags    = {orgánica,formulación,múltiple,2B},
      topics  = {química orgánica,formulación,nomenclatura},
      source  = {Química 2B SAN 2016, p391, e16},
    ]
    Nombra los siguientes compuestos:

  \begin{enumerate}
    \item \ch{CH3-O-CH3}
    \item \chemfig{CH_3-CH(-[6]CH_3)-CO-CH_3}
    \item \chemfig{CH_3-CH(-[6]CH_3)-COOH}
    \item \ch{CH3-COO-CH3}
    \item \ch{CH2OH-CH3}
    \item \ch{CH3-NH2}
    \item \ch{CH3-CH2-CO-CH2-CH3}
    \item \ch{CH3-CHO}
  \end{enumerate}
\end{exercise}

\begin{solution}
  \begin{enumerate}
    \item dimetil éter / metoximetano
    \item 3-metilbutan-2-ona / isopropil metil cetona
    \item ácido 2-metilpropanoico
    \item etanoato de metilo / acetato de metilo
    \item hidroxietanal
    \item metanamina
    \item dietil cetona / pentan-3-ona
    \item etanal
  \end{enumerate}
\end{solution}
