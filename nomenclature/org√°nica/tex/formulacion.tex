\section{Ejercicios de formulación}

\begin{exercise}[
    tags    = {orgánica,nomenclatura,múltiple,2B},
    topics  = {química orgánica,formulación,nomenclatura},
    source  = {Química 2B SAN 2016, p391, e17},
  ]
  Formula los siguientes compuestos:

  \begin{enumerate}
    \item ácido 2-metilpropanoico
    \item etanoato de potasio
    \item 1,2-bencenodiol
    \item naftaleno
    \item trifenilamina
    \item clorobenceno
    \item ácido 2-metilpentanoico
    \item metilamina
  \end{enumerate}
\end{exercise}

\begin{solution}
  \begin{enumerate}
    \item \ch{CH3-CH(CH3)-COOH}
    \item \ch{CH3-COO-K}
    \item \chemfig{[:-120]**6(----(-OH)-(-OH)-)}
    \item \chemfig{[:-120]**6(----**6(------)--)}
    \item \chemfig{[:-120]**6(----(-N(-[:-90]**6(------))-[:30]**6(------))--)}
    \item \chemfig{[:-120]**6(-----(-Cl)-)}
    \item \ch{CH3-CH2-CH2-CH(CH3)-COOH}
    \item \ch{CH3-NH2}
  \end{enumerate}
\end{solution}


\begin{exercise}[
    tags    = {orgánica,nomenclatura,múltiple,2B},
    topics  = {química orgánica,formulación,nomenclatura},
    source  = {Química 2B SAN 2016, p391, e17},
  ]
  Formula los siguientes compuestos:

  \begin{enumerate}
    \item ácido propenoico
    \item butil metil amina
    \item etil propil éter
    \item 2-buteno
    \item 4-metilhexan-1-ol
    \item tolueno
    \item 2,3-dimetilbut-1-eno
    \item propanamida
  \end{enumerate}
\end{exercise}

\begin{solution}
  \begin{enumerate}
    \item \ch{CH2=CH-COOH}
    \item \ch{CH3-CH2-CH2-CH2-NH-CH3}
    \item \ch{CH3-CH2-O-CH2-CH2-CH3}
    \item \ch{CH3-CH=CH-CH3}
    \item \ch{CH2OH-CH2-CH2-CH(CH3)-CH2-CH3}
    \item \chemfig{[:-120]**6(-----(-CH_3)-)}
    \item \ch{CH2=C(CH3)-CH(CH3)-CH3}
    \item \ch{CH3-CH2-CONH2}
  \end{enumerate}
\end{solution}


\begin{exercise}[
    tags    = {orgánica,nomenclatura,múltiple,2B},
    topics  = {química orgánica,formulación,nomenclatura},
    source  = {Química 2B SAN 2016, p392, e19},
  ]
  Formula los siguientes compuestos:

  \begin{enumerate}
    \item 1-bromo-2,2-diclorobutano
    \item 2-metilhexa-1,5-dien-3-ino
    \item trimetilamina
    \item butanoato de 2-metilpropano
    \item 3-etil-4,4-dimetilheptano
    \item \textit{N}-metil-\textit{N}-etilpentanamina
    \item ácido 3-aminohexanoico
    \item \textit{N}-etilbutanamida
  \end{enumerate}
\end{exercise}

\begin{solution}
  \begin{enumerate}
    \item \ch{CH2Br-CCl2-CH2-CH3}
    \item \ch{CH2=C(CH3)-C+C-CH=CH2}
    \item \ch{N(CH3)3}
    \item \ch{CH3-CH2-CH2-COO-CH2-CH(CH3)-CH3}
    \item \ch{CH3-CH2-CH(CH2-CH3)-C(CH3)-CH2-CH2-CH3}
    \item \ch{CH3-CH2-N(CH3)-CH2-CH2-CH2-CH2-CH3}
    \item \ch{CH3-CH2-CH2-CH(NH2)-CH2-COOH}
    \item \ch{CH3-CH2-NH-CO-CH2-CH2-CH3}
  \end{enumerate}
\end{solution}



\begin{exercise}[
    tags    = {orgánica,nomenclatura,múltiple,2B},
    topics  = {química orgánica,formulación,nomenclatura},
    source  = {Química 2B SAN 2016, p392, e20},
  ]
  Formula los siguientes compuestos:

  \begin{enumerate}
    \item trietilamina
    \item 2,2-dimetilbutanamida
    \item but-1-ino
    \item \textit{trans}-but-2-eno
    \item pentan-3-ona
    \item 1,1-difluoro-2,2-dicloropropano
    \item 2,5,6-trimetilnonano
    \item difenilcetona
  \end{enumerate}
\end{exercise}

\begin{solution}
  \begin{enumerate}
    \item \ch{N(CH2-CH3)3}
    \item \ch{CH3-CH2-C(CH3)2-CONH2}
    \item \ch{CH+C-CH2-CH3}
    \item \chemfig{H-[:-60]C(-[:-120]H_3C)=C(-[:-60]H)-[:60]CH_3}
    \item \ch{CH3-CH2-CO-CH2-CH3}
    \item \ch{CHF2-CCl2-CH3}
    \item \ch{CH3-CH(CH3)-CH2-CH2-CH(CH3)-CH(CH3)-CH2-CH2-CH3}
    \item \chemfig{CO(-[:-150]**6(------))-[:-30]**6(------)}
  \end{enumerate}
\end{solution}



\begin{exercise}[
    tags    = {orgánica,nomenclatura,múltiple,2B},
    topics  = {química orgánica,formulación,nomenclatura},
    source  = {Química 2B SAN 2016, p392, e21},
  ]
  Formula los siguientes compuestos:

  \begin{enumerate}
    \item pentan-2-ol
    \item acetato de etilo
    \item propan-2-ol
    \item pent-2-eno
    \item 1-cloro-1,1-difluoroetano
    \item 5,6-dietil-3-metildecano
    \item dietil metil amina
    \item 2-metilpropan-2-ol
  \end{enumerate}
\end{exercise}

\begin{solution}
  \begin{enumerate}
    \item \ch{CH3-CH2-CH2-CHOH-CH3}
    \item \ch{CH3-COO-CH2-CH3}
    \item \ch{CH3-CHOH-CH3}
    \item \ch{CH3-CH=CH-CH2-CH3}
    \item \ch{CClF2-CH3}
    \item \chemfig{CH_3-CH_2-CH(-[6]CH_3)-CH_2-CH(-[6]CH_2-[6]CH_3)-CH(-[6]CH_2-[6]CH_3)-CH_2-CH_2-CH_2-CH_3}
    \item \ch{CH3-CH2-N(CH3)-CH2-CH3}
    \item \ch{CH3-C(CH3)OH-CH3}
  \end{enumerate}
\end{solution}



\begin{exercise}[
    tags    = {orgánica,nomenclatura,múltiple,2B},
    topics  = {química orgánica,formulación,nomenclatura},
    source  = {Química 2B SAN 2016, p392, e22},
  ]
  Formula los siguientes compuestos:

  \begin{enumerate}
    \item 3,4-dimetilhept-2-eno
    \item penta-1,3-dieno
    \item etanamina
    \item butan-2-ol
    \item dietilamina
    \item antraceno
    \item propino
    \item \textit{o}-dimetilbenceno
  \end{enumerate}
\end{exercise}

\begin{solution}
  \begin{enumerate}
    \item \ch{CH3-CH=C(CH3)-CH(CH3)-CH2-CH2-CH3}
    \item \ch{CH2=CH-CH=CH-CH3}
    \item \ch{CH3-CH2-NH2}
    \item \ch{CH3-CH2-CHOH-CH3}
    \item \ch{CH3-CH2-NH-CH2-CH3}
    \item \chemfig{[:-120]**6(----**6(--**6(------)----)--)}
    \item \ch{CH3-C+CH}
    \item \chemfig{[:-120]**6(----(-CH_3)-(-CH_3)-)}
  \end{enumerate}
\end{solution}



\begin{exercise}[
    tags    = {orgánica,nomenclatura,múltiple,2B},
    topics  = {química orgánica,formulación,nomenclatura},
    source  = {Química 2B SAN 2016, p392, e23},
  ]
  Formula los siguientes compuestos:

  \begin{enumerate}
    \item nitrobenceno
    \item propanal
    \item metanol
    \item ácido benzoico
    \item but-2-eno
    \item ácido metanoico
    \item propan-1-ol
    \item pentanal
  \end{enumerate}
\end{exercise}

\begin{solution}
  \begin{enumerate}
    \item \chemfig{[:-120]**6(-----(-NO_2)-)}
    \item \ch{CH3-CH2-CHO}
    \item \ch{CH3OH}
    \item \chemfig{[:-120]**6(-----(-COOH)-)}
    \item \ch{CH3-CH=CH-CH3}
    \item \ch{HCOOH}
    \item \ch{CH3-CH2-CH2OH}
    \item \ch{CH3-CH2-CH2-CH2-CHO}
  \end{enumerate}
\end{solution}



\begin{exercise}[
    tags    = {orgánica,nomenclatura,múltiple,2B},
    topics  = {química orgánica,formulación,nomenclatura},
    source  = {Química 2B SAN 2016, p392, e24},
  ]
  Formula los siguientes compuestos:

  \begin{enumerate}
    \item etanoato de etilo
    \item 3,4-dimetilpent-1-ino
    \item metilbenceno
    \item dietilamina
    \item metilbutanona
    \item N-metilacetamida
    \item pentanamina
    \item propano-1,2-diol
  \end{enumerate}
\end{exercise}

\begin{solution}
  \begin{enumerate}
    \item \ch{CH3-COO-CH2-CH3}
    \item \ch{CH+C-CH(CH3)-CH(CH3)-CH3}
    \item \chemfig{[:-120]**6(-----(-CH_3)-)}
    \item \ch{CH3-CH2-NH-CH2-CH3}
    \item \ch{CH3-CH(CH3)-CO-CH3}
    \item \ch{CH3-CONH-CH3}
    \item \ch{CH3-CH2-CH2-CH2-CH2-NH2}
    \item \ch{CH3-CHOH-CH2OH}
  \end{enumerate}
\end{solution}
