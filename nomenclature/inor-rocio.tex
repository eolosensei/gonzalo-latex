\documentclass[10pt]{article}
  \title{Formulación inorgánica}
  \author{\textit{Gonzalo Esteban}}

\usepackage{polyglossia}
    \setdefaultlanguage{spanish}
\usepackage{fontspec}
    \setmainfont{Fira Sans}
\usepackage{amsmath, amsthm, amssymb}
\usepackage{unicode-math}
  \unimathsetup{
    math-style  = ISO,
    bold-style  = ISO
  }
  \setmathfont{Fira Math}
\usepackage{multicol}
  \setlength{\columnsep}{1cm}
\usepackage[top=2cm, bottom=3cm, left=2.5cm, right=2cm]{geometry}
\usepackage{xsim}
% Configuración general de xsim
  \loadxsimstyle{layouts}
  \xsimsetup{
    path                  = {xsim-files},
    exercise/template     = {runin},
    exercise/name         = {},
    exercise/print        = {true},
    solution/template     = {runin},
    solution/name         = {S},
    solution/print        = {false},
    exercise/within       = section,
    exercise/the-counter  = \thesection.\arabic{exercise}
  }
  \DeclareExerciseProperty{source}
\usepackage{siunitx}
% Exponent symbol options: \times for the typical cross
  \sisetup{
    per-mode                = symbol,
    output-decimal-marker   = {,},
    exponent-product        = \cdot,
    text-celsius            = ^^b0\kern -\scriptspace C,  % soluciona problemas con el símbolo de grados
    math-celsius            = ^^b0\kern -\scriptspace C,
    list-final-separator    = { y },
    list-pair-separator     = { y },
    range-phrase            = { \translate{to (numerical range)} }
  }
\usepackage[inline]{enumitem}
% Configuración general de enumitem
% Establece la configuración por defecto en a), b), c)
  \setlist[enumerate,1]{
    label   =\alph*),
    itemsep = 0.3\itemsep,
  }
\usepackage{chemformula}
\usepackage{chemfig}
  \setchemfig{
    atom sep = 2em
  }
\usepackage{booktabs}
\usepackage{tasks}
\usepackage{fancyhdr}
  \pagestyle{fancy}
  \lhead{\textbf{Formulación inorgánica}}
  \chead{}
  \rhead{1 BACH}
  \lfoot{}
  \cfoot{\thepage}
  \rfoot{}
  \renewcommand{\headrulewidth}{0.2pt}
  \renewcommand{\footrulewidth}{0pt}




\newenvironment{gexdatos}{
  \noindent\makebox[0pt][r]{\textit{Datos:}}
  }{\vspace{5pt}}

\newcommand{\gexBinRow}[3]{
  \ch{#1} & #2 & #3 \\ \midrule
}














\begin{document}

\maketitle

\begin{multicols*}{2}[
  \section{Ejercicios variados}
  ]

\begin{exercise}[
    tags    = {inorgánica,compuestos binarios,sales binarias,sales,hidruros},
    topics  = {química inorgánica,formulación,nomenclatura},
    source  = {SAN Formulación, p26, e28},
  ]
  Nombra y formula los siguientes compuestos binarios, indicando cuando proceda su nomenclatura de composición (sistemática) y su nombre de Stock:

  \begin{enumerate}
    \item \ch{KH}
    \item yoduro de cesio
    \item hidruro de titanio(III)
    \item \ch{SrCl2}
    \item cloruro de hidrógeno
    \item arseniuro de galio
    \item \ch{Li3N}
    \item \ch{CF4}
    \item yoduro de plomo(II)
    \item seleniuro de hidrógeno
  \end{enumerate}
\end{exercise}

\begin{solution}
  \begin{enumerate}
    \item hidruro de potasio, hidruro de potasio
    \item \ch{CsI}, yoduro de cesio
    \item \ch{TiH3}, trihidruro de titanio
    \item dicloruro de estroncio, cloruro de estroncio
    \item \ch{HCl}, cloruro de hidrógeno / ácido clorhídrico
    \item \ch{GaAs}, arsenuro de galio
    \item nitruro de trilitio, nitruro de litio
    \item tetrafluoruro de carbono, fluoruro de carbono(IV)
    \item \ch{PbI2}, diyoduro de plomo
    \item \ch{H2Se}, selenuro de dihidrógeno
  \end{enumerate}
\end{solution}



\begin{exercise}[
    tags    = {inorgánica,compuestos binarios,sales binarias,sales},
    topics  = {química inorgánica,formulación,nomenclatura},
    source  = {SAN Formulación, p26, e29},
  ]
  Nombra y formula los siguientes compuestos binarios, indicando cuando proceda su nomenclatura de composición (sistemática) y su nombre de Stock:

  \begin{enumerate}
    \item pentacloruro de fósforo
    \item cloruro de antimonio(III)
    \item \ch{ScI3}
    \item bromuro de plata
    \item cloruro de mercurio(II)
    \item \ch{Zn3P2}
    \item teluluro de cadmio
    \item bromuro de platino(II)
    \item \ch{Na2Se}
    \item tricloruro de oro
  \end{enumerate}
\end{exercise}

\begin{solution}
  \begin{enumerate}
    \item \ch{PCl5}, cloruro de fósforo(V)
    \item \ch{SbCl3}, tricloruro de antimonio
    \item triyoduro de escandio, yoduro de escandio
    \item \ch{AgBr}, bromuro de plata
    \item \ch{HgCl2}, dicloruro de mercurio;
    \item difosfuro de trizinc, fosfuro de zinc
    \item \ch{CdTe}, telururo de cadmio
    \item \ch{PtBr2}, dibromuro de platino
    \item selenuro de disodio, selenuro de sodio
    \item \ch{AuCl3}, cloruro de oro(III)
  \end{enumerate}
\end{solution}




\begin{exercise}[
    tags    = {inorgánica,compuestos binarios,sales binarias,sales},
    topics  = {química inorgánica,formulación,nomenclatura},
    source  = {SAN Formulación, p26, e30},
  ]
  Nombra y formula los siguientes compuestos binarios, indicando cuando proceda su nomenclatura de composición (sistemática) y su nombre de Stock:

  \begin{enumerate}
    \item teluluro de hidrógeno
    \item \ch{SiC}
    \item dibromuro de calcio
    \item fosfuro de plata
    \item \ch{NiAs}
    \item trisulfuro de diantimonio
    \item hidruro de bismuto(III)
    \item heptafluoruro de yodo
    \item \ch{H2S}
    \item dihidruro de cobre
  \end{enumerate}
\end{exercise}

\begin{solution}
  \begin{enumerate}
    \item \ch{H2Te}, telururo de dihidrógeno
    \item carburo de silicio, carburo de silicio(IV)
    \item \ch{CaBr2}, bromuro de calcio
    \item \ch{Ag3P}, fosfuro de triplata
    \item arsenuro de níquel, arsenuro de níquel(III)
    \item \ch{Sb2S3}, sulfuro de antimonio(III)
    \item \ch{BiH3}, trihidruro de bismuto
    \item \ch{IF7}, fluoruro de yodo(VII)
    \item sulfuro de dihidrógeno, sulfuro de hidrógeno
    \item \ch{CuH2}, sulfuro de cobre(II)
  \end{enumerate}
\end{solution}




\begin{exercise}[
    tags    = {inorgánica,compuestos binarios,óxidos,peróxidos},
    topics  = {química inorgánica,formulación,nomenclatura},
    source  = {SAN Formulación, p27, e31},
  ]
  Nombra y formula los siguientes compuestos binarios, indicando cuando proceda su nomenclatura de composición (sistemática) y su nombre de Stock:

  \begin{enumerate}
    \item \ch{FeO}
    \item trióxido de bromo
    \item óxido de mercurio(II)
    \item \ch{Li2O2}
    \item dióxido de magnesio
    \item peróxido de aluminio
    \item pentaóxido de dinitrógeno
    \item \ch{SO2}
    \item monóxido de carbono
    \item óxido de selenio(IV)
  \end{enumerate}
\end{exercise}

\begin{solution}
  \begin{enumerate}
    \item monóxido de hierro, óxido de hierro(II)
    \item \ch{Cr2O3}, óxido de cromo(III)
    \item \ch{HgO}, monóxido de mercurio
    \item dióxido de dilitio, peróxido de litio
    \item \ch{MgO2}, peróxido de magnesio
    \item \ch{Al2O6}, hexaóxido de dialuminio
    \item \ch{N2O5}, óxido de nitrógeno(V)
    \item dióxido de azufre, óxido de azufre(IV)
    \item \ch{CO}, óxido de carbono(II)
    \item \ch{SeO3}, trióxido de selenio
  \end{enumerate}
\end{solution}




\begin{exercise}[
    tags    = {inorgánica,compuestos binarios,óxidos,peróxidos},
    topics  = {química inorgánica,formulación,nomenclatura},
    source  = {SAN Formulación, p27, e32},
  ]
  Nombra y formula los siguientes compuestos binarios, indicando cuando proceda su nomenclatura de composición (sistemática) y su nombre de Stock:

  \begin{enumerate}
    \item \ch{O3Cl2}
    \item dióxido de manganeso
    \item óxido de fósforo(V)
    \item \ch{CrO3}
    \item diyoduro de heptaoxígeno
    \item peróxido de estaño(IV)
    \item \ch{H2O2}
    \item dióxido de dilitio
    \item óxido de titanio(II)
    \item \ch{SiO2}
  \end{enumerate}
\end{exercise}

\begin{solution}
  \begin{enumerate}
    \item dicloruro de trioxígeno, -
    \item \ch{MnO2}, óxido de manganeso(IV)
    \item \ch{P2O5}, pentaóxido de difósforo
    \item trióxido de cromo, óxido de cromo(VI)
    \item \ch{O7I2}, -
    \item \ch{SnO4}, tetraóxido de estaño
    \item dióxido de dihidrógeno, peróxido de hidrógen
    \item \ch{Li2O2}, peróxido de litio
    \item \ch{TiO}, monóxido de titanio
    \item dióxido de silicio, óxido de silicio(IV)
  \end{enumerate}
\end{solution}




\begin{exercise}[
    tags    = {inorgánica,compuestos binarios,óxidos,peróxidos,hidróxidos},
    topics  = {química inorgánica,formulación,nomenclatura},
    source  = {SAN Formulación, p27, e33},
  ]
  Nombra y formula los siguientes compuestos binarios, indicando cuando proceda su nomenclatura de composición (sistemática) y su nombre de Stock:

  \begin{enumerate}
    \item dióxido de bario
    \item hidróxido de zinc
    \item \ch{Au(OH)3}
    \item hidróxido de plata
    \item hidróxido de magnesio
    \item \ch{NH4OH}
    \item dióxido de paladio
    \item peróxido de rubidio
    \item \ch{TeO}
    \item trióxido de dicobalto
  \end{enumerate}
\end{exercise}

\begin{solution}
  \begin{enumerate}
    \item \ch{Ba(OH)2}, hidróxido de bario
    \item \ch{Zn(OH)2}, dihidróxido de zinc
    \item trihidróxido de oro, hidróxido de oro(III)
    \item \ch{AgOH}, hidróxido de plata
    \item \ch{Mg(OH)2}, dihidróxido de magnesio
    \item hidróxido de amonio, hidróxido de amonio
    \item \ch{PdO2}, óxido de paladio(IV)
    \item \ch{Rb2O2}, dióxido de dirubidio
    \item monóxido de teluro, óxido de teluro(II)
    \item \ch{Co2O3}, óxido de cobalto(III)
  \end{enumerate}
\end{solution}




\begin{exercise}[
    tags    = {inorgánica,ácidos,ácidos binarios,ácidos ternarios,oxoácidos},
    topics  = {química inorgánica,formulación,nomenclatura},
    source  = {SAN Formulación, p28, e34},
  ]
  Nombra y formula los siguientes compuestos, indicando cuando proceda su nomenclatura de hidrógeno y su nombre tradicional:

  \begin{enumerate}
    \item \ch{HF}
    \item ácido selenhídrico
    \item hidrogeno(oxidoclorato)
    \item dihidrogeno(tetraoxidosulfato)
    \item \ch{HMnO4}
    \item \ch{HI}
    \item dihidrogeno(teluro)
    \item ácido crómico
    \item dihidrogeno(tetraoxidocromato)
    \item hidrogeno(tetraoxidobromato)
  \end{enumerate}
\end{exercise}

\begin{solution}
  \begin{enumerate}
    \item ácido fluorhídrico, hidrogeno(fluoruro)
    \item \ch{H2Se}, dihidrogeno(selenuro)
    \item \ch{HClO}, ácido hipocloroso
    \item \ch{H2SO4}, ácido sulfúrico
    \item ácido permangánico,  hidrogeno(tetraoxidomanganato)
    \item ácido yodhídrico, hidrogeno(yoduro)
    \item \ch{H2Te}, ácido telurhídrico
    \item \ch{H2Cr2O7}, dihidrogeno(heptaoxidodicromato)
    \item \ch{H2CrO4}, ácido crómico
    \item \ch{HBrO4}, ácido perbrómico
  \end{enumerate}
\end{solution}




\begin{exercise}[
  tags    = {inorgánica,ácidos,ácidos binarios,ácidos ternarios,oxoácidos},
  topics  = {química inorgánica,formulación,nomenclatura},
  source  = {SAN Formulación, p28, e35},
  ]
  Nombra y formula los siguientes compuestos, indicando cuando proceda su nomenclatura de hidrógeno y su nombre tradicional:

  \begin{enumerate}
    \item ácido clorhídrico
    \item ácido mangánico
    \item \ch{H2S}
    \item ácido arsénico
    \item trihidrógeno(tetraoxidofosfato)
    \item \ch{H3PO3}
    \item ácido hiponitroso
    \item trihidrógeno(trioxidoarsenato)
    \item \ch{H2TeO2}
    \item bromuro de hidrógeno
  \end{enumerate}
\end{exercise}

\begin{solution}
  \begin{enumerate}
    \item \ch{HCL}, cloruro de hidrógeno
    \item \ch{H2MnO4}, dihidrogeno(tetraoxidomanganato)
    \item ácido sulfhídrico, sulfuro de dihidrógeno
    \item \ch{H3AsO4}, trihidrogeno(tetraoxidoarsenato)
    \item \ch{H3PO4}, ácido fosfórico
    \item ácido fosforoso, trihidrogeno(trioxidofosfato)
    \item \ch{HNO}, hidrogeno(óxidonitrato)
    \item \ch{H3AsO3}, ácido arsenoso
    \item ácido hipoteluroso, dihidrogeno(dióxidotelurato)
    \item \ch{HBr}, ácido bromhídrico
  \end{enumerate}
\end{solution}




\begin{exercise}[
    tags    = {inorgánica,ácidos,ácidos binarios,ácidos ternarios,oxoácidos},
    topics  = {química inorgánica,formulación,nomenclatura},
    source  = {SAN Formulación, p28, e36},
  ]
  Nombra y formula los siguientes compuestos, indicando cuando proceda su nomenclatura de hidrógeno y su nombre tradicional:

  \begin{enumerate}
    \item \ch{H3BO3}
    \item \ch{HNO3}
    \item ácido selenioso
    \item ácido nítrico
    \item \ch{H2SO3}
    \item ácido carbónico
    \item hidrógeno(dioxidoclorato)
    \item \ch{H2SeO4}
    \item ácido hipobromoso
    \item hidrógeno(trioxidoclorato)
  \end{enumerate}
\end{exercise}

\begin{solution}
  \begin{enumerate}
    \item ácido bórico, trihidrogeno(trioxidoborato)
    \item ácido nitroso, hidrogeno(dioxidonitrato)
    \item \ch{H2SeO3}, dihidrogeno(trioxidoselenato)
    \item \ch{HNO3}, hidrogeno(trioxidonitrato)
    \item ácido sulfuroso, dihidrogeno(trioxidosulfato)
    \item \ch{H2CO3}, dihidrogeno(trioxidocarbonato)
    \item \ch{HClO2}, ácido cloroso
    \item ácido selénico, dihidrogeno(tetraoxidoselenato)
    \item \ch{HBrO}, hidrogeno(oxidobromato)
    \item \ch{HCLO3}, ácido clórico
  \end{enumerate}
\end{solution}

\newpage


\begin{exercise}[
    tags    = {inorgánica,sales, sales binarias,sales ternarias},
    topics  = {química inorgánica,formulación,nomenclatura},
    source  = {SAN Formulación, p29, e37},
  ]
  Nombra y formula las siguientes sales, indicando cuando proceda su nomenclatura de composición (sistemática) y su nombre de Stock:

  \begin{enumerate}
    \item \ch{BeCl2}
    \item hidrogenosulfuro de calcio
    \item tetraoxidosulfato de disodio
    \item \ch{KMnO4}
    \item carbonato de amonio
    \item tris(hidrogenoselenuro) de níquel
    \item \ch{CoSO3}
    \item hidrogenosufito de zinc
    \item heptaoxidodicromato de dilitio
    \item \ch{Ca(CN)2}
  \end{enumerate}
\end{exercise}

\begin{solution}
  \begin{enumerate}
    \item cloruro de berilio, dicloruro de berilio
    \item \ch{Ca(HS)2}, bis(hidrogenosulfuro) de calcio
    \item \ch{Na2SO4}, sulfato de sodio
    \item permanganato de potasio, tetraoxidomanganato de potasio
    \item \ch{(NH4)2CO3}, trioxidocarbonato de bis(amonio)
    \item \ch{Ni(HSe)3}, hidrogenoselenuro de níquel(III)
    \item sulfito de cobalto(II), trioxidosulfato de cobalto
    \item \ch{Zn(HSO3)2}, bis[hidrogeno(trioxidosulfato)] de zinc
    \item \ch{Li2Cr2O7}, dicromato de litio
    \item cianuro de calcio, dicianuro de calcio
  \end{enumerate}
\end{solution}




\begin{exercise}[
    tags    = {inorgánica,sales, sales binarias,sales ternarias},
    topics  = {química inorgánica,formulación,nomenclatura},
    source  = {SAN Formulación, p29, e38},
  ]
  Nombra y formula las siguientes sales, indicando cuando proceda su nomenclatura de composición (sistemática) y su nombre de Stock:

  \begin{enumerate}
    \item nitrato de cromo(III)
    \item tris(tetraoxidoyodato) de galio
    \item \ch{Sn(ClO)4}
    \item fosfato de calcio
    \item tris(tetraoxidosulfato) de dialuminio
    \item \ch{Pb(NO2)4}
    \item cromato de cesio
    \item trioxidoborato de níquel
    \item \ch{MnSO4}
    \item fluoruro de bario
  \end{enumerate}
\end{exercise}

\begin{solution}
  \begin{enumerate}
    \item \ch{Cr(NO3)3}, tris(trioxidocromato) de cromo
    \item \ch{Ga(IO4)3}, peryodato de galio
    \item hipoclorito de estaño(IV), tetrakis(oxidoclorato) de estaño
    \item \ch{Ca3(PO4)2}, bis(tetraoxidofosfato) de tricalcio
    \item \ch{Al2(SO4)3}, sulfato de aluminio
    \item nitrito de plomo(IV), tetrakis(dioxidonitrato) de plomo
    \item \ch{Cs2CrO4}, tetraoxidocromato de dicesio
    \item \ch{NiBO3}, borato de níquel(III)
    \item sulfato de manganeso(II), tetraoxidosulfato de manganeso
    \item \ch{BaF2}, difluoruro de bario
  \end{enumerate}
\end{solution}




\begin{exercise}[
    tags    = {inorgánica,sales,sales binarias,sales ternarias},
    topics  = {química inorgánica,formulación,nomenclatura},
    source  = {SAN Formulación, p29, e39},
  ]
  Nombra y formula las siguientes sales, indicando cuando proceda su nomenclatura de composición (sistemática) y su nombre de Stock:

  \begin{enumerate}
    \item hidrogenotelururo de rubidio
    \item \ch{KClO2}
    \item sulfito de titanio(III)
    \item dioxidotelurato de dicobre
    \item \ch{NH4HSO2}
    \item hipoclorito de plata
    \item tetraoxidomanganato de dilitio
    \item \ch{Ca3(AsO3)2}
    \item yodato de magnesio
    \item trioxidonitrato de amonio
  \end{enumerate}
\end{exercise}

\begin{solution}
  \begin{enumerate}
    \item \ch{RbHTe}, hidrogenotelururo de rubidio
    \item clorita de potasio, dioxidoclorato de potasio
    \item \ch{Ti2(SO3)3}, tris(trioxidosulfato) de dititanio
    \item \ch{Cu2TeO2}, hipotelurito de cobre(I)
    \item hidrogenohiposulfito de amonio, hidrogeno(dioxidosulfato) de amonio
    \item \ch{AgClO}, oxidoclorato de plata
    \item \ch{Li2MnO4}, manganato de litio
    \item arsenito de calcio, bis(trioxidoarsenato) de tricalcio
    \item \ch{Mg(IO3)2}, bis(trioxidoyodato) de magnesio
    \item \ch{NH4NO3}, nitrato de amonio
  \end{enumerate}
\end{solution}




\begin{exercise}[
    tags    = {inorgánica,óxidos,compuestos binarios,2B},
    topics  = {química inorgánica,formulación,nomenclatura},
    source  = {Química 2B OXF 2016, p342, e5},
  ]
  A continuación se indica la fórmula o el nombre de una serie de compuestos. Completa la información mostrando la fórmula y el nombre de cada uno tanto con prefijos multiplicadores como con número de oxidación o número de carga.

  \begin{enumerate}
    \item dióxido de silicio
    \item \ch{Li2O}
    \item \ch{ZnO}
    \item óxido de cobre(I)
    \item \ch{CaO}
    \item \ch{Cr2O3}
    \item óxido de boro(III)
    \item óxido de magnesio
    \item \ch{SeO3}
    \item pentaóxido de dinitrógeno
  \end{enumerate}
\end{exercise}




\begin{exercise}[
    tags    = {inorgánica,compuestos ternarios,oxoácidos,2B},
    topics  = {química inorgánica,formulación,nomenclatura},
    source  = {Química 2B OXF 2016, p344, e8 y p345, e9},
    print   = false,
  ]
  Completa la tabla siguiente indicando la fórmula o el nombre de hidrógeno de cada compuesto:

  \begin{tabular}{cll}
    Formula      & Nom. de hidrógeno & Nom. tradicional \\ \toprule
    \gexBinRow{HClO3}{}{}
    \gexBinRow{}{trihidrogeno(trioxidoborato)}{ácido bórico}
    \gexBinRow{H2SO3}{}{}
    \gexBinRow{}{hidrogeno(tetraoxidoyodato)}{ácido peryódico}
    \gexBinRow{H2CrO4}{}{}
    \gexBinRow{}{dihidrogeno(trioxidotelurato)}{ácido teluroso}
    \gexBinRow{HBrO2}{}{}
    \gexBinRow{}{dihidrogeno(tetraoxidomanganato)}{ácido mangánico}
    \gexBinRow{H3AsO4}{}{}
    \gexBinRow{}{trihidrogeno(trioxidofosfato)}{ácido fosforoso}
    \gexBinRow{}{dihidrogeno(trioxidocarbonato)}{ácido carbónico}
    \gexBinRow{}{hidrogeno(tetraoxidomanganato)}{ácido permangánico}
    \gexBinRow{H3PO4}{}{}
    \gexBinRow{}{dihidrogeno(heptaoxidodicromato)}{ácido dicrómico}
    \gexBinRow{HNO}{}{}
    \gexBinRow{}{trihidrogeno(trioxidoarsenato)}{}
    \gexBinRow{HNO3}{}{}
    \gexBinRow{}{hidrogeno(oxidoclorato)}{ácido hipocloroso}
    \gexBinRow{H2SeO2}{}{}
  \end{tabular}
\end{exercise}




\begin{exercise}[
    tags    = {inorgánica,sales, sales ternarias, oxosales,2B},
    topics  = {química inorgánica,formulación,nomenclatura},
    source  = {Química 2B OXF 2016, p347, e12},
  ]
  Formula los siguientes compuestos

  \begin{enumerate}
    \item Hiposulfito de oro(III)
    \item Carbonato de sodio
    \item Dicromato de zinc
    \item Fosfato de aluminio
    \item Bromuro de amonio
    \item Perclorato de cromo(II)
    \item Bistrioxidoborato de trihierro
    \item Trioxidoarsenato de trisamonio
    \item Bis(hidrogenoselenuro) de mercurio
    \item Nitrito de cobalto(II)
    \item Oxidoyodato de oro
    \item Fosfito de magnesio
    \item Hidrogenosulfato de níquel(II)
    \item Clorato de cesio
    \item Sulfato de aluminio
    \item Tris(trioxidoclorato) de cobalto
    \item Hidrogenocarbonato de calcio
    \item Manganato de bismuto(III)
    \item Tris(trioxidonitrato) de cromo
    \item Yoduro de plomo(IV)
  \end{enumerate}
\end{exercise}


\end{multicols*}

\newpage














\begin{multicols*}{2}[
  \section{Ejercicios de nomenclatura}
  ]

\begin{exercise}[
    tags    = {inorgánica,nomenclatura,múltiple,2B},
    topics  = {química inorgánica,formulación,nomenclatura},
    source  = {Química 2B SAN 2016, p372, e6},
  ]
  Nombra las siguientes sustancias

  \begin{enumerate}\begin{multicols}{3}
    \item \ch{PCl3}
    \item \ch{Fe(OH)2}
    \item \ch{Sr(OH)2}
    \item \ch{SO2}
    \item \ch{Al2(SO4)3}
    \item \ch{H2SO4}
    \item \ch{H3PO4}
    \item \ch{Fe(HSO4)2}
    \item \ch{PbO2}
    \item \ch{N2O5}
    \item \ch{Fe(NO3)3}
    \item \ch{CsH}
  \end{multicols}\end{enumerate}
\end{exercise}

\begin{solution}
  \begin{enumerate}
    \item tricloruro de fósforo / cloruro de fósforo(III)
    \item dihidróxido de hierro / hidróxido de hierro(II)
    \item dihidróxido de estroncio / hidróxido de estroncio
    \item dióxido de azufre / óxido de azufre((IV)
    \item sulfato de aluminio
    \item ácido sulfuroso
    \item ácido fosfórico
    \item hidrogenosulfato de hierro(II)
    \item dióxido de plomo / óxido de plomo(IV)
    \item pentaóxido de dinitrógeno / óxido de nitrógeno(V)
    \item nitrato de hierro(III)
    \item hidruro de cesio
  \end{enumerate}
\end{solution}




\begin{exercise}[
    tags    = {inorgánica,nomenclatura,múltiple,2B},
    topics  = {química inorgánica,formulación,nomenclatura},
    source  = {Química 2B SAN 2016, p372, e7},
  ]
  Nombra las siguientes sustancias

  \begin{enumerate}\begin{multicols}{3}
    \item \ch{KMnO4}
    \item \ch{HgS}
    \item \ch{NaHSO4}
    \item \ch{Ag2O}
    \item \ch{NaHCO3}
    \item \ch{Ar}
    \item \ch{Ni(OH)2}
    \item \ch{Fe2S3}
    \item \ch{Ca(OH)2}
    \item \ch{NH4Cl}
    \item \ch{Na2O2}
    \item \ch{HClO4}
  \end{multicols}\end{enumerate}
\end{exercise}

\begin{solution}
  \begin{enumerate}
    \item permanganato de potasio
    \item sulfuro de mercurio / sulfuro de mercurio(II)
    \item hidrogenosulfato de sodio
    \item monóxido de diplata / óxido de plata
    \item hidrogenocarbonato de sodio
    \item argón
    \item dihidróxido de níquel / hidróxido de níquel(II)
    \item trisulfuro de dihierro / sulfuro de hierro(III)
    \item dihidróxido de calcio / hidróxido de calcio
    \item monocloruro de amonio / cloruro de amonio
    \item dióxido de disodio / peróxido de sodio
    \item ácido perclórico
  \end{enumerate}
\end{solution}




\begin{exercise}[
    tags    = {inorgánica,nomenclatura,múltiple,2B},
    topics  = {química inorgánica,formulación,nomenclatura},
    source  = {Química 2B SAN 2016, p372, e8},
  ]
  Nombra las siguientes sustancias

  \begin{enumerate}\begin{multicols}{3}
    \item \ch{PH3}
    \item \ch{HgSO3}
    \item \ch{SiH4}
    \item \ch{As2O5}
    \item \ch{B2O3}
    \item \ch{CaO}
    \item \ch{As2O3}
    \item \ch{NO3-}
    \item \ch{HBrO3}
    \item \ch{HIO3}
    \item \ch{Co(OH)3}
    \item \ch{H4P2O7}
  \end{multicols}\end{enumerate}
\end{exercise}

\begin{solution}
  \begin{enumerate}
    \item trihidruro de fósforo / fosfano
    \item sulfito de mercurio(II)
    \item tetrahidruro de silicio / silano
    \item pentaóxido de diarsénico / óxido de arsénico(V)
    \item trióxido de diboro / óxido de boro
    \item monóxido de calcio / óxido de calcio
    \item trióxido de diarsénico / óxido de arsénico(III)
    \item nitrato
    \item ácido brómico
    \item ácido yódico
    \item trihidróxido de cobalto / hidróxido de cobalto(III)
    \item ácido difosfórico
  \end{enumerate}
\end{solution}




\begin{exercise}[
    tags    = {inorgánica,nomenclatura,múltiple,2B},
    topics  = {química inorgánica,formulación,nomenclatura},
    source  = {Química 2B SAN 2016, p372, e9},
  ]
  Nombra las siguientes sustancias

  \begin{enumerate}\begin{multicols}{3}
    \item \ch{HCO3-}
    \item \ch{H2O2}
    \item \ch{Sn(IO3)2}
    \item \ch{HgCl2}
    \item \ch{Au2O3}
    \item \ch{BeH2}
    \item \ch{V2O5}
    \item \ch{BeO}
    \item \ch{Pt(OH)2}
    \item \ch{Ag3AsO4}
    \item \ch{Sn(OH)2}
    \item \ch{Ba3(PO4)2}
  \end{multicols}\end{enumerate}
\end{exercise}

\begin{solution}
  \begin{enumerate}
    \item hidrogenocarbonato
    \item dióxido de dihidrógeno / peróxido de hidrógeno
    \item yodato de estaño(II)
    \item dicloruro de mercurio / cloruro de mercurio(II)
    \item trióxido de dioro / óxido de oro(III)
    \item dihidruro de berilio / hidruro de berilio
    \item pentaóxido de divanadio / óxido de vanadio(V)
    \item monóxido de berilio / óxido de berilio
    \item dihidróxido de platino / óxido de platino(II)
    \item arseniato de plata
    \item dihidróxido de estaño / hidróxido de estaño(II)
    \item fosfato de bario
  \end{enumerate}
\end{solution}




\begin{exercise}[
    tags    = {inorgánica,nomenclatura,múltiple,2B},
    topics  = {química inorgánica,formulación,nomenclatura},
    source  = {Química 2B SAN 2016, p372, e10},
  ]
  Nombra las siguientes sustancias

  \begin{enumerate}\begin{multicols}{3}
    \item \ch{CrO3}
    \item \ch{CaH2}
    \item \ch{P2O5}
    \item \ch{CO3^2-}
    \item \ch{Hg(IO2)2}
    \item \ch{Ag2CrO4}
    \item \ch{H2S}
    \item \ch{Hg(NO2)2}
    \item \ch{NH4IO4}
    \item \ch{NaClO4}
    \item \ch{PbSO4}
    \item \ch{H2SeO3}
  \end{multicols}\end{enumerate}
\end{exercise}

\begin{solution}
  \begin{enumerate}
    \item trióxido de cromo / óxido de cromo(VI)
    \item dihidruro de calcio / hidruro de calcio
    \item pentaóxido de difósforo / óxido de fósforo(V)
    \item carbonato
    \item yodito de mercurio(II)
    \item cromato de plata
    \item sulfuro de hidrógeno / ácido sulfhídrico
    \item nitrito de mercurio(II)
    \item peryodato de amonio
    \item perclorato de sodio
    \item sulfato de plomo
    \item ácido selenioso
  \end{enumerate}
\end{solution}




\begin{exercise}[
    tags    = {inorgánica,nomenclatura,múltiple,2B},
    topics  = {química inorgánica,formulación,nomenclatura},
    source  = {Química 2B SAN 2016, p372, e11},
  ]
  Nombra las siguientes sustancias

  \begin{enumerate}\begin{multicols}{3}
    \item \ch{Sc2S3}
    \item \ch{Bi2O3}
    \item \ch{Cr2O3}
    \item \ch{Cl4}
    \item \ch{SrO2}
    \item \ch{WO3}
    \item \ch{Ba(MnO4)2}
    \item \ch{NaClO}
  \end{multicols}\end{enumerate}
\end{exercise}

\begin{solution}
  \begin{enumerate}
    \item trisulfuro de diescandio / sulfuro de escandio
    \item trióxido de dibismuto / óxido de bismuto(III)
    \item trióxido de dicromo / óxido de cromo(III)
    \item tetracloruro de carbono / cloruro de carbono(IV)
    \item dióxido de estroncio / peróxido de estroncio
    \item trióxido de wolframio / óxido de wolframio(III)
    \item permanganato de bario
    \item hipoclorito de sodio
  \end{enumerate}
\end{solution}

\end{multicols*}

















\begin{multicols*}{2}[
  \section{Ejercicios de formulación}
  ]

\begin{exercise}[
    tags    = {inorgánica,formulación,múltiple,2B},
    topics  = {química inorgánica,formulación,nomenclatura},
    source  = {Química 2B SAN 2016, p372, e12},
  ]
  Nombra las siguientes sustancias

  \begin{enumerate}
    \item tetracloruro de estaño
    \item hidrogenocarbonato de potasio
    \item cromato de cobre(II)
    \item hidrogenosulfuro de bario
    \item hidróxido de aluminio
    \item óxido de plata
    \item hidróxido de zinc
    \item bromato de calcio
    \item hidruro de berilio
    \item nitrito de plata
  \end{enumerate}
\end{exercise}

\begin{solution}
  \begin{enumerate}\begin{multicols}{3}
    \item \ch{SnCl4}
    \item \ch{KHCO3}
    \item \ch{CuCrO4}
    \item \ch{Ba(HS)2}
    \item \ch{Al(OH)3}
    \item \ch{Ag2O}
    \item \ch{Zn(OH)2}
    \item \ch{CaBrO3}
    \item \ch{BeH2}
    \item \ch{AgNO3}
  \end{multicols}\end{enumerate}
\end{solution}




\begin{exercise}[
    tags    = {inorgánica,formulación,múltiple,2B},
    topics  = {química inorgánica,formulación,nomenclatura},
    source  = {Química 2B SAN 2016, p372, e13},
  ]
  Nombra las siguientes sustancias

  \begin{enumerate}
    \item perbromato de hierro(II)
    \item pentasulfuro de diarsénico
    \item sulfuro de arsénico(V)
    \item monóxido de níquel
    \item pentaóxido de difósforo
    \item bromuro de litio
    \item óxido de níquel(II)
    \item ácido sulfuroso
    \item óxido de fósforo(V)
    \item ácido yodoso
  \end{enumerate}
\end{exercise}

\begin{solution}
  \begin{enumerate}\begin{multicols}{3}
    \item \ch{Fe(BrO4)2}
    \item \ch{As2S5}
    \item \ch{As2S5}
    \item \ch{NiO}
    \item \ch{P2O5}
    \item \ch{LiBr}
    \item \ch{NiO}
    \item \ch{H2SO3}
    \item \ch{P2O5}
    \item \ch{HIO2}
  \end{multicols}\end{enumerate}
\end{solution}




\begin{exercise}[
    tags    = {inorgánica,formulación,múltiple,2B},
    topics  = {química inorgánica,formulación,nomenclatura},
    source  = {Química 2B SAN 2016, p372, e14},
  ]
  Nombra las siguientes sustancias

  \begin{enumerate}
    \item disulfuro de carbono
    \item sulfuro de carbono(IV)
    \item seleniuro de hidrógeno
    \item hidrogenosulfato de sodio
    \item dihidrogenofosfato de calcio
    \item clorita de sodio
    \item arsano
    \item yodato de potasio
    \item ácido fosforoso
    \item sulfato de plata
  \end{enumerate}
\end{exercise}

\begin{solution}
  \begin{enumerate}\begin{multicols}{3}
    \item \ch{CS2}
    \item \ch{CS2}
    \item \ch{H2S}
    \item \ch{NaHSO4}
    \item \ch{Ca(H2PO4)2}
    \item \ch{NaClO2}
    \item \ch{AsH3}
    \item \ch{KIO3}
    \item \ch{H3PO3}
    \item \ch{Ag2SO4}
  \end{multicols}\end{enumerate}
\end{solution}




\begin{exercise}[
    tags    = {inorgánica,formulación,múltiple,2B},
    topics  = {química inorgánica,formulación,nomenclatura},
    source  = {Química 2B SAN 2016, p372, e15},
  ]
  Nombra las siguientes sustancias

  \begin{enumerate}
    \item manganato de potasio
    \item hidrogenosulfato de hierro(II)
    \item tricloruro de bismuto
    \item carbonato de bario
    \item peróxido de potasio
    \item sulfuro de zinc
    \item sulfito de sodio
    \item ácido cloroso
    \item peróxido de sodio
    \item óxido de cobre(II)
  \end{enumerate}
\end{exercise}

\begin{solution}
  \begin{enumerate}\begin{multicols}{3}
    \item \ch{K2MnO4}
    \item \ch{Fe(HSO4)2}
    \item \ch{BiCl3}
    \item \ch{BaCO3}
    \item \ch{K2O2}
    \item \ch{ZnS}
    \item \ch{Na2SO3}
    \item \ch{HClO2}
    \item \ch{Na2O2}
    \item \ch{CuO}
  \end{multicols}\end{enumerate}
\end{solution}




\begin{exercise}[
    tags    = {inorgánica,formulación,múltiple,2B},
    topics  = {química inorgánica,formulación,nomenclatura},
    source  = {Química 2B SAN 2016, p372, e16},
  ]
  Nombra las siguientes sustancias

  \begin{enumerate}
    \item perclorato de potasio
    \item tetrafluoruro de estaño
    \item permanganato de litio
    \item permanganato de sodio
    \item tetrabromuro de carbono
    \item cloruro de amonio
    \item nitrato de hierro(II)
    \item nitrato de rubidio
    \item nitrato de zinc(II)
    \item cloruro de hierro(II)
  \end{enumerate}
\end{exercise}

\begin{solution}
  \begin{enumerate}\begin{multicols}{3}
    \item \ch{KClO4}
    \item \ch{SnF4}
    \item \ch{LiMnO4}
    \item \ch{NaMnO4}
    \item \ch{CBr4}
    \item \ch{NH4Cl}
    \item \ch{Fe(NO3)2}
    \item \ch{RbNO3}
    \item \ch{Zn(NO3)2}
    \item \ch{FeCl2}
  \end{multicols}\end{enumerate}
\end{solution}




\begin{exercise}[
    tags    = {inorgánica,formulación,múltiple,2B},
    topics  = {química inorgánica,formulación,nomenclatura},
    source  = {Química 2B SAN 2016, p372, e17},
  ]
  Nombra las siguientes sustancias

  \begin{enumerate}
    \item clorato de cobalto(III)
    \item fosfato de níquel(II)
    \item hidróxido de paladio(II)
    \item hidróxido de magnesio
    \item hidróxido de plomo(IV)
    \item dióxido de titanio
    \item amoniaco
    \item ácido perclórico
    \item sulfuro de cadmio
    \item óxido de cromo(III)
  \end{enumerate}
\end{exercise}

\begin{solution}
  \begin{enumerate}\begin{multicols}{3}
    \item \ch{Co(ClO3)3}
    \item \ch{Ni3(PO4)2}
    \item \ch{Pd(OH)2}
    \item \ch{Mg(OH)2}
    \item \ch{Pb(OH)4}
    \item \ch{TiO2}
    \item \ch{NH3}
    \item \ch{HClO4}
    \item \ch{CdS}
    \item \ch{Cr2O3}
  \end{multicols}\end{enumerate}
\end{solution}




\begin{exercise}[
    tags    = {inorgánica,formulación,múltiple,2B},
    topics  = {química inorgánica,formulación,nomenclatura},
    source  = {Química 2B SAN 2016, p372, e18},
  ]
  Nombra las siguientes sustancias

  \begin{enumerate}
    \item hidróxido de hierro(III)
    \item carbonato de rubidio
    \item nitrato de magnesio
    \item hidruro de níquel(III)
    \item óxido de molibdeno(IV)
    \item ácido crómico
    \item sulfito de hierro(II)
    \item ácido bromoso
    \item sulfato de hierro(III)
    \item cromato férrico
  \end{enumerate}
\end{exercise}

\begin{solution}
  \begin{enumerate}\begin{multicols}{3}
    \item \ch{Fe(OH)3}
    \item \ch{Rb2CO3}
    \item \ch{Mg(NO3)2}
    \item \ch{NiH3}
    \item \ch{MbO2}
    \item \ch{H2CrO4}
    \item \ch{FeSO3}
    \item \ch{HBrO2}
    \item \ch{Fe2(SO4)3}
    \item \ch{Fe2(CrO4)3}
  \end{multicols}\end{enumerate}
\end{solution}




\begin{exercise}[
    tags    = {inorgánica,formulación,múltiple,2B},
    topics  = {química inorgánica,formulación,nomenclatura},
    source  = {Química 2B SAN 2016, p372, e19},
  ]
  Nombra las siguientes sustancias

  \begin{enumerate}
    \item trihidruro de níquel
    \item sulfuro de plomo(IV)
    \item óxido de arsénico(V)
    \item dihidróxido de hierro
    \item carbonato de calcio
    \item nitrato de amonio
    \item disulfuro de plomo
    \item cromato de potasio
  \end{enumerate}
\end{exercise}

\begin{solution}
  \begin{enumerate}\begin{multicols}{3}
    \item \ch{NiH3}
    \item \ch{PbS2}
    \item \ch{As2O5}
    \item \ch{Fe(OH)2}
    \item \ch{CaCO3}
    \item \ch{NH4NO3}
    \item \ch{PbS2}
    \item \ch{K2CrO4}
  \end{multicols}\end{enumerate}
\end{solution}

\end{multicols*}

\end{document}

% TEMPLATES

  \begin{tasks}(1)
    \task
    \task
    \task
    \task
    \task
    \task
    \task
    \task
    \task
    \task
  \end{tasks}


  \begin{enumerate}
    \item \ch{}
    \item \ch{}
    \item \ch{}
    \item \ch{}
    \item \ch{}
    \item \ch{}
    \item \ch{}
    \item \ch{}
    \item \ch{}
    \item \ch{}
    \item \ch{}
    \item \ch{}
  \end{enumerate}
\end{exercise}
