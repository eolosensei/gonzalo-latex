\documentclass[10pt]{article}
\usepackage{polyglossia}
    \setdefaultlanguage{spanish}
\usepackage{fontspec}
    \setmainfont{Fira Sans}
\usepackage{amsmath, amsthm, amssymb}
\usepackage{unicode-math}
  \unimathsetup{
    math-style  = ISO,
    bold-style  = ISO
  }
  \setmathfont{Fira Math}
\usepackage{multicol}
  \setlength{\columnsep}{1.8cm}
\usepackage[top=2cm, bottom=3cm, left=2.5cm, right=2cm]{geometry}
\usepackage{xsim}
% Configuración general de xsim
  \loadxsimstyle{layouts}
  \xsimsetup{
    solution/print    = {true},
    path              = {xsim-files},
    exercise/template = {default},
    exercise/name     = {E},
    solution/template = {margin},
    solution/name     = {S}
  }
  \DeclareExerciseProperty{source}
\usepackage{siunitx}
% Exponent symbol options: \times for the typical cross
  \sisetup{
    per-mode                = symbol,
    output-decimal-marker   = {,},
    exponent-product        = \cdot,
    text-celsius            = ^^b0\kern -\scriptspace C,  % soluciona problemas con el símbolo de grados
    math-celsius            = ^^b0\kern -\scriptspace C,
    list-final-separator    = { y },
    list-pair-separator     = { y },
    range-phrase            = { \translate{to (numerical range)} }
  }
\usepackage[inline]{enumitem}
% Configuración general de enumitem
% Establece la configuración por defecto en a), b), c)
  \setlist[enumerate,1]{
    label=\alph*),
  }
\usepackage{chemformula}
\usepackage{chemfig}
  \setchemfig{atom sep=2em}
\usepackage{booktabs}
\usepackage{tasks}



\newenvironment{gexdatos}{
  \noindent\makebox[0pt][r]{\textit{Datos:}}
  }{\vspace{5pt}}

\newcommand{\gexBinRow}[3]{
  \ch{#1} & #2 & #3 \\ \midrule
}


\begin{document}


\begin{exercise}[
    tags    = {inorgánica,compuestos binarios,sales binarias,sales,hidruros},
    topics  = {química inorgánica,formulación,nomenclatura},
    source  = {SAN Formulación, p26, e28},
  ]
  Nombra y formula los siguientes compuestos binarios, indicando cuando proceda su nomenclatura de composición (sistemática) y su nombre de Stock:

  \begin{enumerate}
    \item \ch{KH}
    \item yoduro de cesio
    \item hidruro de titanio(III)
    \item \ch{SrCl2}
    \item cloruro de hidrógeno
    \item arseniuro de galio
    \item \ch{Li3N}
    \item \ch{CF4}
    \item yoduro de plomo(II)
    \item seleniuro de hidrógeno
  \end{enumerate}
\end{exercise}

\begin{exercise}[
    tags    = {inorgánica,compuestos binarios,sales binarias,sales},
    topics  = {química inorgánica,formulación,nomenclatura},
    source  = {SAN Formulación, p26, e29},
  ]

  Nombra y formula los siguientes compuestos binarios, indicando cuando proceda su nomenclatura de composición (sistemática) y su nombre de Stock:

  \begin{enumerate}
    \item pentacloruro de fósforo
    \item cloruro de antimonio(III)
    \item \ch{ScI3}
    \item bromuro de plata
    \item cloruro de mercurio(II)
    \item \ch{Zn3P2}
    \item teluluro de cadmio
    \item bromuro de platino(II)
    \item \ch{Na2Se}
    \item tricloruro de oro
  \end{enumerate}
\end{exercise}

\begin{exercise}[
    tags    = {inorgánica,compuestos binarios,sales binarias,sales},
    topics  = {química inorgánica,formulación,nomenclatura},
    source  = {SAN Formulación, p26, e30},
  ]

  Nombra y formula los siguientes compuestos binarios, indicando cuando proceda su nomenclatura de composición (sistemática) y su nombre de Stock:

  \begin{enumerate}
    \item teluluro de hidrógeno
    \item \ch{SiC}
    \item dibromuro de calcio
    \item fosfuro de plata
    \item \ch{NiAs}
    \item trisulfuro de diantimonio
    \item hidruro de bismuto(III)
    \item heptafluoruro de yodo
    \item \ch{H2S}
    \item dihidruro de cobre
  \end{enumerate}
\end{exercise}


\begin{exercise}[
    tags    = {inorgánica,compuestos binarios,óxidos,peróxidos},
    topics  = {química inorgánica,formulación,nomenclatura},
    source  = {SAN Formulación, p27, e31},
  ]

  Nombra y formula los siguientes compuestos binarios, indicando cuando proceda su nomenclatura de composición (sistemática) y su nombre de Stock:

  \begin{enumerate}
    \item \ch{FeO}
    \item trióxido de bromo
    \item óxido de mercurio(II)
    \item \ch{Li2O2}
    \item dióxido de magnesio
    \item peróxido de aluminio
    \item pentaóxido de dinitrógeno
    \item \ch{SO2}
    \item monóxido de carbono
    \item óxido de selenio(IV)
  \end{enumerate}
\end{exercise}


\begin{exercise}[
    tags    = {inorgánica,compuestos binarios,óxidos,peróxidos},
    topics  = {química inorgánica,formulación,nomenclatura},
    source  = {SAN Formulación, p27, e32},
  ]

  Nombra y formula los siguientes compuestos binarios, indicando cuando proceda su nomenclatura de composición (sistemática) y su nombre de Stock:

  \begin{enumerate}
    \item \ch{O3Cl2}
    \item dióxido de manganeso
    \item óxido de fósforo(V)
    \item \ch{CrO3}
    \item diyoduro de heptaoxígeno
    \item peróxido de estaño(IV)
    \item \ch{H2O2}
    \item dióxido de dilitio
    \item óxido de titanio(II)
    \item \ch{SiO2}
  \end{enumerate}
\end{exercise}


\begin{exercise}[
    tags    = {inorgánica,compuestos binarios,óxidos,peróxidos,hidróxidos},
    topics  = {química inorgánica,formulación,nomenclatura},
    source  = {SAN Formulación, p27, e33},
  ]

  Nombra y formula los siguientes compuestos binarios, indicando cuando proceda su nomenclatura de composición (sistemática) y su nombre de Stock:

  \begin{enumerate}
    \item dióxido de bario
    \item hidróxido de zinc
    \item \ch{Au(OH)3}
    \item hidróxido de plata
    \item hidróxido de magnesio
    \item \ch{NH4OH}
    \item dióxido de paladio
    \item peróxido de rubidio
    \item \ch{TeO}
    \item trióxido de dicobalto
  \end{enumerate}
\end{exercise}


\begin{exercise}[
    tags    = {inorgánica,ácidos,ácidos binarios,ácidos ternarios,oxoácidos},
    topics  = {química inorgánica,formulación,nomenclatura},
    source  = {SAN Formulación, p28, e34},
  ]

  Nombra y formula los siguientes compuestos, indicando cuando proceda su nomenclatura de hidrógeno y su nombre tradicional:

  \begin{enumerate}
    \item \ch{HF}
    \item ácido selenhídrico
    \item hidrogeno(oxidoclorato)
    \item dihidrogeno(tetraoxidosulfato)
    \item \ch{HMnO4}
    \item \ch{HI}
    \item dihidrogeno(teluro)
    \item ácido crómico
    \item dihidrogeno(tetraoxidocromato)
    \item hidrogeno(tetraoxidobromato)
  \end{enumerate}
\end{exercise}


\begin{exercise}[
  tags    = {inorgánica,ácidos,ácidos binarios,ácidos ternarios,oxoácidos},
  topics  = {química inorgánica,formulación,nomenclatura},
  source  = {SAN Formulación, p28, e35},
  ]

  Nombra y formula los siguientes compuestos, indicando cuando proceda su nomenclatura de hidrógeno y su nombre tradicional:

  \begin{enumerate}
    \item ácido clorhídrico
    \item ácido mangánico
    \item \ch{H2S}
    \item ácido arsénico
    \item trihidrógeno(tetraoxidofosfato)
    \item \ch{H3PO3}
    \item ácido hiponitroso
    \item trihidrógeno(trioxidoarsenato)
    \item \ch{H2TeO2}
    \item bromuro de hidrógeno
  \end{enumerate}
\end{exercise}


\begin{exercise}[
    tags    = {inorgánica,ácidos,ácidos binarios,ácidos ternarios,oxoácidos},
    topics  = {química inorgánica,formulación,nomenclatura},
    source  = {SAN Formulación, p28, e36},
  ]

  Nombra y formula los siguientes compuestos, indicando cuando proceda su nomenclatura de hidrógeno y su nombre tradicional:

  \begin{enumerate}
    \item \ch{H3BO3}
    \item \ch{HNO3}
    \item ácido selenioso
    \item ácido nítrico
    \item \ch{H2SO3}
    \item ácido carbónico
    \item hidrógeno(dioxidoclorato)
    \item \ch{H2SeO4}
    \item ácido hipobromoso
    \item hidrógeno(trioxidoclorato)
  \end{enumerate}
\end{exercise}


\begin{exercise}[
    tags    = {inorgánica,sales, sales binarias,sales ternarias},
    topics  = {química inorgánica,formulación,nomenclatura},
    source  = {SAN Formulación, p29, e37},
  ]

  Nombra y formula las siguientes sales, indicando cuando proceda su nomenclatura de composición (sistemática) y su nombre de Stock:

  \begin{enumerate}
    \item \ch{BeCl2}
    \item hidrogenosulfuro de calcio
    \item tetraoxidosulfato de disodio
    \item \ch{KMnO4}
    \item carbonato de amonio
    \item tris(hidrogenoselenuro) de níquel
    \item \ch{CoSO3}
    \item hidrogenosufito de zinc
    \item heptaoxidodicromato de dilitio
    \item \ch{Ca(CN)2}
  \end{enumerate}
\end{exercise}


\begin{exercise}[
    tags    = {inorgánica,sales, sales binarias,sales ternarias},
    topics  = {química inorgánica,formulación,nomenclatura},
    source  = {SAN Formulación, p29, e38},
  ]

  Nombra y formula las siguientes sales, indicando cuando proceda su nomenclatura de composición (sistemática) y su nombre de Stock:

  \begin{enumerate}
    \item nitrato de cromo(III)
    \item tris(tetraoxidoyodato) de galio
    \item \ch{Sn(ClO)4}
    \item fosfato de calcio
    \item tris(tetraoxidosulfato) de dialuminio
    \item \ch{Pb(NO2)4}
    \item cromato de cesio
    \item trioxidoborato de níquel
    \item \ch{MnSO4}
    \item fluoruro de bario
  \end{enumerate}
\end{exercise}


\begin{exercise}[
    tags    = {inorgánica,sales,sales binarias,sales ternarias},
    topics  = {química inorgánica,formulación,nomenclatura},
    source  = {SAN Formulación, p29, e39},
  ]

  Nombra y formula las siguientes sales, indicando cuando proceda su nomenclatura de composición (sistemática) y su nombre de Stock:

  \begin{enumerate}
    \item hidrogenotelururo de rubidio
    \item \ch{KClO2}
    \item sulfito de titanio(III)
    \item dioxidotelurato de dicobre
    \item \ch{NH4HSO2}
    \item hipoclorito de plata
    \item tetraoxidomanganato de dilitio
    \item \ch{Ca3(AsO3)2}
    \item yodato de magnesio
    \item trioxidonitrato de amonio
  \end{enumerate}
\end{exercise}


\begin{exercise}[
    tags    = {inorgánica,óxidos,compuestos binarios,2B},
    topics  = {química inorgánica,formulación,nomenclatura},
    source  = {Química 2B OXF 2016, p342, e5},
  ]

  A continuación se indica la fórmula o el nombre de una serie de compuestos. Completa la información mostrando la fórmula y el nombre de cada uno tanto con prefijos multiplicadores como con número de oxidación o número de carga.

  \begin{enumerate}
    \item dióxido de silicio
    \item \ch{Li2O}
    \item \ch{ZnO}
    \item óxido de cobre(I)
    \item \ch{CaO}
    \item \ch{Cr2O3}
    \item óxido de boro(III)
    \item óxido de magnesio
    \item \ch{SeO3}
    \item pentaóxido de dinitrógeno
  \end{enumerate}
\end{exercise}


\begin{exercise}[
    tags    = {inorgánica,compuestos ternarios,oxoácidos,2B},
    topics  = {química inorgánica,formulación,nomenclatura},
    source  = {Química 2B OXF 2016, p344, e8 y p345, e9},
    print   = false,
  ]

  Completa la tabla siguiente indicando la fórmula o el nombre de hidrógeno de cada compuesto:

  \begin{tabular}{cll}
    Formula      & Nom. de hidrógeno & Nom. tradicional \\ \toprule
    \gexBinRow{HClO3}{}{}
    \gexBinRow{}{trihidrogeno(trioxidoborato)}{ácido bórico}
    \gexBinRow{H2SO3}{}{}
    \gexBinRow{}{hidrogeno(tetraoxidoyodato)}{ácido peryódico}
    \gexBinRow{H2CrO4}{}{}
    \gexBinRow{}{dihidrogeno(trioxidotelurato)}{ácido teluroso}
    \gexBinRow{HBrO2}{}{}
    \gexBinRow{}{dihidrogeno(tetraoxidomanganato)}{ácido mangánico}
    \gexBinRow{H3AsO4}{}{}
    \gexBinRow{}{trihidrogeno(trioxidofosfato)}{ácido fosforoso}
    \gexBinRow{}{dihidrogeno(trioxidocarbonato)}{ácido carbónico}
    \gexBinRow{}{hidrogeno(tetraoxidomanganato)}{ácido permangánico}
    \gexBinRow{H3PO4}{}{}
    \gexBinRow{}{dihidrogeno(heptaoxidodicromato)}{ácido dicrómico}
    \gexBinRow{HNO}{}{}
    \gexBinRow{}{trihidrogeno(trioxidoarsenato)}{}
    \gexBinRow{HNO3}{}{}
    \gexBinRow{}{hidrogeno(oxidoclorato)}{ácido hipocloroso}
    \gexBinRow{H2SeO2}{}{}
  \end{tabular}
\end{exercise}


\begin{exercise}[
    tags    = {inorgánica,sales, sales ternarias, oxosales,2B},
    topics  = {química inorgánica,formulación,nomenclatura},
    source  = {Química 2B OXF 2016, p347, e12},
  ]

  Formula los siguientes compuestos

  \begin{enumerate}
    \item Hiposulfito de oro(III)
    \item Carbonato de sodio
    \item Dicromato de cinc
    \item Fosfato de aluminio
    \item Bromuro de amonio
    \item Perclorato de cromo(II)
    \item Bistrioxidoborato de trihierro
    \item Trioxidoarsenato de trisamonio
    \item Bis(hidrogenoselenuro) de mercurio
    \item Nitrito de cobalto(II)
    \item Oxidoyodato de oro
    \item Fosfito de magnesio
    \item Hidrogenosulfato de níquel(II)
    \item Clorato de cesio
    \item Sulfato de aluminio
    \item Tris(trioxidoclorato) de cobalto
    \item Hidrogenocarbonato de calcio
    \item Manganato de bismuto(III)
    \item Tris(trioxidonitrato) de cromo
    \item Yoduro de plomo(IV)
  \end{enumerate}
\end{exercise}



\begin{multicols}{2}[
  \section{Ejercicios de nomenclatura}
  ]

\begin{exercise}[
    tags    = {inorgánica,nomenclatura,múltiple,2B},
    topics  = {química inorgánica,formulación,nomenclatura},
    source  = {Química 2B SAN 2016, p372, e6},
  ]

  Nombra las siguientes sustancias dadas por su fórmula

  \begin{tasks}(2)
    \task \ch{PCl3}
    \task \ch{Fe(OH)2}
    \task \ch{Sr(OH)2}
    \task \ch{SO2}
    \task \ch{Al2(SO4)3}
    \task \ch{H2SO4}
    \task \ch{H3PO4}
    \task \ch{Fe(HSO4)2}
    \task \ch{PbO2}
    \task \ch{N2O5}
    \task \ch{Fe(NO3)3}
    \task \ch{CsH}
  \end{tasks}
\end{exercise}


\begin{exercise}[
    tags    = {inorgánica,nomenclatura,múltiple,2B},
    topics  = {química inorgánica,formulación,nomenclatura},
    source  = {Química 2B SAN 2016, p372, e7},
  ]

  Nombra las siguientes sustancias dadas por su fórmula

  \begin{tasks}(2)
    \task \ch{KMnO4}
    \task \ch{HgS}
    \task \ch{NaHSO4}
    \task \ch{Ag2O}
    \task \ch{NaHCO3}
    \task \ch{Ar}
    \task \ch{Ni(OH)2}
    \task \ch{Fe2S3}
    \task \ch{Ca(OH)2}
    \task \ch{NH4Cl}
    \task \ch{Na2O2}
    \task \ch{HClO4}
  \end{tasks}
\end{exercise}

\begin{exercise}[
    tags    = {inorgánica,nomenclatura,múltiple,2B},
    topics  = {química inorgánica,formulación,nomenclatura},
    source  = {Química 2B SAN 2016, p372, e8},
  ]

  Nombra las siguientes sustancias dadas por su fórmula

  \begin{tasks}(2)
    \task \ch{PH3}
    \task \ch{HgSO3}
    \task \ch{SiH4}
    \task \ch{As2O5}
    \task \ch{B2O3}
    \task \ch{CaO}
    \task \ch{As2O3}
    \task \ch{NO3-}
    \task \ch{HBrO3}
    \task \ch{HIO3}
    \task \ch{Co(OH)3}
    \task \ch{H4P2O7}
  \end{tasks}
\end{exercise}

\begin{exercise}[
    tags    = {inorgánica,nomenclatura,múltiple,2B},
    topics  = {química inorgánica,formulación,nomenclatura},
    source  = {Química 2B SAN 2016, p372, e9},
  ]

  Nombra las siguientes sustancias dadas por su fórmula

  \begin{tasks}(2)
    \task \ch{HCO3-}
    \task \ch{H2O2}
    \task \ch{Sn(IO3)2}
    \task \ch{HgCl2}
    \task \ch{Au2O3}
    \task \ch{BeH2}
    \task \ch{V2O5}
    \task \ch{BeO}
    \task \ch{Pt(OH)2}
    \task \ch{Ag3AsO4}
    \task \ch{Sn(OH)2}
    \task \ch{Ba3(PO4)2}
  \end{tasks}
\end{exercise}


\begin{exercise}[
    tags    = {inorgánica,nomenclatura,múltiple,2B},
    topics  = {química inorgánica,formulación,nomenclatura},
    source  = {Química 2B SAN 2016, p372, e10},
  ]

  Nombra las siguientes sustancias dadas por su fórmula

  \begin{tasks}(2)
    \task \ch{CrO3}
    \task \ch{CaH2}
    \task \ch{P2O5}
    \task \ch{CO3^2-}
    \task \ch{Hg(IO2)2}
    \task \ch{Ag2CrO4}
    \task \ch{H2S}
    \task \ch{Hg(NO2)2}
    \task \ch{NH4IO4}
    \task \ch{NaClO4}
    \task \ch{PbSO4}
    \task \ch{H2SeO3}
  \end{tasks}
\end{exercise}

\begin{exercise}[
    tags    = {inorgánica,nomenclatura,múltiple,2B},
    topics  = {química inorgánica,formulación,nomenclatura},
    source  = {Química 2B SAN 2016, p372, e11},
  ]

  Nombra las siguientes sustancias dadas por su fórmula

  \begin{tasks}(2)
    \task \ch{Sc2S3}
    \task \ch{Bi2O3}
    \task \ch{Cr2O3}
    \task \ch{Cl4}
    \task \ch{SrO2}
    \task \ch{WO3}
    \task \ch{Ba(MnO4)2}
    \task \ch{NaClO}
  \end{tasks}
\end{exercise}

\end{multicols}

\begin{multicols}{2}[
  \section{Ejercicios de formulación}
  ]

  \begin{exercise}[
      tags    = {inorgánica,formulación,múltiple,2B},
      topics  = {química inorgánica,formulación,nomenclatura},
      source  = {Química 2B SAN 2016, p372, e12},
    ]

    Nombra las siguientes sustancias dadas por su nombre

    \begin{tasks}(1)
      \task tetracloruro de estaño
      \task hidrogenocarbonato de potasio
      \task cromato de cobre(II)
      \task hidrogenosulfuro de bario
      \task hidróxido de aluminio
      \task óxido de plata
      \task hidróxido de zinc
      \task bromato de calcio
      \task hidruro de berilio
      \task nitrito de plata
    \end{tasks}
  \end{exercise}


  \begin{exercise}[
      tags    = {inorgánica,formulación,múltiple,2B},
      topics  = {química inorgánica,formulación,nomenclatura},
      source  = {Química 2B SAN 2016, p372, e13},
    ]

    Nombra las siguientes sustancias dadas por su nombre

    \begin{tasks}(1)
      \task perbromato de hierro(II)
      \task pentasulfuro de diarsénico
      \task sulfuro de arsénico(V)
      \task monóxido de níquel
      \task pentaóxido de difósforo
      \task bromuro de litio
      \task óxido de níquel(II)
      \task ácido sulfuroso
      \task óxido de fósforo(V)
      \task ácido yodoso
    \end{tasks}
  \end{exercise}

  \begin{exercise}[
      tags    = {inorgánica,formulación,múltiple,2B},
      topics  = {química inorgánica,formulación,nomenclatura},
      source  = {Química 2B SAN 2016, p372, e14},
    ]

    Nombra las siguientes sustancias dadas por su nombre

    \begin{tasks}(1)
      \task disulfuro de carbono
      \task sulfuro de carbono(IV)
      \task seleniuro de hidrógeno
      \task hidrogenosulfato de sodio
      \task dihidrogenofosfato de calcio
      \task clorita de sodio
      \task arsano
      \task yodato de potasio
      \task ácido fosforoso
      \task sulfato de plata
    \end{tasks}
  \end{exercise}


  \begin{exercise}[
      tags    = {inorgánica,formulación,múltiple,2B},
      topics  = {química inorgánica,formulación,nomenclatura},
      source  = {Química 2B SAN 2016, p372, e15},
    ]

    Nombra las siguientes sustancias dadas por su nombre

    \begin{tasks}(1)
      \task manganato de potasio
      \task hidrogenosulfato de hierro(II)
      \task tricloruro de bismuto
      \task carbonato de bario
      \task peróxido de potasio
      \task sulfuro de cinc
      \task sulfito de sodio
      \task ácido cloroso
      \task peróxido de sodio
      \task óxido de cobre(II)
    \end{tasks}
  \end{exercise}


  \begin{exercise}[
      tags    = {inorgánica,formulación,múltiple,2B},
      topics  = {química inorgánica,formulación,nomenclatura},
      source  = {Química 2B SAN 2016, p372, e16},
    ]

    Nombra las siguientes sustancias dadas por su nombre

    \begin{tasks}(1)
      \task perclorato de potasio
      \task tetrafluoruro de estaño
      \task permanganato de litio
      \task permanganato de sodio
      \task tetrabromuro de carbono
      \task cloruro de amonio
      \task nitrato de hierro(II)
      \task nitrato de rubidio
      \task nitrato de cinc(II)
      \task cloruro de hierro(II)
    \end{tasks}
  \end{exercise}


  \begin{exercise}[
      tags    = {inorgánica,formulación,múltiple,2B},
      topics  = {química inorgánica,formulación,nomenclatura},
      source  = {Química 2B SAN 2016, p372, e17},
    ]

    Nombra las siguientes sustancias dadas por su nombre

    \begin{tasks}(1)
      \task clorato de cobalto(III)
      \task fosfato de níquel(II)
      \task hidróxido de paladio(II)
      \task hidróxido de magnesio
      \task hidróxido de plomo(IV)
      \task dióxido de titanio
      \task amoniaco
      \task ácido perclórico
      \task sulfuro de cadmio
      \task óxido de cromo(III)
    \end{tasks}
  \end{exercise}

  \begin{exercise}[
      tags    = {inorgánica,formulación,múltiple,2B},
      topics  = {química inorgánica,formulación,nomenclatura},
      source  = {Química 2B SAN 2016, p372, e18},
    ]

    Nombra las siguientes sustancias dadas por su nombre

    \begin{tasks}(1)
      \task hidróxido de hierro(III)
      \task carbonato de rubidio
      \task nitrato de magnesio
      \task hidruro de níquel(III)
      \task óxido de molibdeno(IV)
      \task ácido crómico
      \task sulfito de hierro(II)
      \task ácido bromoso
      \task sulfato de hierro(III)
      \task cromato férrico
    \end{tasks}
  \end{exercise}

  \begin{exercise}[
      tags    = {inorgánica,formulación,múltiple,2B},
      topics  = {química inorgánica,formulación,nomenclatura},
      source  = {Química 2B SAN 2016, p372, e19},
    ]

    Nombra las siguientes sustancias dadas por su nombre

    \begin{tasks}(1)
      \task trihidruro de níquel
      \task sulfuro de plomo(IV)
      \task óxido de arsénico(V)
      \task dihidróxido de hierro
      \task carbonato de calcio
      \task nitrato de amonio
      \task disulfuro de plomo
      \task cromato de potasio
    \end{tasks}
  \end{exercise}


\end{multicols}

\end{document}

% TEMPLATES

  \begin{tasks}(1)
    \task
    \task
    \task
    \task
    \task
    \task
    \task
    \task
    \task
    \task
  \end{tasks}


  \begin{enumerate}
    \item \ch{}
    \item \ch{}
    \item \ch{}
    \item \ch{}
    \item \ch{}
    \item \ch{}
    \item \ch{}
    \item \ch{}
    \item \ch{}
    \item \ch{}
    \item \ch{}
    \item \ch{}
  \end{enumerate}
\end{exercise}
