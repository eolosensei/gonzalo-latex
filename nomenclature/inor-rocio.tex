\documentclass[10pt]{article}
\usepackage{polyglossia}
    \setdefaultlanguage{spanish}
\usepackage{fontspec}
    \setmainfont{Fira Sans}
\usepackage{amsmath, amsthm, amssymb}
\usepackage{unicode-math}
  \unimathsetup{
    math-style  = ISO,
    bold-style  = ISO
  }
  \setmathfont{Fira Math}
\usepackage{multicol}
  \setlength{\columnsep}{1.8cm}
\usepackage[top=2cm, bottom=3cm, left=2.5cm, right=2cm]{geometry}
\usepackage{xsim}
% Configuración general de xsim
  \loadxsimstyle{layouts}
  \xsimsetup{
    solution/print    = {true},
    path              = {xsim-files},
    exercise/template = {margin},
    exercise/name     = {E},
    solution/template = {margin},
    solution/name     = {S}
  }
  \DeclareExerciseProperty{source}
\usepackage{siunitx}
% Exponent symbol options: \times for the typical cross
  \sisetup{
    per-mode                = symbol,
    output-decimal-marker   = {,},
    exponent-product        = \cdot,
    text-celsius            = ^^b0\kern -\scriptspace C,  % soluciona problemas con el símbolo de grados
    math-celsius            = ^^b0\kern -\scriptspace C,
    list-final-separator    = { y },
    list-pair-separator     = { y },
    range-phrase            = { \translate{to (numerical range)} }
  }
\usepackage[inline]{enumitem}
% Configuración general de enumitem
% Establece la configuración por defecto en a), b), c)
  \setlist[enumerate,1]{
    label=\alph*),
  }
\usepackage{chemformula}
\usepackage{chemfig}
  \setchemfig{atom sep=2em}




\newenvironment{gexdatos}{
  \noindent\makebox[0pt][r]{\textit{Datos:}}
  }{\vspace{5pt}}




\begin{document}

\begin{exercise}[
    tags    = {inorgánica,compuestos binarios,sales,hidruros},
    topics  = {química inorgánica,formulación,nomenclatura},
    source  = {SAN Formulación, p26, e28},
  ]
  Nombra y formula los siguientes compuestos binarios, indicando cuando proceda su nomenclatura de composición (sistemática) y su nombre de Stock:

  \begin{enumerate}
    \item \ch{KH}
    \item yoduro de cesio
    \item hidruro de titanio(III)
    \item \ch{SrCl2}
    \item cloruro de hidrógeno
    \item arseniuro de galio
    \item \ch{Li3N}
    \item \ch{CF4}
    \item yoduro de plomo(II)
    \item seleniuro de hidrógeno
  \end{enumerate}
\end{exercise}

\begin{exercise}[
    tags    = {inorgánica,compuestos binarios,sales},
    topics  = {química inorgánica,formulación,nomenclatura},
    source  = {SAN Formulación, p26, e29},
  ]

  Nombra y formula los siguientes compuestos binarios, indicando cuando proceda su nomenclatura de composición (sistemática) y su nombre de Stock:

  \begin{enumerate}
    \item pentacloruro de fósforo
    \item cloruro de antimonio(III)
    \item \ch{ScI3}
    \item bromuro de plata
    \item cloruro de mercurio(II)
    \item \ch{Zn3P2}
    \item teluluro de cadmio
    \item bromuro de platino(II)
    \item \ch{Na2Se}
    \item tricloruro de oro
  \end{enumerate}
\end{exercise}

\begin{exercise}[
    tags    = {inorgánica,compuestos binarios,sales},
    topics  = {química inorgánica,formulación,nomenclatura},
    source  = {SAN Formulación, p26, e30},
  ]

  Nombra y formula los siguientes compuestos binarios, indicando cuando proceda su nomenclatura de composición (sistemática) y su nombre de Stock:

  \begin{enumerate}
    \item teluluro de hidrógeno
    \item \ch{SiC}
    \item dibromuro de calcio
    \item fosfuro de plata
    \item \ch{NiAs}
    \item trisulfuro de diantimonio
    \item hidruro de bismuto(III)
    \item heptafluoruro de yodo
    \item \ch{H2S}
    \item dihidruro de cobre
  \end{enumerate}
\end{exercise}

\end{document}
