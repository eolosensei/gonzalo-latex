\documentclass[10pt]{article}
\usepackage{polyglossia}
    \setdefaultlanguage{spanish}
\usepackage{fontspec}
    \setmainfont{Fira Sans}
\usepackage{amsmath, amsthm, amssymb}
\usepackage{unicode-math}
  \unimathsetup{
    math-style  = ISO,
    bold-style  = ISO
  }
  \setmathfont{Fira Math}
\usepackage{multicol}
  \setlength{\columnsep}{1.8cm}
\usepackage[top=2cm, bottom=3cm, left=2.5cm, right=2cm]{geometry}
\usepackage{xsim}
% Configuración general de xsim
  \loadxsimstyle{layouts}
  \xsimsetup{
    solution/print    = {true},
    path              = {xsim-files},
    exercise/template = {default},
    exercise/name     = {E},
    solution/template = {margin},
    solution/name     = {S}
  }
  \DeclareExerciseProperty{source}
\usepackage{siunitx}
% Exponent symbol options: \times for the typical cross
  \sisetup{
    per-mode                = symbol,
    output-decimal-marker   = {,},
    exponent-product        = \cdot,
    text-celsius            = ^^b0\kern -\scriptspace C,  % soluciona problemas con el símbolo de grados
    math-celsius            = ^^b0\kern -\scriptspace C,
    list-final-separator    = { y },
    list-pair-separator     = { y },
    range-phrase            = { \translate{to (numerical range)} }
  }
\usepackage[inline]{enumitem}
% Configuración general de enumitem
% Establece la configuración por defecto en a), b), c)
  \setlist[enumerate,1]{
    label=\alph*),
  }
\usepackage{chemformula}
\usepackage{chemfig}
  \setchemfig{atom sep=2em}
\usepackage{booktabs}
\usepackage{tasks}



\newenvironment{gexdatos}{
  \noindent\makebox[0pt][r]{\textit{Datos:}}
  }{\vspace{5pt}}

\newcommand{\gexBinRow}[3]{
  \ch{#1} & #2 & #3 \\ \midrule
}


\begin{document}

\begin{multicols}{2}[
  \section{Ejercicios de nomenclatura}
  ]

  \begin{exercise}[
      tags    = {inorgánica,formulación,múltiple,2B},
      topics  = {química inorgánica,formulación,nomenclatura},
      source  = {Química 2B SAN 2016, p391, e11},
    ]

    Nombra las siguientes sustancias dadas por su nombre

    \begin{tasks}(1)
      \task \chemfig{CH_2=CH-CH(-[6]CH_3-CH_3}
      \task \chemfig{CH_3-CH_2-O-CH_2-CH_3}
      \task \chemfig{CH_3-CH(-[2]CH_3)-CH_2-CH_3}
      \task \chemfig{CH_3-CH2-C~C-CH_3}
      \task \chemfig{CH_3-CH_2-CH_2-CH_2-CH_2-CH_2-NH_2}
      \task \chemfig{CH_3-CO-CH_3}
      \task \chemfig{CH_3-CH_2OH}
      \task \chemfig{CH_3-COO-CH_2-CH_3}
    \end{tasks}
  \end{exercise}

\end{multicols}

\end{document}

% TEMPLATES

  \begin{tasks}(1)
    \task \chemfig{}
    \task \chemfig{}
    \task \chemfig{}
    \task \chemfig{}
    \task \chemfig{}
    \task \chemfig{}
    \task \chemfig{}
    \task \chemfig{}
    \task \chemfig{}
    \task \chemfig{}
  \end{tasks}


  \begin{enumerate}
    \item \ch{}
    \item \ch{}
    \item \ch{}
    \item \ch{}
    \item \ch{}
    \item \ch{}
    \item \ch{}
    \item \ch{}
    \item \ch{}
    \item \ch{}
    \item \ch{}
    \item \ch{}
  \end{enumerate}
\end{exercise}
