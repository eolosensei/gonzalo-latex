\documentclass[10pt]{article}
\usepackage{polyglossia}
    \setdefaultlanguage{spanish}
\usepackage{fontspec}
    \setmainfont{Fira Sans}
\usepackage{amsmath, amsthm, amssymb}
\usepackage{unicode-math}
  \unimathsetup{
    math-style  = ISO,
    bold-style  = ISO
  }
  \setmathfont{Fira Math}
\usepackage{multicol}
  \setlength{\columnsep}{1cm}
\usepackage[top=2cm, bottom=3cm, left=2.5cm, right=2cm]{geometry}
\usepackage{xsim}
% Configuración general de xsim
  \loadxsimstyle{layouts}
  \xsimsetup{
    path              = {xsim-files},
    exercise/template = {runin},
    exercise/name     = {E},
    exercise/print    = {true},
    solution/template = {runin},
    solution/name     = {S},
    solution/print    = {false},
  }
  \DeclareExerciseProperty{source}
\usepackage{siunitx}
% Exponent symbol options: \times for the typical cross
  \sisetup{
    per-mode                = symbol,
    output-decimal-marker   = {,},
    exponent-product        = \cdot,
    text-celsius            = ^^b0\kern -\scriptspace C,  % soluciona problemas con el símbolo de grados
    math-celsius            = ^^b0\kern -\scriptspace C,
    list-final-separator    = { y },
    list-pair-separator     = { y },
    range-phrase            = { \translate{to (numerical range)} }
  }
\usepackage[inline]{enumitem}
% Configuración general de enumitem
% Establece la configuración por defecto en a), b), c)
  \setlist[enumerate,1]{
    label   =\alph*),
    itemsep = 0.3\itemsep,
  }
\usepackage{chemformula}
\usepackage{chemfig}
  \setchemfig{
    atom sep = 2em
  }
\usepackage{booktabs}
\usepackage{tasks}



\newenvironment{gexdatos}{
  \noindent\makebox[0pt][r]{\textit{Datos:}}
  }{\vspace{5pt}}

\newcommand{\gexBinRow}[3]{
  \ch{#1} & #2 & #3 \\ \midrule
}


\begin{document}

\begin{multicols}{2}[
  \section{Ejercicios de nomenclatura}
  ]

  \begin{exercise}[
      tags    = {orgánica,formulación,múltiple,2B},
      topics  = {química orgánica,formulación,nomenclatura},
      source  = {Química 2B SAN 2016, p391, e11},
    ]
    Nombra los siguientes compuestos:
  %  \renewcommand\printatom[1]{\fbox{\ensuremath{\mathrm#1}}}
    \begin{enumerate}
      \item \chemfig{CH_2=CH-CH(-[6]CH_3)-CH_3}
      \item \ch{CH3-CH2-O-CH2-CH3}
      \item \chemfig{CH_3-CH(-[2]CH_3)-CH_2-CH_3}
      \item \ch{CH3-CH2-C+C-CH3}
      \item \ch{CH3-CH2-CH2-CH2-CH2-CH2-NH2}
      \item \ch{CH3-CO-CH3}
      \item \ch{CH3-CH2OH}
      \item \ch{CH3-COO-CH2-CH3}
    \end{enumerate}
  \end{exercise}

  \begin{solution}
    \begin{enumerate}
      \item 3-metilbut-1-eno
      \item dietil éter / etoxietano
      \item metilbutano
      \item pent-2-ino
      \item hexanamina
      \item dimetil cetona / propanona
      \item etanol
      \item etanoato de etilo / acetato de etilo
    \end{enumerate}
  \end{solution}

  \begin{exercise}[
      tags    = {orgánica,formulación,múltiple,2B},
      topics  = {química orgánica,formulación,nomenclatura},
      source  = {Química 2B SAN 2016, p391, e12},
    ]
    Nombra los siguientes compuestos:

    \begin{enumerate}
      \item \ch{CH3-CH2-CH2-COOH}
      \item \ch{CH3-CHCl2}
      \item \ch{CH+CH}
      \item \ch{CH3-CO-(CH2)4-CH3}
      \item \ch{CH3-CHOH-CH3}
      \item \chemfig{[:-120]**6(-----(-COOH)-)}
      \item \chemfig{CH_3-CH(-[6]CH_3)-CH(-[6]CH_3)-CH_2-CH(-[6]CH_3)-CH_3}
      \item \ch{CHCl=CH-CH_3}
    \end{enumerate}
  \end{exercise}

\begin{solution}
  \begin{enumerate}
    \item ácido butanoico
    \item 1,1-dicloroetano
    \item etino
    \item pentil metil cetona / heptan-2-ona
    \item propan-2-ol
    \item ácido benzoico / ácido bencenocarboxílico
    \item 2,3,5-trimetilhexano
    \item 1-cloropropeno
  \end{enumerate}
\end{solution}


\begin{exercise}[
    tags    = {orgánica,formulación,múltiple,2B},
    topics  = {química orgánica,formulación,nomenclatura},
    source  = {Química 2B SAN 2016, p391, e13},
  ]
  Nombra los siguientes compuestos:

  \begin{enumerate}
    \item \chemfig{H_3C-CH(-[6]CH_3)-CH(=[6]O)-CH_2-CH(-[6]CH_3)-CH_3}
    \item \ch{CH3-CH2-CO-CH3}
    \item \ch{CH2=CHOH}
    \item \ch{CH2=H2}
    \item \ch{CH3-CH2-NH-CH3}
    \item \ch{CH3-CH=CH-CO-CH3}
    \item \ch{CH3-CH2-CO-NH2}
    \item \ch{CH3-CH2-CH=CH-C+C-CH=CH-CH3}
  \end{enumerate}
\end{exercise}

\begin{solution}
  \begin{enumerate}
    \item 2,5-dimetilhexan-3-ona / isobutil isopropil cetona
    \item butanona / etil metil cetona
    \item etenol
    \item eteno
    \item etil metil amina / N-metiletanamina
    \item pent-3-en-2-ona / metil prop-1-enil cetona
    \item propanamida
    \item nona-2,6-dien-4-ino
  \end{enumerate}
\end{solution}

\begin{exercise}[
    tags    = {orgánica,formulación,múltiple,2B},
    topics  = {química orgánica,formulación,nomenclatura},
    source  = {Química 2B SAN 2016, p391, e14},
  ]
  Nombra los siguientes compuestos:

  \begin{enumerate}
    \item \ch{CH3-CH2-CH2-CH2-CHO}
    \item \ch{CH3-COOH}
    \item \ch{CH2=CH-CH=CH-CH2-COOH}
    \item \chemfig{CH_3-C(-[2]OH)(-[6]OH)-CH_2-CH_2-CHO}
    \item \ch{CH3-CH2-NH-CH2-CH3}
    \item \chemfig{CH_3-CH(-[6]CH_3)-COO-CH_2-CH_3}
    \item \ch{CH2=CH-CH=CH-CHO}
    \item \ch{CH+C-CH2-COOH}
  \end{enumerate}
\end{exercise}

\begin{solution}
  \begin{enumerate}
    \item pentanal
    \item ácido etanoico / ácido acético
    \item ácido hexa-3,5-dienoico
    \item 4,4-dihidroxipentanal
    \item dietil amina / N-etiletanamina
    \item 2-metilpropanoato de etilo
    \item penta-2,4-dienal
    \item ácido but-3-inoico
  \end{enumerate}
\end{solution}


\begin{exercise}[
    tags    = {orgánica,formulación,múltiple,2B},
    topics  = {química orgánica,formulación,nomenclatura},
    source  = {Química 2B SAN 2016, p391, e15},
  ]
  Nombra los siguientes compuestos:

  \begin{enumerate}
    \item \ch{CH3-CO-CO-CH3}
    \item \ch{CH3-CH2-CHOH-CONH2}
    \item \chemfig{CH~C-CH(-[6]CH_3)-CO-CH_2-CH_3}
    \item \ch{CH3-CH2-CH2-NH2}
    \item \ch{CH+C-CH=CH-CHCl-C+CH}
    \item \chemfig{NC-CH_2-CH(-[6]CH_3)-CH_3}
    \item \ch{CH3-CO-CHOH-CH3}
    \item \ch{CH2=CH-CH2-CH=CH2}
  \end{enumerate}
  \end{exercise}

  \begin{solution}
    \begin{enumerate}
      \item butanodiona
      \item 2-hidroxibutanamida
      \item 4-metilhex-5-in-3-ona
      \item propanamina
      \item 5-clorohept-3-eno-1,6-diino
      \item 3-metilbutanonitrilo
      \item 3-hidroxibutanona
      \item penta-1,4-dieno
    \end{enumerate}
  \end{solution}


  \begin{exercise}[
      tags    = {orgánica,formulación,múltiple,2B},
      topics  = {química orgánica,formulación,nomenclatura},
      source  = {Química 2B SAN 2016, p391, e16},
    ]
    Nombra los siguientes compuestos:

  \begin{enumerate}
    \item \ch{CH3-O-CH3}
    \item \chemfig{CH_3-CH(-[6]CH_3)-CO-CH_3}
    \item \chemfig{CH_3-CH(-[6]CH_3)-COOH}
    \item \ch{CH3-COO-CH3}
    \item \ch{CH2OH-CH3}
    \item \ch{CH3-NH2}
    \item \ch{CH3-CH2-CO-CH2-CH3}
    \item \ch{CH3-CHO}
  \end{enumerate}
\end{exercise}

\begin{solution}
  \begin{enumerate}
    \item dimetil éter / metoximetano
    \item 3-metilbutan-2-ona / isopropil metil cetona
    \item ácido 2-metilpropanoico
    \item etanoato de metilo / acetato de metilo
    \item hidroxietanal
    \item metanamina
    \item dietil cetona / pentan-3-ona
    \item etanal
  \end{enumerate}
\end{solution}

\end{multicols}











\begin{multicols}{2}[
  \section{Ejercicios de formulación}
  ]

\begin{exercise}[
    tags    = {orgánica,nomenclatura,múltiple,2B},
    topics  = {química orgánica,formulación,nomenclatura},
    source  = {Química 2B SAN 2016, p391, e17},
  ]
  Formula los siguientes compuestos:

  \begin{enumerate}
    \item ácido 2-metilpropanoico
    \item etanoato de potasio
    \item 1,2-bencenodiol
    \item naftaleno
    \item trifenilamina
    \item clorobenceno
    \item ácido 2-metilpentanoico
    \item metilamina
  \end{enumerate}
\end{exercise}

\begin{solution}
  \begin{enumerate}
    \item \ch{CH3-CH(CH3)-COOH}
    \item \ch{CH3-COO-K}
    \item \chemfig{[:-120]**6(----(-OH)-(-OH)-)}
    \item \chemfig{[:-120]**6(----**6(------)--)}
    \item \chemfig{[:-120]**6(----(-N(-[:-90]**6(------))-[:30]**6(------))--)}
    \item \chemfig{[:-120]**6(-----(-Cl)-)}
    \item \ch{CH3-CH2-CH2-CH(CH3)-COOH}
    \item \ch{CH3-NH2}
  \end{enumerate}
\end{solution}


\begin{exercise}[
    tags    = {orgánica,nomenclatura,múltiple,2B},
    topics  = {química orgánica,formulación,nomenclatura},
    source  = {Química 2B SAN 2016, p391, e17},
  ]
  Formula los siguientes compuestos:

  \begin{enumerate}
    \item ácido propenoico
    \item butil metil amina
    \item etil propil éter
    \item 2-buteno
    \item 4-metilhexan-1-ol
    \item tolueno
    \item 2,3-dimetilbut-1-eno
    \item propanamida
  \end{enumerate}
\end{exercise}

\begin{solution}
  \begin{enumerate}
    \item \ch{CH2=CH-COOH}
    \item \ch{CH3-CH2-CH2-CH2-NH-CH3}
    \item \ch{CH3-CH2-O-CH2-CH2-CH3}
    \item \ch{CH3-CH=CH-CH3}
    \item \ch{CH2OH-CH2-CH2-CH(CH3)-CH2-CH3}
    \item \chemfig{[:-120]**6(-----(-CH_3)-)}
    \item \ch{CH2=C(CH3)-CH(CH3)-CH3}
    \item \ch{CH3-CH2-CONH2}
  \end{enumerate}
\end{solution}


\begin{exercise}[
    tags    = {orgánica,nomenclatura,múltiple,2B},
    topics  = {química orgánica,formulación,nomenclatura},
    source  = {Química 2B SAN 2016, p392, e19},
  ]
  Formula los siguientes compuestos:

  \begin{enumerate}
    \item 1-bromo-2,2-diclorobutano
    \item 2-metilhexa-1,5-dien-3-ino
    \item trimetilamina
    \item butanoato de 2-metilpropano
    \item 3-etil-4,4-dimetilheptano
    \item \textit{N}-metil-\textit{N}-etilpentanamina
    \item ácido 3-aminohexanoico
    \item \textit{N}-etilbutanamida
  \end{enumerate}
\end{exercise}

\begin{solution}
  \begin{enumerate}
    \item \ch{CH2Br-CCl2-CH2-CH3}
    \item \ch{CH2=C(CH3)-C+C-CH=CH2}
    \item \ch{N(CH3)3}
    \item \ch{CH3-CH2-CH2-COO-CH2-CH(CH3)-CH3}
    \item \ch{CH3-CH2-CH(CH2-CH3)-C(CH3)-CH2-CH2-CH3}
    \item \ch{CH3-CH2-N(CH3)-CH2-CH2-CH2-CH2-CH3}
    \item \ch{CH3-CH2-CH2-CH(NH2)-CH2-COOH}
    \item \ch{CH3-CH2-NH-CO-CH2-CH2-CH3}
  \end{enumerate}
\end{solution}



\begin{exercise}[
    tags    = {orgánica,nomenclatura,múltiple,2B},
    topics  = {química orgánica,formulación,nomenclatura},
    source  = {Química 2B SAN 2016, p392, e20},
  ]
  Formula los siguientes compuestos:

  \begin{enumerate}
    \item trietilamina
    \item 2,2-dimetilbutanamida
    \item but-1-ino
    \item \textit{trans}-but-2-eno
    \item pentan-3-ona
    \item 1,1-difluoro-2,2-dicloropropano
    \item 2,5,6-trimetilnonano
    \item difenilcetona
  \end{enumerate}
\end{exercise}

\begin{solution}
  \begin{enumerate}
    \item \ch{N(CH2-CH3)3}
    \item \ch{CH3-CH2-C(CH3)2-CONH2}
    \item \ch{CH+C-CH2-CH3}
    \item \chemfig{H-[:-60]C(-[:-120]H_3C)=C(-[:-60]H)-[:60]CH_3}
    \item \ch{CH3-CH2-CO-CH2-CH3}
    \item \ch{CHF2-CCl2-CH3}
    \item \ch{CH3-CH(CH3)-CH2-CH2-CH(CH3)-CH(CH3)-CH2-CH2-CH3}
    \item \chemfig{CO(-[:-150]**6(------))-[:-30]**6(------)}
  \end{enumerate}
\end{solution}



\begin{exercise}[
    tags    = {orgánica,nomenclatura,múltiple,2B},
    topics  = {química orgánica,formulación,nomenclatura},
    source  = {Química 2B SAN 2016, p392, e21},
  ]
  Formula los siguientes compuestos:

  \begin{enumerate}
    \item pentan-2-ol
    \item acetato de etilo
    \item propan-2-ol
    \item pent-2-eno
    \item 1-cloro-1,1-difluoroetano
    \item 5,6-dietil-3-metildecano
    \item dietil metil amina
    \item 2-metilpropan-2-ol
  \end{enumerate}
\end{exercise}

\begin{solution}
  \begin{enumerate}
    \item \ch{CH3-CH2-CH2-CHOH-CH3}
    \item \ch{CH3-COO-CH2-CH3}
    \item \ch{CH3-CHOH-CH3}
    \item \ch{CH3-CH=CH-CH2-CH3}
    \item \ch{CClF2-CH3}
    \item \chemfig{CH_3-CH_2-CH(-[6]CH_3)-CH_2-CH(-[6]CH_2-[6]CH_3)-CH(-[6]CH_2-[6]CH_3)-CH_2-CH_2-CH_2-CH_3}
    \item \ch{CH3-CH2-N(CH3)-CH2-CH3}
    \item \ch{CH3-C(CH3)OH-CH3}
  \end{enumerate}
\end{solution}



\begin{exercise}[
    tags    = {orgánica,nomenclatura,múltiple,2B},
    topics  = {química orgánica,formulación,nomenclatura},
    source  = {Química 2B SAN 2016, p392, e22},
  ]
  Formula los siguientes compuestos:

  \begin{enumerate}
    \item 3,4-dimetilhept-2-eno
    \item penta-1,3-dieno
    \item etanamina
    \item butan-2-ol
    \item dietilamina
    \item antraceno
    \item propino
    \item \textit{o}-dimetilbenceno
  \end{enumerate}
\end{exercise}

\begin{solution}
  \begin{enumerate}
    \item \ch{CH3-CH=C(CH3)-CH(CH3)-CH2-CH2-CH3}
    \item \ch{CH2=CH-CH=CH-CH3}
    \item \ch{CH3-CH2-NH2}
    \item \ch{CH3-CH2-CHOH-CH3}
    \item \ch{CH3-CH2-NH-CH2-CH3}
    \item \chemfig{[:-120]**6(----**6(--**6(------)----)--)}
    \item \ch{CH3-C+CH}
    \item \chemfig{[:-120]**6(----(-CH_3)-(-CH_3)-)}
  \end{enumerate}
\end{solution}



\begin{exercise}[
    tags    = {orgánica,nomenclatura,múltiple,2B},
    topics  = {química orgánica,formulación,nomenclatura},
    source  = {Química 2B SAN 2016, p392, e23},
  ]
  Formula los siguientes compuestos:

  \begin{enumerate}
    \item nitrobenceno
    \item propanal
    \item metanol
    \item ácido benzoico
    \item but-2-eno
    \item ácido metanoico
    \item propan-1-ol
    \item pentanal
  \end{enumerate}
\end{exercise}

\begin{solution}
  \begin{enumerate}
    \item \chemfig{[:-120]**6(-----(-NO_2)-)}
    \item \ch{CH3-CH2-CHO}
    \item \ch{CH3OH}
    \item \chemfig{[:-120]**6(-----(-COOH)-)}
    \item \ch{CH3-CH=CH-CH3}
    \item \ch{HCOOH}
    \item \ch{CH3-CH2-CH2OH}
    \item \ch{CH3-CH2-CH2-CH2-CHO}
  \end{enumerate}
\end{solution}



\begin{exercise}[
    tags    = {orgánica,nomenclatura,múltiple,2B},
    topics  = {química orgánica,formulación,nomenclatura},
    source  = {Química 2B SAN 2016, p392, e24},
  ]
  Formula los siguientes compuestos:

  \begin{enumerate}
    \item etanoato de etilo
    \item 3,4-dimetilpent-1-ino
    \item metilbenceno
    \item dietilamina
    \item metilbutanona
    \item N-metilacetamida
    \item pentanamina
    \item propano-1,2-diol
  \end{enumerate}
\end{exercise}

\begin{solution}
  \begin{enumerate}
    \item \ch{CH3-COO-CH2-CH3}
    \item \ch{CH+C-CH(CH3)-CH(CH3)-CH3}
    \item \chemfig{[:-120]**6(-----(-CH_3)-)}
    \item \ch{CH3-CH2-NH-CH2-CH3}
    \item \ch{CH3-CH(CH3)-CO-CH3}
    \item \ch{CH3-CONH-CH3}
    \item \ch{CH3-CH2-CH2-CH2-CH2-NH2}
    \item \ch{CH3-CHOH-CH2OH}
  \end{enumerate}
\end{solution}


\end{multicols}

\end{document}

% TEMPLATES

\begin{enumerate}
  \item \ch{}
  \item \ch{}
  \item \ch{}
  \item \ch{}
  \item \ch{}
  \item \ch{}
  \item \ch{}
  \item \ch{}
\end{enumerate}
\end{exercise}

\begin{solution}
\begin{enumerate}
  \item
  \item
  \item
\end{enumerate}
\end{solution}


  \begin{enumerate}
    \item \ch{}
    \item \ch{}
    \item \ch{}
    \item \ch{}
    \item \ch{}
    \item \ch{}
    \item \ch{}
    \item \ch{}
    \item \ch{}
    \item \ch{}
    \item \ch{}
    \item \ch{}
  \end{enumerate}
\end{exercise}
