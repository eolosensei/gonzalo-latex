\section{Ejercicios de formulación}

\begin{exercise}[
    tags    = {inorgánica,formulación,múltiple,2B},
    topics  = {química inorgánica,formulación,nomenclatura},
    source  = {Química 2B SAN 2016, p372, e12},
  ]
  Nombra las siguientes sustancias

  \begin{enumerate}
    \item tetracloruro de estaño
    \item hidrogenocarbonato de potasio
    \item cromato de cobre(II)
    \item hidrogenosulfuro de bario
    \item hidróxido de aluminio
    \item óxido de plata
    \item hidróxido de zinc
    \item bromato de calcio
    \item hidruro de berilio
    \item nitrito de plata
  \end{enumerate}
\end{exercise}

\begin{solution}
  \begin{enumerate}\begin{multicols}{2}
    \item \ch{SnCl4}
    \item \ch{KHCO3}
    \item \ch{CuCrO4}
    \item \ch{Ba(HS)2}
    \item \ch{Al(OH)3}
    \item \ch{Ag2O}
    \item \ch{Zn(OH)2}
    \item \ch{CaBrO3}
    \item \ch{BeH2}
    \item \ch{AgNO3}
  \end{multicols}\end{enumerate}
\end{solution}




\begin{exercise}[
    tags    = {inorgánica,formulación,múltiple,2B},
    topics  = {química inorgánica,formulación,nomenclatura},
    source  = {Química 2B SAN 2016, p372, e13},
  ]
  Nombra las siguientes sustancias

  \begin{enumerate}
    \item perbromato de hierro(II)
    \item pentasulfuro de diarsénico
    \item sulfuro de arsénico(V)
    \item monóxido de níquel
    \item pentaóxido de difósforo
    \item bromuro de litio
    \item óxido de níquel(II)
    \item ácido sulfuroso
    \item óxido de fósforo(V)
    \item ácido yodoso
  \end{enumerate}
\end{exercise}

\begin{solution}
  \begin{enumerate}\begin{multicols}{2}
    \item \ch{Fe(BrO4)2}
    \item \ch{As2S5}
    \item \ch{As2S5}
    \item \ch{NiO}
    \item \ch{P2O5}
    \item \ch{LiBr}
    \item \ch{NiO}
    \item \ch{H2SO3}
    \item \ch{P2O5}
    \item \ch{HIO2}
  \end{multicols}\end{enumerate}
\end{solution}




\begin{exercise}[
    tags    = {inorgánica,formulación,múltiple,2B},
    topics  = {química inorgánica,formulación,nomenclatura},
    source  = {Química 2B SAN 2016, p372, e14},
  ]
  Nombra las siguientes sustancias

  \begin{enumerate}
    \item disulfuro de carbono
    \item sulfuro de carbono(IV)
    \item seleniuro de hidrógeno
    \item hidrogenosulfato de sodio
    \item dihidrogenofosfato de calcio
    \item clorita de sodio
    \item arsano
    \item yodato de potasio
    \item ácido fosforoso
    \item sulfato de plata
  \end{enumerate}
\end{exercise}

\begin{solution}
  \begin{enumerate}\begin{multicols}{2}
    \item \ch{CS2}
    \item \ch{CS2}
    \item \ch{H2S}
    \item \ch{NaHSO4}
    \item \ch{Ca(H2PO4)2}
    \item \ch{NaClO2}
    \item \ch{AsH3}
    \item \ch{KIO3}
    \item \ch{H3PO3}
    \item \ch{Ag2SO4}
  \end{multicols}\end{enumerate}
\end{solution}




\begin{exercise}[
    tags    = {inorgánica,formulación,múltiple,2B},
    topics  = {química inorgánica,formulación,nomenclatura},
    source  = {Química 2B SAN 2016, p372, e15},
  ]
  Nombra las siguientes sustancias

  \begin{enumerate}
    \item manganato de potasio
    \item hidrogenosulfato de hierro(II)
    \item tricloruro de bismuto
    \item carbonato de bario
    \item peróxido de potasio
    \item sulfuro de zinc
    \item sulfito de sodio
    \item ácido cloroso
    \item peróxido de sodio
    \item óxido de cobre(II)
  \end{enumerate}
\end{exercise}

\begin{solution}
  \begin{enumerate}\begin{multicols}{2}
    \item \ch{K2MnO4}
    \item \ch{Fe(HSO4)2}
    \item \ch{BiCl3}
    \item \ch{BaCO3}
    \item \ch{K2O2}
    \item \ch{ZnS}
    \item \ch{Na2SO3}
    \item \ch{HClO2}
    \item \ch{Na2O2}
    \item \ch{CuO}
  \end{multicols}\end{enumerate}
\end{solution}




\begin{exercise}[
    tags    = {inorgánica,formulación,múltiple,2B},
    topics  = {química inorgánica,formulación,nomenclatura},
    source  = {Química 2B SAN 2016, p372, e16},
  ]
  Nombra las siguientes sustancias

  \begin{enumerate}
    \item perclorato de potasio
    \item tetrafluoruro de estaño
    \item permanganato de litio
    \item permanganato de sodio
    \item tetrabromuro de carbono
    \item cloruro de amonio
    \item nitrato de hierro(II)
    \item nitrato de rubidio
    \item nitrato de zinc(II)
    \item cloruro de hierro(II)
  \end{enumerate}
\end{exercise}

\begin{solution}
  \begin{enumerate}\begin{multicols}{2}
    \item \ch{KClO4}
    \item \ch{SnF4}
    \item \ch{LiMnO4}
    \item \ch{NaMnO4}
    \item \ch{CBr4}
    \item \ch{NH4Cl}
    \item \ch{Fe(NO3)2}
    \item \ch{RbNO3}
    \item \ch{Zn(NO3)2}
    \item \ch{FeCl2}
  \end{multicols}\end{enumerate}
\end{solution}




\begin{exercise}[
    tags    = {inorgánica,formulación,múltiple,2B},
    topics  = {química inorgánica,formulación,nomenclatura},
    source  = {Química 2B SAN 2016, p372, e17},
  ]
  Nombra las siguientes sustancias

  \begin{enumerate}
    \item clorato de cobalto(III)
    \item fosfato de níquel(II)
    \item hidróxido de paladio(II)
    \item hidróxido de magnesio
    \item hidróxido de plomo(IV)
    \item dióxido de titanio
    \item amoniaco
    \item ácido perclórico
    \item sulfuro de cadmio
    \item óxido de cromo(III)
  \end{enumerate}
\end{exercise}

\begin{solution}
  \begin{enumerate}\begin{multicols}{2}
    \item \ch{Co(ClO3)3}
    \item \ch{Ni3(PO4)2}
    \item \ch{Pd(OH)2}
    \item \ch{Mg(OH)2}
    \item \ch{Pb(OH)4}
    \item \ch{TiO2}
    \item \ch{NH3}
    \item \ch{HClO4}
    \item \ch{CdS}
    \item \ch{Cr2O3}
  \end{multicols}\end{enumerate}
\end{solution}




\begin{exercise}[
    tags    = {inorgánica,formulación,múltiple,2B},
    topics  = {química inorgánica,formulación,nomenclatura},
    source  = {Química 2B SAN 2016, p372, e18},
  ]
  Nombra las siguientes sustancias

  \begin{enumerate}
    \item hidróxido de hierro(III)
    \item carbonato de rubidio
    \item nitrato de magnesio
    \item hidruro de níquel(III)
    \item óxido de molibdeno(IV)
    \item ácido crómico
    \item sulfito de hierro(II)
    \item ácido bromoso
    \item sulfato de hierro(III)
    \item cromato férrico
  \end{enumerate}
\end{exercise}

\begin{solution}
  \begin{enumerate}\begin{multicols}{2}
    \item \ch{Fe(OH)3}
    \item \ch{Rb2CO3}
    \item \ch{Mg(NO3)2}
    \item \ch{NiH3}
    \item \ch{MbO2}
    \item \ch{H2CrO4}
    \item \ch{FeSO3}
    \item \ch{HBrO2}
    \item \ch{Fe2(SO4)3}
    \item \ch{Fe2(CrO4)3}
  \end{multicols}\end{enumerate}
\end{solution}




\begin{exercise}[
    tags    = {inorgánica,formulación,múltiple,2B},
    topics  = {química inorgánica,formulación,nomenclatura},
    source  = {Química 2B SAN 2016, p372, e19},
  ]
  Nombra las siguientes sustancias

  \begin{enumerate}
    \item trihidruro de níquel
    \item sulfuro de plomo(IV)
    \item óxido de arsénico(V)
    \item dihidróxido de hierro
    \item carbonato de calcio
    \item nitrato de amonio
    \item disulfuro de plomo
    \item cromato de potasio
  \end{enumerate}
\end{exercise}

\begin{solution}
  \begin{enumerate}\begin{multicols}{2}
    \item \ch{NiH3}
    \item \ch{PbS2}
    \item \ch{As2O5}
    \item \ch{Fe(OH)2}
    \item \ch{CaCO3}
    \item \ch{NH4NO3}
    \item \ch{PbS2}
    \item \ch{K2CrO4}
  \end{multicols}\end{enumerate}
\end{solution}
