\section{Ejercicios variados}

\begin{exercise}[
    tags    = {inorgánica,compuestos binarios,sales binarias,sales,hidruros},
    topics  = {química inorgánica,formulación,nomenclatura},
    source  = {SAN Formulación, p26, e28},
  ]
  Nombra y formula los siguientes compuestos binarios, indicando cuando proceda su nomenclatura de composición (sistemática) y su nombre de Stock:

  \begin{enumerate}
    \item \ch{KH}
    \item yoduro de cesio
    \item hidruro de titanio(III)
    \item \ch{SrCl2}
    \item cloruro de hidrógeno
    \item arseniuro de galio
    \item \ch{Li3N}
    \item \ch{CF4}
    \item yoduro de plomo(II)
    \item seleniuro de hidrógeno
  \end{enumerate}
\end{exercise}

\begin{solution}
  \begin{enumerate}
    \item hidruro de potasio, hidruro de potasio
    \item \ch{CsI}, yoduro de cesio
    \item \ch{TiH3}, trihidruro de titanio
    \item dicloruro de estroncio, cloruro de estroncio
    \item \ch{HCl}, cloruro de hidrógeno / ácido clorhídrico
    \item \ch{GaAs}, arsenuro de galio
    \item nitruro de trilitio, nitruro de litio
    \item tetrafluoruro de carbono, fluoruro de carbono(IV)
    \item \ch{PbI2}, diyoduro de plomo
    \item \ch{H2Se}, selenuro de dihidrógeno
  \end{enumerate}
\end{solution}



\begin{exercise}[
    tags    = {inorgánica,compuestos binarios,sales binarias,sales},
    topics  = {química inorgánica,formulación,nomenclatura},
    source  = {SAN Formulación, p26, e29},
  ]
  Nombra y formula los siguientes compuestos binarios, indicando cuando proceda su nomenclatura de composición (sistemática) y su nombre de Stock:

  \begin{enumerate}
    \item pentacloruro de fósforo
    \item cloruro de antimonio(III)
    \item \ch{ScI3}
    \item bromuro de plata
    \item cloruro de mercurio(II)
    \item \ch{Zn3P2}
    \item teluluro de cadmio
    \item bromuro de platino(II)
    \item \ch{Na2Se}
    \item tricloruro de oro
  \end{enumerate}
\end{exercise}

\begin{solution}
  \begin{enumerate}
    \item \ch{PCl5}, cloruro de fósforo(V)
    \item \ch{SbCl3}, tricloruro de antimonio
    \item triyoduro de escandio, yoduro de escandio
    \item \ch{AgBr}, bromuro de plata
    \item \ch{HgCl2}, dicloruro de mercurio;
    \item difosfuro de trizinc, fosfuro de zinc
    \item \ch{CdTe}, telururo de cadmio
    \item \ch{PtBr2}, dibromuro de platino
    \item selenuro de disodio, selenuro de sodio
    \item \ch{AuCl3}, cloruro de oro(III)
  \end{enumerate}
\end{solution}




\begin{exercise}[
    tags    = {inorgánica,compuestos binarios,sales binarias,sales},
    topics  = {química inorgánica,formulación,nomenclatura},
    source  = {SAN Formulación, p26, e30},
  ]
  Nombra y formula los siguientes compuestos binarios, indicando cuando proceda su nomenclatura de composición (sistemática) y su nombre de Stock:

  \begin{enumerate}
    \item teluluro de hidrógeno
    \item \ch{SiC}
    \item dibromuro de calcio
    \item fosfuro de plata
    \item \ch{NiAs}
    \item trisulfuro de diantimonio
    \item hidruro de bismuto(III)
    \item heptafluoruro de yodo
    \item \ch{H2S}
    \item dihidruro de cobre
  \end{enumerate}
\end{exercise}

\begin{solution}
  \begin{enumerate}
    \item \ch{H2Te}, telururo de dihidrógeno
    \item carburo de silicio, carburo de silicio(IV)
    \item \ch{CaBr2}, bromuro de calcio
    \item \ch{Ag3P}, fosfuro de triplata
    \item arsenuro de níquel, arsenuro de níquel(III)
    \item \ch{Sb2S3}, sulfuro de antimonio(III)
    \item \ch{BiH3}, trihidruro de bismuto
    \item \ch{IF7}, fluoruro de yodo(VII)
    \item sulfuro de dihidrógeno, sulfuro de hidrógeno
    \item \ch{CuH2}, sulfuro de cobre(II)
  \end{enumerate}
\end{solution}




\begin{exercise}[
    tags    = {inorgánica,compuestos binarios,óxidos,peróxidos},
    topics  = {química inorgánica,formulación,nomenclatura},
    source  = {SAN Formulación, p27, e31},
  ]
  Nombra y formula los siguientes compuestos binarios, indicando cuando proceda su nomenclatura de composición (sistemática) y su nombre de Stock:

  \begin{enumerate}
    \item \ch{FeO}
    \item trióxido de bromo
    \item óxido de mercurio(II)
    \item \ch{Li2O2}
    \item dióxido de magnesio
    \item peróxido de aluminio
    \item pentaóxido de dinitrógeno
    \item \ch{SO2}
    \item monóxido de carbono
    \item óxido de selenio(IV)
  \end{enumerate}
\end{exercise}

\begin{solution}
  \begin{enumerate}
    \item monóxido de hierro, óxido de hierro(II)
    \item \ch{Cr2O3}, óxido de cromo(III)
    \item \ch{HgO}, monóxido de mercurio
    \item dióxido de dilitio, peróxido de litio
    \item \ch{MgO2}, peróxido de magnesio
    \item \ch{Al2O6}, hexaóxido de dialuminio
    \item \ch{N2O5}, óxido de nitrógeno(V)
    \item dióxido de azufre, óxido de azufre(IV)
    \item \ch{CO}, óxido de carbono(II)
    \item \ch{SeO3}, trióxido de selenio
  \end{enumerate}
\end{solution}




\begin{exercise}[
    tags    = {inorgánica,compuestos binarios,óxidos,peróxidos},
    topics  = {química inorgánica,formulación,nomenclatura},
    source  = {SAN Formulación, p27, e32},
  ]
  Nombra y formula los siguientes compuestos binarios, indicando cuando proceda su nomenclatura de composición (sistemática) y su nombre de Stock:

  \begin{enumerate}
    \item \ch{O3Cl2}
    \item dióxido de manganeso
    \item óxido de fósforo(V)
    \item \ch{CrO3}
    \item diyoduro de heptaoxígeno
    \item peróxido de estaño(IV)
    \item \ch{H2O2}
    \item dióxido de dilitio
    \item óxido de titanio(II)
    \item \ch{SiO2}
  \end{enumerate}
\end{exercise}

\begin{solution}
  \begin{enumerate}
    \item dicloruro de trioxígeno, -
    \item \ch{MnO2}, óxido de manganeso(IV)
    \item \ch{P2O5}, pentaóxido de difósforo
    \item trióxido de cromo, óxido de cromo(VI)
    \item \ch{O7I2}, -
    \item \ch{SnO4}, tetraóxido de estaño
    \item dióxido de dihidrógeno, peróxido de hidrógen
    \item \ch{Li2O2}, peróxido de litio
    \item \ch{TiO}, monóxido de titanio
    \item dióxido de silicio, óxido de silicio(IV)
  \end{enumerate}
\end{solution}




\begin{exercise}[
    tags    = {inorgánica,compuestos binarios,óxidos,peróxidos,hidróxidos},
    topics  = {química inorgánica,formulación,nomenclatura},
    source  = {SAN Formulación, p27, e33},
  ]
  Nombra y formula los siguientes compuestos binarios, indicando cuando proceda su nomenclatura de composición (sistemática) y su nombre de Stock:

  \begin{enumerate}
    \item dióxido de bario
    \item hidróxido de zinc
    \item \ch{Au(OH)3}
    \item hidróxido de plata
    \item hidróxido de magnesio
    \item \ch{NH4OH}
    \item dióxido de paladio
    \item peróxido de rubidio
    \item \ch{TeO}
    \item trióxido de dicobalto
  \end{enumerate}
\end{exercise}

\begin{solution}
  \begin{enumerate}
    \item \ch{Ba(OH)2}, hidróxido de bario
    \item \ch{Zn(OH)2}, dihidróxido de zinc
    \item trihidróxido de oro, hidróxido de oro(III)
    \item \ch{AgOH}, hidróxido de plata
    \item \ch{Mg(OH)2}, dihidróxido de magnesio
    \item hidróxido de amonio, hidróxido de amonio
    \item \ch{PdO2}, óxido de paladio(IV)
    \item \ch{Rb2O2}, dióxido de dirubidio
    \item monóxido de teluro, óxido de teluro(II)
    \item \ch{Co2O3}, óxido de cobalto(III)
  \end{enumerate}
\end{solution}




\begin{exercise}[
    tags    = {inorgánica,ácidos,ácidos binarios,ácidos ternarios,oxoácidos},
    topics  = {química inorgánica,formulación,nomenclatura},
    source  = {SAN Formulación, p28, e34},
  ]
  Nombra y formula los siguientes compuestos, indicando cuando proceda su nomenclatura de hidrógeno y su nombre tradicional:

  \begin{enumerate}
    \item \ch{HF}
    \item ácido selenhídrico
    \item hidrogeno(oxidoclorato)
    \item dihidrogeno(tetraoxidosulfato)
    \item \ch{HMnO4}
    \item \ch{HI}
    \item dihidrogeno(teluro)
    \item ácido crómico
    \item dihidrogeno(tetraoxidocromato)
    \item hidrogeno(tetraoxidobromato)
  \end{enumerate}
\end{exercise}

\begin{solution}
  \begin{enumerate}
    \item ácido fluorhídrico, hidrogeno(fluoruro)
    \item \ch{H2Se}, dihidrogeno(selenuro)
    \item \ch{HClO}, ácido hipocloroso
    \item \ch{H2SO4}, ácido sulfúrico
    \item ácido permangánico,  hidrogeno(tetraoxidomanganato)
    \item ácido yodhídrico, hidrogeno(yoduro)
    \item \ch{H2Te}, ácido telurhídrico
    \item \ch{H2Cr2O7}, dihidrogeno(heptaoxidodicromato)
    \item \ch{H2CrO4}, ácido crómico
    \item \ch{HBrO4}, ácido perbrómico
  \end{enumerate}
\end{solution}




\begin{exercise}[
  tags    = {inorgánica,ácidos,ácidos binarios,ácidos ternarios,oxoácidos},
  topics  = {química inorgánica,formulación,nomenclatura},
  source  = {SAN Formulación, p28, e35},
  ]
  Nombra y formula los siguientes compuestos, indicando cuando proceda su nomenclatura de hidrógeno y su nombre tradicional:

  \begin{enumerate}
    \item ácido clorhídrico
    \item ácido mangánico
    \item \ch{H2S}
    \item ácido arsénico
    \item trihidrógeno(tetraoxidofosfato)
    \item \ch{H3PO3}
    \item ácido hiponitroso
    \item trihidrógeno(trioxidoarsenato)
    \item \ch{H2TeO2}
    \item bromuro de hidrógeno
  \end{enumerate}
\end{exercise}

\begin{solution}
  \begin{enumerate}
    \item \ch{HCL}, cloruro de hidrógeno
    \item \ch{H2MnO4}, dihidrogeno(tetraoxidomanganato)
    \item ácido sulfhídrico, sulfuro de dihidrógeno
    \item \ch{H3AsO4}, trihidrogeno(tetraoxidoarsenato)
    \item \ch{H3PO4}, ácido fosfórico
    \item ácido fosforoso, trihidrogeno(trioxidofosfato)
    \item \ch{HNO}, hidrogeno(óxidonitrato)
    \item \ch{H3AsO3}, ácido arsenoso
    \item ácido hipoteluroso, dihidrogeno(dióxidotelurato)
    \item \ch{HBr}, ácido bromhídrico
  \end{enumerate}
\end{solution}




\begin{exercise}[
    tags    = {inorgánica,ácidos,ácidos binarios,ácidos ternarios,oxoácidos},
    topics  = {química inorgánica,formulación,nomenclatura},
    source  = {SAN Formulación, p28, e36},
  ]
  Nombra y formula los siguientes compuestos, indicando cuando proceda su nomenclatura de hidrógeno y su nombre tradicional:

  \begin{enumerate}
    \item \ch{H3BO3}
    \item \ch{HNO3}
    \item ácido selenioso
    \item ácido nítrico
    \item \ch{H2SO3}
    \item ácido carbónico
    \item hidrógeno(dioxidoclorato)
    \item \ch{H2SeO4}
    \item ácido hipobromoso
    \item hidrógeno(trioxidoclorato)
  \end{enumerate}
\end{exercise}

\begin{solution}
  \begin{enumerate}
    \item ácido bórico, trihidrogeno(trioxidoborato)
    \item ácido nitroso, hidrogeno(dioxidonitrato)
    \item \ch{H2SeO3}, dihidrogeno(trioxidoselenato)
    \item \ch{HNO3}, hidrogeno(trioxidonitrato)
    \item ácido sulfuroso, dihidrogeno(trioxidosulfato)
    \item \ch{H2CO3}, dihidrogeno(trioxidocarbonato)
    \item \ch{HClO2}, ácido cloroso
    \item ácido selénico, dihidrogeno(tetraoxidoselenato)
    \item \ch{HBrO}, hidrogeno(oxidobromato)
    \item \ch{HCLO3}, ácido clórico
  \end{enumerate}
\end{solution}



\newpage



\begin{exercise}[
    tags    = {inorgánica,sales, sales binarias,sales ternarias},
    topics  = {química inorgánica,formulación,nomenclatura},
    source  = {SAN Formulación, p29, e37},
  ]
  Nombra y formula las siguientes sales, indicando cuando proceda su nomenclatura de composición (sistemática) y su nombre de Stock:

  \begin{enumerate}
    \item \ch{BeCl2}
    \item hidrogenosulfuro de calcio
    \item tetraoxidosulfato de disodio
    \item \ch{KMnO4}
    \item carbonato de amonio
    \item tris(hidrogenoselenuro) de níquel
    \item \ch{CoSO3}
    \item hidrogenosufito de zinc
    \item heptaoxidodicromato de dilitio
    \item \ch{Ca(CN)2}
  \end{enumerate}
\end{exercise}

\begin{solution}
  \begin{enumerate}
    \item cloruro de berilio, dicloruro de berilio
    \item \ch{Ca(HS)2}, bis(hidrogenosulfuro) de calcio
    \item \ch{Na2SO4}, sulfato de sodio
    \item permanganato de potasio, tetraoxidomanganato de potasio
    \item \ch{(NH4)2CO3}, trioxidocarbonato de bis(amonio)
    \item \ch{Ni(HSe)3}, hidrogenoselenuro de níquel(III)
    \item sulfito de cobalto(II), trioxidosulfato de cobalto
    \item \ch{Zn(HSO3)2}, bis[hidrogeno(trioxidosulfato)] de zinc
    \item \ch{Li2Cr2O7}, dicromato de litio
    \item cianuro de calcio, dicianuro de calcio
  \end{enumerate}
\end{solution}




\begin{exercise}[
    tags    = {inorgánica,sales, sales binarias,sales ternarias},
    topics  = {química inorgánica,formulación,nomenclatura},
    source  = {SAN Formulación, p29, e38},
  ]
  Nombra y formula las siguientes sales, indicando cuando proceda su nomenclatura de composición (sistemática) y su nombre de Stock:

  \begin{enumerate}
    \item nitrato de cromo(III)
    \item tris(tetraoxidoyodato) de galio
    \item \ch{Sn(ClO)4}
    \item fosfato de calcio
    \item tris(tetraoxidosulfato) de dialuminio
    \item \ch{Pb(NO2)4}
    \item cromato de cesio
    \item trioxidoborato de níquel
    \item \ch{MnSO4}
    \item fluoruro de bario
  \end{enumerate}
\end{exercise}

\begin{solution}
  \begin{enumerate}
    \item \ch{Cr(NO3)3}, tris(trioxidocromato) de cromo
    \item \ch{Ga(IO4)3}, peryodato de galio
    \item hipoclorito de estaño(IV), tetrakis(oxidoclorato) de estaño
    \item \ch{Ca3(PO4)2}, bis(tetraoxidofosfato) de tricalcio
    \item \ch{Al2(SO4)3}, sulfato de aluminio
    \item nitrito de plomo(IV), tetrakis(dioxidonitrato) de plomo
    \item \ch{Cs2CrO4}, tetraoxidocromato de dicesio
    \item \ch{NiBO3}, borato de níquel(III)
    \item sulfato de manganeso(II), tetraoxidosulfato de manganeso
    \item \ch{BaF2}, difluoruro de bario
  \end{enumerate}
\end{solution}




\begin{exercise}[
    tags    = {inorgánica,sales,sales binarias,sales ternarias},
    topics  = {química inorgánica,formulación,nomenclatura},
    source  = {SAN Formulación, p29, e39},
  ]
  Nombra y formula las siguientes sales, indicando cuando proceda su nomenclatura de composición (sistemática) y su nombre de Stock:

  \begin{enumerate}
    \item hidrogenotelururo de rubidio
    \item \ch{KClO2}
    \item sulfito de titanio(III)
    \item dioxidotelurato de dicobre
    \item \ch{NH4HSO2}
    \item hipoclorito de plata
    \item tetraoxidomanganato de dilitio
    \item \ch{Ca3(AsO3)2}
    \item yodato de magnesio
    \item trioxidonitrato de amonio
  \end{enumerate}
\end{exercise}

\begin{solution}
  \begin{enumerate}
    \item \ch{RbHTe}, hidrogenotelururo de rubidio
    \item clorita de potasio, dioxidoclorato de potasio
    \item \ch{Ti2(SO3)3}, tris(trioxidosulfato) de dititanio
    \item \ch{Cu2TeO2}, hipotelurito de cobre(I)
    \item hidrogenohiposulfito de amonio, hidrogeno(dioxidosulfato) de amonio
    \item \ch{AgClO}, oxidoclorato de plata
    \item \ch{Li2MnO4}, manganato de litio
    \item arsenito de calcio, bis(trioxidoarsenato) de tricalcio
    \item \ch{Mg(IO3)2}, bis(trioxidoyodato) de magnesio
    \item \ch{NH4NO3}, nitrato de amonio
  \end{enumerate}
\end{solution}




\begin{exercise}[
    tags    = {inorgánica,óxidos,compuestos binarios,2B},
    topics  = {química inorgánica,formulación,nomenclatura},
    source  = {Química 2B OXF 2016, p342, e5},
  ]
  A continuación se indica la fórmula o el nombre de una serie de compuestos. Completa la información mostrando la fórmula y el nombre de cada uno tanto con prefijos multiplicadores como con número de oxidación o número de carga.

  \begin{enumerate}
    \item dióxido de silicio
    \item \ch{Li2O}
    \item \ch{ZnO}
    \item óxido de cobre(I)
    \item \ch{CaO}
    \item \ch{Cr2O3}
    \item óxido de boro(III)
    \item óxido de magnesio
    \item \ch{SeO3}
    \item pentaóxido de dinitrógeno
  \end{enumerate}
\end{exercise}




\begin{exercise}[
    tags    = {inorgánica,compuestos ternarios,oxoácidos,2B},
    topics  = {química inorgánica,formulación,nomenclatura},
    source  = {Química 2B OXF 2016, p344, e8 y p345, e9},
    print   = false,
    use     = false,
  ]
  Completa la tabla siguiente indicando la fórmula o el nombre de hidrógeno de cada compuesto:

  \begin{tabular}{cll}
    Formula      & Nom. de hidrógeno & Nom. tradicional \\ \toprule
    \gexBinRow{HClO3}{}{}
    \gexBinRow{}{trihidrogeno(trioxidoborato)}{ácido bórico}
    \gexBinRow{H2SO3}{}{}
    \gexBinRow{}{hidrogeno(tetraoxidoyodato)}{ácido peryódico}
    \gexBinRow{H2CrO4}{}{}
    \gexBinRow{}{dihidrogeno(trioxidotelurato)}{ácido teluroso}
    \gexBinRow{HBrO2}{}{}
    \gexBinRow{}{dihidrogeno(tetraoxidomanganato)}{ácido mangánico}
    \gexBinRow{H3AsO4}{}{}
    \gexBinRow{}{trihidrogeno(trioxidofosfato)}{ácido fosforoso}
    \gexBinRow{}{dihidrogeno(trioxidocarbonato)}{ácido carbónico}
    \gexBinRow{}{hidrogeno(tetraoxidomanganato)}{ácido permangánico}
    \gexBinRow{H3PO4}{}{}
    \gexBinRow{}{dihidrogeno(heptaoxidodicromato)}{ácido dicrómico}
    \gexBinRow{HNO}{}{}
    \gexBinRow{}{trihidrogeno(trioxidoarsenato)}{}
    \gexBinRow{HNO3}{}{}
    \gexBinRow{}{hidrogeno(oxidoclorato)}{ácido hipocloroso}
    \gexBinRow{H2SeO2}{}{}
  \end{tabular}
\end{exercise}




\begin{exercise}[
    tags    = {inorgánica,sales, sales ternarias, oxosales,2B},
    topics  = {química inorgánica,formulación,nomenclatura},
    source  = {Química 2B OXF 2016, p347, e12},
  ]
  Formula los siguientes compuestos

  \begin{enumerate}
    \item Hiposulfito de oro(III)
    \item Carbonato de sodio
    \item Dicromato de zinc
    \item Fosfato de aluminio
    \item Bromuro de amonio
    \item Perclorato de cromo(II)
    \item Bistrioxidoborato de trihierro
    \item Trioxidoarsenato de trisamonio
    \item Bis(hidrogenoselenuro) de mercurio
    \item Nitrito de cobalto(II)
    \item Oxidoyodato de oro
    \item Fosfito de magnesio
    \item Hidrogenosulfato de níquel(II)
    \item Clorato de cesio
    \item Sulfato de aluminio
    \item Tris(trioxidoclorato) de cobalto
    \item Hidrogenocarbonato de calcio
    \item Manganato de bismuto(III)
    \item Tris(trioxidonitrato) de cromo
    \item Yoduro de plomo(IV)
  \end{enumerate}
\end{exercise}
